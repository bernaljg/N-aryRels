\section{Conclusion}
In this paper we have introduced the first massively distributed architecture for deep reinforcement learning. The \emph{Gorila} architecture acts and learns in parallel, using a distributed replay memory and distributed neural network. We applied Gorila to  an asynchronous variant of the state-of-the-art DQN algorithm. A single machine had previously achieved state-of-the-art results in the challenging suite of Atari 2600 games, but it was not previously known whether the good performance of DQN would continue to scale with additional computation. By leveraging massive parallelism, Gorila DQN significantly outperformed single-GPU DQN on 41 out of 49 games; achieving by far the best results in this domain to date. Gorila takes a further step towards fulfilling the promise of deep learning in RL: a scalable architecture that performs better and better with increased computation and memory.

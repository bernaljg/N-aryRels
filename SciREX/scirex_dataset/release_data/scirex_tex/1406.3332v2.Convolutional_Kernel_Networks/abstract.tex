An important goal in visual recognition is to devise image representations that
are invariant to particular transformations. In this paper, we address this
goal with a new type of convolutional neural network (CNN) whose invariance is
encoded by a reproducing kernel. Unlike traditional approaches where neural
networks are learned either to represent data or for solving a classification
task, our network learns to approximate the kernel feature map on training data.

Such an approach enjoys several benefits over classical ones.  First, by
teaching CNNs to be invariant, we obtain simple network architectures that
achieve a similar accuracy to more complex ones, while being easy to train and
robust to overfitting. Second, we bridge a gap between the neural network
literature and kernels, which are natural tools to model invariance.  We
evaluate our methodology on visual recognition tasks where CNNs have proven to
perform well, \eg, digit recognition with the MNIST dataset, and the more
challenging CIFAR-10 and STL-10 datasets, where our accuracy is competitive
with the state of the art.

% Template for ICASSP-2018 paper; to be used with:
%          spconf.sty  - ICASSP/ICIP LaTeX style file, and
%          IEEEbib.bst - IEEE bibliography style file.
% --------------------------------------------------------------------------
\pdfoutput=1
\documentclass{article}
\usepackage{authblk}
\usepackage{bm}
\usepackage{booktabs}
\usepackage{float}
\usepackage[hang,flushmargin]{footmisc}
\usepackage{makecell}
\usepackage{microtype}
\usepackage{multirow}
\usepackage{paralist}
\usepackage{spconf,amsmath,graphicx}
\usepackage{todonotes}
\usepackage{xspace}

\newcommand{\eg}{e.g.,\xspace}
\newcommand{\etc}{etc.\xspace}
\newcommand{\ie}{i.e.,\xspace}

\makeatletter
\renewcommand\AB@affilsepx{, \protect\Affilfont}
\makeatother
\renewcommand{\thefootnote}{\fnsymbol{footnote}}
\interfootnotelinepenalty=10000



% Example definitions.
% --------------------
\def\x{{\mathbf x}}
\def\L{{\cal L}}

% Title.
% ------
\title{\large \bf Natural TTS Synthesis By Conditioning WaveNet On Mel Spectrogram Predictions}

\renewcommand*{\Authfont}{\itshape}
\author[1]{Jonathan Shen}
\author[1]{Ruoming Pang}
\author[1]{Ron J. Weiss}
\author[1]{Mike Schuster}
\author[1]{Navdeep Jaitly}
\author[2]{Zongheng Yang\thanks{Work done while at Google.}}
\author[1]{Zhifeng Chen}
\author[1]{Yu Zhang}
\author[1]{Yuxuan Wang}
\author[1]{RJ Skerry-Ryan}
\author[1]{Rif A. Saurous}
\author[1]{Yannis Agiomyrgiannakis}
\author[1]{Yonghui Wu}
\affil[1]{Google, Inc.}
\affil[2]{University of California, Berkeley}
\affil[ ]{\texttt{\{jonathanasdf,rpang,yonghui\}@google.com}}
\date{}

\begin{document}
\ninept
%
\maketitle
%

\begin{abstract}
This paper describes Tacotron~2,
a neural network architecture for speech synthesis directly
from text. The system is composed of a recurrent sequence-to-sequence feature
prediction network that maps character embeddings to
mel-scale spectrograms, followed by a modified WaveNet model acting as a vocoder
to synthesize time-domain waveforms from those spectrograms. Our model achieves
a mean opinion score (MOS) of $4.53$ comparable to a MOS of $4.58$ for
professionally recorded speech.
To validate our design choices, we present ablation studies of key components of
our system and evaluate the impact of using mel spectrograms as the conditioning
input to WaveNet instead of linguistic, duration, and $F_0$ features. We further
show that using this compact acoustic intermediate representation allows
for a significant reduction in the size of the WaveNet architecture.
\end{abstract}
%
\begin{keywords}
Tacotron 2, WaveNet, text-to-speech
\end{keywords}
%

%%%%%%%%%%%%%%%%%%%%%
\section{Introduction}

Humans use different forms of communications such as speech, hand gestures and emotions. Being able to understand one's emotions and the encoded feelings is an important factor for an appropriate and correct understanding.


With the ongoing research in the field of robotics, especially in the field of humanoid robots, it becomes interesting to integrate these capabilities into machines allowing for a more diverse and natural way of communication. One example is the Software called EmotiChat~\cite{Anderson06areal-time}. This is a chat application with emotion recognition. The user is monitored and whenever an emotion is detected (smile, etc.), an emoticon is inserted into the chat window. Besides Human Computer Interaction other fields like surveillance or driver safety could also profit from it. Being able to detect the mood of the driver could help to detect the level of attention, so that automatic systems can adapt.\\
\let\thefootnote\relax\footnote{*F. Trier and P. Burkert contributed equally to this work.}


Many methods rely on extraction of the facial region. This can be realized through manual inference~\cite{4032815} or an automatic detection approach~\cite{Anderson06areal-time}.
Methods often involve the Facial Action Coding System (FACS) which describes the facial expression using Action Units (AU). An Action Unit is a facial action like "raising the Inner Brow". Multiple activations of AUs describe the facial expression~\cite{kumar2009face}. Being able to correctly detect AUs is a helpful step, since it allows making a statement about the activation level of the corresponding emotion. \\
Handcrafted facial landmarks can be used such as done by Kotsia et al.~\cite{4032815}. Detecting such landmarks can be hard, as the distance between them differs depending on the person~\cite{6998925}. Not only AUs can be used to detect emotions, but also texture. When a face shows an emotion the structure changes and different filters can be applied to detect this~\cite{6998925}.\\


\begin{figure}
   \centering
        \includegraphics[width=\columnwidth]{Fig1}
   \caption{Example images from the MMI (top) and CKP (bottom). The emotions from left to right are: \textit{Anger}, \textit{Sadness}, \textit{Disgust}, \textit{Happiness}, \textit{Fear}, \textit{Surprise}. The emotion \textit{Contempt} of the CKP set is not displayed.}\label{fig:example_images}
\end{figure}




The presented approach uses Artificial Neural Networks (ANN). ANNs differ, as they are trained on the data with less need for manual interference. 
Convolutional Neural Networks are a special kind of ANN and have been shown to work well as feature extractor when using images as input~\cite{donahue2013decaf} and are real-time capable. This allows for the usage of the raw input images without any pre- or postprocessing.\\
GoogleNet~\cite{DBLP:journals/corr/SzegedyLJSRAEVR14} is a deep neural network architecture that relies on CNNs. It has been introduced during the Image Net Large Scale Visual Recognition Challenge(ILSVRC) 2014. This challenge analyses the quality of different image classification approaches submitted by different groups. The images are separated into 1000 different classes organized by the WordNet hierarchy. In the challenge "object detection with additional training data" GoogleNet has achieved about 44\% precision~\cite{LSVRC-results}. These results have demonstrated the potential which lies in this kind of architecture. Therefore it has been used as inspiration for the proposed architecture.\\
The proposed network has been evaluated on the Extended Cohn-Kanade Dataset (Section~\ref{sec:ckp}) and on the MMI Dataset (Section~\ref{sec:mmi}). Typical pictures of persons showing emotions can be seen in Fig.~\ref{fig:example_images}.
The emotion \textit{Contempt} of the CKP set is not shown as no subject with consent for publication and an annotated emotion is part of the dataset. Results of experiments on these datasets demonstrate the success of using a deep layered neural network structure. With a 10-fold cross-validation a recognition accuracy of 99.6\% has been achieved. \\

The paper is arranged as follows: After this introduction, Related Work (Section~\ref{sec:related}) is presented which focuses on Emotion/Expression recognition and the various approaches scientists have taken. Next is Section~\ref{sec:background}, Background, which focuses on the main components of the architecture proposed in this article. Section~\ref{sec:datasets} contains a summary of the used Datasets. In Section~\ref{sec:architecture} the architecture is presented. This is followed by the experiments and its results (Section~\ref{sec:experiments}) . Finally, Section~\ref{sec:conclusion} summarizes the article and concludes the article.
%%%%%%%%%%%%%%%%%%%%%


%%%%%%%%%%%%%%%%%%%%%
\section{Model Architecture}
\label{sec:arch}

Our proposed system consists of two components, shown in Figure~\ref{fig:TTSArchitecture}:
\begin{inparaenum}[(1)]
  \item a recurrent sequence-to-sequence feature prediction network with
    attention which predicts a sequence of mel spectrogram frames from an
    input character sequence, and
  \item a modified version of WaveNet which generates time-domain waveform samples
    conditioned on the predicted mel spectrogram frames.
\end{inparaenum}

\subsection{Intermediate Feature Representation}

In this work we choose a low-level acoustic representation: mel-frequency
spectrograms, to bridge the two components. Using a representation
that is easily computed from time-domain waveforms allows us to train the two
components separately. This representation is also smoother than waveform
samples and is easier to train using a squared error loss because it is
invariant to phase within each frame.

A mel-frequency spectrogram is related to the linear-frequency spectrogram, \ie
the short-time Fourier transform (STFT) magnitude. It is obtained by applying a
nonlinear transform to the frequency axis of the STFT, inspired by measured
responses from the human auditory system, and summarizes the frequency content
with fewer dimensions.
%
Using such an auditory frequency scale has the effect of emphasizing details in
lower frequencies, which are critical to speech intelligibility, while
de-emphasizing high frequency details, which are dominated by fricatives and
other noise bursts and generally do not need to be modeled with high fidelity.
%
Because of these properties, features derived from the mel scale have
been used as an underlying representation for speech recognition for
many decades \cite{davis:mel}.

While linear spectrograms discard phase information (and are therefore lossy),
algorithms such as Griffin-Lim \cite{Griffin84signalestimation} are capable of
estimating this discarded information, which enables time-domain conversion
via the inverse short-time Fourier transform. Mel spectrograms discard even more
information, presenting a challenging inverse problem.
%
However, in comparison to the linguistic and acoustic features used in
WaveNet, the mel spectrogram is a simpler, lower-level acoustic
representation of audio signals. It should therefore be straightforward for a
similar WaveNet model conditioned on mel spectrograms to generate audio,
essentially as a neural vocoder.
%
% Furthermore, using 80 frequency buckets to compute a mel spectrogram with a
% frame hop of 12.5~ms, only 80 values were needed to represent each frame as
% compared to the 300 samples in a waveform sampled at 24~kHz, which should make it
% easier to predict.
%
Indeed, we will show that it is possible to generate high quality audio from mel
spectrograms using a modified WaveNet architecture.


\subsection{Spectrogram Prediction Network}
\label{ssec:c2f}

% char2mel params: http://cnsviewer2/cns/jn-d/home/jonathanasdf/brain/rs=6.3/char2mel_169147260/train/params.txt

%\subsubsection{Features}
%task.input.waveform_processor.frame_size_ms : 50.0
%task.input.waveform_processor.frame_step_ms : 12.5
%task.input.waveform_processor.magnitude_floor : 0.01
%task.input.waveform_processor.mel_channels : 80
%task.input.waveform_processor.mel_lower_edge_hertz : 125.0
%task.input.waveform_processor.mel_upper_edge_hertz : 7600.0
% wow: so wavenet extends the bandwidth!!!
%
%NOTE: pre_emphasis is *not* used in WaveformProcessor.LogMelScaleFilterbankEnergies.
%task.input.waveform_processor.pre_emphasis : 0.97
As in Tacotron, mel spectrograms are computed
through a short-time Fourier transform (STFT) using a 50~ms frame size, 12.5~ms
frame hop, and a Hann window function. We experimented with a 5~ms frame hop to
match the frequency of the conditioning inputs in the original WaveNet, but
the corresponding increase in temporal resolution resulted in significantly more
pronunciation issues.

% TODO(ronw): Is the below true and do we need to note it?
% Note that we do not use preemphasis.
%
We transform the STFT magnitude to the mel scale using an 80 channel
mel filterbank spanning 125~Hz to 7.6~kHz, followed by log dynamic
range compression.
%
Prior to log compression, the filterbank output magnitudes are clipped to a
minimum value of 0.01 in order to limit dynamic range in the logarithmic domain.

% \subsubsection{Encoder}
%task.encoder.emb_dim : 512
The network is composed of an encoder and a decoder with attention.
The encoder converts a character sequence into a hidden
feature representation which the decoder consumes to predict a
spectrogram.
%
Input characters are represented using a learned 512-dimensional character
embedding, which are passed through
%task.encoder.conv_dropout_prob : 0.5
%task.encoder.conv_layers : 3
%task.encoder.conv_tpl.activation : 'RELU'
%task.encoder.conv_tpl.batch_norm : True
%conv_p.filter_shape = [5, 1, p.emb_dim, p.emb_dim]
a stack of 3 convolutional layers each containing 512 filters with shape
$5\times1$, \ie where each filter spans 5 characters, followed by batch
normalization \cite{ioffe2015batch} and ReLU activations.
%
As in Tacotron, these convolutional layers model longer-term
context (\eg $N$-grams) in the input character sequence.
%task.encoder.num_lstm_layers : 1
%task.encoder.lstm_cell_size : 256
%task.encoder.lstm_tpl.zo_prob : 0.1
%
The output of the final convolutional layer is passed into a single
bi-directional \cite{Schuster:1997:BRN:2198065.2205129} LSTM
\cite{Hochreiter:1997:LSM:1246443.1246450} layer containing 512 units
(256 in each direction) to generate the encoded features.

% \subsubsection{Attention}
The encoder output is consumed by an attention network which
summarizes the full encoded sequence as a fixed-length context vector
for each decoder output step.
%
We use the location-sensitive attention from
\cite{chorowski2015attention}, which extends the additive attention
mechanism \cite{bahdanau2014neural} to use cumulative attention
weights from previous decoder time steps as an additional feature.
This encourages the model to move forward consistently through the
input, mitigating potential failure modes where some
subsequences are repeated or ignored by the decoder.
%
%task.decoder.attention.hidden_dim : 128
%task.decoder.attention.location_filter_size : 31
%task.decoder.attention.location_num_filters : 32
Attention probabilities are computed after projecting inputs % TODO: does this need an equation
and location features to 128-dimensional hidden representations.
Location features are computed using 32 1-D convolution filters of
length 31.

% \subsubsection{Decoder}
The decoder is an autoregressive recurrent neural network which
predicts a mel spectrogram from the encoded input sequence one
frame at a time.
%
%task.decoder.target_pre_net.activation : 'RELU'
%task.decoder.target_pre_net.batch_norm : False
%task.decoder.target_pre_net.dropout_prob : 0.5
%task.decoder.target_pre_net.hidden_layer_dims : [256, 256]
The prediction from the previous time step is first passed through a
small \emph{pre-net} containing 2 fully connected layers of 256 hidden ReLU units.
We found that the pre-net acting as an information bottleneck was essential for
learning attention.
%
%task.decoder.rnn_layers : 2
%task.decoder.rnn_cell_dim : 1024
%task.decoder.rnn_cell_tpl.zo_prob : 0.1
The pre-net output and attention context vector are concatenated and
passed through a stack of 2 uni-directional LSTM layers with 1024 units.
%
The concatenation of the LSTM output and the attention context vector is
projected through a linear transform to predict the target
spectrogram frame.
%task.decoder.post_edit_convnet_filter_shapes : [[5, 1, None, 512], [5, 1, 512, 512], [5, 1, 512, 512], [5, 1, 512, 512], [5, 1, 512, None]]
Finally, the predicted mel spectrogram is passed through a 5-layer convolutional
\emph{post-net} which predicts a residual to add to the prediction to improve the
overall reconstruction.
%
Each post-net layer is comprised of 512 filters with shape $5\times1$ with
batch normalization, followed by $\tanh$ activations on all but the final layer.

We minimize the summed mean squared error (MSE) from before and after the
post-net to aid convergence.  We also experimented with a log-likelihood loss by
modeling the output distribution with a Mixture Density Network
\cite{Bishop94mixturedensity,Schuster99onsupervised} to avoid assuming
a constant variance over time, but found that these were more difficult to
train and they did not lead to better sounding samples.

%->eos prediction
% https://cs.corp.google.com/piper///depot/google3/learning/brain/research/babelfish/tts/decoder.py?l=226
In parallel to spectrogram frame prediction, the concatenation of
decoder LSTM output and the attention context
is projected down to a scalar and passed through a sigmoid activation
to predict the probability that the output sequence has completed.
This ``stop token'' prediction is used during inference to allow the model to
dynamically determine when to terminate generation instead of always generating
for a fixed duration.
Specifically, generation completes at the first frame for which this probability
exceeds a threshold of 0.5.

The convolutional layers in the network are regularized using dropout
\cite{srivastava2014dropout} with probability 0.5, and LSTM layers are
regularized using zoneout \cite{krueger2016zoneout} with probability 0.1. In
order to introduce output variation at inference time, dropout with probability
0.5 is applied only to layers in the pre-net of the autoregressive decoder.

In contrast to the original Tacotron, our model uses simpler
building blocks, using vanilla LSTM and convolutional layers in
the encoder and decoder instead of ``CBHG'' stacks and GRU recurrent
layers.
%
We do not use a ``reduction factor'', \ie each decoder step
corresponds to a single spectrogram frame.


\begin{figure}[t!]
\centering
\includegraphics[width=0.98\columnwidth]{TTSArchitecture.pdf}
\caption{Block diagram of the Tacotron 2 system architecture.}
\label{fig:TTSArchitecture}
\end{figure}


\subsection{WaveNet Vocoder}
\label{ssec:wavenet}

We use a modified version of the WaveNet architecture from \cite{45774} to
invert the mel spectrogram feature representation into time-domain waveform
samples.
%
As in the original architecture, there are 30 dilated convolution layers,
grouped into 3 dilation cycles, \ie the dilation rate of layer k
($k=0\ldots 29$) is $2^{k\pmod{10}}$.
%
To work with the 12.5~ms frame hop of the spectrogram frames, only 2 upsampling
layers are used in the conditioning stack instead of 3 layers.

Instead of predicting discretized buckets with a softmax layer,
we follow PixelCNN++ \cite{DBLP:journals/corr/SalimansKCK17} and
Parallel WaveNet \cite{FasterWaveNet} and use a 10-component
mixture of logistic distributions (MoL) to generate 16-bit samples at 24~kHz.
%
To compute the logistic mixture distribution, the WaveNet stack output is passed
through a ReLU activation followed by a linear projection to predict
parameters (mean, log scale, mixture weight) for each mixture component.
%
The loss is computed as the negative log-likelihood of the ground truth sample.

%%%%%%%%%%%%%%%%%%%%%


%%%%%%%%%%%%%%%%%%%%%
% !TEX root = ../multi_task.tex

We evaluate the presented MTL method on a number of problems. First, we use MultiMNIST \citep{multi_mnist}, an MTL adaptation of MNIST \citep{mnist}. Next, we tackle multi-label classification on the CelebA dataset \citep{celeba} by considering each label as a distinct binary classification task. These problems include both classification and regression, with the number of tasks ranging from 2 to 40. Finally, we experiment with scene understanding, jointly tackling the tasks of semantic segmentation, instance segmentation, and depth estimation on the Cityscapes dataset \citep{cityscapes}. We discuss each experiment separately in the following subsections.

The baselines we consider are (i) \textbf{uniform scaling:} minimizing a uniformly weighted sum of loss functions \mbox{$\frac{1}{T}\sum_t \lL^t$}, \mbox{(ii) \textbf{single task:}} solving tasks independently, \mbox{(iii) \textbf{grid search:}} exhaustively trying various values from $\{ c^t \in [0,1] | \sum_t c^t = 1\}$ and optimizing for $\frac{1}{T}\sum_t c^t \lL^t$, \mbox{(iv) \textbf{\citet{Kendall2018}:}} using the uncertainty weighting proposed by \citet{Kendall2018}, and \mbox{(v) \textbf{GradNorm:}} using the normalization proposed by \citet{Chen2018}.



\subsection{MultiMNIST}
\label{sec:multi_mnist_exp}

Our initial experiments are on MultiMNIST, an MTL version of the MNIST dataset \citep{multi_mnist}. In order to convert digit classification into a multi-task problem, \citet{multi_mnist} overlaid multiple images together. We use a similar construction. For each image, a different one is chosen uniformly in random. Then one of these images is put at the top-left and the other one is at the bottom-right. The resulting tasks are: classifying the digit on the top-left (task-L) and classifying the digit on the bottom-right (task-R). We use 60K examples and directly apply existing single-task MNIST models. The MultiMNIST dataset is illustrated in the supplement.

We use the LeNet architecture \citep{mnist}. We treat all layers except the last as the representation function $g$ and put two fully-connected layers as task-specific functions (see the supplement for details). We visualize the performance profile as a scatter plot of accuracies on task-L and task-R in Figure~\ref{fig:multi_mnist_performance_curve}, and list the results in Table~\ref{tab:multi_mnist}.

In this setup, any static scaling results in lower accuracy than solving each task separately (the single-task baseline). The two tasks appear to compete for model capacity, since increase in the accuracy of one task results in decrease in the accuracy of the other. Uncertainty weighting \citep{Kendall2018} and GradNorm \citep{Chen2018} find solutions that are slightly better than grid search but distinctly worse than the single-task baseline. In contrast, our method finds a solution that efficiently utilizes the model capacity and yields accuracies that are as good as the single-task solutions. This experiment demonstrates the effectiveness of our method as well as the necessity of treating MTL as multi-objective optimization. Even after a large hyper-parameter search, \emph{any} scaling of tasks does not approach the effectiveness of our method.



\subsection{Multi-Label Classification}

\begin{figure}[t]
\includegraphics[width=\textwidth]{radar_full_new}
\vspace{1mm}
\caption{Radar charts of percentage error per attribute on CelebA \citep{celeba}. Lower is better. We divide attributes into two sets for legibility: easy on the left, hard on the right. Zoom in for details.}
\label{fig:multi_label_radar}
\end{figure}


\begin{wraptable}{r}{0.3\textwidth}
%\vspace{-4mm}
\captionof{table}{Mean of error per category of MTL algorithms in multi-label classification on CelebA \citep{celeba}.}
\begin{tabular}{r@{\hspace{2mm}}c@{}}
\toprule
& Average  \\
&  error \\
\midrule
Single task & $8.77$ \\
Uniform scaling & $9.62$ \\
\citealt{Kendall2018} & $9.53$ \\
GradNorm & $8.44$ \\
Ours & $\mathbf{8.25}$  \\
\bottomrule
\end{tabular}
\label{table:multi_label_bar}
%\vspace{-5mm}
\end{wraptable}

Next, we tackle multi-label classification. Given a set of attributes, multi-label classification calls for deciding whether each attribute holds for the input. We use the CelebA dataset \citep{celeba}, which includes 200K face images annotated with 40 attributes. Each attribute gives rise to a binary classification task and we cast this as a 40-way MTL problem. We use ResNet-18 \citep{resnet} without the final layer as a shared representation function, and attach a linear layer for each attribute (see the supplement for further details).


We plot the resulting error for each binary classification task as a radar chart in Figure~\ref{fig:multi_label_radar}. The average over them is listed in Table~\ref{table:multi_label_bar}. We skip grid search since it is not feasible over 40 tasks. Although uniform scaling is the norm in the multi-label classification literature, single-task performance is significantly better. Our method outperforms baselines for significant majority of tasks and achieves comparable performance in rest. This experiment also shows that our method remains effective when the number of tasks is high.


\subsection{Scene Understanding}

To evaluate our method in a more realistic setting, we use scene understanding. Given an RGB image, we solve three tasks: semantic segmentation (assigning pixel-level class labels), instance segmentation (assigning pixel-level instance labels), and monocular depth estimation (estimating continuous disparity per pixel). We follow the experimental procedure of \citet{Kendall2018} and use an encoder-decoder architecture. The encoder is based on ResNet-50 \citep{resnet} and is shared by all three tasks. The decoders are task-specific and are based on the pyramid pooling module \citep{pspnet} (see the supplement for further implementation details).

Since the output space of instance segmentation is unconstrained (the number of instances is not known in advance), we use a proxy problem as in \citet{Kendall2018}. For each pixel, we estimate the location of the center of mass of the instance that encompasses the pixel. These center votes can then be clustered to extract the instances. In our experiments, we directly report the MSE in the proxy task. Figure~\ref{fig:cityscapes_performance_profile} shows the performance profile for each pair of tasks, although we perform all experiments on all three tasks jointly. The pairwise performance profiles shown in Figure~\ref{fig:cityscapes_performance_profile} are simply 2D projections of the three-dimensional profile, presented this way for legibility. The results are also listed in Table~\ref{tab:cityscapes_results}.

MTL outperforms single-task accuracy, indicating that the tasks cooperate and help each other. Our method outperforms all baselines on all tasks.


\subsection{Role of the Approximation}

In order to understand the role of the approximation proposed in Section~\ref{sec:approximation}, we compare the final performance and training time of our algorithm with and without the presented approximation in Table~\ref{tab:approximation_tradeoff} (runtime measured on a single Titan Xp GPU). For a small number of tasks (3 for scene understanding), training time is reduced by 40\%. For the multi-label classification experiment (40 tasks), the presented approximation accelerates learning by a factor of 25.

On the accuracy side, we expect both methods to perform similarly as long as the full-rank assumption is satisfied. As expected, the accuracy of both methods is very similar. Somewhat surprisingly, our approximation results in slightly improved accuracy in all experiments. While counter-intuitive at first, we hypothesize that this is related to the use of SGD in the learning algorithm. Stability analysis in convex optimization suggests that if gradients are computed with an error $\hat{\nabla}_\btheta \mathcal{L}^t = \nabla_\btheta \mathcal{L}^t + \mathbf{e}^t$ ($\btheta$ corresponds to $\btheta^{sh}$ in (\ref{eq:kkt_opt})), as opposed to $\mathbf{Z}$ in the approximate problem in \ref{eq:approx}, the error in the solution is bounded as $\|\hat{\mathbf{\alpha}} - \mathbf{\alpha} \|_2 \leq \mathcal{O}(\max_t \|\mathbf{e}^t\|_2)$. Considering the fact that the gradients are computed over the full parameter set (millions of dimensions) for the original problem and over a smaller space for the approximation (batch size times representation which is in the thousands), the dimension of the error vector is significantly higher in the original problem. We expect the $l_2$ norm of such a random vector to depend on the dimension.

In summary, our quantitative analysis of the approximation suggests that (i) the approximation does not cause an accuracy drop and (ii) by solving an equivalent problem in a lower-dimensional space, our method achieves both better computational efficiency and higher stability.

  {\small
  \begin{table}[t]
%  \vspace{-4mm}
  \caption{Effect of the MGDA-UB approximation. We report the final accuracies as well as training times for our method with and without the approximation.}
  %\vspace{1mm}
  \centering
  \begin{tabular}{@{}r@{\hspace{3mm}}c@{\hspace{3mm}}c@{\hspace{2mm}}c@{\hspace{2mm}}c@{}c@{\hspace{5mm}}c@{\hspace{2mm}}c@{}}
  \toprule
  & \multicolumn{4}{c}{Scene understanding (3 tasks)} &  & \multicolumn{2}{c}{Multi-label (40 tasks)}  \\
  \cmidrule(r){2-5} \cmidrule(lr){7-8}
                  & Training & Segmentation & Instance  & Disparity      & & Training & Average \\
                 & time     &  mIoU [\%]       & error [px] & error [px] & & time (hour)      & error \\
  \midrule
  Ours (w/o approx.) & $38.6$ & $66.13$ & $10.28$ & $2.59$ & & $429.9$ & $8.33$ \\
  Ours & $\mathbf{23.3}$ & $\mathbf{66.63}$ & $\mathbf{10.25}$ & $\mathbf{2.54}$  & & $\mathbf{16.1}$ & $\mathbf{8.25}$ \\
  \bottomrule
  \end{tabular}
  %\vspace{-2mm}
  \label{tab:approximation_tradeoff}
  \end{table}}

%%%%%%%%%%%%%%%%%%%%%


%%%%%%%%%%%%%%%%%%%%%
\section{Conclusion}\label{sec:conclusion}
%\vspace{-.1in}
In this work, we apply the attentional encoder-decoder for the task of abstractive summarization with very promising results, outperforming state-of-the-art results significantly on two different datasets. Each of our proposed novel models addresses a specific problem in abstractive summarization, yielding further improvement in performance. We also propose a new dataset for multi-sentence summarization and establish benchmark numbers on it. As part of our future work, we plan to focus our efforts on this data and build more robust models for summaries consisting of multiple sentences.


%Our results strongly demonstrate that sequence-to-sequence models are extremely promising for summarization. Some of the other lessons we learned from our experiments are: (i) the LVT-trick is very useful for summarization as it improves training speed while not sacrificing performance; (ii) traditional methods such as vocabulary expansion and syntax-based features can boost performance of deep learning based models as well. As part of our ongoing work, we are investigating on ways to effectively generate rare words in the summary, which appears to be a glaring weakness in the existing models.  

%%%%%%%%%%%%%%%%%%%%%

\section{Acknowledgments}
The authors thank Jan Chorowski, Samy Bengio, A{\"a}ron van den Oord, and the
  WaveNet and Machine Hearing teams for their helpful discussions and advice, as
  well as Heiga Zen and the Google TTS team for their feedback and assistance
  with running evaluations.
  %
  The authors are also grateful to the very thorough reviewers.
% References should be produced using the bibtex program from suitable
% BiBTeX files (here: strings, refs, manuals). The IEEEbib.bst bibliography
% style file from IEEE produces unsorted bibliography list.
% -------------------------------------------------------------------------
\bibliographystyle{IEEEbib}
\bibliography{ms}

\end{document}

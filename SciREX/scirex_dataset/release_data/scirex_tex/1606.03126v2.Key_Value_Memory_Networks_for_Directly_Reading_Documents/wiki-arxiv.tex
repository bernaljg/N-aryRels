%
% File emnlp2016.tex
%

\documentclass[11pt,letterpaper]{article}
\usepackage{emnlp2016}
\usepackage{times}
\usepackage{latexsym}


\usepackage{color}
\usepackage[utf8]{inputenc} % allow utf-8 input
\usepackage[T1]{fontenc}    % use 8-bit T1 fonts
\usepackage{hyperref}       % hyperlinks
\usepackage{url}            % simple URL typesetting
\usepackage{booktabs}       % professional-quality tables
\usepackage{amsfonts}       % blackboard math symbols
\usepackage{nicefrac}       % compact symbols for 1/2, etc.
\usepackage{microtype}      % microtypography
\usepackage{graphicx}
\usepackage{epsfig}
\usepackage{natbib}
\usepackage{xspace}

\usepackage{array}
\newcolumntype{L}{>{\arraybackslash}m{10cm}}

\newcommand{\mcb}[1]{\multicolumn{2}{c|}{#1}}
\newcommand{\mc}[1]{\multicolumn{2}{c}{#1}}
\newcommand{\WikiMovies}{{\sc WikiMovies}\xspace}
\definecolor{dgreen}{rgb}{0.0,0.4,0.0}
\definecolor{dred}{rgb}{1,0.0,0.0}
\usepackage{amsmath}
\usepackage{amssymb}
\DeclareMathOperator*{\argmax}{arg\,max}
\DeclareMathOperator*{\softmax}{softmax}


% Uncomment this line for the final submission:
\emnlpfinalcopy

%  Enter the EMNLP Paper ID here:
\def\emnlppaperid{679}

% To expand the titlebox for more authors, uncomment
% below and set accordingly.
% \addtolength\titlebox{.5in}

\newcommand\BibTeX{B{\sc ib}\TeX}

\title{Key-Value Memory Networks for Directly Reading Documents}

% Author information can be set in various styles:
% For several authors from the same institution:
 \author{
Alexander H. Miller$^{1}$ \mbox{~~}
Adam Fisch$^{1}$ \mbox{~~}
Jesse Dodge$^{1,2}$ \mbox{~~}
Amir-Hossein Karimi$^{1}$ \\
{\bf Antoine Bordes$^{1}$} \mbox{~~}
{\bf Jason Weston$^{1}$} \\
$^{1}$Facebook AI Research, 770 Broadway, New York, NY, USA\\
$^{2}$Language Technologies Institute, Carnegie Mellon University, Pittsburgh, PA, USA\\
{\tt \{ahm,afisch,jessedodge,ahkarimi,abordes,jase\}@fb.com}
 }

% \author{Author 1 \and ... \and Author n \\
%         Address line \\ ... \\ Address line}
% if the names do not fit well on one line use
%         Author 1 \\ {\bf Author 2} \\ ... \\ {\bf Author n} \\
% For authors from different institutions:
% \author{Author 1 \\ Address line \\  ... \\ Address line
%         \And  ... \And
%         Author n \\ Address line \\ ... \\ Address line}
% To start a seperate ``row'' of authors use \AND, as in
% \author{Author 1 \\ Address line \\  ... \\ Address line
%         \AND
%         Author 2 \\ Address line \\ ... \\ Address line \And
%         Author 3 \\ Address line \\ ... \\ Address line}
% If the title and author information does not fit in the area allocated,
% place \setlength\titlebox{<new height>} right after
% at the top, where <new height> can be something larger than 2.25in
%\author{Siddharth Patwardhan \and Daniele Pighin\\
%  {\tt publication@emnlp2016.net}}

\date{}

\begin{document}

\maketitle

\begin{abstract}
Directly reading documents and being able to answer questions from them is an unsolved challenge.
To avoid its inherent difficulty, question answering (QA) has been directed towards
using Knowledge Bases (KBs) instead,
which has proven effective.
Unfortunately KBs often suffer from being too restrictive, as the schema cannot support certain types of answers,
and too sparse, e.g. Wikipedia contains much more information than Freebase.
In this work we introduce a new method, Key-Value Memory Networks,
that makes reading documents more viable
by utilizing different encodings in the addressing  and output stages of the memory read operation.
To compare using  KBs, information extraction or Wikipedia documents directly in a single
framework we construct
 an analysis tool, {\sc WikiMovies}, a QA dataset that contains raw text alongside a preprocessed KB, in the domain of movies.
Our method reduces the gap between all three settings.
It also achieves state-of-the-art results on the existing {\sc WikiQA} benchmark.
\end{abstract}

\section{Introduction}
\vspace{-0.5ex}
\section{Introduction}
\label{sec:intro}

Language modeling is among the important problems that require modeling long-term dependency, with successful applications such as unsupervised pretraining~\citep{dai2015semi,peters2018deep,radford2018improving,devlin2018bert}.
However, it has been a challenge to equip neural networks with the capability to model long-term dependency in sequential data.
Recurrent neural networks (RNNs), in particular Long Short-Term Memory (LSTM) networks~\citep{hochreiter1997long}, have been a standard solution to language modeling and obtained strong results on multiple benchmarks.
Despite the wide adaption, RNNs are difficult to optimize due to gradient vanishing and explosion~\citep{hochreiter2001gradient}, and the introduction of gating in LSTMs and the gradient clipping technique~\citep{graves2013generating} might not be sufficient to fully address this issue.
% ,pascanu2012understanding
Empirically, previous work has found that LSTM language models use 200 context words on average~\citep{khandelwal2018sharp}, indicating room for further improvement.

On the other hand, the direct connections between long-distance word pairs baked in attention mechanisms might ease optimization and enable the learning of long-term dependency~\citep{bahdanau2014neural,vaswani2017attention}.
Recently, \citet{al2018character} designed a set of auxiliary losses to train deep Transformer networks for character-level language modeling, which outperform LSTMs by a large margin.
Despite the success, the LM training in~\citet{al2018character} is performed on separated fixed-length segments of a few hundred characters, without any information flow across segments.
As a consequence of the fixed context length, the model cannot capture any longer-term dependency beyond the predefined context length.
In addition, the fixed-length segments are created by selecting a consecutive chunk of symbols without respecting the sentence or any other semantic boundary.
Hence, the model lacks necessary contextual information needed to well predict the first few symbols, leading to inefficient optimization and inferior performance.
We refer to this problem as \textit{context fragmentation}.

%However, the context length is fixed to hundreds of characters and thus it is not possible to model longer-term dependency. Moreover, it is not clear how the model performs on word-level language modeling data, as the granularity changes.

% Moreover, using auxiliary losses brings additional challenges such as properly tuning the mixture weights and the loss decay schedule.

To address the aforementioned limitations of fixed-length contexts, we propose a new architecture called Transformer-XL (meaning extra long).
We introduce the notion of recurrence into our deep self-attention network. In particular, instead of computing the hidden states from scratch for each new segment, we reuse the hidden states obtained in previous segments.
The reused hidden states serve as memory for the current segment, which builds up a recurrent connection between the segments.
As a result, modeling very long-term dependency becomes possible because information can be propagated through the recurrent connections.
Meanwhile, passing information from the previous segment can also resolve the problem of context fragmentation.
More importantly, we show the necessity of using relative positional encodings rather than absolute ones, in order to enable state reuse without causing temporal confusion.
Hence, as an additional technical contribution, we introduce a simple but more effective relative positional encoding formulation that generalizes to attention lengths longer than the one observed during training.

Transformer-XL obtained strong results on five datasets, varying from word-level to character-level language modeling.
Transformer-XL is also able to generate relatively coherent long text articles with \textit{thousands of} tokens (see Appendix \ref{sec:gen}), trained on only 100M tokens.
% Transformer-XL improves the previous state-of-the-art (SoTA) results from 1.06 to 0.99 in bpc on enwiki8, from 1.13 to 1.08 in bpc on text8, from 20.5 to 18.3 in perplexity on WikiText-103, and from 23.7 to 21.8 in perplexity on One Billion Word.
% Transformer-XL improves the previous state-of-the-art (SoTA) results to 0.99 in bpc on enwiki8, 1.08 in bpc on text8, 18.3 in perplexity on WikiText-103, and 21.8 in perplexity on One Billion Word.
% On small data, Transformer-XL also achieves a perplexity of 54.5 on Penn Treebank without finetuning, which is SoTA when comparable settings are considered.

Our main technical contributions include introducing the notion of recurrence in a purely self-attentive model and deriving a novel positional encoding scheme. These two techniques form a complete set of solutions, as any one of them alone does not address the issue of fixed-length contexts. Transformer-XL is the first self-attention model that achieves substantially better results than RNNs on both character-level and word-level language modeling.

% On WikiText-103, Transformer-XL improves the previous state-of-the-art (SoTA) results from 33 perplexity to 24, with a relative reduction of 27\%. On enwiki8 character-level language modeling, Transformer-XL achieves a SoTA bpc of 1.03, which outperforms \cite{al2018character} by 0.03 with 60+\% fewer parameters. Given a more common model size with 40+M parameters, Transformer-XL achieves a bpc of 1.06, compared to 1.11 by \cite{al2018character}. Transformer-XL also achieves perplexities of 54.5 on Penn Treebank and 29.4 on One Billion Word, which are SoTA when comparable settings are considered.

% Due to the ability of modeling long-range context, our best model uses attention lengths of 1,600 and 3,800 on WikiText-103 and enwiki8 respectively. We also devise a metric called \textit{Relative Effective Context Length} (RECL) that aims to fairly compare the ability of long-range dependency modeling.
% % perform a fair comparison of the gains brought by increasing the context lengths for different models.
% In this setting, Transformer-XL learns a RECL of 900 words on WikiText-103, while the numbers for recurrent networks and Transformer are only 500 and 128.

% We use two methods to quantitatively study the effective lengths of Transformer-XL and the baselines. Similar to \cite{khandelwal2018sharp}, we gradually increase the attention length at test time until no further noticeable improvement ($\sim$0.1\% relative gains) can be observed. Our best model in this settings use attention lengths of 1,600 and 3,800 on WikiText-103 and enwiki8 respectively.
% %In addition, since the effective context length of Transformer-XL can be longer than the attention length due to our recurrent formulation, we devise a metric called \textit{Relative Effective Context Length} (RECL) that aims to perform a fair comparison of the gains brought by increasing the context lengths for different models.
% In addition, we devise a metric called \textit{Relative Effective Context Length} (RECL) that aims to perform a fair comparison of the gains brought by increasing the context lengths for different models.
% In this setting, Transformer-XL learns a RECL of 900 words on WikiText-103, while the numbers for recurrent networks and Transformer are only 500 and 128.


%\vspace{-0.25ex}
\section{Related Work}
\vspace{-0.5ex}

Early QA systems were based on information retrieval and were designed
to return snippets of text containing an answer
\citep{voorhees2000building,banko2002askmsr}, with limitations in terms of question
complexity and response coverage.
%
The creation of large-scale KBs
\citep{auer2007dbpedia,bollacker2008freebase} have led to the
development of a new class of QA methods based on semantic parsing
\citep{berant2013semantic,kwiatkowski-EtAl:2013:EMNLP,fader2014open,yih2015semantic}
that can return precise answers to complicated compositional questions.
%
Due to the sparsity of KB data, however, the main challenge
shifts from finding answers to developing efficient information
extraction methods to populate KBs automatically
\citep{craven2000learning,carlson2010toward}---not an easy
problem.

For this reason, recent initiatives are returning to the original
setting of directly answering from text using
datasets like {\sc TrecQA} \citep{wang2007jeopardy},
which is based on classical {\sc Trec} resources \citep{voorhees1999trec},
and {\sc WikiQA} \citep{yang2015wikiqa}, which is extracted from Wikipedia.
%
Both benchmarks are organized around the task of answer sentence
selection, where a system must identify the sentence containing
the correct answer in a collection of documents, but need not return the
actual answer as a KB-based system would do.
%
Unfortunately, these datasets are very small (hundreds of
examples) and, because of their answer selection setting, do not
offer the option to directly compare answering from a KB against answering from pure text.
%
Using similar resources as the dialog dataset
of \cite{dodge2015evaluating}, our new benchmark {\sc WikiMovies}
addresses both deficiencies by providing a substantial
corpus of question-answer pairs that can be answered by either using a
KB or a corresponding set of documents.




Even though standard pipeline QA systems like AskMR
\citep{banko2002askmsr} have been recently revisited
\citep{tsai2015web},
%
the best published results on {\sc TrecQA} and {\sc WikiQA} have been
obtained by either convolutional neural networks
\citep{santos2016attentive,yin2015convolutional,wang2016sentence} or
recurrent neural networks \citep{miao2015neural}---both usually with
attention mechanisms inspired by \citep{bahdanau2014neural}.
%
In this work, we introduce KV-MemNNs, a Memory Network model that operates a symbolic memory structured as $(key, value)$ pairs.
Such structured memory is not employed
 in any existing attention-based neural network architecture for QA.
As we will show, it gives the model greater
flexibility for encoding knowledge sources
% to retrieve the answers from
and helps shrink the gap between
directly reading documents and answering from a KB.
%versus answering from a KB.


%\citep{hill2015goldilocks}
%\citep{nips15_hermann}


%\citep{bordes2014question}
%\citep{bordes2015large}


%\vspace{-0.5ex}
\section{Key-Value Memory Networks} \label{sec:models}
\vspace{-0.5ex}
\begin{figure*}[t]
    \centering
    \begin{subfigure}[t]{0.5\textwidth}
        \centering
        \includegraphics[height=3.5cm]{Images/O_1_Models.png}
        \caption{\label{fig:o_1_models}Constant time late combining models}
    \end{subfigure}%
    ~ 
    \begin{subfigure}[t]{0.25\textwidth}
        \centering
        \includegraphics[height=3cm]{Images/O_n_Models.png}
        \caption{\label{fig:o_n_models}Linear time early combining models}
    \end{subfigure}%
    ~
    \begin{subfigure}[t]{0.20\textwidth}
        \centering
        \includegraphics[height=2.7cm]{Images/O_n2_Models.png}
        \caption{\label{fig:o_n2_models}Quadratic time early combining model}
    \end{subfigure}%
\end{figure*}



%\vspace{-1ex}
\section{The WikiMovies Benchmark} \label{sec:data}
\vspace{-0.5ex}
The \WikiMovies ~benchmark consists of question-answer pairs in the domain of movies.
It was built with the following goals in mind:
(i) machine learning techniques should have ample training examples for learning;
and (ii) one can analyze easily the performance of different representations of knowledge and break
down the results by question type.
%
The dataset can be downloaded from \url{http://fb.ai/babi}.

\newcolumntype{L}{>{\arraybackslash}m{10cm}}

\definecolor{dblue}{rgb}{0.0,0.0,0.6}
\definecolor{dred}{rgb}{0.3,0.0,0.0}
\definecolor{die}{rgb}{0.6,0.6,0.0}
\definecolor{dgreen}{rgb}{0.0,0.6,0.0}

%\newcolumntype{L}{>{\arraybackslash}m{10cm}}


\begin{table}[t]
\begin{small}
\begin{center}
 \resizebox{1\linewidth}{!}{
 {
%\begin{tabular}{l}
\begin{tabular}{|L|}
\hline
{\bf Doc: Wikipedia Article for Blade Runner (partially shown)}\\
\vspace{1mm}
\textcolor{dblue}{Blade Runner is a 1982 American neo-noir dystopian science fiction film
 directed by Ridley Scott and starring Harrison Ford, Rutger Hauer, Sean Young, and Edward James Olmos. The screenplay, written by Hampton Fancher and David Peoples, is a modified film adaptation of the 1968 novel ``Do Androids Dream of Electric Sheep?''  by Philip K. Dick. The film depicts a dystopian Los Angeles in November 2019 in which genetically engineered replicants, which are visually indistinguishable from adult humans, are manufactured by the powerful Tyrell Corporation as well as by other ``mega-corporations'' around the world.   %\dots}\\
Their use on Earth is banned and replicants are exclusively used for dangerous, menial, or leisure work on off-world colonies. Replicants who defy the ban and return to Earth are hunted down and ``retired'' by special police operatives known as ``Blade Runners''. \dots}
\\
\hline
%\end{tabular}
%\begin{tabular}{|L|}
{\text{\bf KB entries for Blade Runner (subset)}}\\
\vspace{1mm}
\textcolor{red}{Blade Runner {\em directed\_by} Ridley Scott}\\
\textcolor{red}{Blade Runner {\em written\_by} Philip K. Dick, Hampton Fancher}\\
\textcolor{red}{Blade Runner {\em starred\_actors} Harrison Ford, Sean Young, \dots} \\% Edward James Olmos\\
\textcolor{red}{Blade Runner {\em release\_year} 1982}\\
\textcolor{red}{Blade Runner {\em has\_tags} dystopian, noir, police, androids, \dots}
%dystopia, cult film, police, future, \dots %harrison ford, library, national film registry, philip k. dick, los angeles, ridley scott, androids, noir, visual, 2, rutger hauer, dystopian, edward james olmos, director's cut, sean young, android\\
\\
%\textcolor{red}{\dots} \\
\hline
{\text{\bf IE entries for Blade Runner (subset)}}\\
\vspace{1mm}
\textcolor{die}{Blade Runner, Ridley Scott {\em directed}    dystopian, science fiction, film}\\
\textcolor{die}{Hampton Fancher {\em written}    Blade Runner}\\
%\textcolor{die}{Blade Runner {\em brought}    Philip K. Dick}\\
\textcolor{die}{Blade Runner {\em starred}   Harrison Ford, Rutger Hauer, Sean Young\dots}\\ % Edward James Olmos}\\
\textcolor{die}{Blade Runner {\em labelled}    1982 neo noir}\\
\textcolor{die}{special police, Blade {\em retired} Blade Runner}\\
\textcolor{die}{Blade Runner, special police {\em known} Blade}
\\
\hline
{\bf Questions for Blade Runner (subset)}\\
\vspace{1mm}
\textcolor{dgreen}{Ridley Scott directed which films?}\\
\textcolor{dgreen}{What year was the movie Blade Runner released?}\\
\textcolor{dgreen}{Who is the writer of the film Blade Runner?}\\
   %What films can be described by ridley scott?\\
   %Which films can be described by sean young?\\
   %Which movies are about edward james olmos?\\
   %What movies are about harrison ford?\\
   %Which movies are about noir?\\
   %What films can be described by android?\\
\textcolor{dgreen}{Which films can be described by dystopian?}\\
   %Which movie did Hampton Fancher write?\\
\textcolor{dgreen}{Which movies was Philip K. Dick the writer of?}\\
\textcolor{dgreen}{Can you describe movie Blade Runner in a few words?}
\\
%\textcolor{dgreen}{The movie Blade Runner starred who?}\\
%\textcolor{dgreen}{Who directed the movie Blade Runner?}\\
   %What movies did Harrison Ford star in?\\
%\textcolor{dgreen}{What movies was Sean Young an actor in?}\\
   %What does Edward James Olmos star in?\\
%\dots \\


%%1 what films can be described by ridley scott?  Gladiator, Alien, Prometheus, Blade Runner, American Gangster, Black Hawk Down, Kingdom of Heaven, Robin Hood, Hannibal, Body of Lies, Matchstick Men, The Counselor, A Good Year, G.I. Jane, Legend, Black Rain, White Squall, The Duellists, Someone to Watch Over Me
%%1 what movies can be described with visual?     The Matrix, Avatar, Sin City, 300, Blade Runner, Manhunter, Halloween II
%%1 which films can be described by sean young?   Blade Runner, No Way Out, Fire Birds
%%1 which movies are about edward james olmos?    Blade Runner, Selena, Stand and Deliver, American Me, Zoot Suit
%%1 which movies are about noir?  Pulp Fiction, Sin City, Blade Runner, Drive, L.A. Confidential, Chinatown, Dark City, The Third Man, The Maltese Falcon, Double Indemnity, The Man Who Wasn't There, Brick, Touch of Evil, Following, The Big Sleep, The Killing, Laura, Body Heat, Out of the Past, White Heat, Gilda, The Asphalt Jungle, The Last Seduction, In a Lonely Place, Dark Passage, Stray Dog, D.O.A., Kansas City Confidential
%%1 which movies are about 2?     Forrest Gump, The Sixth Sense, Sin City, The Truman Show, Catch Me If You Can, Blade Runner, American Psycho, Oldboy, There's Something About Mary, Panic Room, Annie Hall, Corpse Bride, 25th Hour, Chicken Run, Rushmore, Miami Vice, Ghost in the Shell, Husbands and Wives, Gridlock'd, Nadja
%%1 what movies are about harrison ford?  Raiders of the Lost Ark, Indiana Jones and the Last Crusade, Blade Runner, Indiana Jones and the Kingdom of the Crystal Skull, The Fugitive, Ender's Game, Air Force One, What Lies Beneath, Patriot Games, The Conversation, Clear and Present Danger, Witness, 42, American Graffiti, Six Days Seven Nights, Morning Glory, Firewall, The Devil's Own, Working Girl, Frantic, Hollywood Homicide, Sabrina, Presumed Innocent, Regarding Henry, The Mosquito Coast, Random Hearts, Extraordinary Measures
%%1 what films can be described by android?       Blade Runner, A.I. Artificial Intelligence
%%1 which movies are about director's cut?        Donnie Darko, Blade Runner, Daredevil
%%1 what movies can be described by rutger hauer? Sin City, Blade Runner, Hobo with a Shotgun, The Hitcher, Ladyhawke, Nighthawks, Blind Fury, Split Second, The Osterman Weekend
%%1 which films can be described by dystopian?    V for Vendetta, Blade Runner, Brazil, Sleep Dealer
%%1 which films can be described by philip k. dick?       Blade Runner, Minority Report, Total Recall, The Adjustment Bureau, Next, Paycheck, A Scanner Darkly, Impostor, Screamers, Radio Free Albemuth
%%1 which movie did Hampton Fancher write?        Blade Runner, The Minus Man, The Mighty Quinn
%%1 which movies was Philip K. Dick the writer of?        Blade Runner, Minority Report, Total Recall, The Adjustment Bureau, Next, Paycheck, A Scanner Darkly, Impostor, Screamers, Radio Free Albemuth
%%1 can you describe movie Blade Runner in a few words?   dystopia, cult film, r, police, future, harrison ford, library, national film registry, philip k. dick, los angeles, ridley scott, androids, noir, visual, 2, rutger hauer, dystopian, edward james olmos, director's cut, sean young, android
%%1 what year was the movie Blade Runner released?        1982
%%1 who is the writer of the film Blade Runner?   Philip K. Dick, Hampton Fancher
%%1 Ridley Scott directed which films?    Gladiator, Alien, Prometheus, Blade Runner, American Gangster, Black Hawk Down, Kingdom of Heaven, Robin Hood, Hannibal, Body of Lies, Matchstick Men, The Counselor, A Good Year, G.I. Jane, Legend, Black Rain, White Squall, The Duellists, Someone to Watch Over Me
%%1 the movie Blade Runner starred who?   Harrison Ford, Sean Young, Rutger Hauer, Edward James Olmos
%%1 who directed the movie Blade Runner?  Ridley Scott
%%1 what movies did Harrison Ford star in?        Raiders of the Lost Ark, Indiana Jones and the Last Crusade, Blade Runner, Indiana Jones and the Kingdom of the Crystal Skull, The Fugitive, Ender's Game, Air Force One, The Expendables 3, Patriot Games, Clear and Present Danger, Witness, 42, Six Days Seven Nights, Firewall, The Devil's Own, Working Girl, Frantic, Hollywood Homicide, Sabrina, Presumed Innocent, Paranoia, Regarding Henry, The Mosquito Coast, Crossing Over, Random Hearts, Extraordinary Measures, Force 10 from Navarone, The Frisco Kid
%%1 what movies was Sean Young an actor in?       Blade Runner, No Way Out, Fatal Instinct, Fire Birds, A Kiss Before Dying, Cousins, Young Doctors in Love, Dr. Jekyll and Ms. Hyde, The Boost
%%1 what does Edward James Olmos star in? Blade Runner, Stand and Deliver, Wolfen, My Family, In the Time of the Butterflies, Triumph of the Spirit, Caught, Zoot Suit, Talent for the Game, The Wonderful Ice Cream Suit
%%


%~~What movies are about open source?   \textcolor{dred}{Revolution OS}\\
%~~Ruggero Raimondi appears in which movies?     \textcolor{dred}{Carmen}\\
%~~What movies did Darren McGavin star in?       \textcolor{dred}{Billy Madison, The Night Stalker, Mrs. Pollifax-Spy}\\
%~~Can you name a film directed by Stuart Ortiz? \textcolor{dred}{Grave Encounters}\\
%~~Who directed the film White Elephant?  \textcolor{dred}{Pablo Trapero}\\
%~~What is the genre of the film Dial M for Murder?   \textcolor{dred}{Thriller, Crime}\\
%~~What language is Whity in?     \textcolor{dred}{German}\\
\hline
\end{tabular}
}}
\caption{
\label{fig:blade}
%{\bf WikiMovies}: Questions and KB, Wikipedia sources.}
{\bf \WikiMovies}: Questions, Doc, KB and IE sources.}
\end{center}
\end{small}
\vspace{-1ex}
\end{table}





%The dataset has around 100,000 question-answer pairs that are split between train, dev and test sets.
%This is much larger than most existing datasets, for example the WikiQA dataset \citep{yang2015wikiqa}
% for which we also conduct experiments in Sec. \ref{sec:wikiqa} has only $\sim$1000 training pairs.
%Being able to separate the problem of severe overfitting from the type of knowledge representation..

\subsection{Knowledge Representations} \label{sec:kr}

We construct three forms of knowledge representation:
(i) Doc: raw Wikipedia documents consisting of the pages of the movies mentioned;
(ii) KB: a classical graph-based KB consisting of entities
and relations created from the Open Movie Database (OMDb) and MovieLens;
and (iii) IE: information extraction performed on the Wikipedia pages to
build a KB in a similar form as (ii).
We take care to construct QA pairs such that they are all potentially answerable
from either the KB from (ii) or the original  Wikipedia documents from (i) to
eliminate data sparsity issues. However, it should
 be noted that the advantage of working from raw documents in real applications
is that data sparsity is less of a concern than for a KB, while on the other hand the KB
has the information already parsed in a form amenable to manipulation by machines.
This dataset can help analyze what  methods we need
to close the gap between all three settings, and in particular what
are the best methods for reading documents when a KB is not available.
A sample of the dataset is shown in Table~\ref{fig:blade}.

\paragraph{Doc}
We selected a set of Wikipedia articles about movies
by identifying a set of movies from OMDb\footnote{\tiny{\url{http://beforethecode.com/projects/omdb/download.aspx}}}
that had an associated article by title match.
%We identified a set of movies from OMDb\footnote{Downloaded from \tiny{\url{http://beforethecode.com/projects/omdb/download.aspx}}.}
%that had an associated Wikipedia article by title match,
We keep the title and the first section (before the contents box) for each article.
This gives $\sim$17k documents (movies) which comprise the set of documents our
models will read from in order to answer questions.

\paragraph{KB}
Our set of movies were also matched to
the MovieLens dataset\footnote{\tiny{\url{http://grouplens.org/datasets/movielens/}}}.
We built a KB using OMDb and MovieLens metadata with entries for each movie and nine different relation types:
director, writer, actor, release year, language, genre, tags, IMDb rating and IMDb votes,
with $\sim$10k related actors, $\sim$6k directors and
$\sim$43k  entities in total.
The KB is stored as triples; see Table~\ref{fig:blade} for examples.
% such as
 %{\small{ {\sc (Young Frankenstein, starred\_actors, Gene Wilder)}}} and
%{\small {\sc (The Little Mermaid, has\_tags, Disney Animation)}}.
IMDb ratings and votes
are originally real-valued but are binned and converted to text
 (``unheard of'', ``unknown'', ``well known'', ``highly watched'', ``famous'').
We finally only retain KB triples where the entities also appear in the Wikipedia
articles\footnote{The dataset also
includes the slightly larger version without this constraint.}
to try to
guarantee that all QA pairs will be equally answerable by either the KB or Wikipedia document
sources.

\paragraph{IE} As an alternative to directly reading documents,
we explore leveraging information extraction  techniques to
transform documents into a KB format.
%Constraining the memories solely to facts identified by an IE system introduces a few tradeoffs.
%Processing each Wikipedia entry into a series of semi-structured facts mimics
%some of the attractive attributes of the KB,
An IE-KB representation has attractive properties
such as more precise and compact expressions of facts
and logical key-value pairings based on subject-verb-object groupings.
This can come at the cost of lower recall due to malformed or completely missing triplets.
%
For IE we use standard open-source software followed by some task-specific
engineering to improve the results.
We first employ coreference resolution via the Stanford NLP Toolkit \citep{manning2014stanford} to reduce ambiguity by replacing pronominal (``he'', ``it'') and nominal (``the film'') references with their representative entities. Next we use the SENNA semantic role labeling tool \citep{senna_collobert} to uncover the grammatical structure of each sentence and pair verbs with their arguments. Each triplet is cleaned of words that are not recognized entities,
and lemmatization is done to collapse different inflections of important task-specific verbs to one form (e.g. stars, starring, star $\rightarrow$ starred).
Finally, we append the movie title to each triple similar to the ``Window + Title''
representation of Sec. \ref{sec:featuremap}, which improved results.


\subsection{Question-Answer Pairs}
%\paragraph{QA Pairs}
%The dataset has more than 100,000 question-answer pairs.
%Being able to separate the problem of severe overfitting from the type of knowledge representation..
%
Within the dataset's more than 100,000 question-answer pairs, we distinguish 13
classes of question corresponding to different kinds of edges in our KB.
They range in scope from specific---such as
{\em actor to movie}:~``What movies did Harrison Ford star in?'' and
{\em movie to actors}:~``Who starred in Blade Runner?''---to more general,
such as {\em tag to movie}:~``Which films can be described by {\em dystopian}?'';
see Table \ref{table:breakdown} for the full list.
For some question there can be multiple correct answers.
%
%The topics correspond to different edges in our KB, and range in scope from specific ({\em movie to actors}  -- ``Who starred in Blade Runner?'') to more general ({\em tag to movie} -- ``Which films can be described by {\em dystopian}?''). For each question type there is a set of possible answers.
%
%as shown in Table \ref{table:breakdown}.
%he topics correspond to different edges in our KB, and range in scope from specific ({\em movie to actors}  -- ``Who starred in Blade Runner?'') to more general ({\em tag to movie} -- ``Which films can be described by {\em dystopian}?'').
%For each question type there is a set of possible answers.
%corresponding to different kinds of edges in our KB:
%{\em actor to movie} (``What movies did Harrison Ford star in?''),
%{\em movie to actors} (``Who starred in Blade Runner?''),
%{\em movie to director}, {\em director to movie},
%{\em movie to writer}, {\em writer to movie},
%{\em movie to tags}, {\em tag to movie},
%{\em movie to year}, {\em movie to genre}, {\em movie to language},
%{\em movie to IMDb rating} and
%{\em movie to IMDb votes}.

Using SimpleQuestions \citep{bordes2015large},
an existing open-domain question answering dataset based on Freebase,
we identified the subset of questions posed by human annotators that covered
our question types.
%We expanded this set to cover all of our KB by substituting the entities
%in those questions to also apply them to other questions.
We created our question set by substituting the entities in those questions
with entities from all of our KB triples.
%
For example, if the original question written by an annotator was
``What movies did Harrison Ford star in?'', we created a pattern
``What movies did [@actor] star in?'', which we substitute for any
other actors in our set, and repeat this for all annotations.
%We removed {\em tag to movie} questions with more than 50 answers,
%and
We split the questions into disjoint training, development and test sets
with $\sim$96k, 10k and 10k examples, respectively.
The same question (even worded differently) cannot appear in
both train and test sets.
Note that this is much larger than most existing datasets;
for example, the {\sc WikiQA} dataset \citep{yang2015wikiqa}
for which we also conduct experiments
in Sec. \ref{sec:wikiqa} has only $\sim$1000 training pairs.


%\vspace{-0.5ex}
\section{Experiments}\label{sec:exp}
\vspace{-0.5ex}
\begin{table}[t!]
	\begin{center}
	\label{tab:WikiMoviesres}
   	\resizebox{1\linewidth}{!}{{
	\renewcommand{\arraystretch}{1.0}
    	\begin{tabular}{l|c|c|c|}
      		Method &  KB   &  IE  & Doc \\
      		\hline
		\citep{bordes2014question} QA system & 93.5 & 56.5 & N/A \\
		Supervised Embeddings         & 54.4 & 54.4 & 54.4 \\
		Memory Network                    & 78.5 & 63.4 & 69.9 \\ %69.9
		Key-Value Memory Network  & \textbf{93.9} & \textbf{68.3} & \textbf{76.2} \\ % 73.9
    	\end{tabular}
    	}}
    	\caption{
          \label{table:main-res}
          { Test results (\% hits@1)
on \WikiMovies, comparing human-annotated KB (KB), information extraction-based KB (IE),
and directly reading Wikipedia documents (Doc).}}
  	\end{center}
%  \vspace{-3ex}
\end{table}


\begin{table}[t!]
	\begin{center}
    	\resizebox{1\linewidth}{!}{{
	\renewcommand{\arraystretch}{1.0}
    	\begin{tabular}{l|c|c|c|}
      		Memory Representation  &   Doc \\
      		\hline
		Sentence-level               &  52.4 \\  % dev-kv:52.4        dev-memnn:?
		Window-level                 &  66.8 \\  % dev-kv:66.77 test-kv:66.4   dev-memnn:?
		Window-level + Title      &  74.1 \\  % dev-kv:74.1 test-kv:73.9      dev-memnn:?
		Window-level + Center Encoding + Title & \textbf{76.9} \\ % dev-kv:76.9 test-kv:76.2 dev-memnn:?
    	\end{tabular}
    	}}
    	\caption{
          \label{table:memkv-res}
{Development set performance (\% hits@1) with different document memory representations for KV-MemNNs. }}
  	\end{center}
%  \vspace{-3ex}
\end{table}







This section describes our experiments %with KV-MemNNs
 on \WikiMovies and
{\sc WikiQA}.

\subsection{WikiMovies} \label{sec:WikiMovies}

We conducted experiments on the \WikiMovies~ dataset described
in Sec. \ref{sec:data}. Our main goal is to
compare the performance of KB, IE and Wikipedia (Doc) sources when
trying varying learning methods.
%and the ability of different learning methods on them.
We compare four approaches:
(i) the QA system of
\cite{bordes2014question} that performs well on existing datasets
WebQuestions \citep{berant2013semantic} and SimpleQuestions \citep{bordes2015large} that use KBs only; % \citep{bordes2015large},
(ii) supervised embeddings that do not make use of a KB at all
but learn question-to-answer embeddings directly
and hence act as a sanity check \citep{dodge2015evaluating};
(iii) Memory Networks; and (iv) Key-Value
Memory Networks.
Performance is reported using the accuracy of the top hit (single answer)
over all possible answers (all entities), i.e. the hits@1 metric measured in percent.
In all cases hyperparameters are optimized on the development set, including
the memory representations of Sec. \ref{sec:featuremap} for MemNNs and KV-MemNNs.
As MemNNs do not support key-value pairs, we concatenate key and value together
when they differ instead.
%
%The best choice of hops was $H=1$ for MemNNs and $H=2$ for KV-MemNNs.


The main results are given in Table \ref{table:main-res}.
The  QA system of \cite{bordes2014question} outperforms Supervised Embeddings
and Memory Networks for KB and IE-based KB representations, but is designed
to work with a KB, not with documents (hence the N/A in that column).
%Key-Value Memory Networks outperform Memory Networks and all other methods
%on all three data source types.
However, Key-Value Memory Networks outperform all other methods
on all three data source types.
Reading from Wikipedia documents directly (Doc) outperforms an IE-based KB (IE),
which is an encouraging result towards automated machine reading though a
gap to a human-annotated KB still remains (93.9 vs. 76.2).
The best memory representation for directly reading
documents uses ``Window-level + Center Encoding + Title''
($W=7$ and $H=2$);
see Table \ref{table:memkv-res} for a comparison of results for different
representation types.
Both center encoding and title features help the window-level representation, while
sentence-level is inferior.

\begin{table}[t!]
	\begin{center}
    	\resizebox{0.75\linewidth}{!}{{
    	\begin{tabular}{l|c|c|c|}
      		Question Type  & KB & IE & Doc \\
      		\hline
		Writer to Movie           &  97  &  72  &  91  \\
		Tag to Movie               &  85  &  35  &  49  \\
		Movie to Year             &  95   &  75  &  89  \\
		Movie to Writer           &  95   &  61  &  64  \\
		Movie to Tags             &  94   &  47  &  48  \\
		Movie to Language     &  96   &  62  &  84  \\
		Movie to IMDb Votes   &  92   &  92  &  92  \\
		Movie to IMDb Rating  &  94   &  75  &  92  \\
		Movie to Genre           &  97   &  84  &  86  \\
		Movie to Director        &  93   &  76  &  79  \\
		Movie to Actors           &  91   &  64  &  64  \\
		Director to Movie        &  90   &  78  &   91  \\
		Actor to Movie            &  93   &   66  &  83  \\
    	\end{tabular}
    	}}
    	\caption{
	\label{table:breakdown}
{Breakdown of test results (\% hits@1) on \WikiMovies for
Key-Value Memory Networks
%KV-MemNNs
using different
knowledge representations.}}
 	\end{center}
%  \vspace{-3ex}
\end{table}



\paragraph{QA Breakdown}
A breakdown by question type comparing the different data sources for KV-MemNNs is given
in Table \ref{table:breakdown}. IE loses out especially to Doc (and KB) on
Writer, Director and Actor to Movie, perhaps because coreference is difficult in these cases --
although it has other losses elsewhere too. Note that only 56\% of
subject-object pairs in IE match the triples in the original KB, so losses are expected.
Doc loses out to KB particularly on Tag to Movie, Movie to Tags, Movie to Writer and
Movie to Actors. Tag questions
are hard because they can reference more or less any word in the entire
Wikipedia document; see Table \ref{fig:blade}. Movie to Writer/Actor are hard
because there is likely only one or a few references to the answer across all documents, whereas
for Writer/Actor to Movie there are more possible answers to find.


\begin{table}[t!]
	\begin{center}
          \begin{small}
  	\begin{tabular}{l|c|c|}
      	    Knowledge Representation   &  KV-MemNN  \\
     	    \hline
       	KB                               &   93.9    \\
      	One Template Sentence            &   82.9    \\
       	All Templates Sentences          &   80.0        \\
       	One Template + Coreference      &   76.0         \\
       	One Template + Conjunctions       &   74.0        \\
       	All Templates + Conj. + Coref.   &   72.5       \\
       	Wikipedia Documents              &    76.2   \\
  	 \end{tabular}
 %   	}}
    	\caption{
	\label{tab:templateres}
              { Analysis of test set results (\% hits@1) for KB vs. Synthetic Docs on \WikiMovies.}}
          \end{small}
  	\end{center}
%  \vspace{-3ex}
\end{table}



\paragraph{KB vs. Synthetic Document Analysis}
%\textcolor{red}{TODO: talk about templates (or not)}
To further understand the difference between using a KB versus reading documents directly,
we conducted an experiment where we constructed synthetic documents using the KB.
For a given movie, we use a simple grammar to construct a synthetic ``Wikipedia'' document
based on the KB triples: for each relation type we have a set of template phrases
(100 in total)  used to generate the fact, e.g.
``Blade Runner came out in 1982'' for the entry {\sc Blade Runner release\_year 1982}.
We can then parameterize the complexity of our synthetic documents:
(i) using one template, or all of them;
(ii) using conjunctions to combine facts into single sentences or not;
and (iii) using coreference between sentences where we replace the movie name with ``it''.\footnote{This data is also part of the \WikiMovies benchmark.}
The purpose of this experiment is to find which aspects are responsible for the gap
in performance to a KB.
The results are given in Table \ref{tab:templateres}.
They indicate that some of the loss (93.9\% for KB to 82.9\% for One Template Sentence)
 in performance is due directly to representing in sentence form, making the subject, relation
and object harder to extract.
Moving to a larger number of templates does not deteriorate performance much (80\%).
 The remaining performance drop seems to be split roughly
equally between conjunctions (74\%) and coreference (76\%).
%When combined, which is the hardest synthetic dataset (All Templates + Conj. + Coref.), this
The hardest synthetic dataset combines these (All Templates + Conj. + Coref.) and
is actually harder than using the real Wikipedia documents (72.5\% vs. 76.2\%).
This is possibly because the amount of conjunctions and coreferences we make are artificially
too high (50\% and 80\% of the time, respectively).


\begin{table}[t!]
	\begin{center}
    	\resizebox{1\linewidth}{!}{{
	\begin{tabular}{l|c|c|}
      		Method & MAP & MRR \\
      		\hline
		Word Cnt                                                 &  0.4891 & 0.4924 \\
		Wgt Word Cnt                                          &  0.5099 & 0.5132 \\
		2-gram CNN    \citep{yang2015wikiqa}    & 0.6520 & 0.6652 \\
		AP-CNN  \citep{santos2016attentive}      & 0.6886 & 0.6957  \\
		Attentive LSTM \citep{miao2015neural}          & 0.6886  & 0.7069 \\
		Attentive CNN \citep{yin2015convolutional}    & 0.6921  & 0.7108 \\
		L.D.C. \citep{wang2016sentence}  & 0.7058  & 0.7226 \\
		\hline
      		Memory Network                     &       0.5170  &       0.5236  \\
      		Key-Value Memory Network   & {\bf 0.7069} & {\bf 0.7265} \\
    	\end{tabular}
    	}}
    	\caption{
	\label{tab:wikiqares}
              { Test results on WikiQA.}}
  	\end{center}
%  \vspace{-3ex}
\end{table}


\subsection{WikiQA} \label{sec:wikiqa}

{\sc WikiQA} \citep{yang2015wikiqa} is an existing dataset for answer sentence selection
using Wikipedia as the knowledge source. The task is, given a question, to select
the sentence coming from a Wikipedia document that best answers the question,
where performance is measured using mean average precision (MAP) and mean reciprocal rank (MRR)
of the ranked set of answers. The dataset uses a pre-built
information retrieval step and hence provides a fixed set of candidate sentences per question,
 so systems do not have to consider ranking all of Wikipedia.
In contrast to \WikiMovies, the training set size is small ($\sim$1000 examples) while
the topic is much more broad (all of Wikipedia, rather than just movies) and the
questions can only be answered by reading the documents, so no comparison to the use
of KBs can be performed. However, a wide range of methods have already been tried on
{\sc WikiQA}, thus providing a useful benchmark to test if the same results
found on \WikiMovies carry across to {\sc WikiQA}, in particular the performance
of Key-Value Memory Networks.


Due to the size of the training set, following many
other works \citep{yang2015wikiqa,santos2016attentive,miao2015neural}
we pre-trained the word vectors (matrices $A$ and $B$ which are constrained to be identical)
before training KV-MemNNs.
We employed Supervised Embeddings \citep{dodge2015evaluating}
for that goal, training on all of Wikipedia while
treating the input as a random sentence and the target as the subsequent sentence.
We then trained KV-MemNNs with dropout regularization:
we sample words from the question, memory representations and the answers,
choosing the dropout rate using the development set.
Finally, again following other successful methods \citep{yin2015convolutional},
we combine our approach
with exact matching word features between question and answers.
Key hashing was not used as candidates were already pre-selected.
To represent the memories, we used the Window-Level representation (the best choice on
the dev set was $W=7$) as the key and the whole sentence as the value, as the value should match the answer which in this case is a sentence.
Additionally, in the representation
all numbers in the text and the phrase ``how many'' in the question
were replaced with the feature ``\_number\_''.
The best choice of hops was also $H=2$ for KV-MemNNs.

The results are given in Table \ref{tab:wikiqares}.
Key-Value Memory Networks outperform a large set of other methods,
although the results of the L.D.C. method of \citep{wang2016sentence} are very similar.
Memory Networks, which cannot easily pair windows to sentences, perform much worse,
highlighting the importance of key-value memories.


%\vspace{-0.5ex}
\section{Conclusion}
\vspace{-0.5ex}

%Conclusion!

%key,value for graph, duh duh duh
%]


We studied the problem of directly reading documents in order to answer questions,
concentrating our analysis on the gap between such direct methods and using
human-annotated or automatically constructed KBs.
We presented a new model, Key-Value Memory Networks, which helps bridge this gap,
outperforming several other methods across two datasets, \WikiMovies and {\sc WikiQA}.
However, some gap in performance still remains. \WikiMovies serves as an
 analysis tool to shed some light on the causes.
Future work should try to close this gap further.

Key-Value Memory Networks are versatile models for reading documents or KBs and answering
questions about them---allowing to encode prior knowledge about the task at hand
in the key and value memories. These models could be applied to storing and
reading memories
for other tasks as well, and future work should try them in other domains,
such as in a full dialog setting.
%dialog,
%for example dialgog



\bibliography{dialog}
\bibliographystyle{natbib}

\end{document}

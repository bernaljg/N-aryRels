\documentclass{article} % For LaTeX2e
\usepackage{iclr2016_conference,times}
\usepackage{hyperref}
\usepackage{url}
\usepackage{amsmath,amssymb,float}
\usepackage{multirow}
\usepackage{tabularx}
\usepackage{todonotes}
\usepackage{caption}
\usepackage{subcaption}
\DeclareMathOperator{\E}{\mathbb{E}}
\def\!#1{\boldsymbol{#1}}
\def\*#1{\mathbf{#1}}
\newcommand\independent{\protect\mathpalette{\protect\independenT}{\perp}}
\def\independenT#1#2{\mathrel{\rlap{$#1#2$}\mkern2mu{#1#2}}}


\title{The Variational Fair Autoencoder}
  
\author{Christos Louizos$^*$, Kevin Swersky$^\times$, Yujia Li$^\times$, Max Welling$^{*\dagger\ddagger}$, Richard Zemel$^{\times\dagger}$\\
	$^*$ Machine Learning Group, University of Amsterdam\\
	$^\times$Department of Computer Science, University of Toronto\\
	$^\dagger$ Canadian Institute for Advanced Research (CIFAR)\\
	$^\ddagger$ University of California, Irvine\\
	\texttt{C.Louizos@uva.nl, \{kswersky, yujiali\}@cs.toronto.edu}\\
	\texttt{M.Welling@uva.nl, zemel@cs.toronto.edu}}
	
\newcommand{\fix}{\marginpar{FIX}}
\newcommand{\new}{\marginpar{NEW}}

\iclrfinalcopy % Uncomment for camera-ready version

\begin{document}

\maketitle

\begin{abstract}
We investigate the problem of learning representations that are invariant to certain nuisance or sensitive factors of variation in the data while retaining as much of the remaining information as possible. Our model is based on a variational autoencoding architecture~\citep{kingma2013auto, rezende2014stochastic} with priors that encourage independence between sensitive and latent factors of variation. Any subsequent processing, such as classification, can then be performed on this purged latent representation. To remove any remaining dependencies we incorporate an additional penalty term based on the ``Maximum Mean Discrepancy'' (MMD)~\citep{gretton2006kernel} measure. We discuss how these architectures can be efficiently trained on data and show in experiments that this method is more effective than previous work in removing unwanted sources of variation while maintaining informative latent representations. 
\end{abstract}

\section{Introduction} 
In ``Representation Learning'' one tries to find representations of the data that are informative for a particular task while removing the factors of variation that are uninformative and are typically detrimental for the task under consideration. Uninformative dimensions are often called ``noise'' or ``nuisance variables'' while informative dimensions are usually called latent or hidden factors of variation. Many machine learning algorithms can be understood in this way: principal component analysis, nonlinear dimensional reduction and latent Dirichlet allocation are all models that extract informative factors (dimensions, causes, topics) of the data which can often be used to visualize the data. On the other hand, linear discriminant analysis and deep (convolutional) neural nets learn representations that are good for classification. 

In this paper we  consider the case where we wish to learn latent representations where (almost) all of the information about certain known factors of variation are purged from the representation while still retaining as much information about the data as possible. In other words, we want a latent representation $\*z$ that is maximally informative about an observed random variable $\*y$ (e.g., class label) while minimally informative about a \emph{sensitive} or \emph{nuisance} variable $\*s$. By treating $\*s$ as a sensitive variable, i.e. $\*s$ is correlated with our objective, we are dealing with ``fair representations'', a problem previously considered by~\cite{zemel2013learning}. If we instead treat $\*s$ as a nuisance variable we are dealing with ``domain adaptation'', in other words by removing the domain $\*s$ from our representations we will obtain \emph{improved} performance.

In this paper we introduce a novel model based on deep variational autoencoders (VAE)~\citep{kingma2013auto, rezende2014stochastic}. These models can naturally encourage separation between latent variables $\*z$ and sensitive variables $\*s$ by using factorized priors $p(\*s)p(\*z)$. However, some dependencies may still remain when mapping data-cases to their hidden representation using the variational posterior $q(\*z|\*x,\*s)$, which we stamp out using a ``Maximum Mean Discrepancy"~\citep{gretton2006kernel} term that penalizes differences between all order moments of the marginal posterior distributions $q(\*z|\*s=k)$ and $q(\*z|\*s=k')$ (for a discrete RV $\*s$). In experiments we show that this combined approach is highly successful in learning representations that are devoid of unwanted information while retaining as much information as possible from what remains. 

\section{Learning Invariant Representations}
\begin{figure}[ht]
    \begin{center}
        \begin{minipage}[b]{0.48\textwidth}
            \centering
            \includegraphics[scale=0.75]{unsupervised.pdf}
            \caption{Unsupervised model}
        \end{minipage}
        \quad
          \begin{minipage}[b]{0.48\textwidth}
            \centering
            \includegraphics[scale=0.75]{semisupervised.pdf}
            \caption{Semi-supervised model}
        \end{minipage}
    \end{center}
\end{figure}

\subsection{Unsupervised model}
Factoring out undesired variations from the data can be easily formulated as a general probabilistic model which admits two distinct (independent) ``sources''; an observed variable $\*s$, which denotes the variations that we want to remove, and a continuous latent variable $\*z$ which models all the remaining information.  This generative process can be formally defined as:
\begin{align*}
	\*z \sim p(\*z); \qquad \*x \sim p_\theta(\*x| \*z, \*s)
\end{align*}
where $p_\theta(\*x| \*z, \*s)$ is an appropriate probability distribution for the data we are modelling. With this formulation we explicitly encode a notion of `invariance' in our model, since the latent representation is marginally independent of the factors of variation $\*s$. Therefore the problem of finding an invariant representation for a data point $\*x$ and variation $\*s$ can be cast as performing inference on this graphical model and obtaining the posterior distribution of $\*z$, $p(\*z|\*x, \*s)$. 

For our model we will employ a variational autoencoder architecture~\citep{kingma2013auto,rezende2014stochastic}; namely we will parametrize the generative model (decoder) $p_\theta(\*x|\*z,\*s)$ and  the variational posterior (encoder) $q_\phi(\*z|\*x,\*s)$ as (deep) neural networks which accept as inputs $\*z,\*s$ and $\*x,\*s$ respectively and produce the parameters of each distribution after a series of non-linear transformations. Both the model ($\theta$) and variational ($\phi$) parameters will be jointly optimized with the SGVB~\citep{kingma2013auto} algorithm according to a lower bound on the log-likelihood. This parametrization will allow us to capture most of the salient information of $\*x$ in our embedding $\*z$. Furthermore the distributed representation of a neural network would allow us to better resolve the dependencies between $\*x$ and $\*s$ thus yielding a better disentangling between the independent factors $\*z$ and $\*s$. By choosing a Gaussian posterior $q_\phi(\*z|\*x, \*s)$ and standard isotropic Gaussian prior $p(\*z) = \mathcal{N}(\*0, \*I)$ we can obtain the following lower bound:
\begin{align}
	\sum_{n=1}^{N}\log p(\*x_n|\*s_n) & \ge  \sum_{n=1}^{N}\E_{q_\phi(\*z_n|\*x_n,\*s_n)}[\log p_\theta(\*x_n|\*z_n,\*s_n)] - KL(q_\phi(\*z_n|\*x_n,\*s_n) || p(\*z))\\
	                    & = \mathcal{F}(\phi, \theta; \*x_n, \*s_n) \nonumber
\end{align}
with $q_\phi(\*z_n|\*x_n, \*s_n) = \mathcal{N}(\*z_n|\!\mu_n = f_\phi(\*x_n, \*s_n), \!\sigma_n = e^{f_\phi(\*x_n, \*s_n)})$ and $p_\theta(\*x_n|\*z_n, \*s_n) = f_\theta(\*z_n, \*s_n)$ with $f_\theta(\*z_n, \*s_n)$ being an appropriate probability distribution for the data we are modelling.

\subsection{Semi-Supervised model}
Factoring out variations in an unsupervised way can however be harmful in cases where we want to use this invariant representation for a subsequent prediction task. In particular if we have a situation where the nuisance variable $\*s$ and the actual label $\*y$ are correlated, then training an unsupervised model could yield \emph{random} or \emph{degenerate} representations with respect to $\*y$. Therefore it is more appropriate to try to ``inject'' the information about the label during the feature extraction phase. This can be quite simply achieved by introducing a second ``layer'' of latent variables to our generative model where we try to correlate $\*z$ with the prediction task. Assuming that the invariant features are now called $\*z_1$ we enrich the generative story by similarly providing two distinct (independent) sources for $\*z_1$; a discrete (in case of classification)variable $\*y$ which denotes the label of the data point $\*x$ and a continuous latent variable $\*z_2$ which encodes the variation on $\*z_1$ that is not explained by $\*y$ ($\*x$ dependent noise). The process now can be formally defined as: 
\begin{align*}
\*y, \*z_2 \sim \text{Cat}(\*y)p(\*z_2); \qquad  \*z_1 \sim p_\theta(\*z_1|\*z_2, \*y);\qquad \*x \sim p_\theta(\*x| \*z_1,\*s)\nonumber
\end{align*}   
Similarly to the unsupervised case we use a variational auto-encoder and jointly optimize the variational and model parameters. The lower bound now becomes:
\begin{align}
\sum_{n=1}^{N}\log p(\*x_n|\*s_n) & \ge \sum_{n=1}^{N}\E_{q_\phi({\*z_1}_n, {\*z_2}_n, \*y_n| \*x_n, \*s_n)}[\log p(\*z_2) + \log p(\*y_n) + \log p_\theta({\*z_1}_n|{\*z_2}_n, \*y_n) + \nonumber\\&\qquad\qquad\qquad\qquad + \log p_\theta(\*x_n|{\*z_1}_n,\*s_n) - \log q_\phi({\*z_1}_n, {\*z_2}_n, \*y_n|\*x_n, \*s_n)]  
\end{align}
where we assume that the posterior $q_\phi({\*z_1}_n, {\*z_2}_n, \*y_n|\*x_n, \*s_n)$ is factorized as $q_\phi({\*z_1}_n, {\*z_2}_n, \*y_n|\*x_n, \*s_n) = q_\phi({\*z_1}_n|\*x_n, \*s_n)q_\phi(\*y_n|{\*z_1}_n)q_\phi({\*z_2}_n|{\*z_1}_n,\*y_n)$, and where:
\begin{align*}
q_\phi({\*z_1}_n|\*x_n, \*s_n) & = \mathcal{N}({\*z_1}_n|\!\mu_n = f_\phi(\*x_n, \*s_n), \!\sigma_n = e^{f_\phi(\*x_n, \*s_n)}) \\
q_\phi(\*y_n | {\*z_1}_n) & = \text{Cat}(\*y_n|\!\pi_n = \text{softmax}(f_\phi({\*z_1}_n)))\\
q_\phi({\*z_2}_n|{\*z_1}_n, \*y_n) & = \mathcal{N}({\*z_2}_n| \!\mu_n = f_\phi({\*z_1}_n, \*y_n), \!\sigma_n = e^{f_\phi({\*z_1}_n, \*y_n)})\\
p_\theta({\*z_1}_n | {\*z_2}_n, \*y_n) & = \mathcal{N}({\*z_1}_n| \!\mu_n = f_\theta({\*z_2}_n, \*y_n), \!\sigma_n = e^{f_\theta({\*z_2}_n, \*y_n)})\\
p_\theta(\*x_n|{\*z_1}_n, \*s_n) & = f_\theta({\*z_1}_n, \*s_n)
\end{align*}
with $f_\theta({\*z_1}_n, \*s_n)$ again being an appropriate probability distribution for the data we are modelling. The model proposed here can be seen as an extension to the `stacked M1+M2' model originally proposed from~\cite{kingma2014semi}, where we have additionally introduced the nuisance variable $\*s$ during the feature extraction. Thus following~\cite{kingma2014semi} we can also handle the `semi-supervised' case, i.e., missing labels. In situations where the label is observed the lower bound takes the following form (exploiting the fact that we can compute some Kullback-Leibler divergences explicitly in our case):
\begin{align}
\sum_{n=1}^{N}\mathcal{L}_s(\phi, \theta; \*x_n, \*s_n, \*y_n) & =\sum_{n=1}^{N_s} \E_{q_\phi({\*z_1}_n|\*x_n, \*s_n)}[-KL(q_\phi({\*z_2}_n|{\*z_1}_n, \*y_n) || p(\*z_2)) + \log p_\theta(\*x_n|{\*z_1}_n, \*s_n)] +\nonumber\\&\qquad\qquad + \E_{q_\phi({\*z_1}_n|\*x_n, \*s_n)q_\phi({\*z_2}_n|{\*z_1}_n,\*y_n)}[\log p_\theta({\*z_1}|{\*z_2}_n, \*y_n)- \log q_\phi({\*z_1}_n|\*x_n \*s_n)]
\end{align}
and in the case that it is not observed we use $q(\*y_n|{\*z_1}_n)$ to `impute' our data:
\begin{align}
\sum_{m=1}^{M}\mathcal{L}_u(\phi, \theta; \*x_m, \*s_m) & = \sum_{m=1}^{M}\E_{q_\phi({\*z_1}_m|\*x_m,\*s_m)}[ - KL (q(\*y_m|{\*z_1}_m) || p(\*y_m)) + \log p_\theta(\*x_m|{\*z_1}_m, \*s_m)] +\nonumber\\&\qquad\qquad +  \E_{q_\phi({\*z_1}_m, \*y_m|\*x_m,\*s_m)}[ - KL (q_\phi({\*z_2}_m|{\*z_1}_m,\*y_m) || p(\*z_2))] +\nonumber\\&\qquad\qquad+  \E_{q_\phi({\*z_1}_m, \*y_m, {\*z_2}_m|\*x_m, \*s_m)}[\log p_\theta({\*z_1}_m|{\*z_2}_m,\*y_m) -\log q_\phi({\*z_1}_m|\*x_m,\*s_m) ]
\end{align}
therefore the final objective function is:
\begin{align}
    \mathcal{F}_{\text{VAE}}(\phi, \theta; \*x_n, \*x_m, \*s_n, \*s_m, \*y_n) & = \sum_{n=1}^{N}\mathcal{L}_s(\phi, \theta; \*x_n, \*s_n, \*y_n) + \sum_{m=1}^{M}\mathcal{L}_u(\phi, \theta; \*x_m, \*s_m) + \nonumber \\ &\qquad\qquad +  \alpha\sum_{n=1}^{N}\E_{q({\*z_1}_n|\*x_n, \*s_n)}[- \log q_\phi(\*y_n|{\*z_1}_n)]
\end{align}
where the last term is introduced so as to ensure that the predictive posterior $q_\phi(\*y|\*z_1)$ learns from both labeled and unlabeled data. This semi-supervised model will be called ``VAE'' in our experiments.

However, there is a subtle difference between the approach of~\cite{kingma2014semi} and our model. Instead of training separately each layer of stochastic variables we optimize the model jointly. The potential advantages of this approach are two fold: as we previously mentioned if the label $\*y$ and the nuisance information $\*s$ are correlated then training a (conditional) feature extractor separately poses the danger of creating a degenerate representation with respect to the label $\*y$. Furthermore the label information will also better guide the feature extraction towards the more salient parts of the data, thus maintaining most of the (predictive) information.


\subsection{Further invariance via Maximum Mean Discrepancy}
Despite the fact that we have a model that encourages statistical independence between $\*s$ and $\*z_1$ a-priori we might still have some dependence in the (approximate) marginal posterior $q_{\phi}(\*z_1|\*s)$. In particular, this can happen if the label $\*y$ is correlated with the sensitive variable $\*s$, which can allow information about $\*s$ to ``leak'' into the posterior. Thus instead we could maximize a ``penalized'' lower bound where we impose some sort of regularization on the marginal $q_{\phi}(\*z_1|\*s)$. In the following we will describe one way to achieve this regularization through the Maximum Mean Discrepancy (MMD)~\citep{gretton2006kernel} measure.

 
\subsubsection{Maximum Mean Discrepancy}
Consider the problem of determining whether two datasets $\{ \*X \} \sim P_0$ and $\{ \*X' \} \sim P_1$ are drawn from the same distribution, i.e., $P_0 = P_1$. A simple test is to consider the distance between empirical statistics $\psi(\cdot)$ of the two datasets:
\begin{equation}
    \left \| \frac{1}{N_0} \sum_{i=1}^{N_0} \psi(\*x_i) - \frac{1}{N_1} \sum_{i=1}^{N_1} \psi(\*x'_i) \right \|^2. \label{eq:premmd}
\end{equation}
Expanding the square yields an estimator composed only of inner products on which the kernel trick can be applied. The resulting estimator is known as Maximum Mean Discrepancy (MMD)~\citep{gretton2006kernel}:
\begin{align}
    \ell_{\mathrm{MMD}}(\*X, \*X') &= \frac{1}{N_0^2} \sum_{n=1}^{N_0} \sum_{m=1}^{N_0} k(\*x_n, \*x_{m}) + \frac{1}{N_1^2} \sum_{n=1}^{N_1} \sum_{m=1}^{N_1} k(\*x'_n, \*x'_m) - \frac{2}{N_0 N_1} \sum_{n=1}^{N_0} \sum_{m=1}^{N_1} k(\*x_n, \*x'_m). \label{eq:mmd}
\end{align}
Asymptotically, for a universal kernel such as the Gaussian kernel $k(x,x')=e^{-\gamma \| \*x - \*x' \|^2}$, $\ell_{\mathrm{MMD}}(\*X, \*X')$ is $0$ if and only if $P_0 = P_1$. Equivalently, minimizing MMD can be viewed as matching all of the moments of $P_0$ and $P_1$.  Therefore, we can use it as an extra ``regularizer'' and force the model to try to match the moments between the marginal posterior distributions of our latent variables, i.e., $q_{\phi}(\*z_1|s=0)$ and $q_{\phi}(\*z_1| s=1)$ (in the case of binary nuisance information $\*s$\footnote{In case that we have more than two states for the nuisance information $\*s$, we minimize the MMD penalty between each marginal $q(\*z|\*s=k)$ and $q(\*z)$, i.e., $\sum_{k=1}^{K}\ell_{\mathrm{MMD}}({\*Z_1}_{\*s=k}, {\*Z_1})$ for all possible states $K$ of $\*s$.}). By adding the MMD penalty into the lower bound of our aforementioned VAE architecture we obtain our proposed model, the ``Variational Fair Autoencoder'' (VFAE):
\begin{align}
    \mathcal{F}_{\text{VFAE}}(\phi, \theta; \*x_n, \*x_m, \*s_n, \*s_m, \*y_n) & = \mathcal{F}_{\text{VAE}}(\phi, \theta; \*x_n, \*x_m, \*s_n, \*s_m, \*y_n) -\beta \ell_{\mathrm{MMD}}({\*Z_1}_{\*s=0}, {\*Z_1}_{\*s=1})
\end{align}
where:
\begin{align}
  \ell_{\mathrm{MMD}}({\*Z_1}_{\*s=0}, {\*Z_1}_{\*s=1}) & = \|\E_{\tilde{p}(\*x|\*s=0)}[\E_{q(\*z_1|\*x, \*s=0)}[\psi(\*z_1)]] - E_{\tilde{p}(\*x|\*s=1)}[\E_{q(\*z_1|\*x, \*s=1)}[\psi(\*z_1)]]\|^2
\end{align}

\subsection{Fast MMD via Random Fourier Features}
A naive implementation of MMD in minibatch stochastic gradient descent would require computing the $M\times M$ Gram matrix for each minibatch during training, where $M$ is the minibatch size. Instead, we can use random kitchen sinks~\citep{rahimi2009weighted} to compute a feature expansion such that computing the estimator $(\ref{eq:premmd})$ approximates the full MMD (\ref{eq:mmd}). To compute this, we draw a random $K\times D$ matrix $\*W$, where $K$ is the dimensionality of $\*x$, $D$ is the number of random features and each entry of $\*W$ is drawn from a standard isotropic Gaussian.
The feature expansion is then given as:
\begin{align}
\psi_\*W(\*x) &= \sqrt{\frac{2}{D}}\mathrm{cos}\left ( \sqrt{\frac{2}{\gamma}} \*x\*W + \*b \right ).
\end{align}
where $\*b$ is a $D$-dimensional uniform random vector with entries in $[0,2\pi]$. \cite{zhao2014fastmmd} have successfully applied the idea of using random kitchen sinks to approximate MMD. This estimator is fairly accurate, and is typically much faster than the full MMD penalty. We use $D=500$ in our experiments.


\section{Experiments}
% !TEX root = ../multi_task.tex

We evaluate the presented MTL method on a number of problems. First, we use MultiMNIST \citep{multi_mnist}, an MTL adaptation of MNIST \citep{mnist}. Next, we tackle multi-label classification on the CelebA dataset \citep{celeba} by considering each label as a distinct binary classification task. These problems include both classification and regression, with the number of tasks ranging from 2 to 40. Finally, we experiment with scene understanding, jointly tackling the tasks of semantic segmentation, instance segmentation, and depth estimation on the Cityscapes dataset \citep{cityscapes}. We discuss each experiment separately in the following subsections.

The baselines we consider are (i) \textbf{uniform scaling:} minimizing a uniformly weighted sum of loss functions \mbox{$\frac{1}{T}\sum_t \lL^t$}, \mbox{(ii) \textbf{single task:}} solving tasks independently, \mbox{(iii) \textbf{grid search:}} exhaustively trying various values from $\{ c^t \in [0,1] | \sum_t c^t = 1\}$ and optimizing for $\frac{1}{T}\sum_t c^t \lL^t$, \mbox{(iv) \textbf{\citet{Kendall2018}:}} using the uncertainty weighting proposed by \citet{Kendall2018}, and \mbox{(v) \textbf{GradNorm:}} using the normalization proposed by \citet{Chen2018}.



\subsection{MultiMNIST}
\label{sec:multi_mnist_exp}

Our initial experiments are on MultiMNIST, an MTL version of the MNIST dataset \citep{multi_mnist}. In order to convert digit classification into a multi-task problem, \citet{multi_mnist} overlaid multiple images together. We use a similar construction. For each image, a different one is chosen uniformly in random. Then one of these images is put at the top-left and the other one is at the bottom-right. The resulting tasks are: classifying the digit on the top-left (task-L) and classifying the digit on the bottom-right (task-R). We use 60K examples and directly apply existing single-task MNIST models. The MultiMNIST dataset is illustrated in the supplement.

We use the LeNet architecture \citep{mnist}. We treat all layers except the last as the representation function $g$ and put two fully-connected layers as task-specific functions (see the supplement for details). We visualize the performance profile as a scatter plot of accuracies on task-L and task-R in Figure~\ref{fig:multi_mnist_performance_curve}, and list the results in Table~\ref{tab:multi_mnist}.

In this setup, any static scaling results in lower accuracy than solving each task separately (the single-task baseline). The two tasks appear to compete for model capacity, since increase in the accuracy of one task results in decrease in the accuracy of the other. Uncertainty weighting \citep{Kendall2018} and GradNorm \citep{Chen2018} find solutions that are slightly better than grid search but distinctly worse than the single-task baseline. In contrast, our method finds a solution that efficiently utilizes the model capacity and yields accuracies that are as good as the single-task solutions. This experiment demonstrates the effectiveness of our method as well as the necessity of treating MTL as multi-objective optimization. Even after a large hyper-parameter search, \emph{any} scaling of tasks does not approach the effectiveness of our method.



\subsection{Multi-Label Classification}

\begin{figure}[t]
\includegraphics[width=\textwidth]{radar_full_new}
\vspace{1mm}
\caption{Radar charts of percentage error per attribute on CelebA \citep{celeba}. Lower is better. We divide attributes into two sets for legibility: easy on the left, hard on the right. Zoom in for details.}
\label{fig:multi_label_radar}
\end{figure}


\begin{wraptable}{r}{0.3\textwidth}
%\vspace{-4mm}
\captionof{table}{Mean of error per category of MTL algorithms in multi-label classification on CelebA \citep{celeba}.}
\begin{tabular}{r@{\hspace{2mm}}c@{}}
\toprule
& Average  \\
&  error \\
\midrule
Single task & $8.77$ \\
Uniform scaling & $9.62$ \\
\citealt{Kendall2018} & $9.53$ \\
GradNorm & $8.44$ \\
Ours & $\mathbf{8.25}$  \\
\bottomrule
\end{tabular}
\label{table:multi_label_bar}
%\vspace{-5mm}
\end{wraptable}

Next, we tackle multi-label classification. Given a set of attributes, multi-label classification calls for deciding whether each attribute holds for the input. We use the CelebA dataset \citep{celeba}, which includes 200K face images annotated with 40 attributes. Each attribute gives rise to a binary classification task and we cast this as a 40-way MTL problem. We use ResNet-18 \citep{resnet} without the final layer as a shared representation function, and attach a linear layer for each attribute (see the supplement for further details).


We plot the resulting error for each binary classification task as a radar chart in Figure~\ref{fig:multi_label_radar}. The average over them is listed in Table~\ref{table:multi_label_bar}. We skip grid search since it is not feasible over 40 tasks. Although uniform scaling is the norm in the multi-label classification literature, single-task performance is significantly better. Our method outperforms baselines for significant majority of tasks and achieves comparable performance in rest. This experiment also shows that our method remains effective when the number of tasks is high.


\subsection{Scene Understanding}

To evaluate our method in a more realistic setting, we use scene understanding. Given an RGB image, we solve three tasks: semantic segmentation (assigning pixel-level class labels), instance segmentation (assigning pixel-level instance labels), and monocular depth estimation (estimating continuous disparity per pixel). We follow the experimental procedure of \citet{Kendall2018} and use an encoder-decoder architecture. The encoder is based on ResNet-50 \citep{resnet} and is shared by all three tasks. The decoders are task-specific and are based on the pyramid pooling module \citep{pspnet} (see the supplement for further implementation details).

Since the output space of instance segmentation is unconstrained (the number of instances is not known in advance), we use a proxy problem as in \citet{Kendall2018}. For each pixel, we estimate the location of the center of mass of the instance that encompasses the pixel. These center votes can then be clustered to extract the instances. In our experiments, we directly report the MSE in the proxy task. Figure~\ref{fig:cityscapes_performance_profile} shows the performance profile for each pair of tasks, although we perform all experiments on all three tasks jointly. The pairwise performance profiles shown in Figure~\ref{fig:cityscapes_performance_profile} are simply 2D projections of the three-dimensional profile, presented this way for legibility. The results are also listed in Table~\ref{tab:cityscapes_results}.

MTL outperforms single-task accuracy, indicating that the tasks cooperate and help each other. Our method outperforms all baselines on all tasks.


\subsection{Role of the Approximation}

In order to understand the role of the approximation proposed in Section~\ref{sec:approximation}, we compare the final performance and training time of our algorithm with and without the presented approximation in Table~\ref{tab:approximation_tradeoff} (runtime measured on a single Titan Xp GPU). For a small number of tasks (3 for scene understanding), training time is reduced by 40\%. For the multi-label classification experiment (40 tasks), the presented approximation accelerates learning by a factor of 25.

On the accuracy side, we expect both methods to perform similarly as long as the full-rank assumption is satisfied. As expected, the accuracy of both methods is very similar. Somewhat surprisingly, our approximation results in slightly improved accuracy in all experiments. While counter-intuitive at first, we hypothesize that this is related to the use of SGD in the learning algorithm. Stability analysis in convex optimization suggests that if gradients are computed with an error $\hat{\nabla}_\btheta \mathcal{L}^t = \nabla_\btheta \mathcal{L}^t + \mathbf{e}^t$ ($\btheta$ corresponds to $\btheta^{sh}$ in (\ref{eq:kkt_opt})), as opposed to $\mathbf{Z}$ in the approximate problem in \ref{eq:approx}, the error in the solution is bounded as $\|\hat{\mathbf{\alpha}} - \mathbf{\alpha} \|_2 \leq \mathcal{O}(\max_t \|\mathbf{e}^t\|_2)$. Considering the fact that the gradients are computed over the full parameter set (millions of dimensions) for the original problem and over a smaller space for the approximation (batch size times representation which is in the thousands), the dimension of the error vector is significantly higher in the original problem. We expect the $l_2$ norm of such a random vector to depend on the dimension.

In summary, our quantitative analysis of the approximation suggests that (i) the approximation does not cause an accuracy drop and (ii) by solving an equivalent problem in a lower-dimensional space, our method achieves both better computational efficiency and higher stability.

  {\small
  \begin{table}[t]
%  \vspace{-4mm}
  \caption{Effect of the MGDA-UB approximation. We report the final accuracies as well as training times for our method with and without the approximation.}
  %\vspace{1mm}
  \centering
  \begin{tabular}{@{}r@{\hspace{3mm}}c@{\hspace{3mm}}c@{\hspace{2mm}}c@{\hspace{2mm}}c@{}c@{\hspace{5mm}}c@{\hspace{2mm}}c@{}}
  \toprule
  & \multicolumn{4}{c}{Scene understanding (3 tasks)} &  & \multicolumn{2}{c}{Multi-label (40 tasks)}  \\
  \cmidrule(r){2-5} \cmidrule(lr){7-8}
                  & Training & Segmentation & Instance  & Disparity      & & Training & Average \\
                 & time     &  mIoU [\%]       & error [px] & error [px] & & time (hour)      & error \\
  \midrule
  Ours (w/o approx.) & $38.6$ & $66.13$ & $10.28$ & $2.59$ & & $429.9$ & $8.33$ \\
  Ours & $\mathbf{23.3}$ & $\mathbf{66.63}$ & $\mathbf{10.25}$ & $\mathbf{2.54}$  & & $\mathbf{16.1}$ & $\mathbf{8.25}$ \\
  \bottomrule
  \end{tabular}
  %\vspace{-2mm}
  \label{tab:approximation_tradeoff}
  \end{table}}


\section{Related Work}
\paragraph{3D Object Detection from RGB-D Data} Researchers have approached the 3D detection problem by taking various ways to represent RGB-D data.

\emph{Front view image based methods:} ~\cite{chen2016monocular, mousavian20163d, xiang2015data} take monocular RGB images and shape priors or occlusion patterns to infer 3D bounding boxes. ~\cite{li2016vehicle, deng2017amodal} represent depth data as 2D maps and apply CNNs to localize objects in 2D image. In comparison we represent depth as a point cloud and use advanced 3D deep networks (PointNets) that can exploit 3D geometry more effectively.

\emph{Bird's eye view based methods:} MV3D~\cite{cvpr17chen} projects LiDAR point cloud to bird's eye view and trains a region proposal network (RPN~\cite{ren2015faster}) for 3D bounding box proposal. However, the method lags behind in detecting small objects, such as pedestrians and cyclists and cannot easily adapt to scenes with multiple objects in vertical direction.
%Our method shares the idea with~\cite{cvpr17chen} in reducing 3D search cost by 2D search first. What differentiates our method from \cite{cvpr17chen} is that, \hao{???} instead of projecting point cloud to images costing loss in 3D geometry, we directly apply PointNet to point clouds that correspond to the 2D regions. % Besides, our method and MV3D can potentially be combined in the bird's eye setting. 3D proposals from our frustum-based PointNet and MV3D can be combined and our 3D network can also be used for bounding box estimation for point cloud in the bird's eye 2D region.

\emph{3D based methods:} ~\cite{wang2015voting, song2014sliding} train 3D object classifiers by SVMs on hand-designed geometry features extracted from point cloud and then localize objects using sliding-window search. \cite{engelcke2017vote3deep} extends ~\cite{wang2015voting} by replacing SVM with 3D CNN on voxelized 3D grids. \cite{ren2016three} designs new geometric features for 3D object detection in a point cloud. \cite{song2016deep, li20163d} convert a point cloud of the entire scene into a volumetric grid and use 3D volumetric CNN for object proposal and classification. Computation cost for those method is usually quite high due to the expensive cost of 3D convolutions and large 3D search space.
%In comparison, we use 2D region proposals from RGB images to reduce the search space from the entire 3D scenes into 3D frustums. Since the points cloud in the frustums have largely varying depth ranges and can be very sparse, it's not applicable to apply CNN on bird's eye view or apply 3D CNN in grids. Our frustum-based PointNet, on the other hand, suits well for this type of data and is able to accurately estimate 3D bounding box with good efficiency.
Recently, \cite{lahoud20172d} proposes a 2D-driven 3D object detection method that is similar to ours in spirit. However, they use hand-crafted features (based on histogram of point coordinates) with simple fully connected networks to regress 3D box location and pose, which is sub-optimal in both speed and performance. In contrast, we propose a more flexible and effective solution with deep 3D feature learning (PointNets).
%In addition we also get 3D instance segmentation as intermediate outputs. Evaluated on SUN-RGBD we show our method is \emph{8.9\%} better than theirs in mAP and \emph{34x} faster at the same time.


% \begin{enumerate}
%     \item ZOOX~\cite{mousavian20163d} image based
%     \item Vote3Deep~\cite{engelcke2017vote3deep} 3d cnn. Recent LIDAR-based methods place 3D windows in 3D voxel grids to score the point cloud
%     \item Voting for Voting~\cite{wang2015voting} Recent LIDAR-based methods place 3D windows in 3D voxel grids to score the point cloud. apply SVM classifers on 3D grids encoded with geometry features
%     \item MV3D~\cite{cvpr17chen}
%     \item VeloFCN~\cite{li2016vehicle} apply convolutional networks to the front view point map in a dense box prediction scheme
%     \item 3DOP~\cite{chen20153d} image based. reconstructs depth from stereo images and uses an energy minimization approach to generate 3D box proposals, which are fed to an R-CNN [10] pipeline for object recognition
%     \item Mono3D~\cite{chen2016monocular} image based. shares the same pipeline with 3DOP, it generates 3D proposals from monocular images.
%     \item 3DFCN~\cite{li20163d} 3d cnn.
%     \item 3DVP~\cite{xiang2015data} introduces 3D voxel patterns and employ a set of ACF detectors to do 2D detection and 3D pose estimation
%     \item Are Cars just 3D Box?~\cite{zeeshan2014cars} fit model to image patch
%     \item ~\cite{zia2013detailed} fit model to image patch
% \end{enumerate}
% \begin{enumerate}
%     \item SlidingShapes~\cite{song2014sliding} apply SVM classifers on 3D grids encoded with geometry features
%     \item DeepSlidingShapes~\cite{song2015sun} 3d cnn.
%     \item 2D-driven~\cite{lahoud20172d}
%     \item ~\cite{deng2017amodal} rgb-d images
%     \item COG feature~\cite{ren2016three}
%     \item Align 3D model in RGB-D~\cite{gupta2015aligning}
% \end{enumerate}

\paragraph{Deep Learning on Point Clouds}
Most existing works convert point clouds to images or volumetric forms before feature learning. \cite{wu20153d, maturana2015voxnet, qi2016volumetric} voxelize point clouds into volumetric grids and generalize image CNNs to 3D CNNs. ~\cite{li2016fpnn, riegler2016octnet, wang2017cnn, engelcke2017vote3deep} design more efficient 3D CNN or neural network architectures that exploit sparsity in point cloud.
However, these CNN based methods still require quantitization of point clouds with certain voxel resolution.
Recently, a few works~\cite{qi2017pointnet,qi2017pointnetplusplus} propose a novel type of network architectures (PointNets) that directly consumes raw point clouds without converting them to other formats. While PointNets have been applied to single object classification and semantic segmentation, our work explores how to extend the architecture for the purpose of 3D object detection.

\section{Conclusion}
\section{Conclusion}\label{sec:conclusion}
%\vspace{-.1in}
In this work, we apply the attentional encoder-decoder for the task of abstractive summarization with very promising results, outperforming state-of-the-art results significantly on two different datasets. Each of our proposed novel models addresses a specific problem in abstractive summarization, yielding further improvement in performance. We also propose a new dataset for multi-sentence summarization and establish benchmark numbers on it. As part of our future work, we plan to focus our efforts on this data and build more robust models for summaries consisting of multiple sentences.


%Our results strongly demonstrate that sequence-to-sequence models are extremely promising for summarization. Some of the other lessons we learned from our experiments are: (i) the LVT-trick is very useful for summarization as it improves training speed while not sacrificing performance; (ii) traditional methods such as vocabulary expansion and syntax-based features can boost performance of deep learning based models as well. As part of our ongoing work, we are investigating on ways to effectively generate rare words in the summary, which appears to be a glaring weakness in the existing models.  



%\subsubsection*{Acknowledgments}

\bibliography{bibliography}
\bibliographystyle{iclr2016_conference}

\newpage
\appendix

\section{Discrimination metrics}
The Discrimination metric~\citep{zemel2013learning} and the Discrimination metric that takes into account the probabilities of the correct class are mathematically formalized as:
 
\begin{align*}
 \text{Discrimination} & = \bigg|\frac{\sum_{n=1}^{N}\mathbb{I}[y_n^{s=0}]}{N_{s = 0}} - \frac{\sum_{n=1}^{N}\mathbb{I}[y_n^{s=1}]}{N_{s=1}}\bigg| \\  
  \text{Discrimination prob.} & = \bigg|\frac{\sum_{n=1}^{N}p(y_n^{s=0})}{N_{s = 0}} - \frac{\sum_{n=1}^{N}p(y_n^{s=1})}{N_{s=1}}\bigg|
\end{align*}
where $\mathbb{I}[y_n^{s=0}] = 1$ for the predictions $y_n$ that were done on the datapoints with nuisance variable $s = 0$, $N_{s=0}$ denotes the total amount of datapoints that had nuisance variable $s = 0$ and $p(y_n^{s=0})$ denotes the probability of the prediction $y_n$ for the datapoints with $s=0$. For the predictions and their respective probabilities we used a Logistic Regression classifier.

\section{Proxy A-Distance (PAD) for Amazon Reviews dataset}
Similarly to~\cite{2015arXiv150507818G}, we also calculated the Proxy A-distance (PAD)~\citep{ben2007analysis,ben2010theory} scores for the raw data $\*x$ and for the $\*z_1$ representations of VFAE. Briefly, Proxy A-distance is an approximation to the $\mathcal{H}$-divergence measure of domain distinguishability proposed in~\cite{kifer2004detecting} and~\cite{ben2007analysis,ben2010theory}. To compute it we first need to train a learning algorithm on the task of discriminating examples from the source and target domain. Afterwards we can use the test error $\epsilon$ of that algorithm in the following formula:
\begin{align*}
  \text{PAD}(\epsilon)  = 2(1 - 2\epsilon) 
\end{align*}
It is clear that low PAD scores correspond to low discrimination of the source and target domain examples from the classifier. To obtain $\epsilon$ for our model we used Logistic Regression. The resulting plot can be seen in Figure~\ref{fig:pad_reviews}, where we have also added the plot from DANN~\citep{2015arXiv150507818G}, where they used a linear Support Vector Machine for the classifier, as a reference. It can be seen that our VFAE model can factor out the information about $\*s$ better, since the PAD scores on our new representation are, overall, lower than the ones obtained from the DANN architecture.

\begin{figure}[ht]
    \centering
    \begin{subfigure}{.49\textwidth}
    \centering
        \includegraphics[width=1.\textwidth]{pad_vfae_x_reviews.pdf}
  \end{subfigure} %
    \begin{subfigure}{.49\textwidth}
      \centering
        \includegraphics[width=.9\textwidth]{PAD_DANN.pdf}
     \end{subfigure} % 
    \caption{Proxy A-distances (PAD) for the Amazon reviews dataset: left from our VFAE model, right from the DANN model (taken from~\cite{2015arXiv150507818G})}
    \label{fig:pad_reviews}
\end{figure}

\end{document}

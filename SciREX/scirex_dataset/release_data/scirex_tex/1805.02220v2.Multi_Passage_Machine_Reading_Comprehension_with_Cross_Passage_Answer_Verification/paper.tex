%
% File acl2018.tex
%
%% Based on the style files for ACL-2017, with some changes, which were, in turn,
%% Based on the style files for ACL-2015, with some improvements
%%  taken from the NAACL-2016 style
%% Based on the style files for ACL-2014, which were, in turn,
%% based on ACL-2013, ACL-2012, ACL-2011, ACL-2010, ACL-IJCNLP-2009,
%% EACL-2009, IJCNLP-2008...
%% Based on the style files for EACL 2006 by 
%%e.agirre@ehu.es or Sergi.Balari@uab.es
%% and that of ACL 08 by Joakim Nivre and Noah Smith

\documentclass[11pt,a4paper]{article}
\usepackage{authblk}
\usepackage{blindtext}
% \usepackage[draft]{hyperref}
\usepackage[hyperref]{acl2018}
\usepackage{times}
\usepackage{latexsym}
\usepackage{qtree}
\usepackage{graphicx}
\usepackage{amsmath}
\usepackage{amsfonts,amssymb}
\usepackage{multirow}
\usepackage{multicol}
\usepackage{enumitem}
\usepackage{comment}
\usepackage{url}

\aclfinalcopy % Uncomment this line for the final submission
\def\aclpaperid{760} %  Enter the acl Paper ID here

%\setlength\titlebox{5cm}
% You can expand the titlebox if you need extra space
% to show all the authors. Please do not make the titlebox
% smaller than 5cm (the original size); we will check this
% in the camera-ready version and ask you to change it back.
\newcommand{\matr}[1]{\mathbf{#1}}
\renewcommand{\vec}[1]{\mathbf{#1}}

\newcommand{\figref}[1]{Figure \ref{#1}}
\newcommand{\tabref}[1]{Table \ref{#1}}
\newcommand{\secref}[1]{Section \ref{#1}}
% \newcommand{\YZ}[1]{\textcolor{purple}{Yizhong: #1}}

\usepackage{etoolbox}
\makeatletter
\patchcmd{\maketitle}
 {\def\@makefnmark}
 {\def\@makefnmark{}\def\useless@macro}
 {}{}
\makeatother
 

\title{Multi-Passage Machine Reading Comprehension \\ with Cross-Passage Answer Verification}


\author[1 *]{Yizhong Wang\thanks{\llap{\textsuperscript{*}}This work was done while the first author was doing internship at Baidu Inc.}}
\author[2]{Kai Liu}
\author[2]{Jing Liu}
\author[2]{Wei He}
\author[2]{\\Yajuan Lyu}
\author[2]{Hua Wu}
\author[1]{Sujian Li}
\author[2]{Haifeng Wang}
\affil[1]{Key Laboratory of Computational Linguistics, Peking University, MOE, China}
\affil[2]{Baidu Inc., Beijing, China}
% \affil[ ]{\texttt{\{yizhong, lisujian\}@pku.edu.cn}}
\affil[ ]{\tt {\{yizhong, lisujian\}@pku.edu.cn, \{liukai20, liujing46, }}
\affil[ ]{\tt {hewei06, lvyajuan, wu\_hua, wanghaifeng\}@baidu.com}}
\renewcommand\Authands{ and }
\date{}

\begin{document}
\maketitle

\begin{abstract}

Machine reading comprehension (MRC) on real web data usually requires the machine to answer a question by analyzing multiple passages retrieved by search engine. 
Compared with MRC on a single passage, multi-passage MRC is more challenging, since we are likely to get multiple confusing answer candidates from different passages.
To address this problem, we propose an end-to-end neural model that enables those answer candidates from different passages to verify each other based on their content representations.
% To address this problem, we propose an end-to-end neural model by leveraging the content representations of the candidate answers to verify each other. 
% Specifically, our model employs joint training and predicting the final answer by combining the confidence scores from three aspects: the answer boundary, the answer content and the cross-passage answer verification. 
Specifically, we jointly train three modules that can predict the final answer based on three factors: the answer boundary, the answer content and the cross-passage answer verification. 
The experimental results show that our method outperforms the baseline by a large margin and achieves the state-of-the-art performance on the English MS-MARCO dataset and the Chinese DuReader dataset, both of which are designed for MRC in real-world settings. 

% Machine reading comprehension (MRC) on real web data usually requires the model to answer the question given multiple passages retrieved by search engines. 
% % Compared with single-passage MRC,  there are probably multiple confusing answer candidates coming from different passages, which are extremely challenging for MRC models to distinguish. 
% Compared with the MRC on single passage, multi-passage MRC is more challenging since there are likely multiple confusing answer candidates coming from different passages.
% In this work, we propose an end-to-end framework that enables those answer candidates from different passages to verify each other based on their content representation. Our model  predicts the final answer from three aspects: the answer boundary, the answer content and the cross-passage answer verification. Experiments show that our method outperforms the baseline model by a large margin and achieves state-of-the-art performance on both the English MS-MARCO dataset and the Chinese DuReader dataset. 
\end{abstract}

\section{Introduction}

Humans use different forms of communications such as speech, hand gestures and emotions. Being able to understand one's emotions and the encoded feelings is an important factor for an appropriate and correct understanding.


With the ongoing research in the field of robotics, especially in the field of humanoid robots, it becomes interesting to integrate these capabilities into machines allowing for a more diverse and natural way of communication. One example is the Software called EmotiChat~\cite{Anderson06areal-time}. This is a chat application with emotion recognition. The user is monitored and whenever an emotion is detected (smile, etc.), an emoticon is inserted into the chat window. Besides Human Computer Interaction other fields like surveillance or driver safety could also profit from it. Being able to detect the mood of the driver could help to detect the level of attention, so that automatic systems can adapt.\\
\let\thefootnote\relax\footnote{*F. Trier and P. Burkert contributed equally to this work.}


Many methods rely on extraction of the facial region. This can be realized through manual inference~\cite{4032815} or an automatic detection approach~\cite{Anderson06areal-time}.
Methods often involve the Facial Action Coding System (FACS) which describes the facial expression using Action Units (AU). An Action Unit is a facial action like "raising the Inner Brow". Multiple activations of AUs describe the facial expression~\cite{kumar2009face}. Being able to correctly detect AUs is a helpful step, since it allows making a statement about the activation level of the corresponding emotion. \\
Handcrafted facial landmarks can be used such as done by Kotsia et al.~\cite{4032815}. Detecting such landmarks can be hard, as the distance between them differs depending on the person~\cite{6998925}. Not only AUs can be used to detect emotions, but also texture. When a face shows an emotion the structure changes and different filters can be applied to detect this~\cite{6998925}.\\


\begin{figure}
   \centering
        \includegraphics[width=\columnwidth]{Fig1}
   \caption{Example images from the MMI (top) and CKP (bottom). The emotions from left to right are: \textit{Anger}, \textit{Sadness}, \textit{Disgust}, \textit{Happiness}, \textit{Fear}, \textit{Surprise}. The emotion \textit{Contempt} of the CKP set is not displayed.}\label{fig:example_images}
\end{figure}




The presented approach uses Artificial Neural Networks (ANN). ANNs differ, as they are trained on the data with less need for manual interference. 
Convolutional Neural Networks are a special kind of ANN and have been shown to work well as feature extractor when using images as input~\cite{donahue2013decaf} and are real-time capable. This allows for the usage of the raw input images without any pre- or postprocessing.\\
GoogleNet~\cite{DBLP:journals/corr/SzegedyLJSRAEVR14} is a deep neural network architecture that relies on CNNs. It has been introduced during the Image Net Large Scale Visual Recognition Challenge(ILSVRC) 2014. This challenge analyses the quality of different image classification approaches submitted by different groups. The images are separated into 1000 different classes organized by the WordNet hierarchy. In the challenge "object detection with additional training data" GoogleNet has achieved about 44\% precision~\cite{LSVRC-results}. These results have demonstrated the potential which lies in this kind of architecture. Therefore it has been used as inspiration for the proposed architecture.\\
The proposed network has been evaluated on the Extended Cohn-Kanade Dataset (Section~\ref{sec:ckp}) and on the MMI Dataset (Section~\ref{sec:mmi}). Typical pictures of persons showing emotions can be seen in Fig.~\ref{fig:example_images}.
The emotion \textit{Contempt} of the CKP set is not shown as no subject with consent for publication and an annotated emotion is part of the dataset. Results of experiments on these datasets demonstrate the success of using a deep layered neural network structure. With a 10-fold cross-validation a recognition accuracy of 99.6\% has been achieved. \\

The paper is arranged as follows: After this introduction, Related Work (Section~\ref{sec:related}) is presented which focuses on Emotion/Expression recognition and the various approaches scientists have taken. Next is Section~\ref{sec:background}, Background, which focuses on the main components of the architecture proposed in this article. Section~\ref{sec:datasets} contains a summary of the used Datasets. In Section~\ref{sec:architecture} the architecture is presented. This is followed by the experiments and its results (Section~\ref{sec:experiments}) . Finally, Section~\ref{sec:conclusion} summarizes the article and concludes the article.

\subsection{Overview}
The basic architecture of our SyncSpecCNN is similar to the fully convolutional segmentation network as in \cite{long2015fully}, namely, we repeat the operation of convolving the vertex function by kernels and applying non-linear transformation. However, we have several key differences. First, we achieve convolution by modulation in the spectral domain.
%To be specific, we first transform a vertex function to its spectral representation, then modulate this representation by a set of learnable multipliers, and finally transform the spectral representation back to obtain the convolved vertex function. The multipliers essentially defines the set of convolution kernels. 
Second, we parametrize kernels in the spectral domain following a dilated fashion, so that kernel sizes could be effectively enlarged to capture large context information without increasing the number of parameters.
%Large kernels have been proved to be very helpful for image segmentation tasks due to its ability of effectively capturing large context information \cite{yu2015multi}. 
%In \cite{yu2015multi}, the author proposes to use dilated convolution for image segmentation, where kernel size could be enlarged without increasing the number of parameters.
%We achieve a similar goal in the spectral domain by parametrizing the multipliers as modulated exponential windows. 
%The spectral bandwidths of these kernels vary layer by layer, essentially aggregating information at multiple scales (\ref{sdkp}).
Last, we design a Spectral Transformer Network to synchronize the spectral domain of different shapes, allowing better parameter sharing.
%Thirdly, we align different spectral domains to allow valid parameter sharing across different shapes graphs.
%Spectral representation of vertex functions and convolution kernels constructed in the spectral domain both rely on the spectral bases choice. However, the underlying spectral bases of shape graphs are different, making parameter sharing a tricky problem. Thus we design a Spectral Transformer Network to predict function map in an end-to-end fashion, which synchronizes the spectral domain of different shapes and allows better parameter sharing.
\iffalse
\todo{
  \begin{itemize}
    \item the basic architecture of our neural network is similar to the fully convolutional segmentation network as in CITATION, namely, we repeat the operation of convolving the vertex function by kernels and applying non-linear transformation. however, we have several critical differences.
    \item we achieve the convolution by operations in the spectral domain. Namely, we first apply forward transform to the vertex function and get its spectral representation, then modulate this representation by a set of learned multipliers, and finally apply a backward transform to obtain the convolved vertex function. 
    \item the multipliers effectively defines a set of kernels. Large kernels have been proved to be very helpful for image segmentation tasks due to its ability of capturing large context information effectively [CITE DILATED CONV paper]. In [CITE DILATED CONV AGAIN], the author propose to use dilated convolution for image segmentation, where kernel size could be enlarged without increasing the number of parameters. We achieve a similar goal in the spectral domain by parametrising the kernels as modulated exponential windows. The spectral bandwidths of these kernels vary layer by layer, essentially aggregating information at multiple scales (section xxx).
    \item spectral representation of vertex functions and convolution kernels constructed in the spectral domain both rely on the spectral bases choice. However, the underlying spectral bases of shape graphs are different, making parameter sharing a tricky problem. Thus we design a Spectral Transformer Network to predict function map in an end-to-end fashion, which synchronizes the spectral domain of different shapes and allows better parameter sharing.
  \end{itemize}
%     \item We develop our SyncSpecCNN framework so that similar to conventional CNN our network could allow weight sharing across different shape graphs and information aggregation at multiscales.
%     \item We parameterize our convolution kernels in the spectral domain following a dilated fashion. A bunch of sin/cos modulated heat kernel functions are used for parameter
}
\fi

\subsection{Network Architecture}
Similar to conventional CNN, our SyncSpecCNN contains layers including  ReLU, DropOut, 1$\times$1 Convolution \cite{szegedy2015going}, and BatchNormalization, which all operate in the spatial domain on graph vertex functions. The difference comes from our graph convolution operation, which introduces the following modules: Forward Transform, Backward Transform, Spectral Multiplication, and Spectral Transformer Network, as is shown in Figure~\ref{fig:architecture} and summarized in Table~\ref{tab:architecture}. % Forward Transform converts vertex functions on the graph into their spectral representations. Backward Transform converts the spectral representations back to vertex functions. Spectral Multiplication modulates a spectral representation of a vertex function by pixelwise multiplication with multipliers from kernels, which is the counter part of spatial domain convolution. Spectral Transformer Network (SpecTN) takes shape graph $\graph$ as input and outputs a linear functional map which canonicalizes spectral domain of $\graph$.

We provide more details about the newly introduces modules as below.
\begin{figure*}
    \centering
    \includegraphics[width=0.9\linewidth]{./fig/architecture_v3.png}
    \caption{Architecture of our SyncSpecCNN. Spectral convolution is done through first transforming graph vertex functions into their spectral representation and then pointwise modulating it with a set of multipliers. The multiplied signal is transformed back to spatial domain to perform nonlinear operations. We introduce spectral transformer network to synchronize different spectral domains and allow better parameter sharing in spectral convolution. Convolution kernels are parametrized in a dilated fashion for effective multi-scale information aggregation.}
    \label{fig:architecture}
\end{figure*}

\begin{table}[]
\centering
{\footnotesize
\begin{tabular}{@{}p{0.25\linewidth}p{0.015\linewidth}p{0.015\linewidth}p{0.015\linewidth}p{0.015\linewidth}p{0.015\linewidth}p{0.015\linewidth}p{0.017\linewidth}p{0.016\linewidth}p{0.016\linewidth}p{0.015\linewidth}}
\toprule
Layer               & 1  & 2  & 3  & 4  & 5  & 6  & 7   & 8   & 9  & 10 \\ \midrule
Dilation ($\gamma$) & 1  & 1  & 4  & 4  & 16 & 16 & 64   & 64   & 1  & 1  \\
SpecTN              & No & No & No & No & No & No & Yes & Yes & No & No \\
\#Kernel Param      & 7  & 1  & 7  & 1  & 7  & 1  & 45  & 45  & 7  & 1  \\
\#Out Channel    & c  & c  & c  & c  & 2c & 2c & 2c  & 2c  & 2c & 2c\\ \bottomrule
\end{tabular}
}
\caption{Parameters used in different layers of the architecture, including dilation parameter $\gamma$ which controls convolution kernel size, whether use spectral transformer network (SpecTN), the number of learnable parameters in convolution kernels, the number of output channels after each convolution operation.}
\label{tab:architecture}
\end{table}

In a basic convolution block, a vertex function $f$ defined on $\graph$ is first transformed into its spectral representation $\myvec{\alpha}$ through Forward Transform $\myvec{\alpha} = \myvec{B}^Tf$. Then the functional map $C$ predicted by the Spectral Transformer Network will be applied to $\myvec{\alpha}$ and outputs $\myvec{\alpha}'=C\myvec{\alpha}$ for spectral domain synchronization (Sec~\ref{spectn}). A Spectral Multiplication layer is followed, pointwisely multiplying $\myvec \alpha'$ by a set of multipliers and getting $\tilde{\myvec \alpha}' = W\myvec \alpha'$, where $W$ is a diagonal matrix with its diagonal being the set of multipliers, and $\tilde{\myvec \alpha}'$ is used to denote the multiplication result. This is how we conduct convolution in the spectral domain, where spectral dilated kernels are used to capture multiscale information (Sec~\ref{sdkp}). Then we apply the inverse functional map $C_{inv}$ to $\tilde{\myvec \alpha}'$, so that we get the spectral representation $\tilde{\myvec \alpha} = C_{inv}\tilde{\myvec \alpha}'$ in the original spectral domain before canonicalization. $\tilde{\myvec \alpha}$ is then converted back to a graph vertex function through Backward Transform $\myvec \tilde{f} = \myvec{B}\tilde{\myvec \alpha}$. This building block was repeated for several times and forms the backbone of our deep architecture. We also add skip links into our SyncSpecCNN to better facilitate information flow across earlier and later layers. 
% We enforce the output $C$ of our spectral transformer network to be close to orthogonal so that its transpose $C^T$ is used as the approximated inverse functional map $C_{inv}$, as is shown in Figure~\ref{fig:architecture}. 

One interesting observation is worth mentioning: small convolution kernels correspond to smoothly transiting multipliers in the spectral domain, therefore not very sensitive to bases misalignment among shapes graphs in a certain range of spectrum and are more generalizable across graphs. As a result, we omit the spectral transformer network when the convolution kernels are small. %Moreover, while using the spectral transformer network for very large convolution kernels, instead of using our dilated parametrization, we choose to treat every single multiplier as a learnable parameter. To be specific, assuming $C\in\mathbb{R}^{k_1\times k_2}$ and transforms a subspace with dimension $k_1$ to a canonical domain with dimension $k_2$, then we will adopt a set of $k_2$ learnable parameters for spectral multiplication.

\iffalse
\todo{
\begin{itemize}
  \item Similar to conventional CNN, our SyncSpecCNN contains BatchNormalization, ReLU, DropOut, 1x1 Convolution [CITE GOOGLE INCEPTION NETWORK] layers, which are all operated in the spatial domain on graph vertex functions.
  \item Different from conventional CNN, our SyncSpecCNN conducts convolution operation on graphs through the following modules:
    \subitem Forward Transform: converts vertex functions on the graph into its spectral representation under the spectral bases. 
    \subitem Backward Transform: converts the spectral representation back to vertex functions.
    \subitem Spectral Multiplication: pointwise multiplies a spectral representation of a vertex function in the spectral domain. It's the counter part of the spatial domain convolution.
    \subitem Functional Map: linearly transforms the spectral representation of an input function
  \item In a basic convolution block, a vertex function on a shape graph is first transformed into its spectral representation through Forward Transform layer. Then a functional map layer is applied to the spectral representation for spectral domain synchronization (section xxx). A Spectral Multiplication layer is followed, pointwise multiplying the spectral representation by a set of multipliers. This is the part conducting convolution in the spectral domain, where spectral dilated kernels are used to capture multiscale information. Then we convert the spectral representation back to a graph vertex function through Backward Transform. This building block was repeated for several times and forms the backbone of our deep architecture.
\end{itemize}
}
\fi

\subsection{Spectral Dilated Kernel Parameterization}
\label{sdkp}
Yu et al.~\cite{yu2015multi} has proved the effectiveness of multi-scale kernels for aggregating context information at different scales in the task of image segmentation. They propose to use dilated kernels to increase the kernel size without increasing the number of parameters. We parametrize our convolution kernels in a similar flavor but in the spectral domain, which turns out to be straightforward and effective. Essentially, we find that multi-resolution analysis on graphs could be achieved without complicated hierarchical graph clustering.

Before explaining what the exact parametrization is, we first discuss the intuition behind our design. The Spectral Multiplication layer modulates the spectral representation $\myvec \alpha=\{\alpha_i\}$ by a set of multipliers from the kernel, where $\alpha_i$ is the spectral coordinate of vertex function at basis $\myvec{b}_i$. Note that $\lambda_i$ can be interpreted as the frequency of its corresponding eigenbasis $\myvec b_i$, and $\myvec b_i$ itself is a vertex function that captures the intrinsic geometry of the shape. We assume that $\lambda_i$'s are sorted ascendingly and arrange $\myvec{b}_i$'s accordingly.

The multiplers are the spectral representation of convolution kernel. Denote the set of multipliers as $\myvec m=\{m_i\}$, each corresponds to one $\lambda_i$. Regard $\myvec m$ as a function of $\lambda_i$. 

Again, generalized from conventional Fourier analysis, if $\myvec m$ is concentrated in the low-end of the spectrum, the corresponding spatial kernel function is smooth; conversely, if the corresponding spatial functions is localized, $\myvec m$ is smooth. Therefore, to obtain a smoother kernel function as in \cite{yu2015multi}, we constrain the bandwidth of $\myvec m$, enabling us to learn a smaller number of parameters; in addition, varying the smoothness of $\myvec m$ would control the kernel size. 

To be specific, we associate each Spectral Multiplication layer with a dilation parameter $\gamma$ and parameterize $m_i$ as a combination of some modulated exponential window functions, namely

\vspace{-0.25cm}
\begin{align*}
    m_i = \sum_{j=0}^n\omega_{2j+1}\text{e}^{-j\gamma \lambda_i}\text{cos}(j\gamma \lambda_i \pi)
    +\sum_{j=1}^n\omega_{2j}\text{e}^{-j\gamma \lambda_i}\text{sin}(j\gamma \lambda_i \pi)
    \vspace{-0.25cm}
\end{align*}

Here $\myvec \omega$ is a set of $2n+1$ learnable parameters, $n$ is a hyper-parameter controlling the number of learnable parameters. Large $\gamma$ corresponds to rapidly changing multipliers with small bandwidth, thus a smooth kernel with large spatial support. On the other hand, small $\gamma$ corresponds to slowly changing multipliers with large bandwidth, corresponding to kernels with small spatial support. Instead of using an exponential window only, we add $\text{sin}/\text{cos}$ modulation to increase the expressive power of the kernel. Figure~\ref{fig:kernelvis} shows a visualization of modulated exponential window function with different dilation parameter.

Our parametrization has three main advantages: First, it allows aggregating multi-scale information since the size of convolution kernels vary in different layers; Second, large kernels could be easily acquired with a compact set of parameters, which effectively increases the receptive field while mitigates overfitting; Third, reduced parameters allow more efficient computation.
\iffalse
\todo{
\begin{itemize}
  \item We use modulated exponential window to parametrize our convolution kernel in the spectral domain. 
  \item Observing that the spacial support of these kernels increase with a shrink of the exponential window width, we adapt exponential window with different width in different layers of the network. This allows the network to capture information at different spatial scales.
  \item We use cos/sin function to modulate the exponential window, which increases the expressive power of the kernel without increasing its spatial support much. 
\end{itemize}
}
\fi

\vspace{-0.1cm}
\begin{figure}
    \centering
    \includegraphics[width=0.8\linewidth]{./fig/kernelvis4.pdf}
    \caption{Visualization of modulated exponential window function with different dilation parameters in both spectral domain and spatial domain. The same spectral representation could induce spatially different kernel functions, especially when the kernel size is large.}
    \label{fig:kernelvis}
    \vspace{-0.3cm}
\end{figure}

\subsection{Spectral Transformer Network}
\label{spectn}
As is shown in Figure~\ref{fig:kernelvis}, the same spectral parametrization of kernels could lead to very different vertex functions when the underlying spectral domains are different. This problem is especially prominent when the kernel size is large. Therefore, being able to synchronize different spectral domains is the key to allow large kernels sharing parameters across different shape graphs. 

\subsubsection{Basic idea} According to \cite{ovsjanikov2012functional} and \cite{wang2013image}, one way to synchronize the spectral domains of a group of shapes is through a tool named functional map. In the functional map framework, one can find a linear map to pull the spectral domain of each individual shape to a canonical space, so that representations in the individual spectral domains become comparable under a canonical set of bases. Indeed, given each shape $S$, this linear map is as simple as a matrix $C$, which linearly transforms the spectral representation $\myvec \alpha$ on one shape to its counterpart $\myvec \alpha'$ in the canonical space. Note that, {\bf from the synchronization in the spectral domain, one induces a spatial correspondence on the graph, vice versa}. Viewing the spectral domain as the dual space and spatial domain on graph as the primal space, this {\bf primal-dual relationship} is the pivotal idea behind functional map.

Inspired by this idea, we design a Spectral Transformer Network (SpecTN) for the spectral domain synchronization task. Our SpecTN takes a shape $S$ as input and predicts a matrix $C$ for it (see Figure~\ref{fig:architecture}), so that
% \vspace{-0.25cm}
% \begin{align*}
$\myvec \alpha'=C \myvec \alpha$.
% \vspace{-0.25cm}
%\end{align*}
Thus, without SpecTN, $\myvec \alpha$ will be directly passed to subsequent modules of our network; with SpecTN, $\myvec \alpha'$ will be passed. In Figure~\ref{fig:jointbasis}, we show an example of how different spectral domains are synchronized after applying the linear map $C$ predicted from our SpecTN.
% The transformation is linear: $\alpha'=C\alpha$. When plugged into our SyncSpecCNN, SpecTN will learn how to actively transform the spectral domain so that the overall cost function could be minimized.

Our SpecTN draws inspiration from Spatial Transformer Network (STN)~\cite{jaderberg2015spatial}. From a high level, both SpecTN and STN are learned to align data to a canonical form.

\subsubsection{Input to SpecTN}
A proper representation for shape $\shape$ is needed as the input to our SpecTN. To allow SpecTN predicting a transform between different spectral domains, certain depiction about the underlying spectral domain is greatly helpful, i.e. graph laplacian eigenbases in our setting. In addition, since spectral synchronization couples with graph alignment, providing rough shape graph correspondences could facilitate good prediction. 

Based on these, we use voxel functions $\myvec{B}_v$ that is computed from laplacian eigenbases as the input to SpecTN:
% \vspace{-0.25cm}
% \begin{align*}
    $C=\mbox{SpecTN}(\myvec{B}_{v};\Theta).$
%     \vspace{-0.25cm}
% \end{align*} 
% As is explained soon, $\myvec{B}_v$ also implicitly provide correspondences. 
Specifically, $\myvec{B}_v$ is a \emph{volumetric reparameterization} of the graph laplacian eigenbases $\myvec{B}$, defined voxel-wise in 3D volumetric space. The volumetric reparameterization is conducted by converting graph vertex function $\myvec{B}$ into voxel function $\myvec{B}_v$ in a straightforward manner -- we simply assign a vertex function value to the voxel where the vertex lies. Since all $\myvec{B}_v$ live in the same 3D volumetric space, correspondences among them are associated accordingly.

% This representation has two merits: it depicts the underlying spectral domains of each $\shape$ through graph laplacian eigenbases; it provides rough shape correspondences for a set of spatially aligned shapes, which helps spectral alignment.

\subsubsection{Optimization of SpecTN}
Ideally, SpecTN should be learned automatically along with the minimization of the prediction loss, as the case in STN; however, in practice we find that such optimization is extremely challenging. This is because the parameters of $C$ in SpecTN is quadratic w.r.t the number of spectral bases, hundreds of times more than in the affine transformation matrix of STN. 

We address this challenge from three aspects: limit our scope to a reduced set of prominent spectral bases to curtail the parameters of $C$; add regularization to constrain the optimization space; smartly initialize SpecTN with a good starting point.  


\mypara{Reduced bases} Synchronizing the whole spectrum could be a daunting task given its high dimensionality. In particular, free parameters in $C$ grows quadratically as the dimension of spectral domain increases. To favor optimization, we adopt a natural strategy that only synchronizes the prominent part of the spectrum. In our case, the spectral parametrization of large kernels are mainly determined by the low-frequency end of the spectrum, indicating that the synchronization in this part of spectrum is sufficient. In practice, we synchronize the top $15$ bases sorted by the frequency. This idea has been verified to be effective by \cite{ovsjanikov2012functional}.

%Similar to  our SpecTN is a self-contained model that could be trained together with the end task. What's different is that STN transforms in the spatial domain whereas our SpecTN transforms in the spectral domain. This difference, however, renders a much more challenging optimization problem at training time, since the spectral alignment matrix $C$ has hundreds of of SpecTN to be rather tricky, as the 

%There are several factors differentiate SpecTN from STN. SpecTN takes a graph with vertex function as input and the input representation should be properly designed so that it is friendly to functional map prediction. Moreover, SpecTN applies a high dimensional linear transformation in the spectral space, with hundreds times more variables than STN, which makes the optimization especially tricky.

\mypara{Regularization}
Regularizations are used during training to force the output $C$ of SpecTN to be close to an orthogonal map, namely, in the overall loss function we add a term $\|CC^T-I\|_F^2$. With this regularization, $C^T$ can be used to approximate the inverse map. Such a maneuver is more friendly to differentiation and easier to train. 


 
\begin{figure}[t!]
    \centering
    \includegraphics[width=0.4\textwidth]{./fig/visjointbasis2.pdf}
    \caption{Visualization of low frequency eigenbasis functions before and after spectral synchronization. Before synchronization, eigenbasis functions on different shapes are not aligned. After applying the transform predicted from SpecTN, different spectral domain could be synchronized and the eigenbasis functions align.}
    \label{fig:jointbasis}
\end{figure}
%Eigenbasis functions are sorted according to their frequency and same ordered functions are shown in a column.

\mypara{Initialization by precomputed functional map} 
Given the huge optimization space and the non-convex objective, a good starting point helps to avoid optimization from getting stuck in bad local minima. As stated above, our linear transformation $C$ can be interpreted as a functional map; therefore, it is natural for us to initialize $C$ accordingly and then refine it to better serve the end-task. To this end, we first precompute a set of function maps $C_{pre}$ for each shape by an external routine, which roughly align each individual spectral domain of $\shape$ to a canonical domain. Then we pretrain the SpecTN separately in a supervised manner:
\begin{equation*}
\begin{aligned}
\underset{\Theta}{\mbox{minimize}} & &\sum_i \|\mbox{SpecTN}(\myvec{B}_{v, i};\Theta) - C_{pre, i}\|^2
\end{aligned}
\end{equation*}
where $i$ indexes shapes.
This pretrained SpecTN is plugged into the  SyncSpecCNN pipeline and fine-tuned while optimizing a specific task such as shape segmentation.  Validated by our experiment, the pretraining step is crucial. 

Next we introduce how the external routine precomputes a functional map for some shape $\shape$. This functional map aligns the spectral domain of $S$ to a canonical one of an ``average'' shape $\bar{\shape}$. So we start from the construction of the ``average'' shape and then proceed to the computation of the functional map. 

The geometry of $\bar{\shape}$ is not generated explicitly. Instead, $\bar{\shape}$ is represented by its volumetric adjacency matrix $\bar{W}_v$, which depicts the connectivity of voxels in the volumetric space that all shapes are voxelized. $\bar{W}_v$ is obtained by averaging the volumetric adjacency matrices $W_v$ of all shapes. The $W_v$ for each shape $S$ is the adjacency matrix of the corresponding volumetric graph, whose vertices are all the voxels and edges indicate the adjacency of occupied voxels in the volumetric space.

The functional map $C$ from $\shape$ to $\bar{\shape}$ could be induced from the spatial correspondences between $\shape$ and $\bar{\shape}$, by the primal-dual relationship~\cite{ovsjanikov2012functional}. Since we already have the bases of $\shape$ and $\bar{\shape}$, as well as the rough spatial correspondences between them from the volumetric occupancy, this map can then be discovered by the approach proposed in \cite{ovsjanikov2012functional}. To be specific, we use $\myvec{B}_v$ to denote the volumetric reparametrization of graph laplacian eigenbases $\myvec{B}$ for each shape $\shape$, and use $\bar{\myvec{B}}_v$ to denote the grahp laplacian eigenbases of $\bar{\shape}$. $\myvec{B}_v$ and $\bar{\myvec{B}}_v$ both lie in the volumetric space and their spatial correspondence is natural to acquire. The functional map $C_{pre}$ aligning $\myvec{B_v}$ with $\bar{\myvec{B}}_v$ could be computed through simple matrix multiplication $C_{pre}=\bar{\myvec{B}}_v^T\myvec{B}_v$. The computed functional map will serve as supervision and SpecTN is pretrained to minimize the loss function $||C-C_{pre}||_F^2$.

It is worth mentioning that, if the shapes under consideration are diverse in topology and geometry, i.e. shapes from different categories, aligning every shape to a single ``average'' shape might cause unwanted distortion. Therefore we leverage multiple ``average'' shapes $\{\bar{\shape}_i\}_{i=1}^{n}$ and use a combination of their spectral domains as the canonical domain. Specifically, we assign each shape $\shape$ to its closest ``average'' shape under some global similarity measurement (i.e. lightfield descriptor) and use $\{a_i\}_{i=1}^n$ to represent such assignment, namely $a_i=1$ if $\shape$ is assigned to $\bar{\shape}_i$ and $a_i=0$ otherwise. Also we use $\bar{\myvec{B}}_{vi}$ to denote the spectral bases of $\bar{\shape}_i$. Then the functional map $C_{pre}$ for each shape $\shape$ could be computed through $C_{pre}=[a_1\bar{\myvec{B}}_{v1} \;a_2\bar{\myvec{B}}_{v2}\; ...\; a_n\bar{\myvec{B}}_{vn}]^T\myvec{B}_v$. The SpecTN is pretrained to predict a functional map which only synchronizes spectral domain of each shape to its most similar ``average'' shape.

%For more details of how this map is computed and other implementation details of our framework, please refer to our supplementary. % to the voxel function space, spanned by $\bar{\myvec{B}}_v$, where $\myvec{B}$ and $\bar{\myvec{B}}_v$ denotes the laplacian eigenbases of each individual shape graph and the average shape graph respectively. This problem couples with graph alignment and requires rough spatial correspondences to solve. To acquire this, we reparametrize each $\myvec{B}$ as $\myvec{B}_v$ defined in 3D volumetric space, so that the correspondences between each shape graph and the average shape graph can be obtained by a simple nearest neighbor lookup. The functional map $C_v$ aligning $\myvec{B_v}$ with $\bar{\myvec{B}}_v$ could be computed through simple matrix multiplication $C_v=\bar{\myvec{B}}_v^T\bar{\myvec{B}}_v$. The computed functional map will serve as supervision and SpecTN is pretrained to minimize the loss function $||C-C_v||_F^2$.
 
%Another thing worth mentioning, when the shapes under consideration are very diverse, i.e. shapes from different categories, aligning every shape to a single "average" shape might cause unwanted distortion. Therefore we leverage multiple "average" shapes and use a combination of their spectral domains as the canonical domain. The SpecTN is pretrained to predict a functional map which only synchronizes spectral domain of each shape based on its most similar "average" shape.

%For implementation details of our pipeline, we refer the reader to supplementary for details.

\iffalse
\todo{
\begin{itemize}
  \item laplacian bases from different shape graphs vary but are still roughly aligned regarding to the low/high frequency in the spectrum.
  \item small convolution kernels in the spatial domain corresponds to smooth and flatly changing spectral multipliers, which are not sensitive to bases change within a relatively wide spectrum range. Thus generalizability is not a prominent problem for small convolution kernels.
  \item large convolution kernels, on the other hand, corresponds to rapidly changing spectral multipliers, therefore very sensitive to bases change. Spectral domain synchronization becomes a key issue for across graph deep learning.
  \item in our setting, large convolution kernels are constructed by low frequency bases. Therefore we don't need to synchronize the whole spectrum of different shape graphs but only the low frequency end.
  \item we use a functional map to transform spectral representation under different shape graphs into a canonical domain, allowing parameter sharing.
  \item to learn the functional map, which is shape dependent, we design a Spectral Transformer Network, which takes the volumetric representation of each shape as input and predicts a functional map for it. The volumetric representation is not just a binary occupancy grid. Voxel cells are also equipped with spectral bases function, which can be obtained by max pooling the vertex function values falling in each voxel cell.
\end{itemize}
}
\fi

\subsection{Implementation Details}
\label{sec:impl}
In most of our experiments, input shapes are represented as point cloud with around $2000-3000$ points. Given an input shape point cloud, we build a k-nearest neighbor graph $\graph$ first. We use $k=6$ in all our experiments. Then a graph weight matrix $W$ could be constructed in which $W_{i,j}=\frac{1}{d_{i,j}^2}$ if point $i$ and $j$ are connected, $0$ otherwise. We then compute the symmetric normalized graph laplacian $L$ as $L=I-D^{-1/2}WD^{-1/2}$, where $D$ is the degree matrix and $I$ denotes identity matrix. Since many natural functions we care about could be depicted by a small number of low-frequency laplacian eigenbases, we compute and use the smallest $100$ eigenvalues as well as the corresponding eigenbases for each $L$ in all our experiments. 

The choice of dilation parameters $\gamma$, number of output channels after each convolution layer, number of learnable parameters in each convolution kernel are shown in Table~\ref{tab:architecture}. We choose $c=50$ in all of our experiments. As is mentioned, we only consider the problem of synchronizing the low-frequency end of different spectral domains, so we choose to predict a functional map $C\in\mathbb{R}^{15\times45}$ in our experiments, which maps the first $15$ eigenbases of each individual spectral domain into a canonical domain of dimension $45$. Notice the dimension of canonical domain is larger that each individual domain to allow very different shapes to be mapped into different subspaces.

%\subsection{Implementation details}
%In most of our experiments, input shapes are represented as point cloud with around $2000-3000$ points. We build k-nearest neighbor graph $\graph$ for each shape, construct its normalized graph laplacian $L$, compute and use the smallest $100$ eigenvalues as well as the corresponding eigenbases of $L$ in all our experiments. The choice of dilation parameters $\gamma$, number of output channels after each convolution layer, number of learnable parameters in each convolution kernel are shown in Table~\ref{tab:architecture}. We choose $c=50$ in most of our experiments. We refer the reader to supplementary for more details.

%In most of our experiments, input shapes are represented as point cloud with around $2000-3000$ points. Given an input shape point cloud, we build a k-nearest neighbor graph $\graph$ first. We use $k=6$ in all our experiments. Then an graph weight matrix $W$ could be constructed where $W_{i,j}=\frac{1}{d_{i,j}^2}$ if point $i$ and $j$ are connected, $0$ otherwise. We then compute the symmetric normalized graph laplacian $L$ as $L=I-D^{-1/2}WD^{-1/2}$ where $D$ is the degree matrix and $I$ denotes identity matrix. Since most natural functions we care about could be depicted by a small number of low-frequency laplacian eigenbases, we compute and use the smallest $100$ eigenvalues as well as the corresponding eigenbases for each $L$ in all our experiments. The choice of dilation parameters $\gamma$, number of output channels after each convolution layer, number of learnable parameters in each convolution kernel are shown in Table~\ref{tab:architecture}. We choose $c=50$ in most of our experiments. As is mentioned, we only consider the problem of synchronizing low-frequency end of different spectral domains, so we choose to predict a functional map $C\in\mathbb{R}^{15\times45}$ in all our experiments, which maps the first $15$ eigenbases of each individual spectral domain into a canonical domain of dimension $45$. Notice the dimension of canonical domain is larger that each individual domain to allow very different shapes mapped into different subspaces.

\iffalse
\todo{
  \begin{itemize}
    \item how to build the shape graph; compute the normalized laplacian
  \end{itemize}
}
\fi

% !TEX root = ../multi_task.tex

We evaluate the presented MTL method on a number of problems. First, we use MultiMNIST \citep{multi_mnist}, an MTL adaptation of MNIST \citep{mnist}. Next, we tackle multi-label classification on the CelebA dataset \citep{celeba} by considering each label as a distinct binary classification task. These problems include both classification and regression, with the number of tasks ranging from 2 to 40. Finally, we experiment with scene understanding, jointly tackling the tasks of semantic segmentation, instance segmentation, and depth estimation on the Cityscapes dataset \citep{cityscapes}. We discuss each experiment separately in the following subsections.

The baselines we consider are (i) \textbf{uniform scaling:} minimizing a uniformly weighted sum of loss functions \mbox{$\frac{1}{T}\sum_t \lL^t$}, \mbox{(ii) \textbf{single task:}} solving tasks independently, \mbox{(iii) \textbf{grid search:}} exhaustively trying various values from $\{ c^t \in [0,1] | \sum_t c^t = 1\}$ and optimizing for $\frac{1}{T}\sum_t c^t \lL^t$, \mbox{(iv) \textbf{\citet{Kendall2018}:}} using the uncertainty weighting proposed by \citet{Kendall2018}, and \mbox{(v) \textbf{GradNorm:}} using the normalization proposed by \citet{Chen2018}.



\subsection{MultiMNIST}
\label{sec:multi_mnist_exp}

Our initial experiments are on MultiMNIST, an MTL version of the MNIST dataset \citep{multi_mnist}. In order to convert digit classification into a multi-task problem, \citet{multi_mnist} overlaid multiple images together. We use a similar construction. For each image, a different one is chosen uniformly in random. Then one of these images is put at the top-left and the other one is at the bottom-right. The resulting tasks are: classifying the digit on the top-left (task-L) and classifying the digit on the bottom-right (task-R). We use 60K examples and directly apply existing single-task MNIST models. The MultiMNIST dataset is illustrated in the supplement.

We use the LeNet architecture \citep{mnist}. We treat all layers except the last as the representation function $g$ and put two fully-connected layers as task-specific functions (see the supplement for details). We visualize the performance profile as a scatter plot of accuracies on task-L and task-R in Figure~\ref{fig:multi_mnist_performance_curve}, and list the results in Table~\ref{tab:multi_mnist}.

In this setup, any static scaling results in lower accuracy than solving each task separately (the single-task baseline). The two tasks appear to compete for model capacity, since increase in the accuracy of one task results in decrease in the accuracy of the other. Uncertainty weighting \citep{Kendall2018} and GradNorm \citep{Chen2018} find solutions that are slightly better than grid search but distinctly worse than the single-task baseline. In contrast, our method finds a solution that efficiently utilizes the model capacity and yields accuracies that are as good as the single-task solutions. This experiment demonstrates the effectiveness of our method as well as the necessity of treating MTL as multi-objective optimization. Even after a large hyper-parameter search, \emph{any} scaling of tasks does not approach the effectiveness of our method.



\subsection{Multi-Label Classification}

\begin{figure}[t]
\includegraphics[width=\textwidth]{radar_full_new}
\vspace{1mm}
\caption{Radar charts of percentage error per attribute on CelebA \citep{celeba}. Lower is better. We divide attributes into two sets for legibility: easy on the left, hard on the right. Zoom in for details.}
\label{fig:multi_label_radar}
\end{figure}


\begin{wraptable}{r}{0.3\textwidth}
%\vspace{-4mm}
\captionof{table}{Mean of error per category of MTL algorithms in multi-label classification on CelebA \citep{celeba}.}
\begin{tabular}{r@{\hspace{2mm}}c@{}}
\toprule
& Average  \\
&  error \\
\midrule
Single task & $8.77$ \\
Uniform scaling & $9.62$ \\
\citealt{Kendall2018} & $9.53$ \\
GradNorm & $8.44$ \\
Ours & $\mathbf{8.25}$  \\
\bottomrule
\end{tabular}
\label{table:multi_label_bar}
%\vspace{-5mm}
\end{wraptable}

Next, we tackle multi-label classification. Given a set of attributes, multi-label classification calls for deciding whether each attribute holds for the input. We use the CelebA dataset \citep{celeba}, which includes 200K face images annotated with 40 attributes. Each attribute gives rise to a binary classification task and we cast this as a 40-way MTL problem. We use ResNet-18 \citep{resnet} without the final layer as a shared representation function, and attach a linear layer for each attribute (see the supplement for further details).


We plot the resulting error for each binary classification task as a radar chart in Figure~\ref{fig:multi_label_radar}. The average over them is listed in Table~\ref{table:multi_label_bar}. We skip grid search since it is not feasible over 40 tasks. Although uniform scaling is the norm in the multi-label classification literature, single-task performance is significantly better. Our method outperforms baselines for significant majority of tasks and achieves comparable performance in rest. This experiment also shows that our method remains effective when the number of tasks is high.


\subsection{Scene Understanding}

To evaluate our method in a more realistic setting, we use scene understanding. Given an RGB image, we solve three tasks: semantic segmentation (assigning pixel-level class labels), instance segmentation (assigning pixel-level instance labels), and monocular depth estimation (estimating continuous disparity per pixel). We follow the experimental procedure of \citet{Kendall2018} and use an encoder-decoder architecture. The encoder is based on ResNet-50 \citep{resnet} and is shared by all three tasks. The decoders are task-specific and are based on the pyramid pooling module \citep{pspnet} (see the supplement for further implementation details).

Since the output space of instance segmentation is unconstrained (the number of instances is not known in advance), we use a proxy problem as in \citet{Kendall2018}. For each pixel, we estimate the location of the center of mass of the instance that encompasses the pixel. These center votes can then be clustered to extract the instances. In our experiments, we directly report the MSE in the proxy task. Figure~\ref{fig:cityscapes_performance_profile} shows the performance profile for each pair of tasks, although we perform all experiments on all three tasks jointly. The pairwise performance profiles shown in Figure~\ref{fig:cityscapes_performance_profile} are simply 2D projections of the three-dimensional profile, presented this way for legibility. The results are also listed in Table~\ref{tab:cityscapes_results}.

MTL outperforms single-task accuracy, indicating that the tasks cooperate and help each other. Our method outperforms all baselines on all tasks.


\subsection{Role of the Approximation}

In order to understand the role of the approximation proposed in Section~\ref{sec:approximation}, we compare the final performance and training time of our algorithm with and without the presented approximation in Table~\ref{tab:approximation_tradeoff} (runtime measured on a single Titan Xp GPU). For a small number of tasks (3 for scene understanding), training time is reduced by 40\%. For the multi-label classification experiment (40 tasks), the presented approximation accelerates learning by a factor of 25.

On the accuracy side, we expect both methods to perform similarly as long as the full-rank assumption is satisfied. As expected, the accuracy of both methods is very similar. Somewhat surprisingly, our approximation results in slightly improved accuracy in all experiments. While counter-intuitive at first, we hypothesize that this is related to the use of SGD in the learning algorithm. Stability analysis in convex optimization suggests that if gradients are computed with an error $\hat{\nabla}_\btheta \mathcal{L}^t = \nabla_\btheta \mathcal{L}^t + \mathbf{e}^t$ ($\btheta$ corresponds to $\btheta^{sh}$ in (\ref{eq:kkt_opt})), as opposed to $\mathbf{Z}$ in the approximate problem in \ref{eq:approx}, the error in the solution is bounded as $\|\hat{\mathbf{\alpha}} - \mathbf{\alpha} \|_2 \leq \mathcal{O}(\max_t \|\mathbf{e}^t\|_2)$. Considering the fact that the gradients are computed over the full parameter set (millions of dimensions) for the original problem and over a smaller space for the approximation (batch size times representation which is in the thousands), the dimension of the error vector is significantly higher in the original problem. We expect the $l_2$ norm of such a random vector to depend on the dimension.

In summary, our quantitative analysis of the approximation suggests that (i) the approximation does not cause an accuracy drop and (ii) by solving an equivalent problem in a lower-dimensional space, our method achieves both better computational efficiency and higher stability.

  {\small
  \begin{table}[t]
%  \vspace{-4mm}
  \caption{Effect of the MGDA-UB approximation. We report the final accuracies as well as training times for our method with and without the approximation.}
  %\vspace{1mm}
  \centering
  \begin{tabular}{@{}r@{\hspace{3mm}}c@{\hspace{3mm}}c@{\hspace{2mm}}c@{\hspace{2mm}}c@{}c@{\hspace{5mm}}c@{\hspace{2mm}}c@{}}
  \toprule
  & \multicolumn{4}{c}{Scene understanding (3 tasks)} &  & \multicolumn{2}{c}{Multi-label (40 tasks)}  \\
  \cmidrule(r){2-5} \cmidrule(lr){7-8}
                  & Training & Segmentation & Instance  & Disparity      & & Training & Average \\
                 & time     &  mIoU [\%]       & error [px] & error [px] & & time (hour)      & error \\
  \midrule
  Ours (w/o approx.) & $38.6$ & $66.13$ & $10.28$ & $2.59$ & & $429.9$ & $8.33$ \\
  Ours & $\mathbf{23.3}$ & $\mathbf{66.63}$ & $\mathbf{10.25}$ & $\mathbf{2.54}$  & & $\mathbf{16.1}$ & $\mathbf{8.25}$ \\
  \bottomrule
  \end{tabular}
  %\vspace{-2mm}
  \label{tab:approximation_tradeoff}
  \end{table}}


\section{VQA Dataset Analysis}
\label{sec:analysis}
%\vspace{\sectionReduceBot}
%%%%%%%%%%%%%%%%%%%%%%%%%%%%%%%%%%%%%%%%%%%%%%%%%%%%%%%%%%%
%%%%%%%%%%%%%%%%%%%%%%%%%%%%%%%%%%%%%%%%%%%%%%%%%%%%%%%%%%%
\begin{figure*}[t]
\centering
\includegraphics[width=1\linewidth]{figures/QuestionTypes3.pdf}
\caption{Distribution of questions by their first four words for a random sample of 60K questions for real images (left) and all questions for abstract scenes (right). The ordering of the words starts towards the center and radiates outwards. The arc length is proportional to the number of questions containing the word. White areas are words with contributions too small to show. }
%\vspace{-5pt}
\label{fig:QuesCluster}
%\setlength{\belowcaptionskip}{-10pt}
\end{figure*}
%%%%%%%%%%%%%%%%%%%%%%%%%%%%%%%%%%%%%%%%%%%%%%%%%%%%%%%%%%%

In this section, we provide an analysis of the questions and answers in the VQA train dataset.
To gain an understanding of the types of questions asked and answers provided, we visualize
the distribution of question types and answers. We also explore how often the questions may
be answered without the image using just commonsense information. Finally, we analyze whether
the information contained in an image caption is sufficient to answer the questions.

The dataset includes 614,163 questions 
%and a total of 
and 7,984,119 answers (including answers provided by workers with and without 
looking at the image) 
%and without looking at the image) 
for 204,721 images from the MS COCO dataset~\cite{coco} and 150,000 questions with 1,950,000 answers for $50,000$ abstract scenes.

%\textcolor{red}{
%We emphasize that the creation of a dataset of this scale and richness
%is a time consuming process, taking months to complete.
%While the entirety of the dataset has been collected,} at the time of original submission,
%120,520 questions with 270,210 answers for 50,000 MS COCO
%images and 30,000 questions with 79,740 answers for 10,000 abstract scenes had been collected.
%Please refer to the appendix for further details.
%\textcolor{red}{The results in this section still reflect that subset of the final dataset.}
%We emphasize that the creation of a dataset of this scale and richness
%is a time consuming process, taking months to complete.
%By our current estimates,
%approximately 5,000 questions and 40,000 answers are collected per day
%using Amazon Mechanical Turk (AMT).
%The entire dataset will take approximately three months to complete. At the time of submission,
%120,520 questions with 270,210 answers for 50,000 MS COCO
%images and 30,000 questions with 79,740 answers for 10,000 abstract scenes had been collected.
%Please refer to the appendix for further details.


%%%%%%%%%%%%%%%%%%%%%%%%%%%%%%%%%%%%%%%%%%%%%%%%%%%%%%%%%%%
%\vspace{\subsectionReduceTop}
\subsection{Questions}
%\vspace{\subsectionReduceBot}
%%%%%%%%%%%%%%%%%%%%%%%%%%%%%%%%%%%%%%%%%%%%%%%%%%%%%%%%%%%

\textbf{Types of Question.}
Given the structure of questions generated in the English language,
we can cluster questions into different types based on the words that start the question.
\figref{fig:QuesCluster} shows the distribution of questions based on the first four
words of the questions for both the real images (left) and abstract scenes (right).
Interestingly, the distribution of questions is quite similar for both real images and abstract scenes.
This helps demonstrate that the type of questions elicited by the abstract scenes is similar to
those elicited by the real images. There exists a surprising variety of question types,
including ``What is$\ldots$'', ``Is there$\ldots$'', ``How many$\ldots$'', and ``Does the$\ldots$''.
Quantitatively, the percentage of questions for different types is shown in \tableref{tab:typeacc}. Several example questions and answers are shown in \figref{fig:qualResults}.
%\textbf{Sub-Types.}
A particularly interesting type of question is the ``What is$\ldots$'' questions, since they have a
diverse set of possible answers. See the appendix for visualizations for ``What is$\ldots$'' questions.

\textbf{Lengths.}
\figref{fig:QuesLen} shows the distribution of question lengths.
We see that most questions range from four to ten words.


\begin{comment}\begin{table}[h]
{\small
\begin{tabular}{@{\extracolsep{\fill}}p{2cm}|ccccc@{\extracolsep{\fill}}}
%\toprule
Dataset  & Yes & No\\
%\midrule
Real   & 18.21 & 14.06 \\
Abstract & 26.54 & 16.70 \\
\end{tabular}
}
\vspace{5pt}
\caption{Percentage of ``yes'' and ``no'' questions in the real and abstract datasets.}
\label{table:yesno}
%\vspace{\captionReduceBot}
\end{table}
\end{comment}

%%%%%%%%%%%%%%%%%%%%%%%%%%%%%%%%%%%%%%%%%%%%%%%%%%%%%%%%%%%
\begin{figure}[t]
\centering
\includegraphics[width=1\linewidth]{figures/Lengths.pdf}
%\vspace{-9pt}
\caption{Percentage of questions with different word lengths for real images and abstract scenes.}
%\vspace{-5pt}
\label{fig:QuesLen}
%\setlength{\belowcaptionskip}{-10pt}
\end{figure}
%%%%%%%%%%%%%%%%%%%%%%%%%%%%%%%%%%%%%%%%%%%%%%%%%%%%%%%%%%%




\begin{figure*}
\centering
\includegraphics[width=1\linewidth]{figures/answers.pdf}
%\vspace{-5pt}
\caption{Distribution of answers per question type for a random sample of 60K questions for real images when subjects provide answers when given the image (top) and when not given the image (bottom).}
%\vspace{-5pt}
\label{fig:AnsPerQues}
%\setlength{\belowcaptionskip}{-10pt}
\end{figure*}


%%%%%%%%%%%%%%%%%%%%%%%%%%%%%%%%%%%%%%%%%%%%%%%%%%%%%%%%%%%
%\vspace{\subsectionReduceTop}
\subsection{Answers}
%\vspace{\subsectionReduceBot}
%%%%%%%%%%%%%%%%%%%%%%%%%%%%%%%%%%%%%%%%%%%%%%%%%%%%%%%%%%%

%\textbf{Typical Answers for Different Question Types.}
\textbf{Typical Answers.}
%Next, we analyze the answers provided for different question types.
\figref{fig:AnsPerQues} (top) shows the distribution of answers for several question types.
We can see that a number of question types, such as ``Is the\ldots'', ``Are\ldots'', and ``Does\ldots'' are
typically answered using ``yes'' and ``no'' as answers.
%\textcolor{red}{Question types such as ``How many\ldots'' are answered using numbers. $12.31\%$ and $14.48\%$ of the questions are answered using numbers on real images and abstract scenes, respectively.}
Other questions such as ``What is\ldots'' and ``What type\ldots'' have a rich diversity
of responses. Other question types such as ``What color\ldots'' or ``Which\ldots'' have more specialized responses,
such as colors, or ``left'' and ``right''. 
See the appendix for a list of the most popular answers.

\textbf{Lengths.}
Most answers consist of a single word, with the distribution of answers containing one, two, or three words, respectively being $89.32\%$, $6.91\%$, and $2.74\%$ for real images and $90.51\%$, $5.89\%$, and $2.49\%$ for abstract scenes.
%$89.16\%$, $7.00\%$, and $2.77\%$ of answers containing one, two, or three words, respectively.
The brevity of answers is not surprising, since the questions tend to elicit specific
information from the images. This is in contrast with image captions that generically
describe the entire image and hence tend to be longer. The brevity of our answers makes
automatic evaluation feasible. While it may be tempting to believe the brevity of the answers
makes the problem easier, recall that they are human-provided open-ended answers to
open-ended questions. The questions typically require complex reasoning to arrive at these
deceptively simple answers (see \figref{fig:qualResults}).
There are currently 23,234 unique one-word answers in our dataset for real images and 3,770 for abstract scenes.
%There are currently 10,011 unique one-word answers in our dataset.

\textbf{`Yes/No' and `Number' Answers.}
Many questions are answered using either ``yes'' or ``no'' (or sometimes ``maybe'') -- 
$38.37\%$ and $40.66\%$ of the questions on real images and abstract scenes respectively. 
Among these `yes/no' questions, there is a bias towards %answering with 
``yes'' -- %with ``yes'' being preferred %$61.32\%$ and $58.46\%$ 
$58.83\%$ and $55.86\%$ of `yes/no' answers are ``yes'' for real images and abstract scenes. 
Question types such as ``How many\ldots'' are answered using numbers -- 
$12.31\%$ and $14.48\%$ of the questions on real images and abstract scenes are `number' questions. 
``2'' is the most popular answer among the `number' questions, making up 
$26.04\%$ of the `number' answers for real images and $39.85\%$ for abstract scenes. 

\textbf{Subject Confidence.}
When the subjects answered the questions, we asked
``Do you think you were able to answer the question correctly?''.
\figref{fig:ConfScores} shows the distribution of responses. A majority of the answers
were labeled as confident for both real images and abstract scenes. % respectively.

\textbf{Inter-human Agreement.}
Does the self-judgment of confidence correspond to the answer agreement between subjects?
\figref{fig:ConfScores} shows the percentage of questions in which 
(i) $7$ or more, 
(ii) $3-7$, or 
(iii) less than $3$ subjects agree on the answers given their average confidence score 
(0 = not confident, 1 = confident).
As expected, the agreement between subjects increases with confidence.
However, even if all of the subjects are confident the answers may still vary.
This is not surprising since some answers may vary, yet have very similar meaning, such as ``happy'' and ``joyful''.

\begin{figure}[t]
\centering
\includegraphics[width=1\linewidth]{figures/Confidence.pdf}
%\vspace{-5pt}
\caption{Number of questions per average confidence score (0 = not confident, 1 = confident) for real images and abstract scenes (black lines). Percentage of questions where 7 or more answers are same, 3-7 are same, less than 3 are same (color bars). }
%\vspace{-7pt}
\label{fig:ConfScores}
%\setlength{\belowcaptionskip}{-10pt}
\end{figure}

As shown in \tableref{table:commonsense_acc} (Question + Image), there is significant inter-human
agreement in the answers for both real images ($83.30\%$) and abstract scenes ($87.49\%$). 
%when humans are provided both the question and image while answering the question.
Note that on average each question has $2.70$ unique answers for real images and $2.39$ for abstract scenes. 
The agreement is significantly higher ($>95\%$) for \quotes{yes/no} questions and lower for other questions ($<76\%$), possibly due to the fact that we perform exact string matching and do not account for synonyms, plurality, \etc. Note that the automatic determination of synonyms is a difficult problem, since the level of answer granularity can vary across questions.




%%%%%%%%%%%%%%%%%%%%%%%%%%%%%%%%%%%%%%%%%%%%%%%%%%%%%%%%%%%
%\vspace{\subsectionReduceTop}
\subsection{Commonsense Knowledge}
\label{sec:cs}
%\vspace{\subsectionReduceBot}
%%%%%%%%%%%%%%%%%%%%%%%%%%%%%%%%%%%%%%%%%%%%%%%%%%%%%%%%%%%
\begin{figure*}[t]
 \includegraphics[width=\linewidth]{figures/age.pdf}
 \centering
\caption{\small Example questions judged by Mturk workers to be answerable by different age groups. The percentage of questions falling into each age group is shown in parentheses.}
 \label{fig:age}
 \end{figure*}
 	
\textbf{Is the Image Necessary?}
%Can the questions be answered using commonsense knowledge alone without the need for an image,
%\eg, ``What is the color of the sheep?''?
Clearly, some questions can sometimes be
answered correctly using commonsense knowledge alone without the need for an image,
\eg, ``What is the color of the fire hydrant?''.
We explore this issue by asking three subjects to answer
the questions \emph{without seeing the image} (see the examples in blue in \figref{fig:qualResults}).
In \tableref{table:commonsense_acc} (Question), we show the percentage of questions for which
the correct answer is provided over all questions, ``yes/no'' questions, and the other questions that
are not ``yes/no''. For ``yes/no'' questions, the human subjects respond better than chance.
For other questions, humans are only correct about $21\%$ of the time. This demonstrates that
understanding the visual information is critical to VQA and that commonsense information alone is not sufficient.

To show the qualitative difference in answers provided with and without images,
we show the distribution of answers for various question types in \figref{fig:AnsPerQues} (bottom).
The distribution of colors, numbers, and even ``yes/no'' responses is surprisingly different for answers
with and without images.
 
\textbf{Which Questions Require Common Sense?}
In order to identify questions that require commonsense reasoning to answer, we conducted 
two AMT studies (on a subset 10K questions from the real images of VQA trainval) asking subjects --
\begin{compactenum} 
\item Whether or not they believed a question required commonsense to answer the question, and 
\item The youngest age group that they believe a person must be in order to be able to correctly answer the question -- 
toddler (3-4), 
younger child (5-8), 
older child (9-12), 
teenager (13-17), 
adult (18+).
\end{compactenum}
Each question was shown to 10 subjects. We found that 
for $47.43\%$ of questions 3 or more subjects voted `yes' to commonsense, 
($18.14\%$: 6 or more).  
In the `perceived human age required to answer question' study, we found the following distribution of responses: 
toddler: $15.3\%$,
younger child: $39.7\%$, 
older child: $28.4\%$, 
teenager: $11.2\%$, 
adult: $5.5\%$.
In Figure \ref{fig:age} we show several questions for which a majority of subjects picked the specified age range. Surprisingly the perceived age needed to answer the questions is fairly well distributed across the different age ranges. As expected the questions that were judged answerable by an adult (18+) generally need specialized knowledge, whereas those answerable by a toddler (3-4) are more generic.
 
We measure the degree of commonsense required to answer a question as the percentage of subjects (out of 10) who voted ``yes'' in our ``whether or not a question requires commonsense'' study.
A fine-grained breakdown of average age and average degree of common sense (on a scale of $0-100$) required to answer a question is shown in \tableref{tab:typeacc}. The average age and the average degree of commonsense across all questions is $8.92$ and $31.01\%$ respectively. 

%\arxiv{To compute average age and average degree of commonsense across questions, we first compute the average age and average degree of commonsense (binary response scaled to $0-100$) per question (by taking average across 10 subjects for each question) and then take average across questions.} 

It is important to distinguish between:
\begin{compactenum}
\item How old someone needs to be to be able to answer a question correctly,  and
\item How old people \emph{think} someone needs to be to be able to answer a question correctly. 
\end{compactenum}

Our age annotations capture the latter -- perceptions of MTurk workers in an uncontrolled environment. As such, the relative ordering of question types in \tableref{tab:typeacc} is more important than absolute age numbers.
%The relative ordering of question types is more important than the absolute age numbers. It is important to note that the age annotations we have collected are just perceived ages: how old people -- untrained MTurk workers in an uncontrolled environment -- \emph{think} someone needs to be to be able to answer a question correctly.}
The two rankings of questions in terms of common sense required according to the two studies 
were largely correlated (Pearson's rank correlation: 0.58). 

%%%%%%%%%%%%%%%%%%%%%%%%%%%%%%%%%%%%%%%%%%%%%%%%%%%%%%%%%%%
\begin{table}[t]
\setlength{\tabcolsep}{3.2pt}
{\small
\begin{center}
%\begin{tabular}{@{}llccc@{}}
%\toprule
%Dataset & Input & All & Yes/No & Other \\
%%\hline
%\midrule
%    & Question & 40.81 & 67.60 & 21.22 \\
%Real   & Question + Caption* & 57.47 & 78.97 & 44.41 \\
%    & Question + Image & 83.30 & 95.77 & 72.67 \\
%%\hline
%\midrule
% & Question & 43.27 & 66.65 &  23.66 \\
%Abstract & Question + Caption* & 54.34 & 74.70 & 40.18 \\
% & Question + Image & 87.49 & 95.96 & 75.33 \\
%\bottomrule
%\end{tabular}
\begin{tabular}{@{}llcccc@{}}
\toprule
Dataset & Input & All & Yes/No & Number & Other \\
%\hline
\midrule
    & Question & 40.81 & 67.60 & 25.77 & 21.22 \\
Real   & Question + Caption* & 57.47 & 78.97 & 39.68 & 44.41 \\
    & Question + Image & 83.30 & 95.77 & 83.39 & 72.67 \\
%\hline
\midrule
 & Question & 43.27 & 66.65 & 28.52 & 23.66 \\
Abstract & Question + Caption* & 54.34 & 74.70 & 41.19 & 40.18 \\
 & Question + Image & 87.49 & 95.96 & 95.04 & 75.33 \\
\bottomrule
\end{tabular}
\end{center}
}
%\vspace{-7pt}
\caption {Test-standard accuracy of human subjects when asked to answer the 
question without seeing the image (Question), 
seeing just a caption of the image and not the image itself (Question + Caption), 
and seeing the image (Question + Image). 
Results are shown for all questions, ``yes/no'' \& ``number'' questions, and other questions 
that are neither answered ``yes/no'' nor number. 
All answers are free-form and not multiple-choice. 
*\hspace{1pt}These accuracies are evaluated on a subset of 3K train questions (1K images).}
% \textcolor{red}{and are not directly comparable to the corresponding numbers in older version.}}
\label{table:commonsense_acc}
%\vspace{\captionReduceBot}
%\vspace{-5pt}
\end{table}
%%%%%%%%%%%%%%%%%%%%%%%%%%%%%%%%%%%%%%%%%%%%%%%%%%%%%%%%%%%


%%%%%%%%%%%%%%%%%%%%%%%%%%%%%%%%%%%%%%%%%%%%%%%%%%%%%%%%%%%
%\vspace{\subsectionReduceTop}
\subsection{Captions \textbf{\vs} Questions}
%\vspace{\subsectionReduceBot}
%%%%%%%%%%%%%%%%%%%%%%%%%%%%%%%%%%%%%%%%%%%%%%%%%%%%%%%%%%%


Do generic image captions provide enough information to answer the questions?
\tableref{table:commonsense_acc} (Question + Caption) shows the percentage of questions answered
correctly when human subjects are given the question and a human-provided caption
describing the image, but not the image. As expected, the results are better than when humans are shown the questions alone.
However, the accuracies are significantly lower than when subjects are shown the actual image.
This demonstrates that in order to answer the questions correctly, deeper image understanding 
(beyond what image captions typically capture) is necessary. In fact, we find that the distributions of nouns, verbs, and adjectives mentioned in captions is statistically significantly different from those mentioned in our questions + answers (Kolmogorov-Smirnov test, $p<.001$) for both real images and abstract scenes. See the appendix for details. 
%This motivates the VQA task as a way to learn further information about visual scenes.

\paragraph{3D Object Detection from RGB-D Data} Researchers have approached the 3D detection problem by taking various ways to represent RGB-D data.

\emph{Front view image based methods:} ~\cite{chen2016monocular, mousavian20163d, xiang2015data} take monocular RGB images and shape priors or occlusion patterns to infer 3D bounding boxes. ~\cite{li2016vehicle, deng2017amodal} represent depth data as 2D maps and apply CNNs to localize objects in 2D image. In comparison we represent depth as a point cloud and use advanced 3D deep networks (PointNets) that can exploit 3D geometry more effectively.

\emph{Bird's eye view based methods:} MV3D~\cite{cvpr17chen} projects LiDAR point cloud to bird's eye view and trains a region proposal network (RPN~\cite{ren2015faster}) for 3D bounding box proposal. However, the method lags behind in detecting small objects, such as pedestrians and cyclists and cannot easily adapt to scenes with multiple objects in vertical direction.
%Our method shares the idea with~\cite{cvpr17chen} in reducing 3D search cost by 2D search first. What differentiates our method from \cite{cvpr17chen} is that, \hao{???} instead of projecting point cloud to images costing loss in 3D geometry, we directly apply PointNet to point clouds that correspond to the 2D regions. % Besides, our method and MV3D can potentially be combined in the bird's eye setting. 3D proposals from our frustum-based PointNet and MV3D can be combined and our 3D network can also be used for bounding box estimation for point cloud in the bird's eye 2D region.

\emph{3D based methods:} ~\cite{wang2015voting, song2014sliding} train 3D object classifiers by SVMs on hand-designed geometry features extracted from point cloud and then localize objects using sliding-window search. \cite{engelcke2017vote3deep} extends ~\cite{wang2015voting} by replacing SVM with 3D CNN on voxelized 3D grids. \cite{ren2016three} designs new geometric features for 3D object detection in a point cloud. \cite{song2016deep, li20163d} convert a point cloud of the entire scene into a volumetric grid and use 3D volumetric CNN for object proposal and classification. Computation cost for those method is usually quite high due to the expensive cost of 3D convolutions and large 3D search space.
%In comparison, we use 2D region proposals from RGB images to reduce the search space from the entire 3D scenes into 3D frustums. Since the points cloud in the frustums have largely varying depth ranges and can be very sparse, it's not applicable to apply CNN on bird's eye view or apply 3D CNN in grids. Our frustum-based PointNet, on the other hand, suits well for this type of data and is able to accurately estimate 3D bounding box with good efficiency.
Recently, \cite{lahoud20172d} proposes a 2D-driven 3D object detection method that is similar to ours in spirit. However, they use hand-crafted features (based on histogram of point coordinates) with simple fully connected networks to regress 3D box location and pose, which is sub-optimal in both speed and performance. In contrast, we propose a more flexible and effective solution with deep 3D feature learning (PointNets).
%In addition we also get 3D instance segmentation as intermediate outputs. Evaluated on SUN-RGBD we show our method is \emph{8.9\%} better than theirs in mAP and \emph{34x} faster at the same time.


% \begin{enumerate}
%     \item ZOOX~\cite{mousavian20163d} image based
%     \item Vote3Deep~\cite{engelcke2017vote3deep} 3d cnn. Recent LIDAR-based methods place 3D windows in 3D voxel grids to score the point cloud
%     \item Voting for Voting~\cite{wang2015voting} Recent LIDAR-based methods place 3D windows in 3D voxel grids to score the point cloud. apply SVM classifers on 3D grids encoded with geometry features
%     \item MV3D~\cite{cvpr17chen}
%     \item VeloFCN~\cite{li2016vehicle} apply convolutional networks to the front view point map in a dense box prediction scheme
%     \item 3DOP~\cite{chen20153d} image based. reconstructs depth from stereo images and uses an energy minimization approach to generate 3D box proposals, which are fed to an R-CNN [10] pipeline for object recognition
%     \item Mono3D~\cite{chen2016monocular} image based. shares the same pipeline with 3DOP, it generates 3D proposals from monocular images.
%     \item 3DFCN~\cite{li20163d} 3d cnn.
%     \item 3DVP~\cite{xiang2015data} introduces 3D voxel patterns and employ a set of ACF detectors to do 2D detection and 3D pose estimation
%     \item Are Cars just 3D Box?~\cite{zeeshan2014cars} fit model to image patch
%     \item ~\cite{zia2013detailed} fit model to image patch
% \end{enumerate}
% \begin{enumerate}
%     \item SlidingShapes~\cite{song2014sliding} apply SVM classifers on 3D grids encoded with geometry features
%     \item DeepSlidingShapes~\cite{song2015sun} 3d cnn.
%     \item 2D-driven~\cite{lahoud20172d}
%     \item ~\cite{deng2017amodal} rgb-d images
%     \item COG feature~\cite{ren2016three}
%     \item Align 3D model in RGB-D~\cite{gupta2015aligning}
% \end{enumerate}

\paragraph{Deep Learning on Point Clouds}
Most existing works convert point clouds to images or volumetric forms before feature learning. \cite{wu20153d, maturana2015voxnet, qi2016volumetric} voxelize point clouds into volumetric grids and generalize image CNNs to 3D CNNs. ~\cite{li2016fpnn, riegler2016octnet, wang2017cnn, engelcke2017vote3deep} design more efficient 3D CNN or neural network architectures that exploit sparsity in point cloud.
However, these CNN based methods still require quantitization of point clouds with certain voxel resolution.
Recently, a few works~\cite{qi2017pointnet,qi2017pointnetplusplus} propose a novel type of network architectures (PointNets) that directly consumes raw point clouds without converting them to other formats. While PointNets have been applied to single object classification and semantic segmentation, our work explores how to extend the architecture for the purpose of 3D object detection.

\section{Conclusion}\label{sec:conclusion}
%\vspace{-.1in}
In this work, we apply the attentional encoder-decoder for the task of abstractive summarization with very promising results, outperforming state-of-the-art results significantly on two different datasets. Each of our proposed novel models addresses a specific problem in abstractive summarization, yielding further improvement in performance. We also propose a new dataset for multi-sentence summarization and establish benchmark numbers on it. As part of our future work, we plan to focus our efforts on this data and build more robust models for summaries consisting of multiple sentences.


%Our results strongly demonstrate that sequence-to-sequence models are extremely promising for summarization. Some of the other lessons we learned from our experiments are: (i) the LVT-trick is very useful for summarization as it improves training speed while not sacrificing performance; (ii) traditional methods such as vocabulary expansion and syntax-based features can boost performance of deep learning based models as well. As part of our ongoing work, we are investigating on ways to effectively generate rare words in the summary, which appears to be a glaring weakness in the existing models.  


\section*{Acknowledgments}
This work is supported by the National Basic Research Program of China (973 program, No. 2014CB340505) and Baidu-Peking University Joint Project.
We thank the Microsoft MSMARCO team for evaluating our results on the anonymous test set. We also thank Ying Chen, Xuan Liu and the anonymous reviewers for their constructive criticism of the manuscript.


% include your own bib file like this:
%\bibliographystyle{acl}
%\bibliography{acl2018}
\bibliography{ref}
\bibliographystyle{acl_natbib}

\end{document}

\documentclass{article} % For LaTeX2e
\usepackage{iclr2017_conference,times}

\usepackage{booktabs}       % professional-quality tables
\usepackage{amsfonts}       % blackboard math symbols
\usepackage{nicefrac}       % compact symbols for 1/2, etc.
\usepackage{microtype}      % microtypography
\usepackage{setspace}

\usepackage{qiangstyle}
\usepackage{url}
\newcommand{\dilinfig}{./figures}
\newcommand{\dilincheck}[1]{#1}%\textcolor{magenta}{#1}}

%\rnewcommand{\figcapsize}{\small}

\title{Learning to Draw Samples: With Application to Amortized MLE for Generative Adversarial Learning} %Training Deep Energy Models}

\author{Dilin Wang, ~~Qiang Liu\\
Department of Computer Science, Dartmouth College\\
%Institute \\
%Dartmouth College\\
\texttt{\{dilin.wang.gr, qiang.liu\}@dartmouth.edu}
}
%\author{
%Qiang Liu \& Dilin Wang \thanks{Use footnote for providing further information
%about author (webpage, alternative address)---\emph{not} for acknowledging
%funding agencies.  Funding acknowledgements go at the end of the paper.} \\
%Department of Computer Science\\
%Dartmouth College\\
%Hanover, NH 03755, USA \\
%\texttt{\{qiang.liu, XXX\}@dartmouth.edu}
%}
% The \author macro works with any number of authors. There are two commands
% used to separate the names and addresses of multiple authors: \And and \AND.
%
% Using \And between authors leaves it to \LaTeX{} to determine where to break
% the lines. Using \AND forces a linebreak at that point. So, if \LaTeX{}
% puts 3 of 4 authors names on the first line, and the last on the second
% line, try using \AND instead of \And before the third author name.

%\newcommand{\fix}{\marginpar{FIX}}
%\newcommand{\new}{\marginpar{NEW}}

%\iclrfinalcopy % Uncomment for camera-ready version


\begin{document}

\maketitle


\begin{abstract}
We propose a simple algorithm to train stochastic 
neural networks 
%black-box procedures, such as neural networks with random inputs, 
to draw samples from given target distributions for probabilistic inference. 
Our method is based on iteratively adjusting the neural network parameters 
so that the output changes along a Stein variational gradient \citep{liu2016stein} that maximumly decreases 
the KL divergence with the target distribution. 
%to decrease the KL divergence between the output random variable and the target distribution, by mimicing a recently proposed Stein variational gradient descent. 
Our method works for any target distribution specified by their unnormalized density function, 
and can train any black-box architectures that are differentiable in terms of the parameters we want to adapt. 
%By allowing to ``learn to draw samples'', our method opens a host of applications. 
%We present two examples in this paper: 
%In addition, we leverage our method to 
As an application of our method, we propose an  \emph{amortized MLE} algorithm for training deep energy model, where a neural sampler is adaptively trained to 
approximate the likelihood function. Our method mimics an adversarial game 
between the deep energy model and the neural sampler, and obtains realistic-looking images competitive with the state-of-the-art results. 
% that attempts to fool the energy model.
%by training neural samplers to approximate the gradient for maximum likelihood by training neural networks to approximate the derivative of the log partition function of unnormalized distributions
%for maximum likelihood training, % models,  
%we propose an \emph{amortized MLE} method for training deep energy model, 
%which admits an adversarial game between 
%turning the traditional MLE algorithm into a generative adversarial  
%between the deep energy model 
%and the generative neural network that attempts to fool the energy model. We show that we can get realistic-looking images competitive with the state-of-the-art results. 
%2) by treating stochastic gradient Langevin dynamics as a black-box sampler, we train to automatically adjust its learning rate to maximize its convergence speed, outperforming the hand-designed learning rate schemes. % such as adagrad. 
%generative adversarial networks. 
\end{abstract}



\section{Introduction}


Modern machine learning increasingly relies on highly complex probabilistic models to reason about uncertainty.  
A key computational challenge is to develop efficient inference techniques to approximate, or draw samples from complex distributions. 
Currently, most inference methods, including MCMC and variational inference, are hand-designed by researchers or domain experts. 
%
This makes it difficult to fully optimize the choice of different methods and their parameters, and exploit the structures in the problems of interest in an automatic way. 
%
The hand-designed algorithm can also be inefficient when it requires to make fast inference repeatedly on a large number of different distributions with similar structures. 
%
This happens, for example, when we need to reason about a number of observed datasets in settings like online learning, 
or need fast inference as inner loops for other algorithms such as maximum likelihood training. 
Therefore, it is highly desirable to develop more intelligent probabilistic inference systems that can adaptively improve its own performance to fully the optimize computational efficiency, and generalize to new tasks with similar structures. 

Specifically, denote by $p(x)$ a probability density of interest specified up to the normalization constant, which we want to draw sample from, or marginalize to estimate its normalization constant. We want to study the following problem: 

\begin{problem}\label{pro:prob1}
Given a distribution with density $p(x)$ %specified up to the normalization constant, 
and a function $f(\eta;~\xi)$ with parameter $\eta$ and random input $\xi$, 
for which we only have assess to draws of the random input $\xi$ (without knowing its true distribution $q_0$), 
and the output values of $f(\eta;~\xi)$ and its derivative $\partial_\eta f(\eta;~\xi)$ given $\eta$ and $\xi$.  
We want to find an optimal parameter $\eta$ so that the density of the random output variable $x = f(\eta;~\xi)$ with $\xi\sim q_0$ closely matches the target density $p(x)$. 
\end{problem}

Because we have no assumption on the structure of $f(\eta;~\xi)$ 
and the distribution of random input,
we can not directly calculate the actual distribution of the output random variable $x = f(\eta;~\xi)$; 
this makes it difficult to solve Problem~\ref{pro:prob1} using the traditional variational inference (VI) methods. 
Recall that traditional VI approximates $p(x)$ using simple proposal distributions $q_\eta(x)$ indexed by parameter $\eta$, 
and finds the optimal $\eta$ by minimizing KL divergence $\KL(q_\eta ~||~p) = \E_{q_\eta}[\log (q_\eta/p)]$, 
which requires to calculate the density $q_\eta(x)$ or its derivative that is not computable by our assumption 
(even when the Monte Carlo gradient estimation and the reparametrization trick \citep{kingma2013auto} are applied).  
%and hence require to calculate the density $q$ or its derivative of the random output $x = f(\eta;~\xi)$, which is unknown or difficult to calculate by our assumption. 

\begin{comment}
\begin{figure}[t]
   \centering
   \scalebox{1}{
   \includegraphics[width=.6\textwidth]{figures/neuralsampler1}%neuralbayesian} % requires the graphicx package
   }
   \begin{picture}(0,0)(0,0)
   \put(-250,-10){\figcapsize \emph{Given distribution}}
      \put(-170,-10){{\figcapsize \emph{Black-box neural sampler}}}
%      \put(-180,-10){{\figcapsize \emph{by amortized SVGD}}}      
%   \put(-180,0){\begin{tabular}{c} {\figcapsize Black-box sampler} \\[-1\baselineskip] {\figcapsize trained by amortized SVGD}    \end{tabular}}
   \put(-50,-10){\figcapsize \emph{Samples}}
   \end{picture}\\[.5em]
   \caption{\figcapsize   
   Our methods ``learn to draw samples'',  
   constructing black-box neural samplers for given distributions. 
   It allows us to automatize the hyper-parameter tuning of Bayesian inference, 
 speed up the inference inner loops of learning algorithms, 
and eventually replace hand-designed inference algorithms with more efficiently one that is trained on past tasks and is improved adaptively over time.}
% Our new amortized SVGD algorithm allows us to perform ``meta-Bayesian inference'', constructing functions that learns to produce posterior samples based on prior and data, significantly improving the efficiency of Bayesian inference at large scale.}
%   Our new amortized SVGD algorithm allows us to perform ``meta-Bayesian inference'', constructing functions that learns to produce posterior samples based on prior and data, significantly improving the efficiency of Bayesian inference at large scale.}
   \label{fig:example}
\end{figure}
\end{comment}


In fact, it is this requirement of calculating $q_\eta(x)$ 
that has been the major constraint for the designing of state-of-the-art variational inference methods with rich approximation families; 
the recent successful algorithms \citep[e.g.,][to name only a few]{rezende2015variational,tran2015variational,ranganath2015hierarchical} 
have to handcraft special variational families 
to ensure the computational tractability of $q_\eta(x)$ and simultaneously obtain high approximation accuracy, 
which require substantial mathematical insights and research effects. 
Methods that do not require to explicitly calculate $q_\eta(x)$ 
can significantly simplify the design and applications of VI methods, 
allowing practical users to focus more on choosing proposals that work best with their specific tasks. %, making the design and applications of VI much easier. 
%We should distinguish 
We will use the term \emph{wild variational inference} to refer to new variants of variational methods that require no tractability $q_\eta(x)$, %\emph{wild variational inference}, 
to distinguish with the \emph{black-box variational inference} \citep{ranganath2013black}
which refers to methods that work for generic target distributions $p(x)$ without significant model-by-model consideration (but still require to calculate the proposal density $q_\eta(x)$). 

A similar problem also appears in importance sampling (IS), 
where it requires to calculate the IS proposal density $q(x)$ in order to calculate the importance weight $w(x) = p(x)/q(x)$. 
However, %it is been shown that it is possible to develop %\emph{black-box importance sampling}  
there exist methods that use no explicit information of $q(x)$, which, seemingly counter-intuitively, give better asymptotic variance or converge rates than the typical IS that uses the proposal information \citep[e.g.,][]{liu2016black, briol2015probabilistic, henmi2007importance, delyon2014integral}.
% at least in the asymptotic sense. 
%This phoemnomin was first observed 
Discussions on this phenomenon dates back to \citet{o1987monte}, who 
argued that ``Monte Carlo (that uses the proposal information) is fundamentally unsound'' for violating the Likelihood Principle, 
and developed Bayesian Monte Carlo \citep{o1991bayes} as 
an example that uses no information on $q(x)$, yet gives better convergence rate than the typical Monte Carlo $\Od(n^{-1/2})$ rate \citep{briol2015probabilistic}. 
Despite the substantial difference between IS and VI, 
these results intuitively suggest the possibility of developing efficient variational inference without calculating $q(x)$ explicitly. 

In this work, we propose a simple algorithm for Problem~\ref{pro:prob1} 
by iteratively adjusting the network parameter $\eta$ to make its output random variable changes along 
a Stein variational gradient direction (SVGD) \citep{liu2016stein} that optimally decreases its KL divergence 
with the target distribution. 
Critically, the SVGD gradient includes a repulsive term to ensure that the generated samples have the right amount of variability that matches $p(x).$ 
In this way, we ``amortize SVGD'' using a neural network, which makes it possible for our method to adaptively improve its own efficiency  
%ccording to which shared inferences are cached and composed together to answer new queries. 
by leveraging fast experience, especially in cases when it needs to perform fast inference repeatedly on a large number of similar tasks. 
%One example of this is in maximum likelihood estimation (MLE) for unnormalized distributions, in which it is required to draw samples to approximate the likelihood at each iteration. 
%referred to as
As an application, we use our method to amortize the MLE training of deep energy models, where a neural sampler is adaptively trained to 
approximate the likelihood function. Our method, which we call \emph{SteinGAN}, mimics an adversarial game 
between the energy model and the neural sampler, and obtains realistic-looking images competitive with the state-of-the-art results produced by generative adversarial networks (GAN) \citep{goodfellow2014generative, radford2015unsupervised}. 

\paragraph {Related Work} 
The idea of amortized inference \citep{gershman2014amortized}
 has been recently applied in various domains of probabilistic reasoning, 
 including both amortized variational inference \citep[e.g.,][]{kingma2013auto, jimenez2015variational}, 
and data-driven proposals for (sequential) Monte Carlo methods \citep[e.g.,][]{paige2016inference}, 
 %and maximum a posteriori \citep{sonderby2016amortised}, 
 to name only a few. Most of these methods, however, 
 require to explicitly calculate $q(x)$ (or its gradient). 
 One exception is a very recent paper \citep{operator} that avoids calculating $q(x)$ using an idea related to Stein discrepancy \citep{gorham2015measuring, liu2016kernelized, oates2014control, chwialkowski2016kernel}.
 %; the method by \citet{operator} may be more computationally expensive than our method because it requires to train an additional neural network to distinguish $q_\eta$ and $p$ and requires to calculate the second derivative of $p(x)$. 
 There is also a raising interest recently on a similar problem of ``learning to optimize'' \citep[e.g.,][]{andrychowicz2016learning, daniel2016learning, li2016learning}, which is technically easier than the more general problem of ``learning to sample''. %that also requires to take the uncertainty into account.  
In fact, we show that our algorithm reduces to ``learning to optimize'' when only one particle is used in SVGD. 

Generative adversarial network (GAN) and its variants have recently gained remarkable success on generating realistic-looking images 
\citep{goodfellow2014generative, salimans2016improved, radford2015unsupervised, li2015generative,dziugaite2015training,nowozin2016f}.  
All these methods are set up to train latent variable models (the generator) under the assistant of the discriminator.  
Our SteinGAN instead performs traditional MLE training for a deep energy model, 
 with the help of a neural sampler that learns to draw samples from the energy model to approximate the likelihood function; 
 this admits an adversarial interpretation: we can view the neural sampler as a generator that attends to fool the deep energy model, which in turn serves as a discriminator 
that distinguishes the real samples and the simulated samples given by the neural sampler. 
This idea of training MLE with neural samplers was first discussed by \citet{kim2016deep}; 
one of the key differences is that the neural sampler in \citet{kim2016deep} 
is trained with the help of a heuristic diversity regularizer based on batch normalization, while 
SVGD enforces the diversity in a more principled way. 
%which, without a principled approach for solving Problem~\ref{pro:prob1},  in which the diversity of simulated samples are enforced with a heuristic motivated by batch normalization.  
Another method by \citet{zhao2016energy} also trains an energy score to distinguish real and simulated samples, but within a non-probabilistic framework (see Section~\ref{sec:gan} for more discussion). 
Other more traditional approaches for training energy-based models  \citep[e.g.,][]{ngiam2011learning, xie2016theory} are often based on variants of MCMC-MLE or contrastive divergence \citep{geyer1991markov, hinton2002training, tieleman2008training}, and have difficulty generating realistic-looking images from scratch. 



\section{Stein Variational Gradient Descent (SVGD)}
%This proposal will focus on integrating Stein's method with variational inference, by leveraging a new variational interpretation of Stein's characterization \eqref{equ:stein00} discovered in my recent work \citep{liu2016stein}, in which a general purpose inference algorithm called Stein variational gradient descent (SVGD) has been proposed. 
Stein variational gradient descent (SVGD) \citep{liu2016stein} is a general purpose Bayesian inference algorithm motivated by 
Stein's method \citep{stein1972, barbour2005introduction} and kernelized Stein discrepancy \citep{liu2016kernelized, chwialkowski2016kernel, oates2014control}. 
It uses an efficient \emph{deterministic} gradient-based update 
to iteratively evolve a set of particles $\{x_i\}_{i=1}^n$ to minimize the KL divergence with the target distribution. 
SVGD has a simple form that reduces to the typical gradient descent for maximizing $\log p$ when using only one particle $(n=1)$, 
and hence can be easily combined with the successful tricks for gradient optimization, 
including stochastic gradient, adaptive learning rates (such as adagrad), and momentum. % making large scale Bayesian inference both much easier and more efficient. 

To give a quick overview of the main idea of SVGD, let $p(x)$ be a positive density function on $\R^d$ which we want to 
approximate with a set of particles 
$\{ x_i\}_{i=1}^n$. 
SVGD initializes the particles by sampling from some simple distribution $q_0$, and 
updates the particles iteratively by 
%performs iterative updates of form 
\begin{align}\label{equ:xxii}
x_i  \gets x_i +  \epsilon \ff(x_i),  ~~~~ \forall i = 1, \ldots, n,  
\end{align}
where $\epsilon$ is a step size, and 
$\ff(x)$ is a ``particle gradient direction'' %  perturbation direction, or velocity field, 
%which roughly speak, is the gradient of $\KL(q_\vx~||~ p)$ w.r.t. $\vx$, 
chosen to maximumly decrease the KL divergence between the distribution of particles and the target distribution, in the sense that  
%decreases with the fastest speed in the sense that 
\begin{align}\label{equ:ff00}
\ff =   \argmax_{\ff \in \F}  \bigg\{  -   \frac{d}{d\epsilon} \KL(q_{[\epsilon\ff]} ~|| ~ p) \big |_{\epsilon = 0}  \bigg\}, 
%\frac{1}{\epsilon}\{ \KL(q_{t+1}~||~ p )  -  \KL(q_{t}~||~ p) \}, 
\end{align}
where $q_{[\epsilon \ff]}$ denotes the density of the updated particle $x^\prime = x + \epsilon \ff(x) $ when the density of the original particle $x$ is $q$, and $\F$ is the set of perturbation directions that we optimize over. 
We choose $\F$ to be the unit ball of a vector-valued reproducing kernel Hilbert space (RKHS) $\H^d = \H \times \cdots \times \H$ with each $\H$ associating with a positive definite kernel $k(x,x')$; 
%this makes $\F$
%this choice of $\F$ is broad, since $\H^d$
note that $\H$ is dense in the space of continuous functions with universal kernels such as the Gaussian RBF kernel. 
%to allow close approximation of arbitrary continuous functions. 
%which is dense in the space of continuous function with typical non-degenerate kernels such as RBF kernel. 
%We take $\F$ to be the unit ball of a reproducing kernel Hilbert space (RKHS) $\H$ associated with a positive definite kernel $k(x,x^\prime)$, 
%which can be dense in the space of continuous functions with universal (non-degenerate?) kernels such as Gaussian. 
%which is dense in the set of continuous functions when $k(x,x^\prime)$ is non-degenerate \red{(?)},  
%and also makes \eqref{equ:ff00} computationally tractable. 
%{
%{https://papers.nips.cc/paper/4168-universal-kernels-on-non-standard-input-spaces.pdf
%For example, for compact X ? Rd, [30] showed that, among a few others, the RKHSs of the Gaussian RBF kernels are universal, that is, they are dense in the space C(X) of continuous functions f : X ? R. With the help of a standard result from measure theory, see e.g. [1, Theorem 29.14], it is then easy to conclude that these RKHS are also dense in all L1(?) for which ? has a compact support. This key result has been extended in a couple of different directions: For example, [18] establishes universality for more classes of kernels on compact X ? Rd, whereas [32] shows the denseness of the Gaussian RKHSs in L1(?) for all distributions ? on Rd. Finally, [7, 8, 28, 29] show that universal kernels are closely related to so-called characteristic kernels that can be used to distinguish distributions. In addition, all these papers contain sufficient or necessary conditions for universality of kernels on arbitrary compact metric spaces X, and [32] further shows that the compact metric spaces are exactly the compact topological spaces on which there exist universal spaces.
%}

Critically, the gradient of KL divergence in \eqref{equ:ff00} 
equals a simple linear functional of $\ff$, allowing us to obtain a closed form solution for the optimal $\ff$. 
\citet{liu2016stein} showed that 
%A key observation is that the objective in \eqref{equ:ff00} is a linear functional of $\ff$, in fact, we have
\begin{align}\label{equ:klstein00}
&- \frac{d}{d\epsilon} \KL(q_{[\epsilon\ff]} ~|| ~ p) \big |_{\epsilon = 0}  = \E_{x\sim q}[\sumstein_p \ff(x)], \\[.5\baselineskip]
&~~~~\text{with}~~~~~ \sumstein_p \ff(x)  = \nabla_x \log p(x) ^\top \ff (x)+ \nabla_x \cdot \ff(x),  
\end{align}
where $\sumstein_p$ is considered as a linear operator acting on function $\ff$ and is called the Stein operator in connection with Stein's identity which shows that 
the RHS of \eqref{equ:klstein00} equals zero if $p = q$: 
\begin{align}\label{equ:steinid}
\E_{p}[\sumstein_p \ff] =\E_{p}[ \nabla_x \log p ^\top \ff + \nabla_x \cdot \ff] = 0. 
\end{align}
This is a result of integration by parts assuming the value of $p(x)\ff(x)$ vanishes on the boundary of the integration domain.   
%Obviously, $\F$ should be taken as broad as possible, best with infinite dimension, while still allows tractable solution. 
%
%Critically, we show that derivative in \eqref{equ:f00} yields a closed form representation:
%$$  \frac{d}{d\epsilon} \KL(q_{[\epsilon\ff]} ~|| ~ p) \big |_{\epsilon = 0}  = \E_{x\sim q} [\trace(\stein_p \ff(x))], $$Critically, we show that $\F$ to be the unit ball of a reproducing kernel Hilbert space (RKHS) $\H$
%
%
%%$\KL(q_{\vx^t} ~||~ p)$ between the distribution between the distribution $q_{\vx}$ of particles $\vx^t$ and the target distribution decreases with the fastest speed. 
%We take $\F$ to be the unit ball of a reproducing kernel Hilbert space (RKHS) $\H$ associated with positive definite kernel $k(x,x')$. 
%Thanks to an important connection with Stein's identity and kernelized Stein discrepancy, we show that the optimal $\phi$ is given by 
%$\ff(x)$ in RKHS $\H$ associated with positive definite kernel $k(x,x')$, in which case the optimal choice of $\ff(\vx)$ is 

Therefore, the optimization in \eqref{equ:ff00} reduces to 
\begin{align}\label{equ:ksd}
\S(q ~||~ p) \overset{def}{=} \max_{\ff \in \H^d} \{ \E_{x\sim q} [\sumstein_p \ff(x)]  ~~~s.t.~~~~ ||\ff ||_{\H^d} \leq 1\}, 
\end{align}
where $\S(q ~||~ p)$ is the kernelized Stein discrepancy defined in \citet{liu2016kernelized}, which equals zero if and only if $p = q$ under mild regularity conditions. 
Importantly, the optimal solution of \eqref{equ:ksd} yields a closed form% for the optimal solution: of \eqref{equ:ff00}:
 %Further,  it turns out that optimizing \eqref{equ:klstein00} in the unit ball of $\H$ coincides with the variational form of the kernelized Stein discrepancy in \citet{liu2016kernelized},  giving a closed form for the optimal solution of \eqref{equ:ff00}: 
$$
\ff^*(x') \propto  \E_{x\sim q}[\nabla_x \log p(x)k(x,x') + \nabla_x k(x,x')]. 
$$
%This $\ff^*$ can be treated as a gradient of the KL divergence. 
By approximating the expectation under $q$ with the empirical average of the current particles $
\{x_i\}_{i=1}^n$,  SVGD admits a simple form of update: 
\begin{align}\label{equ:update11}
%\begin{split}
%x_i ~ \gets ~ x_i  ~  + ~ \epsilon  \hat\ff^*(x_i) ~~~~~\text{where}~~~~~~ \hat \ff^*(x_i) = \hat \E_{x\in \{x_i\}} [ k(x_i, ~ x) \nabla_{x} \log p(x)  + \nabla_{x} k(x_i, x) ], 
&& &~~~~~~~~~~~ x_i ~ \gets ~ x_i  ~  + ~ \epsilon \Delta x_i, 
~~~~~~~~\forall i = 1, \ldots, n,  \notag
\\
~
&&& \text{where~~~~~}\Delta x_i = \hat \E_{x\in \{x_i\}_{i=1}^n} [  \nabla_{x} \log p(x) k(x, x_i) + \nabla_{x} k(x, x_i) ], 
%x_i ~ \gets ~ x_i  ~  + ~ \epsilon \frac{1}{n} \sum_{x\in \{x_i\}} [ k(x_i, ~ x_j) \nabla_{x} \log p(x)  + \nabla_{x} k(x_i, x) ] . 
%\end{split}
\end{align}
and $\hat\E_{x\sim \{x_i\}_{i=1}^n}[f(x)] = \sum_i f(x_i)/n$. 
The two terms in $\Delta x_i$ play two different roles: 
the term with the gradient $\nabla_x \log p(x)$ drives the particles toward the high probability regions of $p(x)$, 
while the term with $\nabla_x k(x,x_i)$ serves as a repulsive force to encourage diversity;
to see this, consider a stationary kernel $k(x,x') = k(x-x')$, then the second term reduces to $\hat \E_x \nabla_{x} k(x,x_i) = - \hat \E_x \nabla_{x_i} k(x,x_i)$, 
which can be treated as the negative gradient for minimizing the average similarity $\hat \E_x k(x,x_i)$ in terms of $x_i$. 
Overall, this particle update produces diverse points for distributional approximation and uncertainty assessment, and 
also has an interesting ``momentum'' effect in which the particles move collaboratively to escape the local optima.
%See \citet{liu2016stein} for more details. 

It is easy to see from \eqref{equ:update11} that $\Delta x_i$ reduces to the typical gradient $\nabla_x \log p(x_i)$ when there is only a single particle ($n=1$) and $\nabla_x k(x,x_i)$ when $x=x_i$,  
in which case SVGD reduces to the standard gradient ascent for maximizing $\log p(x)$ (i.e., maximum \emph{a posteriori} (MAP)). 

%This update evolves the particles collectively to match the target distribution with the two terms of $\Delta(x_i)$ playing two different roles:    the first term in $\Delta(x_i)$ is theweighted sum of the particles' gradient of the log-density, which drives the particles towards the high probability regions in a collaborative fashion; the second term can be shown to serve as a ``repulsive force'' that makes the particles repel each other to maintain a degree of diversity.
% maintaining a degree of diversity via the second term $\nabla_x k(x_i, x)$ that makes the particles repel each other.

\begin{algorithm}[t]                      % enter the algorithm environment
\caption{Amortized SVGD for Problem~\ref{pro:prob1}}% for Wild Variational Inference}          % give the algorithm a caption
\label{alg:alg1}                           % and a label for \ref{} commands later in the document
\begin{algorithmic}                    % enter the algorithmic environment
\STATE Set batch size $m$, step-size scheme $\{\epsilon_t\}$ and kernel $k(x,x')$. Initialize $\eta^0$. 
\FOR {iteration $t$}
\STATE  Draw random $\{\xi_i\}_{i=1}^m$, calculate $x_i = f(\eta^t;~\xi_i)$, 
and the Stein variational gradient $\Delta x_i$ in \eqref{equ:update11}. 
%\begin{align*}%\label{equ:dxi}
%\Delta x_i = \hat\E_{x\sim\{x_i\}_{i=1}^m} [\log p(x) k(x,x_i) + \nabla_x k(x, x_i)]. 
%\end{align*}
\STATE  Update parameter $\eta$ using \eqref{equ:follow1}, \eqref{equ:follow2} or \eqref{equ:follow3}. 
%$$\eta^{t+1} \gets \eta^{t} + \epsilon ~ \sum_{i=1}^n \partial_\eta f(\eta^t;~\xi_i)  \Delta x_i.$$ 
\ENDFOR
\end{algorithmic}
\end{algorithm}

\section{Amortized SVGD: Towards an Automatic Neural Sampler}
\label{sec:amortizedsvgd}
%\section{Our Method}

%SVGD and other particle-based methods require a large memory to store the particles when  using a large number particles. In addition, 
%Particle methods like SVGD can provide simple and consistent approximation for individual target distributions, 
SVGD and other particle-based methods become inefficient when we need to repeatedly infer
a large number different target distributions for multiple tasks, including online learning or inner loops of other algorithms, 
because they can not improve based on the experience from the past tasks, and may require a large memory to restore a large number of particles. 
We propose to ``amortize SVGD'' by training a neural network $f(\eta;~\xi)$ to mimic the SVGD dynamics, yielding a solution for Problem~\ref{pro:prob1}. 

%We propose to use a neural network $f_{\eta}$ so that the output $x = f(\eta;~\xi)$ with $\xi \sim q_0$ mimics the SVGD dynamics. 
One straightforward way to achieve this is to run SVGD to convergence and 
train $f(\eta;~\xi)$ to fit the SVGD results. 
This, however, requires to run many epochs of fully converged SVGD and can be slow in practice. 
We instead propose an \emph{incremental approach} in which $\eta$ is iteratively adjusted 
so that the network outputs $x = f(\eta;~\xi)$ changes along the Stein variational gradient direction in \eqref{equ:update11} 
in order to decrease the KL divergence between the target and approximation distribution. 
%to follow the change $\Delta x_i$ given by SVGD. 
% to mimic the SVGD updates which can summary 
%the experience from the past tasks concisely, allowing ``learning to draw samples''. 
%In this 
%Many practice applications involve a large number of target distributions with similar structures for which 
%it is inefficient to repeatedly applying SVGD or other tradition inference methods on the individual distributions separately. 
%and it would be highly desirable to intelligent systems that \emph{train itself to make better Bayesian inference over time, and also generalize well to new tasks with similar structures}. 
%becomes inefficient when we have a large number of different target distributions that requires us to run the particle algorithm repeatedly. 
%The different target
%This happens when we need to reason with different observed data or priors, or need to make iterative, real-time inference for online tasks or inner loops of learning algorithms. 
%Therefore, it is highly desirable to develop more intelligent Bayesian inference systems that can adaptively improve its own performance over time, and also generalize well to new tasks with similar structures. 

%Our idea is simple: 
%we want to iteratively adjust $\eta$ such that the black-box output $x = f(\eta;~D, \xi)$ moves along the Stein variational gradient direction $\Delta(x_i)$ in \eqref{equ:update11} in order to decrease the KL divergence between the target and approximation distribution. 
To be specific, denote by $\eta^t$ the estimated parameter at the $t$-th iteration of our method; 
each iteration of our method 
draws a batch of random inputs $\{\xi_i\}_{i=1}^m$ 
and calculate their corresponding output $x_i = f(\eta;~\xi_i)$ based on $\eta^t$; here $m$ is a mini-batch size (e.g., $m=100$). 
%as well as $x'_i  = x_i \epsilon \Delta x_i$%B
The Stein variational gradient $\Delta x_i$ in 
\eqref{equ:update11} would then ensure that $x'_i = x_i + \epsilon \Delta x_i$ forms a better approximation of the target distribution $p$. 
Therefore, we should adjust $\eta$ to make its output matches $\{x'_i\}$, 
that is, we want to update $\eta$ by
\begin{align}\label{equ:follow1}
\eta^{t+1} \gets  \argmin_\eta  \sum_{i=1}^m ||  f(\eta;~\xi_i)  - x_i' ||_2^2,  ~~~~~~ \text{where} ~~~~~~  x_i'  = x_i  + \epsilon \Delta x_i. 
\end{align}
%. 
See Algorithm~\ref{alg:alg1} for the summary of this procedure. If we assume $\epsilon$ is very small, then 
\eqref{equ:follow1} reduces to a least square optimization. To see this, note that 
$f(\eta;~\xi_i) \approx f(\eta^t;~\xi_i) + \partial_\eta f(\eta^t;~\xi_i) (\eta - \eta^t)$ by Taylor expansion. 
Since $x_i = f(\eta^t;~\xi_i)$, we have
%Therefore, 
$$
 ||  f(\eta;~\xi_i)  - x_i' ||_2^2 \approx || \partial_\eta f(\eta^t;~\xi_i) (\eta  - \eta^t)  - \epsilon \Delta x_i  ||_2^2. 
$$
%This allows us to approximate the update i
As a result, \eqref{equ:follow1} reduces to the following least square optimization: 
\begin{align}\label{equ:follow2}
\eta^{t+1} \gets \eta^t + \epsilon \Delta \eta^t, 
\text{~~~~~where~~~~~}
\Delta \eta^t = \argmin_{\delta}  \sum_{i=1}^m   || \partial_\eta f(\eta^t;~\xi_i)  \delta   -  \Delta x_i  ||_2^2. 
\end{align}
%Although $\Delta \eta^t$ can be solved with a least square optimization, 
Update \eqref{equ:follow2} can still be computationally expensive because of the matrix inversion. 
% in each iteration. 
We can derive a further approximation by performing only one step of gradient descent of \eqref{equ:follow1} (or \eqref{equ:follow2}), which gives 
\begin{align}\label{equ:follow3}
\eta^{t+1} \gets \eta^t + \epsilon  \sum_{i=1}^m   \partial_\eta f(\eta^t;~\xi_i) \Delta x_i. 
\end{align}
%based on update \eqref{equ:follow3}. 
%We find this update works efficiently in practice. It is shown in Algorithm~\ref{alg:alg1}

%In this way, everything is nice


Although update \eqref{equ:follow3} is derived as an approximation of \eqref{equ:follow1}-\eqref{equ:follow2}, 
it is computationally faster and we find it works very effectively in practice; 
this is because when $\epsilon$ is small, 
one step of gradient update can be sufficiently close to the optimum.  

%Update \eqref{equ:follow3} has a particularly 
%Intuitively speaking, 
Update \eqref{equ:follow3} also has a simple and intuitive form: \eqref{equ:follow3} can be thought as \emph{a ``chain rule'' that back-propagates the Stein variational gradient to the network parameter $\eta$}. 
%making the random sample generated by $z = f(\eta;~\xi)$ mimic the behavior of SVGD and hence move towards the target distribution. 
%Although $\Delta x_i$ is not a gradient in the typical sense, it can be interpreted as a form of functional gradient with which the update of $\eta$ can be justified theoretically. 
This can be justified by considering the special case when we use only a single particle $(n=1)$ 
in which case $\Delta x_i$ in \eqref{equ:update11} reduces to the typical gradient $\nabla_x \log p(x_i)$ of $\log p(x)$, 
%In particular, note that if we just use a single particle for each $D$, our method reduces 
and update \eqref{equ:follow3} reduces to the typical gradient ascent for maximizing 
%a typical stochastic gradient ascent for maximizing 
$$\E_{\xi}[\log p(f(\eta;~\xi))],$$  
in which case $f(\eta;~\xi)$ is trained to maximize $\log p(x)$ (that is, \emph{learning to optimize}), 
instead of \emph{learning to draw samples from $p$} for which it is crucial to use Stein variational gradient $\Delta x_i$ to diversify the network outputs. 
%instead of drawing sampling from $p$ which requires to diversify the network output. 
%the method by \citet{andrychowicz2016learning} for learning to optimize using gradient descent.

Update \eqref{equ:follow3} also has a close connection with the typical variational inference with the reparameterization trick \citep{kingma2013auto}. Let $q_\eta(x)$ be the density function of $x = f(\eta;~\xi)$, $\xi\sim q_0$. Using the reparameterization trick, the gradient of $\KL(q_\eta~||~p)$ w.r.t. $\eta$ can be shown to be 
$$
\nabla_\eta\KL(q_{\eta}~||~p) = -\E_{\xi \sim q_0}[\partial_\eta f(\eta;~\xi)(\nabla_x \log p(x) - \nabla_x \log q_\eta(x))]. 
$$
With $\{\xi_i\}$ i.i.d. drawn from $q_0$ and $x_i = f(\eta;~\xi_i), ~\forall i$, the standard stochastic gradient descent for minimizing the KL divergence is 
%can be performed by 
\begin{align}\label{equ:rep}
\eta^{t+1} \gets \eta^t +  \sum_i \partial_\eta f(\eta^t;~\xi_i) \tilde \Delta x_i, ~~~~ \text{where} ~~~~ 
\tilde \Delta x_i = \nabla_x \log p(x_i) - \nabla_x \log q_\eta(x_i). 
\end{align}
This is similar with \eqref{equ:follow3}, but replaces the Stein gradient $\Delta x_i$ defined in \eqref{equ:update11} with 
$\tilde \Delta x_i$. The advantage of using $\Delta x_i$ is that it does not require to explicitly calculate $q_\eta$,
and hence admits a solution to Problem 1 in which $q_\eta$ is not computable for complex network $f(\eta; ~\xi)$ and unknown input distribution $q_0$. 
Further insights can be obtained by noting that 
\begin{align}
\label{equ:tmp}
\Delta x_i 
& \approx \E_{x\sim q}[\nabla_x \log p(x)k(x,x_i) + \nabla_xk(x,x_i)] \notag \\
& =  \E_{x\sim q}[(\nabla_x \log p(x) - \nabla_x \log q(x))k(x,x_i)]  \\
& = \E_{x\sim q} [( \tilde \Delta x) k(x, x_i)],  \notag
%& = \E_{x\sim q} [k(x, x_i) \tilde \Delta x]  \notag
%\\&\approx  \sum_j k(x_j, x_i) \tilde \Delta x_i \notag
\end{align}
where \eqref{equ:tmp} is obtained by using Stein's identity \eqref{equ:steinid}. 
Therefore, %$\Delta x_i$ is approximately $\Delta x_i$ multiplied by a positive definite matrix $[k(x_i, x_j)]_{ij}$ and hence with positive inner product.% $\Delta x_i, \tilde \Delta x_i\la $. 
$\Delta x_i$ can be treated as a kernel smoothed version of $\tilde \Delta x_i $. 
%It is also possible to get $q_\eta(x)$-free (wild) variational inference by directly approximating $\nabla_x\log q_\eta(x)$ based on 




\section{Amortized MLE for Generative Adversarial Training}
Our method allows us to design efficient approximate sampling methods 
adaptively and automatically, and enables a host of novel applications. 
In this paper, we apply it in an amortized MLE method for training deep generative models.   
%(2) automatic hyper-parameter tuning for Bayesian inference. 
%including optimizing hyperparameters in traditional inference algorithms, 
%speeding up repeated inference tasks for online or streaming settings, or inner loops of the algorithms (such as maximum likelihood learning). 
%By leveraging state-of-the-art network architectures such as RNN and ResNet and fully optimizing the  
%or eventually replacing hand-designed infernece methods with more effiicent ones that trained on past tasks and improve adaptively over time. 

%\emph{Amortized maximum likelihood estimation (MLE) for Deep Generative Models} %Generative Adversarial Networks based
%\subsection{Amortized MLE for Generative Adversarial Training} %Generative Adversarial Networks based
 Maximum likelihood estimator (MLE) provides a fundamental approach for learning probabilistic models from data, 
but can be computationally prohibitive on distributions for which drawing samples or computing likelihood is intractable due to the normalization constant. 
Traditional methods such as MCMC-MLE use hand-designed methods (e.g., MCMC) to approximate the intractable likelihood function but do not work efficiently in practice. 
We propose to adaptively train a generative neural network to draw samples from the distribution during MLE training, which not only provides computational advantage, and also allows us to generate realistic-looking images competitive with, or better than the state-of-the-art generative adversarial networks (GAN) \citep{goodfellow2014generative, radford2015unsupervised} (see Figure~\ref{fig:mnist}-\ref{fig:facemore}).  
%See our empirical results in Figure~\ref{fig:face}.
% computationally intractable normalization constant. %(or its derivative). 

To be specific, denote by $\{x_{i,obs}\}$ a set of observed data. 
We consider the maximum likelihood training of energy-based models of form 
$$
p(x|\theta) = \exp(-\phi(x, \theta) - \Phi(\theta)), ~~~~~ \Phi(\theta) = \log \int \exp(-\phi(x,\theta))dx,
$$
where $\phi(x; ~\theta)$ is an energy function for $x$ indexed by parameter $\theta$ and $\Phi(\theta)$ is the log-normalization constant. %partition function. 
The log-likelihood function of $\theta$ is %based on maximizing the log likelihood function, 
$$
L(\theta)  =\frac{1}{n}\sum_{i=1}^n\log p(x_{i,obs} | \theta),
$$
whose gradient is %ascent update is 
\begin{align*}%
\nabla_\theta L(\theta) = -  \hat\E_{obs} [\partial_\theta \phi (x; \theta) ] +    \E_\theta [\partial_\theta \phi(x; \theta) ], 
%\nabla_\theta L(\theta) =  \la\nabla_\theta f(x; \theta) \ra_{Data} -   \la  \nabla_\theta f(x; \theta) \ra_{\theta}, 
\end{align*}
where $\hat\E_{obs}[\cdot]$ and $\E_\theta[\cdot]$ denote the empirical average on the observed data $\{x_{i,obs}\}$
and the expectation under model $p(x |\theta)$, respectively.  
The key computational difficulty is to approximate the model expectation $\E_{\theta}[\cdot]$. 
To address this problem, 
we use a generative neural network $x = f(\eta;~\xi)$ trained by Algorithm~\ref{alg:alg1} 
to approximately sample from $p(x|\theta)$, yielding a gradient update for $\theta$ of form
\begin{align}\label{equ:updatetheta}
\theta \gets \theta + \epsilon \hat\nabla_\theta L(\theta),  &\text{}& 
 \hat \nabla_\theta L(\theta) =  - \hat\E_{obs} [\partial_\theta \phi (x; \theta) ] +   \hat\E_{\eta} [\partial_\theta \phi(x; \theta) ], 
%\nabla_\theta L(\theta) =  \la\nabla_\theta f(x; \theta) \ra_{Data} -   \la  \nabla_\theta f(x; \theta) \ra_{\theta}, 
\end{align}
where $\hat \E_{\eta}$ denotes the empirical average on $\{x_i\}$ where $x_i = f(\eta;~\xi_i)$, 
$\{\xi_i\}\sim q_0$. 
As $\theta$ is updated by gradient ascent, 
$\eta$ is successively updated via Algorithm~\ref{alg:alg1} to \emph{follow} $p(x|\theta)$. 
See Algorithm \ref{alg:gan}. 

We call our method \emph{SteinGAN}, because it can be intuitively
%Our process can be 
interpreted as an adversarial game between 
the generative network $f(\eta;~\xi)$ and 
the energy model $p(x|\theta)$ which serves as a discriminator:
The MLE gradient update of $p(x|\theta)$ effectively decreases the energy of the training data and increases the energy of the simulated data from $f(\eta;~\xi)$, while the SVGD update of $f(\eta;~\xi)$ decreases the energy of the simulated data to fit better with $p(x|\theta)$.  
%Meanwhile, our procedure is still a principled maximum likelihood procedure and can be more stable than the original GAN \citep{goodfellow2014generative} that attends to find a Nash equilibrium. 
Compared with the traditional methods based on MCMC-MLE or contrastive divergence, we \emph{amortize the sampler as we train}, which gives much faster speed and simultaneously provides a high quality generative neural network that can generate realistic-looking images; see \citet{kim2016deep} for a similar idea and discussions. 
%We find that 
%Traditional approaches \citep{ngiam2011learning, xie2016theory} for training energy-based models are often based on variants of MCMC-MLE or contrastive divergence \citep{geyer1991markov, hinton2002training, tieleman2008training} can not generate realistic-looking images. 
%We will perform comprehensive tests on various applications, including image generation and semi-supervised learning. 

\begin{algorithm}[t]                      % enter the algorithm environment
\caption{Amortized MLE as Generative Adversarial Learning}% for Wild Variational Inference}          % give the algorithm a caption
\label{alg:gan}                           % and a label for \ref{} commands later in the document
\begin{algorithmic}                    % enter the algorithmic environment
\STATE {\bf Goal:} MLE training for energy model $p(x|\theta) = \exp(-\phi(x,\theta) - \Phi(\theta))$.
%the regularized likelihood $\sum_i \log \exp(-\phi(x, \theta)-\Phi(\theta)) + R(\theta)$. 
\STATE Initialize $\eta$ and $\theta$. 
\FOR {iteration $t$}
%\FOR {iteration $s$ (inner loop for updating $\eta$)}
\STATE {\bf Updating $\eta$:} Draw $\xi_i\sim q_0$, $x_i = f(\eta;~\xi_i)$; update $\eta$ using \eqref{equ:follow1}, \eqref{equ:follow2} or \eqref{equ:follow3} with $p(x)=p(x|\theta)$. Repeat several times when needed. 
%\ENDFOR
\STATE 
 {\bf Updating $\theta$:}  Draw a mini-batch of observed data $\{x_{i,obs}\}$, and simulated data $x_i = f(\eta;~\xi_i)$, update $\theta$ by \eqref{equ:updatetheta}. 
%\begin{align}\label{equ:updatetheta}
%\theta \gets \theta - \hat\E_{obs}[ \nabla_\theta \phi (x,\theta) ] + 
%\hat\E_{\eta} [\nabla_\theta \phi(x, \theta)]. % + \nabla R(\theta).  
%\end{align}
\ENDFOR
\end{algorithmic}
\end{algorithm}


\section{Experiments}

In this section, we first demonstrate the effectiveness of DU-Net through its comparison with the stacked U-Nets. Then we explore the relation between the prediction accuracy and $order$-$K$ connectivity. After that, we evaluate the iterative refinement to halve DU-Net parameters. Finally, we test the network quantization. Different combinations of bit-widths to find appropriate ones which balance accuracy, model size and memory consumption. The general comparisons are given at last. Some qualitative results are shown in Figure \ref{fig:pose-face-qualitive}.

{\bf Network.} The input resolution is normalized to 256$\times$256. Before the DU-Net, a Conv($7\times 7$) filter with stride 2 and a max pooling would produce 128 features with resolution 64$\times$64. Hence, the maximum resolution of DU-Net is 64$\times$64. Each block in DU-Net has a bottleneck structure as shown on the right side of Figure \ref{fig:framework}. At the beginning of each bottleneck, features from different connections are concatenated and stored in a shared memory. Then the concatenated features are compressed by the Conv($1\times 1$) to 128 features. At last, the Conv($3\times 3$) further produces 32 new features. The batch norm and ReLU are used before the convolutions. 

{\bf Training.} We implement the DU-Net using the PyTorch. The DU-Net is trained by the optimizer RMSprop. When training human pose estimators, the initial learning rate is $2.5\times 10^{-4}$ which is decayed to $5\times 10^{-5}$ after 100 epochs. The whole training takes 200 epochs. The facial landmark localizers are easier to train. Also starting from $2.5\times 10^{-4}$, its learning rate is divided by 5, 2 and 2 at epoch 30, 60 and 90 respectively. The above settings remain the same for quantized
DU-Net. In order to match the pace of dataflow, we set the same bit-width for gradients and inputs. We quantize dataflows and parameters all over the DU-Net except the first and last convolutional layers, since localization is a fine-grained task requires high precision of heatmaps.

{\bf Human Pose Datasets.} We use two benchmark human pose estimation datasets: MPII Human Pose \cite{andriluka14cvpr} and Leeds Sports Pose (LSP) \cite{johnson2010lsp}. The {\bf MPII} is collected from YouTube videos with a broad range of human activities. It has 25K images and 40K annotated persons, which are split into a training set of 29K and a test set of 11K. Following \cite{tompson2015efficient}, 3K samples are chosen from the training set as validation set. Each person has 16 labeled joints. The {\bf LSP} dataset contains images from many sport scenes. Its extended version has 11K training samples and 1K testing samples. Each person in LSP has 14 labeled joints. Since there are usually multiple people in one image, we crop around each person and resize it to 256x256. We also use scaling (0.75-1.25), rotation (-/+30) and random flip to augment the data.

{\bf Facial Landmark Datasets.} The experiments of the facial lanmark localization are conducted on the composite of HELEN, AFW, LFPW and IBUG which are re-annotated in the 300-W challenge \cite{sagonas2013300}. Each face has 68 landmarks. Following \cite{zhu2015face} and \cite{lv2017deep}, we use the training images of HELEN, LFPW and all images of AFW, totally 3148 images, as the training set. The testing is done on the common subset (testing images of HELEN and LFPW), challenge subset (all images from IBUG) and their union. We use the provided bounding boxes from the 300-W challenge to crop faces. The same augmentations of scaling and rotation as in human pose estimation are applied.

{\bf Metric.} We use the standard metrics in both human pose estimation and face alignment. Specifically, Percentage of Correct Keypoints (PCK) is used to evaluate approaches for human pose estimation. And the normalized mean error (NME) is employed to measure the performance of localizing facial landmarks. Following the convention of 300-W challenge, we use the inter-ocular distance to normalize mean error. For network quantization, we propose the balance index (BI) to examine the trade-off between performance and efficiency.

% \begin{table}[t!]
% \begin{center}
% \small
% % \setlength\tabcolsep{1.5pt}
% \caption{DU-Net {\it v.s.} stacked U-Nets on MPII validation set measured by PCKh and number of model parameters. With the same number of U-Nets, DU-Net achieves comparable performance as stacked U-Nets. But it has only about 30\% parameters of stacked U-Nets. The feature reuse across U-Nets make each U-Net become light-weighted.}\label{tb:hg-vs-du-nets}
% \begin{tabular}{|c|c|c|c|}
% \hline
% Method & PCKh & \# Parameters & \# Para. Ratio\\
% \hline
% \hline
% Stacked U-Nets (16) & - & 50.5M & 100\% \\
% DU-Net (16) & 89.9 & 15.9M & 31.5\% \\
% \hline
% \hline
% Stacked U-Nets (8) & 89.3 & 25.5M & 100\%\\
% DU-Net (8) & 89.5 & 7.9M & 31.0\% \\
% \hline
% \hline
% Stacked U-Nets (4) & 88.3 & 12.9M & 100\%\\
% DU-Net (4) & 88.2 & 3.9M & 30.2\% \\
% \hline
% \end{tabular}
% \end{center}
% % \vspace{-10pt}
% % \vspace{-2mm}
% \end{table}

\begin{table}[t]
\minipage[t]{0.49\textwidth}
\centering
%   \small
% \setlength\tabcolsep{1.5pt}
\caption{$Order$-1 DU-Net {\it v.s.} stacked U-Nets on MPII validation set measured by PCKh(\%) and parameter number. $Order$-1 DU-Net achieves comparable performance as stacked U-Nets. But it has only about 30\% parameters of stacked U-Nets. The feature reuse across U-Nets make each U-Net become light-weighted.}\label{tb:hg-vs-du-nets}
% \vspace*{2mm}
\begin{adjustbox}{width=1\textwidth}
\begin{tabular}{lccc}
\toprule
\multirow{2}{*}{Method} & PCKh & \#  & Parameter \\
& & Parameters & Ratio\\
\hline
% \hline
Stacked U-Nets(16) & - & 50.5M & 100\% \\
DU-Net(16) & 89.9 & 15.9M & 31.5\% \\
\hline
% \hline
Stacked U-Nets(8) & 89.3 & 25.5M & 100\%\\
DU-Net(8) & 89.5 & 7.9M & 31.0\% \\
\hline
% \hline
Stacked U-Nets(4) & 88.3 & 12.9M & 100\%\\
DU-Net(4) & 88.2 & 3.9M & 30.2\% \\
\bottomrule
\end{tabular}
\end{adjustbox}
\endminipage \hfill
\minipage[t]{0.49\textwidth}
\centering
% \small
\caption{NME(\%) on 300-W using $Order$-1 DU-Net(4) with iterative refinement, detection and regression supervisions. The top two and bottom three rows are non-iterative and iterative results. Iterative refinement could lower localization errors. Besides, the regression supervision outperforms the detection supervision.}
\label{tb:iter}
% \small
\begin{adjustbox}{width=1\textwidth}
\begin{tabular}{lcccc}
\toprule
\multirow{2}{*}{Method} & Easy  & Hard  & Full & \#\\
&Subset & Subset & Set & Para.\\
\hline
Detection only & 3.63 & 5.60 & 4.01 & 3.9M\\
Regression only &  2.91 & 5.12 & 3.34 & 3.9M\\
\hline
Detection Detection & 3.52 & 5.59 & 3.93 & 4.1M\\
Detection Regression & 2.95 & 5.12 & 3.37 & 4.1M\\
Regression Regression  & 2.87 & 4.97 & 3.28 & 4.1M\\
\bottomrule
\end{tabular} \hfill
\end{adjustbox}
\endminipage
\end{table}


% \begin{figure}[t!]
% \centering
%   \includegraphics[width=\linewidth]{figures/exp-order-k-cropped.pdf}
% \caption{Relation of PCKh, \# parameters and $order$-$K$ connectivity on MPII validation set. The parameter number of DU-Net grows approximately linearly with the order of connectivity. However, the PCKh first increases and then decreases. A small order 1 or 2 would be a good balance for prediction accuracy and parameter efficiency.}
% \label{fig:exp-order-k}
% \end{figure}

%\subsection{Comparison with Stacked Hourglasses}
\subsection{DU-Net {\it vs.} Stacked U-Nets}
To demonstrate the advantages of DU-Net, we first compare it with traditional stacked U-Nets. This experiment is done on the MPII validation set. All DU-Nets use the $order$-$1$ connectivity and intermediate supervisions. Table \ref{tb:hg-vs-du-nets} shows three pairs of comparisons with 4, 8 and 16 U-Nets. Both their PCKh and number of convolution parameters are reported. We could observe that, with the same number of U-Nets, DU-Net could obtain comparable or even better accuracy. More importantly, the number of parameters in DU-Net is decreased by about 70\% of that in stacked U-Nets. The feature reuse across U-Nets make each U-Net in DU-Net become light-weighted. Besides, the high parameter efficiency makes it possible to train 16 $order$-$1$ connected U-Nets in a 12G GPU with batch size 16. In contrast, training 16 stacked U-Nets is infeasible. Thus, $order$-$1$ together with intermediate supervisions could make DU-Net obtain accurate prediction as well as high parameter efficiency, compared with stacked U-Nets.
% This demonstrates that the parameter efficiency is significantly improved by the order-1 connectivity.
% \begin{figure}[hbt]This is because each U-Net in DU-Net has much fewer parameters than that in stacked hourglasses.
% \centering
%   \includegraphics[width=\linewidth]{figures/unets-vs-hgs-cropped.pdf}
% \caption{}
% \label{fig:dunets-vs-hgs}
% \end{figure}

% \begin{table}[t!]
% \begin{center}
% \caption{NME(\%) on 300-W using DU-Net(4) with iterative refinement, detection and regression supervisions. The top two and bottom three rows are non-iterative and iterative results. Iterative refinement could lower facial landmark localization errors. Besides, regression supervision outperforms detection supervision.}
% \label{tb:iter}
% \small
% % \setlength\tabcolsep{1.5pt}
% \begin{tabular}{lcccc}
% \toprule
% \multirow{2}{*}{Method} & Easy  & Hard  & Full & \#\\
% &Subset & Subset & Set & Para.\\
% \hline
% Detection only & 3.63 & 5.60 & 4.01 & 3.9M\\
% Regression only &  2.91 & 5.12 & 3.34 & 3.9M\\
% \hline
% Detection Detection & 3.52 & 5.59 & 3.93 & 4.1M\\
% Detection Regression & 2.95 & 5.12 & 3.37 & 4.1M\\
% Regression Regression  & 2.87 & 4.97 & 3.28 & 4.1M\\
% \bottomrule
% \end{tabular}
% \end{center}
% \vspace{-10pt}
% % \vspace{-2mm}
% \end{table}

%\subsection{Comparison of Variable Order connectivity}
\subsection{Evaluation of $Order$-$K$ connectivity}
The proposed $order$-$K$ connectivity is key to improve the parameter efficiency of DU-Net. In this experiment, we investigate how the PCKh and convolution parameter number change along with the order value. Figure \ref{fig:exp-order-k} gives the results from MPII validation set. The left and right figures show results of DU-Net with 8 and 16 U-Nets. It is clear that the convolution parameter number increases as the order becomes larger. However, the left and right PCKh curves have a similar shape of first increasing and then decreasing. $Order$-$1$ connectivity is always better than $order$-$0$. 

However, very dense connections may not be a good choice, which is kind of counter-intuitive. This is because the intermediate supervisions already provide additional gradients. Too dense connections make gradients accumulate too much, causing the overfitting of training set. Further evidence of overfitting is shown in Table \ref{tb:overfitting}. The $order$-7 connectivity has the higher training PCKh the $order$-1 in all training epochs. But its validation PCKh is a little lower in the last training epochs. Thus, small orders are recommended in DU-Net.

% \begin{table}[htb]
% \begin{center}
% \small
% \setlength\tabcolsep{1.5pt}
% \begin{tabular}{@{}lcccccccc@{}}
% % \begin{tabular}{@{}p{2.5cm} p{0.4cm} p{0.4cm}  p{0.4cm}  p{0.4cm} p{0.4cm} p{0.4cm} p{0.4cm} p{0.4cm}@{}}
% \toprule
% Order & Head & Sho. & Elb. & Wri. & Hip & Knee & Ank. & Mean\\
% \hline
% K=0 & 96.5 & 95.4 & 89.3 & 84.4 & 88.7 & 84.7 & 80.8 & 88.5\\
% K=1 & 96.9 & 95.9 & 90.3 & 85.7 & 89.3 & 86.0 & 82.2 & 89.5\\
% K=2 & 96.8 & 95.6 & 90.6 & 85.3 & 88.9 & 84.6 & 81.6 & 89.1\\
% K=4 & 96.7 & 95.6 & 90.1 & 85.6 & 89.1 & 85.4 & 82.2 & 89.2\\
% K=7 & 96.7 & 96.0 & 90.0 & 85.3 & 88.4 & 85.0 & 82.1 & 89.1\\
% \bottomrule
% \end{tabular}
% \end{center}
% \vspace{-10pt}
% \caption{}
% \label{tb:max7}
% % \vspace{-2mm}
% \end{table}

% \begin{table}[htb]
% \begin{center}
% \small
% \setlength\tabcolsep{1.5pt}
% \begin{tabular}{@{}lcccccccc@{}}
% % \begin{tabular}{@{}p{2.5cm} p{0.4cm} p{0.4cm}  p{0.4cm}  p{0.4cm} p{0.4cm} p{0.4cm} p{0.4cm} p{0.4cm}@{}}
% \toprule
% Order & Head & Sho. & Elb. & Wri. & Hip & Knee & Ank. & Mean\\
% \hline
% K=0 & 96.5 & 95.9 & 90.0 & 85.1 & 89.5 & 85.3 & 82.0 & 89.2\\
% K=1 & 96.8 & 96.1 & 90.7 & 86.1 & 89.6 & 87.0 & 82.9 & 89.9\\
% K=2 & 96.8 & 96.0 & 91.0 & 86.8 & 89.4 & 87.1 & 83.3 & 90.1\\
% K=4 & 96.7 & 96.1 & 90.9 & 86.7 & 89.5 & 86.2 & 82.7 & 89.8\\
% K=8 & 96.7 & 96.0 & 90.9 & 86.6 & 89.3 & 86.7 & 82.4 & 89.8\\
% K=15 & 96.4 & 96.1 & 90.6 & 86.3 & 89.0 & 86.1 & 83.0 & 89.6\\
% \bottomrule
% \end{tabular}
% \end{center}
% \vspace{-10pt}
% \caption{}
% \label{tb:max15}
% % \vspace{-2mm}
% \end{table}

\begin{figure}[t]
\minipage[t]{0.49\textwidth}
\centering
  \includegraphics[width=1\linewidth]{figures/exp-order-k-cropped.pdf}
%   \vspace*{-1mm}
  \caption{Relation of PCKh(\%), \# parameters and $order$-$K$ connectivity on MPII validation set. The parameter number of DU-Net grows approximately linearly with the order of connectivity. However, the PCKh first increases and then decreases. A small order 1 or 2 would be a good balance for prediction accuracy and parameter efficiency.}
\label{fig:exp-order-k}
\endminipage \hfill
\minipage[t]{0.49\textwidth}
\centering
  \includegraphics[width=0.9\linewidth]{figures/exp-naive-vs-efficient-cropped.pdf}
%   \vspace*{1mm}
  \caption{Naive implementation {\it v.s.} memory-efficient implementation. The $order$-$1$ connectivity, batch size 16 and a 12GB GPU are used. The naive implementation can only support 9 U-Nets at most. In contrast, the memory-efficient implementation allows to train 16 U-Nets, which nearly doubles the depth of DU-Net.}
  \label{fig:exp-naive-vs-efficient} \hfill
\endminipage
\end{figure}

% \begin{figure}[t!]
% \centering
%   \includegraphics[width=.9\linewidth]{figures/exp-naive-vs-efficient-cropped.pdf}
% \caption{Naive implementation {\it v.s.} memory-efficient implementation. The $order$-$1$ connectivity, batch size 16 and a 12GB GPU are used. The naive implementation can only support 9 U-Nets at most. In contrast, the memory-efficient implementation allows to train 16 U-Nets, which nearly doubles the depth of DU-Net.}
% % Because of the fixed order connectivity, the training memory of naive implementation also grows linearly, but with an obviously larger slope, compared with that of memory-efficient implementation.
% \label{fig:exp-naive-vs-efficient}
% \end{figure}

\begin{table}[t]
\minipage[t]{0.49\textwidth}
\centering
%   \small
\setlength\tabcolsep{4pt}
\caption{$Order$-1 DU-Net(8) {\it v.s.} $order$-7 DU-Net(8), measured by training and validation PCKhs(\%) on MPII. $Order$-7 DU-Net(8) overfits the training set a little bit. Its validation PCKh is lower at last, though it always has higher training PCKh.}\label{tb:overfitting}
% \vspace*{2mm}
\begin{adjustbox}{width=1\textwidth}
\begin{tabular}{l|cccc}
\toprule
\multicolumn{5}{c}{PCKh on training set}\\
\hline
Epoch & 1 & 50 & 100 & 150 \\
\hline
$Order$-1 DU-Net(8) & 20.3 & 83.2 & 87.7 & 91.7 \\
$Order$-7 DU-Net(8) & {\bf 25.2} & {\bf 84.7} & {\bf 89.3} & {\bf 93.1} \\
\hline
\multicolumn{5}{c}{PCKh on validation set}\\
\hline
Epoch & 1 & 50 & 100 & 150 \\
\hline
$Order$-1 DU-Net(8) & 29.4 & 82.8 & {\bf 85.7} & {\bf 87.1}\\
$Order$-7 DU-Net(8) & {\bf 36.6} & {\bf 84.0} & 85.1 & 86.7\\
\bottomrule
\end{tabular}
\end{adjustbox}
\endminipage \hfill
\minipage[t]{0.49\textwidth}
\centering
% \small
\caption{Iterative $order$-1 DU-Net(4) {\it v.s.} non-iterative $order$-1 DU-Net(8) on 300-W measured by NME(\%). Iterative DU-Net(4), with few additional parameters on DU-Net(4), achieves comparable performance as DU-Net(8). Thus, the iterative refinement has the potential to halve parameters of DU-Net but still maintain comparable performance.}
% \small
\begin{adjustbox}{width=1\textwidth}
\label{tb:iter4-vs-8}
\begin{tabular}{lcccc}
\toprule
\multirow{2}{*}{Method} & Easy  & Hard  & Full & \#\\
&Subset & Subset & Set & Parameters\\
\hline
DU-Net(4) &  2.91 & 5.12 & 3.34 & 3.9M\\
Iter. DU-Net(4)  & 2.87 & 4.97 & 3.28 & 4.1M\\
DU-Net(8) & 2.82 & 5.07 & 3.26 & 7.9M\\
\bottomrule
\end{tabular} \hfill
\end{adjustbox}
\endminipage
\end{table}

\subsection{Evaluation of Efficient Implementation}

The memory-efficient implementation makes it possible to train very deep DU-Net. Figure \ref{fig:exp-naive-vs-efficient} shows the training memory consumption of both naive and memory-efficient implementations of DU-Net with order-1 connectivity. The linear growths of training memory along with number of U-Nets is because of the fixed order connectivity. But the memory growth of efficient implementation is much slower than that of the naive one. With batch size 16, we could train a DU-Net with 16 U-Nets in 12GB GPU. Under the same setting, the naive implementation could accept only 9 U-Nets.


%\subsection{Exploration of Iterative Estimation}
\subsection{Evaluation of Iterative Refinement}
% The iterative refinement is proposed to improve the landmark localization accuracy with few additional parameters.
The iterative refinement is designed to make DU-Net more parameter efficient. First, experiments are done on the 300-W dataset using DU-Net(4). Results are shown in Table \ref{tb:iter}. For both detection and regression supervisions, adding an iteration could lower the localization errors, demonstrating effectiveness of the iterative refinement. Meanwhile, the model parameters only increase 0.2M, making DU-Net even more parameter efficient. Besides, the regression supervision outperforms the detection one no matter in the iterative or non-iterative setting, making it a better choice for landmark localization. 
% Thus, regression supervision is more suitable for landmark localization tasks. 

Further, we compare iterative DU-Net(4) with non-iterative DU-Net(8). Table \ref{tb:iter4-vs-8} gives the comparison. We could find that, the iterative DU-Net(4) could obtain comparable NME as DU-Net(8). However, DU-Net(8) has double parameters of  DU-Net(4) whereas iterative DU-Net(4) increases only 0.2M additional parameters on DU-Net(4).


\subsection{Evaluation of Network Quantization}

Through network quantization, high precision operations and parameters can be efficiently represented by a few discrete values. 
In order to find appropriate choices of bit-widths, we try a series of bit-width combinations on the 300-W dataset based on $order$-$1$ DU-Net(4). The performance and balance ability of these combinations on several methods are shown in Table \ref{tb:IWG-QUAN}, where DU-Net(4) is DU-Net with 4 blocks, BW and TW respectively represents binarized weight and ternarized weight without $\alpha$, BW-$\alpha$ is binarized weight with float scaling factor $\alpha$, the suffix QIG means quantized inputs and gradients. 

For mobile devices with limited computational resources, slightly performance drop is tolerable provided that corresponding large efficiency enhancement. For the evaluation purpose, we propose a balance index (BI) to better examine the trade-off between performance and efficiency: 
\begin{equation} \label{eq:bi}
BI = NME^2 \cdot TM \cdot MS  
\end{equation} 
where $TM$ and $MS$ is respectively short for training memory and model size compression ratios to the original network without quantization. The square of $NME$ is calculated in the above formula to emphasize the prior importance of performance. For BI, the smaller the value, the better the ability of balance.

According to Table \ref{tb:IWG-QUAN}, BW-QIG(818) could achieve the best balance between performance and model efficiency among all the combinations. BW-QIG(818) could reduce more than 4$\times$ training memory and 32$\times$ model size while reach a better performance than TSR \cite{lv2017deep}. Besides, BW-$\alpha$-QIG(818),  BW-QIG(616) and TW-QIG(626) also have small balance index. Among all the combinations, the binarized network with scaling factor $\alpha$, i.e. BW-$\alpha$ gets the closest error to the original network DU-Net(4).


For BW-$\alpha$-QIG(818), the performance is not better than BW-QIG(818). This is mainly because that BW-$\alpha$ is heavily rely on the parameter $\alpha$. However, the quantization of dataflow could reduce the approximation ability of $\alpha$. TW and TW-QIG usually gets better results than BW and BW-QIG, since they have more choices in terms of weight value. The above results proves the effectiveness of network quantization, yet a correct combination of bit-widths is a crucial factor.


\begin{table}[t!]
\begin{center}
\caption{Performance and balance ability of different combinations of bit-width values on the 300-W dataset measured by NME(\%), all quantized networks are based on $order$-$1$ DU-Net(4). BW and TW is short for binarized and ternarized weight, $\alpha$ represents float scaling factor, QIG is short for quantized inputs and gradients. $Bit_I$, $Bit_W$, $Bit_G$ represents the bit-width of inputs, weights, gradients respectively. Training memory and model size is represented by the compression ratio to the original DU-Net(4). Balance index is calculated by equation \ref{eq:bi}. Comparable error rate could be achieved by binarized the model parameters. Further quantizing the inputs and gradients could substantially reduce the training memory with some increase of detection error. The balance index is a indicator for balancing the quantization and accuracy.}
%The binarized network with scaling factor $\alpha$ gets the closest error to the original network. Taking the quantization of inputs and gradients into consideration, binarized network trained with 8-bits quantized inputs and gradients has the lowest balance index.

\small
% \setlength\tabcolsep{1.5pt}
% \label{tb:IWG-QUAN}
% \begin{tabular}{lccccccc}
% \toprule
% % Method & $Bit_I$  & $Bit_W$  & $Bit_G$ & NME(\%)  \\
% \multirow{2}{*}{Method} & {$Bit_I$}  & $Bit_W$  & $Bit_G$ & NME & Training  &  Model & Balance\\
% & & & & (\%) & Memory & Size & Index\\

% % & Training Memory & 
% \hline
% DU-Net(4)  &  32  & 32  & 32  & 3.38 & 1.00 & 1.00 & 11.4 \\
% \hline
% BW-QIG     &  6   & 1   & 6   & 5.93 & 0.17 & 0.03 & 0.18	\\
% BW-QIG     &  8   & 1   & 8   & 4.30 & 0.25 & 0.03 & \bf{0.14}	    \\
% BW-$\alpha$-QIG    &  8   & 1   & 8   & 4.47 & 0.25 & 0.03 & 0.15  \\
% BW         &  32  & 1   & 32  & 3.75 & 1.00 & 0.03 & 0.42	\\
% BW-$\alpha$        &  32  & 1   & 32  & {\bf 3.58} & 1.00 & 0.03 & 0.38  \\
% TW         &  32  & 2   & 32  & 3.73 & 1.00 & 0.06 & 0.83  \\
% TW-QIG     &  6   & 2   & 6   & 4.27 & 0.17 & 0.06 & 0.19	\\
% TW-QIG     &  8   & 2   & 8   & 4.13 & 0.25 & 0.06 & 0.26   \\


% \bottomrule
% \end{tabular}

\label{tb:IWG-QUAN}
\begin{tabular}{lccccccccc}
\toprule
% Method & $Bit_I$  & $Bit_W$  & $Bit_G$ & NME(\%)  \\
\multirow{2}{*}{Method} & {$Bit_I$}  & $Bit_W$  & $Bit_G$ & NME(\%) & NME(\%) & NME(\%) & Training  &  Model & Balance\\
& & & & Full set & Easy set & Hard set & Memory & Size & Index\\

% & Training Memory & 
\hline
DU-Net(4)  &  32  & 32  & 32  & 3.38 & 2.95 & 5.13 & 1.00 & 1.00 & 11.4 \\
\hline
BW-QIG     &  6   & 1   & 6   & 5.93 & 5.10 & 9.34 & 0.17 & 0.03 & 0.18	\\
BW-QIG     &  8   & 1   & 8   & 4.30 & 3.67 & 6.86 & 0.25 & 0.03 & \bf{0.14}	    \\
BW-$\alpha$-QIG    &  8   & 1   & 8   & 4.47 & 3.75 & 7.40 & 0.25 & 0.03 & 0.15  \\
BW         &  32  & 1   & 32  & 3.75 & 3.20 & 5.99 & 1.00 & 0.03 & 0.42	\\
BW-$\alpha$        &  32  & 1   & 32  & {\bf 3.58} & 3.12 & 5.45 & 1.00 & 0.03 & 0.38  \\
TW         &  32  & 2   & 32  & 3.73 & 3.21 & 5.85 & 1.00 & 0.06 & 0.83  \\
TW-QIG     &  6   & 2   & 6   & 4.27 & 3.70 & 6.59 & 0.17 & 0.06 & 0.19	\\
TW-QIG     &  8   & 2   & 8   & 4.13 & 3.55 & 6.50 & 0.25 & 0.06 & 0.26   \\
\bottomrule
\end{tabular}

\end{center}
% \vspace{-10pt}
% \vspace{-2mm}
\end{table}

\begin{table}[htb]
\begin{center}
\small
\caption{Comparison of convolution parameter number (Million) and model size (Megabyte) with state-of-the-art methods. DU-Net(16) has 27\%-62\% parameters of other methods. Its binarized version DU-Net-BW-$\alpha$(16) has less than {\bf 2\%} model size.}\label{tb:para-num}
\setlength\tabcolsep{0.5pt}
\begin{tabular}{lccccc|cc}
\toprule
\multirow{2}{*}{Method} & Yang  & Wei & Bulat  & Chu & Newell  & $Order$-1 & $Order$-1 DU-\\
& {\it et al.}\cite{yang2017learning} & {\it et al.}\cite{wei2016convolutional}  
& {\it et al.}\cite{bulat2016human} & {\it et al.}\cite{chu2017multi} & {\it et al.}\cite{newell2016stacked} & DU-Net(16) & Net-BW-$\alpha$(16)\\
\hline
\# Parameters & 28.0M & 29.7M & 58.1M & 58.1M & 25.5M & {\bf 15.9M} & {\bf 15.9M}\\
Model Size & 110.2MB & 116.9MB & 228.7MB & 228.7MB & 100.5MB & 62.6MB & {\bf 2.0MB}\\

% \hline
% \cite{yang2017learning} & \\
% \cite{wei2016convolutional} & \\
% \cite{bulat2016human} & \\
% \cite{chu2017multi} & \\
% % Carreira \textit{et al.}\cite{carreira2016human} & 10.0M\\
% % \cite{belagiannis2017recurrent} & 15.4M\\
% \hline
% Stacked HGs(8) \cite{newell2016stacked} & \\
% DU-Net(8) & \\
% DU-Net(16) & \\
\bottomrule
\end{tabular}
\end{center}
% \vspace{-10pt}
% \vspace{-2mm}
\end{table}

\subsection{Comparison with State-of-the-art Methods}

% We compare the DU-Net with state-of-the-art approaches for both human pose estimation and facial landmark localization.

{\bf Human Pose Estimation.}
Tables \ref{tb:mpii} and \ref{tb:lsp} show comparisons of human pose estimation on MPII and LSP test sets. The $order$-$1$  DU-Net-BW-$\alpha$(16) achieves comparable state-of-the-art performances. In contrast, as shown in Table \ref{tb:para-num}, it has only 27\%-62\% parameters and less than 2\% model size of other recent state-of-the-art methods. The DU-Net is concise and simple. Other state-of-the-art methods use stacked U-Nets with either sophisticated modules \cite{yang2017learning}, graphical models \cite{chu2017multi} or adversarial networks \cite{yu2017adversarial}.

{\bf Facial Landmark Localization.}
The DU-Net is also compared with other state-of-the-art facial landmark localization methods on 300-W. Please refer to Table \ref{tb:300w}. We uses a smaller network $order$-1 DU-Net(8) than that in human pose estimation, since localizing the facial landmarks is easier. The $order$-1 DU-Net-BW-$\alpha$(8) gets comparable errors state-of-the-art method \cite{newell2016stacked}. However, $order$-1 DU-Net-BW-$\alpha$(8) has only $\sim$2\% model size.

\begin{figure*}[t!]
\centering
  \includegraphics[width=0.9\linewidth]{figures/pose-face-qualitive-results-cropped.pdf}
\caption{Qualitative results of human pose estimation and facial landmark localization. DU-Net could handle a wide range of human poses, even with occlusions. It could also detect accurate facial landmarks with various head poses and expressions.}
\label{fig:pose-face-qualitive}
\end{figure*}

\begin{table}[t!]
\begin{center}
\small
\setlength\tabcolsep{1.5pt}
\caption{PCKh(\%) comparison on MPII test sets. $Order$-$1$ DU-Net could achieve comparable performance as state-of-the-art methods. More importantly, DU-Net-BW-$\alpha$(16) has at least $\sim${\bf 30\%} parameters and at most $\sim${\bf 2\%} model size.}\label{tb:mpii}
\begin{tabular}{@{}lcccccccc@{}}
\toprule
Method & Head & Sho. & Elb. & Wri. & Hip & Knee & Ank. & Mean\\
\hline
Pishchulin \textit{et al.} ICCV'13 \cite{pishchulin2013strong} & 74.3 & 49.0 & 40.8 & 34.1 & 36.5 & 34.4 & 35.2 & 44.1\\
Tompson \textit{et al. } NIPS'14 \cite{tompson2014joint} & 95.8 & 90.3 & 80.5 & 74.3 & 77.6 & 69.7 & 62.8 & 79.6\\
Carreira \textit{et al.} CVPR'16 \cite{carreira2016human} & 95.7 & 91.7 & 81.7 & 72.4 & 82.8 & 73.2 & 66.4 & 81.3\\
Tompson \textit{et al.} CVPR'15 \cite{tompson2015efficient}& 96.1 & 91.9 & 83.9 & 77.8 & 80.9 & 72.3 & 64.8 & 82.0\\
Hu \textit{et al.} CVPR'16 \cite{hu2016bottom}& 95.0 & 91.6 & 83.0 & 76.6 & 81.9 & 74.5 & 69.5 & 82.4\\
Pishchulin \textit{et al.} CVPR'16 \cite{pishchulin2016deepcut}&94.1 & 90.2 & 83.4 & 77.3 & 82.6 & 75.7 & 68.6 & 82.4\\
Lifshitz \textit{et al.} ECCV'16 \cite{lifshitz2016human} & 97.8 & 93.3 & 85.7 & 80.4 & 85.3 & 76.6 & 70.2 & 85.0\\
Gkioxary \textit{et al.} ECCV'16 \cite{gkioxari2016chained} & 96.2 & 93.1 & 86.7 & 82.1 & 85.2 & 81.4 & 74.1 & 86.1\\
Rafi \textit{et al.} BMVC'16 \cite{rafi2016efficient} & 97.2 & 93.9 & 86.4 & 81.3 & 86.8 & 80.6 & 73.4 & 86.3\\
Belagiannis \textit{et al.} FG'17 \cite{belagiannis2017recurrent}&97.7 & 95.0 & 88.2 & 83.0 & 87.9 & 82.6 & 78.4 & 88.1\\
Insafutdinov \textit{et al.} ECCV'16 \cite{insafutdinov2016deepercut}&96.8 & 95.2 & 89.3 & 84.4 & 88.4 & 83.4 & 78.0 & 88.5\\
Wei \textit{et al.} CVPR'16 \cite{wei2016convolutional} & 97.8 & 95.0 & 88.7 & 84.0 & 88.4 & 82.8 & 79.4 & 88.5\\
Bulat \textit{et al.} ECCV'16 \cite{bulat2016human} & 97.9 & 95.1 & 89.9 & 85.3 & 89.4 & 85.7 & 81.7 & 89.7\\
Newell {\it et al.}  ECCV'16 \cite{newell2016stacked} & 98.2 & 96.3 & 91.2 & 87.1 & 90.1 & 87.4 & 83.6 & 90.9\\
Chu \textit{et al.} CVPR'17 \cite{chu2017multi} & {\bf 98.5} & 96.3 & 91.9 & {\bf 88.1} & {\bf 90.6} & {\bf 88.0} & {\bf 85.0} & {\bf 91.5}\\
\hline
$Order$-$1$ DU-Net(16) & 97.4  & {\bf 96.4}  & {\bf 92.1}  & 87.7  & 90.2  & 87.7 & 84.3 & 91.2\\
$Order$-$1$ DU-Net-BW-$\alpha$(16) & 97.6  & 96.4  & 91.7  & 87.3  & 90.4  & 87.3 & 83.8 & 91.0\\
\bottomrule
\end{tabular}
\end{center}
% \vspace{-10pt}
% \vspace{-2mm}
\end{table}

\begin{table}[t]
\begin{center}
\caption{NME(\%) comparison with state-of-the-art facial landmark localization methods on 300-W dataset. DU-Net-BW-$\alpha$ refers to the DU-Net with binarized weights and scaling factor $\alpha$ The binarized DU-Net obtains comparable performance with state-of-the-art method \cite{newell2016stacked}. But it has $\sim${\bf 50}$\times$ smaller model size.}\label{tb:300w}
\small
\setlength\tabcolsep{0.5pt}
\begin{tabular}{lccccccc|ccc}
\toprule
\multirow{2}{*}{Method} & CFAN  & Deep  & CFSS  
& TCDCN  & MDM  & TSR & HGs(4) & $Order$-1 & $Order$-1 DU-\\
& \cite{zhang2014coarse} & Reg \cite{shi2014deep} & \cite{zhu2015face} & \cite{zhang2014facial} &  \cite{trigeorgis2016mnemonic} & \cite{lv2017deep} & \cite{newell2016stacked} & DU-Net(8)& Net(8)-BW-$\alpha$\\
\hline
Easy subset  & 5.50 & 4.51  &  4.73 & 4.80 & 4.83  & 4.36 & 2.90 & {\bf 2.82} & 3.00\\ 
Hard subset  & 16.78 &  13.80  & 9.98 & 8.60 & 10.14 &  7.56 & 5.15 &{\bf 5.07} & 5.36\\
Full set   & 7.69 & 6.31 & 5.76 & 5.54 & 5.88 & 4.99 & 3.35 & {\bf 3.26} & 3.46\\
% \hline
% Stacked HGs(4) \cite{} & 2.85 & 4.88 & 3.24\\
% Stacked HGs(4) \cite{newell2016stacked} &   &  & \\
% +$CoCoNet$ & {\bf 2.90} & {\bf 5.15} & {\bf 3.37}\\
\bottomrule
\end{tabular}
\end{center}
% \vspace{-10pt}
% \vspace{-2mm}
\end{table}

\begin{table}[htb]
\begin{center}
\small
\caption{PCK(\%) comparison on LSP test set. The $Order$-$1$ DU-Net could also obtain comparable state-of-the-art performance. But DU-Net-BW-$\alpha$(16) has at most $\sim${\bf 70}\% fewer parameters and $\sim${\bf 50}$\times$ smaller model size than other state-of-the-art methods.}\label{tb:lsp}
\setlength\tabcolsep{1.5pt}
\begin{tabular}{@{}lcccccccc@{}}
\toprule
Method & Head & Sho. & Elb. & Wri. & Hip & Knee & Ank. & Mean\\
\hline
Belagiannis \textit{et al.} FG'17 \cite{belagiannis2017recurrent} & 95.2 & 89.0 & 81.5 & 77.0 & 83.7 & 87.0 & 82.8 & 85.2\\
Lifshitz \textit{et al.} ECCV'16 \cite{lifshitz2016human} & 96.8 & 89.0 & 82.7 & 79.1 & 90.9 & 86.0 & 82.5 & 86.7\\
Pishchulin \textit{et al.} CVPR'16 \cite{pishchulin2016deepcut} &  97.0 & 91.0 & 83.8 & 78.1 & 91.0 & 86.7 & 82.0 & 87.1\\
Insafutdinov \textit{et al.} ECCV'16 \cite{insafutdinov2016deepercut}& 97.4 & 92.7 & 87.5 & 84.4 & 91.5 & 89.9 & 87.2 & 90.1\\
Wei \textit{et al.} CVPR'16 \cite{wei2016convolutional}& 97.8 & 92.5 & 87.0 & 83.9 & 91.5 & 90.8 & 89.9 & 90.5\\
Bulat \textit{et al.} ECCV'16 \cite{bulat2016human}& 97.2 & 92.1 & 88.1 & 85.2 & 92.2 & 91.4 & 88.7 & 90.7\\
Chu \textit{et al.} CVPR'17 \cite{chu2017multi}& 98.1 & 93.7 & 89.3 & 86.9 &  93.4 & 94.0 & 92.5 & 92.6\\
% Chou \textit{et al.}\cite{chou2017self} & 98.2 & 94.9 & 92.2 & 89.5 & 94.2 & 95.0 & 94.1 & 94.0\\
Newell {\it et al.} ECCV'16 \cite{newell2016stacked} & {\bf 98.2} & 94.0 & 91.2 & 87.2 & 93.5 & {\bf 94.5} & 92.6 & 93.0\\
Yang \textit{et al.} ICCV'17 \cite{yang2017learning} & {\bf 98.3} & 94.5 & 92.2 & 88.9 & {\bf 94.4} & 95.0 & 93.7 & 93.9\\
% Chou \textit{et al.}\cite{chou2017self} & 98.2 & 94.9 & 92.2 & 89.5 & 94.2 & 95.0 & 94.1 & 94.0\\
\hline
% $Order$-$1$ DU-Net(8) & 97.1 &  94.7 &  91.6 & 89.0  & 93.7  & 94.2 &  93.7  & 93.4\\
$Order$-$1$ DU-Net(16) &  97.5 & {\bf 95.0} & {\bf 92.5} & {\bf 90.1} &  93.7 &  {\bf 95.2} & 94.2 & {\bf 94.0}\\
$Order$-$1$ DU-Net-BW-$\alpha$(16) &  97.8 & 94.3 & 91.8 & 89.3 &  93.1 &  94.9 & {\bf 94.4} & 93.6\\

\bottomrule
\end{tabular}
\end{center}
% \vspace{-10pt}
% \vspace{-2mm}
\end{table}





\section{Conclusion}
We propose a new method to train neural samplers for given distributions, together with a new SteinGAN method for generative adversarial training. 
Future directions involve more applications and theoretical understandings for training neural samplers. 

\newpage\clearpage
\bibliographystyle{iclr2016_conference}
{\small
\bibliography{bibrkhs_stein}
}

%\newpage
\appendix
\begin{figure}[h]
\centering
\includegraphics[width=0.9\textwidth]{\dilinfig/faces/vgd_gan-20.pdf}  
\caption{More images generated by SteinGAN on CelebA.}
\label{fig:facemore}
\end{figure}


\end{document}

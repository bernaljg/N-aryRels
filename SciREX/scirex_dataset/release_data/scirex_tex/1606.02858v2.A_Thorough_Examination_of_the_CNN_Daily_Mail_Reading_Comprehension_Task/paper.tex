%
% File ACL2016.tex
%

\documentclass[11pt]{article}

\usepackage{microtype}
\usepackage{acl2016}
\usepackage{url}

% \makeatletter
% \def\url@itstyle{%
%  \@ifundefined{selectfont}{\def\UrlFont{\it}}{\def\UrlFont{\itshape}}}
% \makeatother
% \urlstyle{it}

\makeatletter
\newcommand{\@BIBLABEL}{\@emptybiblabel}
\newcommand{\@emptybiblabel}[1]{}
\makeatother
\interfootnotelinepenalty=10000

\usepackage[hidelinks]{hyperref}

% \usepackage{newtxtext,newtxmath}
\usepackage{times}
\usepackage{color}
\usepackage{enumerate}
\usepackage{amsmath}
\usepackage{amssymb}
\usepackage{amsfonts}
\usepackage{multirow}
\usepackage{latexsym}
\usepackage{array}
\usepackage{tikz}
\usepackage{booktabs}



\aclfinalcopy % Uncomment this line for the final submission
% \def\aclpaperid{***} %  Enter the acl Paper ID here

\newcommand\BibTeX{B{\sc ib}\TeX}

%\title{Towards A Better Understanding on Reading Comprehension}
\title{A Thorough Examination of the \\ CNN\slash Daily Mail Reading Comprehension Task}

\author{Danqi Chen \and Jason Bolton \and Christopher D. Manning\\
            Computer Science
	    Stanford University\\
	    Stanford, CA 94305-9020, USA\\
	    {\tt \{danqi,jebolton,manning\}@cs.stanford.edu}}

\DeclareMathOperator*{\argmin}{arg\,min}
\DeclareMathOperator*{\argmax}{arg\,max}
\DeclareMathOperator*{\sigm}{sigm}
\DeclareMathOperator*{\softmax}{softmax}
\newcommand{\notate}[1]{\textcolor{red}{\textbf{[#1]}}}
\newcommand{\comment}[1]{\textcolor{gray}{\textbf{[#1]}}}
\newcommand{\mf}[1]{\mathbf{#1}}
\newcommand{\tf}[1]{\textbf{#1}}
\newcommand{\ti}[1]{\textit{#1}}
\newcommand{\ttt}[1]{\texttt{#1}}
\newcommand\R{\mathbb{R}}
\newcommand{\specialcell}[2][c]
\newcommand{\finaldm}{76.6\%}

\newcolumntype{C}[1]{>{\centering\let\newline\\\arraybackslash\hspace{0pt}}m{#1}}
\newcommand{\figref}[1]{Figure~\ref{fig:#1}}
\hyphenation{At-ten-tive-Reader}
\hyphenation{Lamb-da-MART}

\renewcommand\floatpagefraction{.9}
\renewcommand\dblfloatpagefraction{.9}
\renewcommand\textfraction{.05}


\date{}

\def\allfiles{}

\begin{document}
\def\tinyweeny{\fontsize{6pt}{8pt}\selectfont}

\maketitle

\begin{abstract}

Enabling a computer to understand a document so that it can answer comprehension questions is a central, yet unsolved goal of NLP\@. A key factor impeding its solution by machine learned systems is the limited availability of human-annotated data. \newcite{hermann2015teaching} seek to solve this problem by creating over a million training examples by pairing \ti{CNN} and \ti{Daily Mail} news articles with their summarized bullet points, and show that a neural network can then be trained to give good performance on this task. In this paper, we conduct a thorough examination of this new reading comprehension task. Our primary aim is to understand what depth of language understanding is required to do well on this task. We approach this from one side by doing a careful hand-analysis of a small subset of the problems and from the other by showing that simple, carefully designed systems can obtain accuracies of {\finalcnn} and {\finaldm} on these two datasets, exceeding current state-of-the-art results by {7}--{10}\%  and approaching what we believe is the ceiling for performance on this task.\footnote{Our code is available at \url{https://github.com/danqi/rc-cnn-dailymail}.}
\end{abstract}

\section{Introduction}
\label{sec:intro}

Language modeling is among the important problems that require modeling long-term dependency, with successful applications such as unsupervised pretraining~\citep{dai2015semi,peters2018deep,radford2018improving,devlin2018bert}.
However, it has been a challenge to equip neural networks with the capability to model long-term dependency in sequential data.
Recurrent neural networks (RNNs), in particular Long Short-Term Memory (LSTM) networks~\citep{hochreiter1997long}, have been a standard solution to language modeling and obtained strong results on multiple benchmarks.
Despite the wide adaption, RNNs are difficult to optimize due to gradient vanishing and explosion~\citep{hochreiter2001gradient}, and the introduction of gating in LSTMs and the gradient clipping technique~\citep{graves2013generating} might not be sufficient to fully address this issue.
% ,pascanu2012understanding
Empirically, previous work has found that LSTM language models use 200 context words on average~\citep{khandelwal2018sharp}, indicating room for further improvement.

On the other hand, the direct connections between long-distance word pairs baked in attention mechanisms might ease optimization and enable the learning of long-term dependency~\citep{bahdanau2014neural,vaswani2017attention}.
Recently, \citet{al2018character} designed a set of auxiliary losses to train deep Transformer networks for character-level language modeling, which outperform LSTMs by a large margin.
Despite the success, the LM training in~\citet{al2018character} is performed on separated fixed-length segments of a few hundred characters, without any information flow across segments.
As a consequence of the fixed context length, the model cannot capture any longer-term dependency beyond the predefined context length.
In addition, the fixed-length segments are created by selecting a consecutive chunk of symbols without respecting the sentence or any other semantic boundary.
Hence, the model lacks necessary contextual information needed to well predict the first few symbols, leading to inefficient optimization and inferior performance.
We refer to this problem as \textit{context fragmentation}.

%However, the context length is fixed to hundreds of characters and thus it is not possible to model longer-term dependency. Moreover, it is not clear how the model performs on word-level language modeling data, as the granularity changes.

% Moreover, using auxiliary losses brings additional challenges such as properly tuning the mixture weights and the loss decay schedule.

To address the aforementioned limitations of fixed-length contexts, we propose a new architecture called Transformer-XL (meaning extra long).
We introduce the notion of recurrence into our deep self-attention network. In particular, instead of computing the hidden states from scratch for each new segment, we reuse the hidden states obtained in previous segments.
The reused hidden states serve as memory for the current segment, which builds up a recurrent connection between the segments.
As a result, modeling very long-term dependency becomes possible because information can be propagated through the recurrent connections.
Meanwhile, passing information from the previous segment can also resolve the problem of context fragmentation.
More importantly, we show the necessity of using relative positional encodings rather than absolute ones, in order to enable state reuse without causing temporal confusion.
Hence, as an additional technical contribution, we introduce a simple but more effective relative positional encoding formulation that generalizes to attention lengths longer than the one observed during training.

Transformer-XL obtained strong results on five datasets, varying from word-level to character-level language modeling.
Transformer-XL is also able to generate relatively coherent long text articles with \textit{thousands of} tokens (see Appendix \ref{sec:gen}), trained on only 100M tokens.
% Transformer-XL improves the previous state-of-the-art (SoTA) results from 1.06 to 0.99 in bpc on enwiki8, from 1.13 to 1.08 in bpc on text8, from 20.5 to 18.3 in perplexity on WikiText-103, and from 23.7 to 21.8 in perplexity on One Billion Word.
% Transformer-XL improves the previous state-of-the-art (SoTA) results to 0.99 in bpc on enwiki8, 1.08 in bpc on text8, 18.3 in perplexity on WikiText-103, and 21.8 in perplexity on One Billion Word.
% On small data, Transformer-XL also achieves a perplexity of 54.5 on Penn Treebank without finetuning, which is SoTA when comparable settings are considered.

Our main technical contributions include introducing the notion of recurrence in a purely self-attentive model and deriving a novel positional encoding scheme. These two techniques form a complete set of solutions, as any one of them alone does not address the issue of fixed-length contexts. Transformer-XL is the first self-attention model that achieves substantially better results than RNNs on both character-level and word-level language modeling.

% On WikiText-103, Transformer-XL improves the previous state-of-the-art (SoTA) results from 33 perplexity to 24, with a relative reduction of 27\%. On enwiki8 character-level language modeling, Transformer-XL achieves a SoTA bpc of 1.03, which outperforms \cite{al2018character} by 0.03 with 60+\% fewer parameters. Given a more common model size with 40+M parameters, Transformer-XL achieves a bpc of 1.06, compared to 1.11 by \cite{al2018character}. Transformer-XL also achieves perplexities of 54.5 on Penn Treebank and 29.4 on One Billion Word, which are SoTA when comparable settings are considered.

% Due to the ability of modeling long-range context, our best model uses attention lengths of 1,600 and 3,800 on WikiText-103 and enwiki8 respectively. We also devise a metric called \textit{Relative Effective Context Length} (RECL) that aims to fairly compare the ability of long-range dependency modeling.
% % perform a fair comparison of the gains brought by increasing the context lengths for different models.
% In this setting, Transformer-XL learns a RECL of 900 words on WikiText-103, while the numbers for recurrent networks and Transformer are only 500 and 128.

% We use two methods to quantitatively study the effective lengths of Transformer-XL and the baselines. Similar to \cite{khandelwal2018sharp}, we gradually increase the attention length at test time until no further noticeable improvement ($\sim$0.1\% relative gains) can be observed. Our best model in this settings use attention lengths of 1,600 and 3,800 on WikiText-103 and enwiki8 respectively.
% %In addition, since the effective context length of Transformer-XL can be longer than the attention length due to our recurrent formulation, we devise a metric called \textit{Relative Effective Context Length} (RECL) that aims to perform a fair comparison of the gains brought by increasing the context lengths for different models.
% In addition, we devise a metric called \textit{Relative Effective Context Length} (RECL) that aims to perform a fair comparison of the gains brought by increasing the context lengths for different models.
% In this setting, Transformer-XL learns a RECL of 900 words on WikiText-103, while the numbers for recurrent networks and Transformer are only 500 and 128.


\section{The Reading Comprehension Task}

% \subsection{The Problem Setup}

The RC datasets introduced in \cite{hermann2015teaching} are made from articles on the news websites \ti{CNN} and \ti{Daily Mail}, utilizing articles and their bullet point summaries.\footnote{The datasets are available at \url{https://github.com/deepmind/rc-data}.} Figure~\ref{fig:example} demonstrates an example\footnote{The original article can be found at \url{http://www.cnn.com/2015/03/10/entertainment/feat-star-wars-gay-character/}.}: it consists of a passage $p$, a question $q$ and an answer $a$, where the passage is a news article, the question is a cloze-style task, in which one of the article's bullet points has had one entity replaced by a placeholder, and the answer is this questioned entity. The goal is to infer the missing entity (answer $a$) from all the possible entities which appear in the passage. A news article is usually associated with a few (e.g., 3--5) bullet points and each of them highlights one aspect of its content.

The text has been run through a Google NLP pipeline. It it tokenized, lowercased, and named entity recognition and coreference resolution have been run. For each coreference chain containing at least one named entity, all items in the chain are replaced by an @entity$n$ marker, for a distinct index $n$. \newcite{hermann2015teaching} argue convincingly that such a strategy is necessary to ensure that systems approach this task by understanding the passage in front of them, rather than by using world knowledge or a language model to answer questions without needing to understand the passage. However, this also gives the task a somewhat artificial character. On the one hand, systems are greatly helped by entity recognition and coreference having already been performed; on the other, they suffer when either of these modules fail, as they do (in \figref{example}, ``the character'' should probably be coreferent with @entity14; clearer examples of failure appear later on in our data analysis). Moreover, this inability to use world knowledge also makes it much more difficult for a human to do this task -- occasionally it is very difficult or impossible for a human to determine the correct answer when presented with an item anonymized in this way.

\begin{table}
\centering
\begin{tabular}{@{} l r  r @{}}
\toprule
& \tf{CNN} & \tf{Daily Mail} \\
\hline
\# Train & 380,298 & 879,450 \\
\# Dev & 3,924 & 64,835 \\
\# Test & 3,198 & 53,182 \\
\midrule
Passage: avg.\ tokens & 761.8 & 813.1 \\
Passage: avg.\ sentences & 32.3 & 28.9 \\
Question: avg.\ tokens & 12.5 & 14.3 \\
\hline
Avg. \# entities & 26.2 & 26.2 \\
\bottomrule
\end{tabular}
\caption{Data statistics of the \ti{CNN} and \ti{Daily Mail} datasets. The avg.\ tokens and sentences in the passage, the avg.\ tokens in the query, and the number of entities are based on statistics from the training set, but they are similar on the development and test sets.}
\label{table:data_stat}
\end{table}

The creation of the datasets benefits from the sheer volume of news articles available online, so they offer a large and realistic testing ground for statistical models. Table~\ref{table:data_stat} provides some statistics on the two datasets: there are 380k and 879k training examples for \ti{CNN} and \ti{Daily Mail} respectively. The passages are around 30 sentences and 800 tokens on average, while each question contains around 12--14 tokens.

In the following sections, we seek to more deeply understand the nature of this dataset. We first build some straightforward systems in order to get a better idea of a lower-bound for the performance of current NLP systems. Then we turn to data analysis of a sample of the items to examine their nature and an upper bound on performance.

% \documentclass{article}

\usepackage{amsmath}
\usepackage{tikz}
\usetikzlibrary{positioning,decorations.markings,calc}
\usepackage{amssymb}
\usepackage{xcolor}

\definecolor{mygray}{RGB}{50,49,51}
\definecolor{mygray2}{RGB}{70,70,70}
\definecolor{mywhite}{RGB}{197,192,195}
\definecolor{myblue}{RGB}{0, 161, 241}
\definecolor{myyellow}{RGB}{255, 187, 0}
\definecolor{mygreen}{RGB}{124, 187, 0}
\definecolor{myred}{RGB}{246, 83, 20}

\begin{document}
\begin{center}
\begin{tikzpicture}[%
	%common options for blocks
	block/.style = {draw,fill=lightgray, align=center, anchor=west, minimum height=0.8cm, inner sep=0},
	largeblock/.style = {draw,fill=lightgray, align=center, anchor=west, minimum height=1.6cm, inner sep=0},
	dashblock/.style = {draw=gray,dashed,fill=lightgray, align=center, anchor=west, minimum height=1.6cm, inner sep=0},
	ball/.style = {circle, draw, align=center, anchor=north, inner sep=0, text width=1cm}%
]
   
\node[dashblock,anchor=north,text width=15.5cm,fill=white, opacity=0.7] (input) at (0,5.6) {};
\node[anchor=north west,text width=6cm, text=mygray] (tasknote) at (-2, 5.6){Task-Specific Representation};
   
\node[block,anchor=north,text width=15cm, fill=gray,text=white] (input) at (0,0) {$X$: Input};
\node[largeblock,anchor=north,text width=15cm, fill=gray, font=10, text=white] (rep) at (0,3) {Semantic Representation};

\node[block,anchor=north,text width=3cm, fill=myred, text=white] (tc) at (-6,5) {Text Classification};

\node[block,anchor=north,text width=3cm, fill=mygreen, text=white] (ae) at (-2,5) {Autoencoder};

\node[block,anchor=north,text width=3cm, fill=myblue, text=white] (lm) at (2,5) {Lanauge Model};

\node[block,anchor=north,text width=3cm, fill=myyellow, text=white] (ot) at (6,5) {Other tasks};

%\draw[thick,->,draw=black] (input.north) to (rep.south) node[anchor=east] at (0, 0.8) {$\mathbf{W_0}$};
\draw[thick,->,draw=black] (input.north) to (rep.south) node[anchor=east] at (0, 0.8) {};

\draw[thick,->,draw=myred] (-6, 3) to (tc.south) node[anchor=east] at (-6, 3.5) {};

\draw[thick,->,draw=mygreen] (-2, 3) to (ae.south) node[anchor=east] at (-2, 3.5) {};

\draw[thick,->,draw=myblue] (2, 3) to (lm.south) node[anchor=east] at (2, 3.5) {};

\draw[thick,->,draw=myyellow] (6, 3) to (ot.south) node[anchor=east] at (6, 3.5) {};


\node[ball, anchor=north,text width=1.2cm,fill=myred, text=white] (t1o) at (-6, 7) {};
\node[ball,anchor=north,text width=1.2cm,fill=mygreen, text=white] (t2o) at (-2,7) {};
\node[ball,anchor=north,text width=1.2cm,fill=myblue] (t3o) at (2,7) {};
\node[ball,anchor=north,text width=1.2cm,fill=myyellow] (t4o) at (6,7) {};

\draw[thick,->,draw=myred] (tc.north) to (t1o.south);
\draw[thick,->,draw=mygreen] (ae.north) to (t2o.south);
\draw[thick,->,draw=myblue] (lm.north) to (t3o.south);
\draw[thick,->,draw=myyellow] (ot.north) to (t4o.south);

\draw[thick,->,draw=myred,text=myred] (t1o.north) to (-6, 7.8) node[anchor=west] at (-6.8, 8) {$P(C|D)$};
\draw[thick,->,draw=mygreen,text=mygreen] (t2o.north) to (-2, 7.8) node[anchor=west] at (-2.8, 8) {$P(X'|X)$};

\draw[thick,->,draw=myblue,text=myblue] (t3o.north) to (2, 7.8) node[anchor=west] at (0.8, 8) {$P(X_{t}|X_{t-1})$};

\draw[thick,->,draw=myyellow,text=myyellow] (t4o.north) to (6, 7.8) node[anchor=west] at (5.7, 8) {$...$};


\node[anchor=north west,text width=4cm, text=myred] (cnote) at (-8, 9.5){Text classification posterior probability};

\node[anchor=north west,text width=4cm, text=mygreen] (cnote) at (-4, 9.5){Autoencoder reconstruction};

\node[anchor=north west,text width=4cm, text=myblue] (cnote) at (0, 9.5){Next word probability};

\node[anchor=north west,text width=4cm, text=myyellow] (cnote) at (4, 9.5){Other objectives};

\end{tikzpicture}
\end{center}
\end{document}
\section{Our Systems}

In this section, we describe two systems we implemented -- a conventional entity-centric classifier and an end-to-end neural network. While \newcite{hermann2015teaching} do provide several baselines for performance on the RC task, we suspect that their baselines are not that strong. They attempt to use a frame-semantic parser, and we feel that the poor coverage of that parser undermines the results, and is not representative of what a straightforward NLP system -- based on standard approaches to factoid question answering and relation extraction developed over the last 15 years -- can achieve. Indeed, their frame-semantic model is markedly inferior to another baseline they provide, a heuristic word distance model. At present just two papers are available presenting results on this RC task, both presenting neural network approaches: \cite{hermann2015teaching} and \cite{hill2016goldilocks}. While the latter is wrapped in the language of end-to-end memory networks, it actually presents a fairly simple window-based neural network classifier running on the CNN data. Its success again raises questions about the true nature and complexity of the RC task provided by this dataset, which we seek to clarify by building a simple attention-based neural net classifier.

Given the (passage, question, answer) triple $(p, q, a)$, $p = \{p_1, \ldots, p_{m}\}$ and $q = \{q_1, \ldots, q_{l}\}$ are sequences of tokens for the passage and question sentence, with $q$ containing exactly one ``@placeholder'' token. The goal is to infer the correct entity $a \in p \cap E$ that the placeholder corresponds to, where $E$ is the set of all abstract entity markers. Note that the correct answer entity must appear in the passage $p$.

\subsection{Entity-Centric Classifier}

We first build a conventional feature-based classifier, aiming to explore what features are effective for this task. This is similar in spirit to \cite{wang2015machine}, which at present has very competitive performance on the MCTest RC dataset \cite{richardson2013mctest}. The setup of this system is to design a feature vector $f_{p, q}(e)$ for each candidate entity $e$, and to learn a weight vector $\theta$ such that the correct answer $a$ is expected to rank higher than all other candidate entities:
\begin{equation}
\theta^{\intercal}f_{p, q}(a) > \theta^{\intercal}f_{p, q}(e), \forall e \in E \cap p \setminus \{a\}
\end{equation}

We employ the following feature templates:
\begin{enumerate}[1.]
    \setlength\itemsep{-0.1em}
    \item
        Whether entity $e$ occurs in the passage.
    \item
        Whether entity $e$ occurs in the question.
    \item
        The frequency of entity $e$ in the passage.
    \item
        The first position of occurence of entity $e$ in the passage.
    \item
        $n$-gram exact match: whether there is an exact match between the text surrounding the placeholder and the text surrounding entity $e$. We have features for all combinations of matching left and/or right one or two words.
    \item
        Word distance: we align the placeholder with each occurrence of entity $e$, and compute the average minimum distance of each non-stop question word from the entity in the passage.
    \item
        Sentence co-occurrence: whether entity $e$ co-occurs with another entity or verb that appears in the question, in some sentence of the passage.
    \item
        Dependency parse match: we dependency parse both the question and all the sentences in the passage, and extract an indicator feature of whether $w \xrightarrow{r} \text{@placeholder}$ and $w \xrightarrow{r} e$ are both found; similar features are constructed for $\text{@placeholder} \xrightarrow{r} w$ and $e \xrightarrow{r} w$.
\end{enumerate}



\subsection{End-to-end Neural Network}

Our neural network system is based on the \ti{AttentiveReader} model proposed by \cite{hermann2015teaching}. The framework can be described in the following three steps (see Figure \ref{fig:framework}):

\begin{figure*}[!ht]
\centering
    \includegraphics[scale=0.37]{figures/fig_model.pdf}
\caption{Our neural network architecture for the reading comprehension task.}
\label{fig:framework}
\end{figure*}

\begin{description}
    \item[\tf{Encoding:}] First, all the words are mapped to $d$-dimensional vectors via an embedding matrix $E \in \R^{d \times |\mathcal{V}|}$; therefore we have $p$: $\mf{p}_1, \ldots, \mf{p}_m \in R^d$ and $q: \mf{q}_1, \ldots, \mf{q}_{l} \in R^d$.

        Next we use a shallow bi-directional recurrent neural network (RNN) with hidden size $\tilde{h}$ to encode contextual embeddings $\tilde{\mf{p}}_i$ of each word in the passage,
        \begin{eqnarray*}
            \overrightarrow{\mf{h}}_i & = & \text{RNN}(\overrightarrow{\mf{h}}_{i-1}, \mf{p}_i), i = 1, \ldots, m\\
            \overleftarrow{\mf{h}}_i & = & \text{RNN}(\overleftarrow{\mf{h}}_{i+1}, \mf{p}_i), i = m, \ldots, 1
        \end{eqnarray*}
        and $\tilde{\mf{p}}_i = \text{concat}(\overrightarrow{\mf{h}}_i, \overleftarrow{\mf{h}}_i) \in \R^{h}$, where $h = 2 \tilde{h}$.
        Meanwhile, we use another bi-directional RNN to map the question $\mf{q}_1, \ldots, \mf{q}_l$ to an embedding $\mf{q} \in \R^h$. We choose to use Gated Recurrent Unit (GRU) \cite{cho2014learning} in our experiments because it performs similarly but is computationally cheaper than LSTM.

    \item[\tf{Attention:}] In this step, the goal is to compare the question embedding and all the contextual embeddings, and \ti{select} the pieces of information that are relevant to the question. We compute a probability distribution $\alpha$ depending on the degree of relevance between word $p_i$ (in its context) and the question $q$ and then produce an output vector $\mf{o}$ which is a weighted combination of all contextual embeddings $\{\tilde{\mf{p}}_i\}$:
        \begin{eqnarray}
            \alpha_i & = & \softmax\nolimits_i \mf{q} ^{\intercal} \mf{W}_{s} \tilde{\mf{p}}_i  \\
            \mf{o} & = & \sum\nolimits_{i}{\alpha_i \tilde{\mf{p}}_i}
        \end{eqnarray}
        $\mf{W_s} \in \R^{h \times h}$ is used in a bilinear term, which allows us to compute a similarity between $\mf{q}$ and $\tilde{\mf{p}}_i$ more flexibly than with just a dot product.
    \item[\tf{Prediction:}] Using the \ti{output} vector $\mf{o}$, the system outputs the most likely answer using:
        \begin{equation}
            a = \argmax\nolimits_{a \in p \cap E}{W_a ^{\intercal} \mf{o}}
        \end{equation}
        Finally, the system adds a softmax function on top of $W_a ^{\intercal} \mf{o}$ and adopts a negative log-likelihood objective for training.
\end{description}

\paragraph*{Differences from \cite{hermann2015teaching}.}
Our model basically follows the \ti{AttentiveReader}. However, to our surprise, our experiments observed nearly \tf{7} --\tf{10}\% improvement over the original \ti{AttentiveReader} results on \ti{CNN} and \ti{Daily Mail} datasets (discussed in Sec.~\ref{sec:experiments}). Concretely, our model has the following differences:
\begin{itemize}
\item
We use a bilinear term, instead of a $\tanh$ layer to compute the relevance (attention) between question and contextual embeddings. The effectiveness of the simple bilinear attention function has been shown previously for neural machine translation by \cite{luong2015effective}.
\item
After obtaining the weighted contextual embeddings $\mf{o}$, we use $\mf{o}$ for direct prediction. In contrast, the original model in \cite{hermann2015teaching} combined $\mf{o}$ and the question embedding $\mf{q}$ via another non-linear layer before making final predictions. We found that we could remove this layer without harming performance. We believe it is sufficient for the model to learn to return the entity to which it maximally gives attention.
\item
The original model considers all the words from the vocabulary $\mathcal{V}$ in making predictions. We think this is unnecessary, and only predict among entities which appear in the passage.
\end{itemize}
Of these changes, only the first seems important; the other two just aim at keeping the model simple.

\paragraph*{Window-based MemN2Ns \cite{hill2016goldilocks}.}
Another recent neural network approach proposed by \cite{hill2016goldilocks} is based on a memory network architecture \cite{weston2015memory}. We think it is highly similar in spirit. The biggest difference is their way of encoding passages: they demonstrate that it is most effective to only use a 5-word context window when evaluating a candidate entity and they use a positional unigram approach to encode the contextual embeddings: if a window consists of 5 words $x_1, \ldots, x_5$, then it is encoded as $\sum_{i=1}^{5}{E_i(x_i)}$, resulting in $5$ separate embedding matrices to learn. They encode the 5-word window surrounding the placeholder in a similar way and all other words in the question text are ignored. In addition, they simply use a dot product to compute the ``relevance'' between the question and a contextual embedding. This simple model nevertheless works well, showing the extent to which this RC task can be done by very local context matching.

% !TEX root = ../multi_task.tex

We evaluate the presented MTL method on a number of problems. First, we use MultiMNIST \citep{multi_mnist}, an MTL adaptation of MNIST \citep{mnist}. Next, we tackle multi-label classification on the CelebA dataset \citep{celeba} by considering each label as a distinct binary classification task. These problems include both classification and regression, with the number of tasks ranging from 2 to 40. Finally, we experiment with scene understanding, jointly tackling the tasks of semantic segmentation, instance segmentation, and depth estimation on the Cityscapes dataset \citep{cityscapes}. We discuss each experiment separately in the following subsections.

The baselines we consider are (i) \textbf{uniform scaling:} minimizing a uniformly weighted sum of loss functions \mbox{$\frac{1}{T}\sum_t \lL^t$}, \mbox{(ii) \textbf{single task:}} solving tasks independently, \mbox{(iii) \textbf{grid search:}} exhaustively trying various values from $\{ c^t \in [0,1] | \sum_t c^t = 1\}$ and optimizing for $\frac{1}{T}\sum_t c^t \lL^t$, \mbox{(iv) \textbf{\citet{Kendall2018}:}} using the uncertainty weighting proposed by \citet{Kendall2018}, and \mbox{(v) \textbf{GradNorm:}} using the normalization proposed by \citet{Chen2018}.



\subsection{MultiMNIST}
\label{sec:multi_mnist_exp}

Our initial experiments are on MultiMNIST, an MTL version of the MNIST dataset \citep{multi_mnist}. In order to convert digit classification into a multi-task problem, \citet{multi_mnist} overlaid multiple images together. We use a similar construction. For each image, a different one is chosen uniformly in random. Then one of these images is put at the top-left and the other one is at the bottom-right. The resulting tasks are: classifying the digit on the top-left (task-L) and classifying the digit on the bottom-right (task-R). We use 60K examples and directly apply existing single-task MNIST models. The MultiMNIST dataset is illustrated in the supplement.

We use the LeNet architecture \citep{mnist}. We treat all layers except the last as the representation function $g$ and put two fully-connected layers as task-specific functions (see the supplement for details). We visualize the performance profile as a scatter plot of accuracies on task-L and task-R in Figure~\ref{fig:multi_mnist_performance_curve}, and list the results in Table~\ref{tab:multi_mnist}.

In this setup, any static scaling results in lower accuracy than solving each task separately (the single-task baseline). The two tasks appear to compete for model capacity, since increase in the accuracy of one task results in decrease in the accuracy of the other. Uncertainty weighting \citep{Kendall2018} and GradNorm \citep{Chen2018} find solutions that are slightly better than grid search but distinctly worse than the single-task baseline. In contrast, our method finds a solution that efficiently utilizes the model capacity and yields accuracies that are as good as the single-task solutions. This experiment demonstrates the effectiveness of our method as well as the necessity of treating MTL as multi-objective optimization. Even after a large hyper-parameter search, \emph{any} scaling of tasks does not approach the effectiveness of our method.



\subsection{Multi-Label Classification}

\begin{figure}[t]
\includegraphics[width=\textwidth]{radar_full_new}
\vspace{1mm}
\caption{Radar charts of percentage error per attribute on CelebA \citep{celeba}. Lower is better. We divide attributes into two sets for legibility: easy on the left, hard on the right. Zoom in for details.}
\label{fig:multi_label_radar}
\end{figure}


\begin{wraptable}{r}{0.3\textwidth}
%\vspace{-4mm}
\captionof{table}{Mean of error per category of MTL algorithms in multi-label classification on CelebA \citep{celeba}.}
\begin{tabular}{r@{\hspace{2mm}}c@{}}
\toprule
& Average  \\
&  error \\
\midrule
Single task & $8.77$ \\
Uniform scaling & $9.62$ \\
\citealt{Kendall2018} & $9.53$ \\
GradNorm & $8.44$ \\
Ours & $\mathbf{8.25}$  \\
\bottomrule
\end{tabular}
\label{table:multi_label_bar}
%\vspace{-5mm}
\end{wraptable}

Next, we tackle multi-label classification. Given a set of attributes, multi-label classification calls for deciding whether each attribute holds for the input. We use the CelebA dataset \citep{celeba}, which includes 200K face images annotated with 40 attributes. Each attribute gives rise to a binary classification task and we cast this as a 40-way MTL problem. We use ResNet-18 \citep{resnet} without the final layer as a shared representation function, and attach a linear layer for each attribute (see the supplement for further details).


We plot the resulting error for each binary classification task as a radar chart in Figure~\ref{fig:multi_label_radar}. The average over them is listed in Table~\ref{table:multi_label_bar}. We skip grid search since it is not feasible over 40 tasks. Although uniform scaling is the norm in the multi-label classification literature, single-task performance is significantly better. Our method outperforms baselines for significant majority of tasks and achieves comparable performance in rest. This experiment also shows that our method remains effective when the number of tasks is high.


\subsection{Scene Understanding}

To evaluate our method in a more realistic setting, we use scene understanding. Given an RGB image, we solve three tasks: semantic segmentation (assigning pixel-level class labels), instance segmentation (assigning pixel-level instance labels), and monocular depth estimation (estimating continuous disparity per pixel). We follow the experimental procedure of \citet{Kendall2018} and use an encoder-decoder architecture. The encoder is based on ResNet-50 \citep{resnet} and is shared by all three tasks. The decoders are task-specific and are based on the pyramid pooling module \citep{pspnet} (see the supplement for further implementation details).

Since the output space of instance segmentation is unconstrained (the number of instances is not known in advance), we use a proxy problem as in \citet{Kendall2018}. For each pixel, we estimate the location of the center of mass of the instance that encompasses the pixel. These center votes can then be clustered to extract the instances. In our experiments, we directly report the MSE in the proxy task. Figure~\ref{fig:cityscapes_performance_profile} shows the performance profile for each pair of tasks, although we perform all experiments on all three tasks jointly. The pairwise performance profiles shown in Figure~\ref{fig:cityscapes_performance_profile} are simply 2D projections of the three-dimensional profile, presented this way for legibility. The results are also listed in Table~\ref{tab:cityscapes_results}.

MTL outperforms single-task accuracy, indicating that the tasks cooperate and help each other. Our method outperforms all baselines on all tasks.


\subsection{Role of the Approximation}

In order to understand the role of the approximation proposed in Section~\ref{sec:approximation}, we compare the final performance and training time of our algorithm with and without the presented approximation in Table~\ref{tab:approximation_tradeoff} (runtime measured on a single Titan Xp GPU). For a small number of tasks (3 for scene understanding), training time is reduced by 40\%. For the multi-label classification experiment (40 tasks), the presented approximation accelerates learning by a factor of 25.

On the accuracy side, we expect both methods to perform similarly as long as the full-rank assumption is satisfied. As expected, the accuracy of both methods is very similar. Somewhat surprisingly, our approximation results in slightly improved accuracy in all experiments. While counter-intuitive at first, we hypothesize that this is related to the use of SGD in the learning algorithm. Stability analysis in convex optimization suggests that if gradients are computed with an error $\hat{\nabla}_\btheta \mathcal{L}^t = \nabla_\btheta \mathcal{L}^t + \mathbf{e}^t$ ($\btheta$ corresponds to $\btheta^{sh}$ in (\ref{eq:kkt_opt})), as opposed to $\mathbf{Z}$ in the approximate problem in \ref{eq:approx}, the error in the solution is bounded as $\|\hat{\mathbf{\alpha}} - \mathbf{\alpha} \|_2 \leq \mathcal{O}(\max_t \|\mathbf{e}^t\|_2)$. Considering the fact that the gradients are computed over the full parameter set (millions of dimensions) for the original problem and over a smaller space for the approximation (batch size times representation which is in the thousands), the dimension of the error vector is significantly higher in the original problem. We expect the $l_2$ norm of such a random vector to depend on the dimension.

In summary, our quantitative analysis of the approximation suggests that (i) the approximation does not cause an accuracy drop and (ii) by solving an equivalent problem in a lower-dimensional space, our method achieves both better computational efficiency and higher stability.

  {\small
  \begin{table}[t]
%  \vspace{-4mm}
  \caption{Effect of the MGDA-UB approximation. We report the final accuracies as well as training times for our method with and without the approximation.}
  %\vspace{1mm}
  \centering
  \begin{tabular}{@{}r@{\hspace{3mm}}c@{\hspace{3mm}}c@{\hspace{2mm}}c@{\hspace{2mm}}c@{}c@{\hspace{5mm}}c@{\hspace{2mm}}c@{}}
  \toprule
  & \multicolumn{4}{c}{Scene understanding (3 tasks)} &  & \multicolumn{2}{c}{Multi-label (40 tasks)}  \\
  \cmidrule(r){2-5} \cmidrule(lr){7-8}
                  & Training & Segmentation & Instance  & Disparity      & & Training & Average \\
                 & time     &  mIoU [\%]       & error [px] & error [px] & & time (hour)      & error \\
  \midrule
  Ours (w/o approx.) & $38.6$ & $66.13$ & $10.28$ & $2.59$ & & $429.9$ & $8.33$ \\
  Ours & $\mathbf{23.3}$ & $\mathbf{66.63}$ & $\mathbf{10.25}$ & $\mathbf{2.54}$  & & $\mathbf{16.1}$ & $\mathbf{8.25}$ \\
  \bottomrule
  \end{tabular}
  %\vspace{-2mm}
  \label{tab:approximation_tradeoff}
  \end{table}}

\section{VQA Dataset Analysis}
\label{sec:analysis}
%\vspace{\sectionReduceBot}
%%%%%%%%%%%%%%%%%%%%%%%%%%%%%%%%%%%%%%%%%%%%%%%%%%%%%%%%%%%
%%%%%%%%%%%%%%%%%%%%%%%%%%%%%%%%%%%%%%%%%%%%%%%%%%%%%%%%%%%
\begin{figure*}[t]
\centering
\includegraphics[width=1\linewidth]{figures/QuestionTypes3.pdf}
\caption{Distribution of questions by their first four words for a random sample of 60K questions for real images (left) and all questions for abstract scenes (right). The ordering of the words starts towards the center and radiates outwards. The arc length is proportional to the number of questions containing the word. White areas are words with contributions too small to show. }
%\vspace{-5pt}
\label{fig:QuesCluster}
%\setlength{\belowcaptionskip}{-10pt}
\end{figure*}
%%%%%%%%%%%%%%%%%%%%%%%%%%%%%%%%%%%%%%%%%%%%%%%%%%%%%%%%%%%

In this section, we provide an analysis of the questions and answers in the VQA train dataset.
To gain an understanding of the types of questions asked and answers provided, we visualize
the distribution of question types and answers. We also explore how often the questions may
be answered without the image using just commonsense information. Finally, we analyze whether
the information contained in an image caption is sufficient to answer the questions.

The dataset includes 614,163 questions 
%and a total of 
and 7,984,119 answers (including answers provided by workers with and without 
looking at the image) 
%and without looking at the image) 
for 204,721 images from the MS COCO dataset~\cite{coco} and 150,000 questions with 1,950,000 answers for $50,000$ abstract scenes.

%\textcolor{red}{
%We emphasize that the creation of a dataset of this scale and richness
%is a time consuming process, taking months to complete.
%While the entirety of the dataset has been collected,} at the time of original submission,
%120,520 questions with 270,210 answers for 50,000 MS COCO
%images and 30,000 questions with 79,740 answers for 10,000 abstract scenes had been collected.
%Please refer to the appendix for further details.
%\textcolor{red}{The results in this section still reflect that subset of the final dataset.}
%We emphasize that the creation of a dataset of this scale and richness
%is a time consuming process, taking months to complete.
%By our current estimates,
%approximately 5,000 questions and 40,000 answers are collected per day
%using Amazon Mechanical Turk (AMT).
%The entire dataset will take approximately three months to complete. At the time of submission,
%120,520 questions with 270,210 answers for 50,000 MS COCO
%images and 30,000 questions with 79,740 answers for 10,000 abstract scenes had been collected.
%Please refer to the appendix for further details.


%%%%%%%%%%%%%%%%%%%%%%%%%%%%%%%%%%%%%%%%%%%%%%%%%%%%%%%%%%%
%\vspace{\subsectionReduceTop}
\subsection{Questions}
%\vspace{\subsectionReduceBot}
%%%%%%%%%%%%%%%%%%%%%%%%%%%%%%%%%%%%%%%%%%%%%%%%%%%%%%%%%%%

\textbf{Types of Question.}
Given the structure of questions generated in the English language,
we can cluster questions into different types based on the words that start the question.
\figref{fig:QuesCluster} shows the distribution of questions based on the first four
words of the questions for both the real images (left) and abstract scenes (right).
Interestingly, the distribution of questions is quite similar for both real images and abstract scenes.
This helps demonstrate that the type of questions elicited by the abstract scenes is similar to
those elicited by the real images. There exists a surprising variety of question types,
including ``What is$\ldots$'', ``Is there$\ldots$'', ``How many$\ldots$'', and ``Does the$\ldots$''.
Quantitatively, the percentage of questions for different types is shown in \tableref{tab:typeacc}. Several example questions and answers are shown in \figref{fig:qualResults}.
%\textbf{Sub-Types.}
A particularly interesting type of question is the ``What is$\ldots$'' questions, since they have a
diverse set of possible answers. See the appendix for visualizations for ``What is$\ldots$'' questions.

\textbf{Lengths.}
\figref{fig:QuesLen} shows the distribution of question lengths.
We see that most questions range from four to ten words.


\begin{comment}\begin{table}[h]
{\small
\begin{tabular}{@{\extracolsep{\fill}}p{2cm}|ccccc@{\extracolsep{\fill}}}
%\toprule
Dataset  & Yes & No\\
%\midrule
Real   & 18.21 & 14.06 \\
Abstract & 26.54 & 16.70 \\
\end{tabular}
}
\vspace{5pt}
\caption{Percentage of ``yes'' and ``no'' questions in the real and abstract datasets.}
\label{table:yesno}
%\vspace{\captionReduceBot}
\end{table}
\end{comment}

%%%%%%%%%%%%%%%%%%%%%%%%%%%%%%%%%%%%%%%%%%%%%%%%%%%%%%%%%%%
\begin{figure}[t]
\centering
\includegraphics[width=1\linewidth]{figures/Lengths.pdf}
%\vspace{-9pt}
\caption{Percentage of questions with different word lengths for real images and abstract scenes.}
%\vspace{-5pt}
\label{fig:QuesLen}
%\setlength{\belowcaptionskip}{-10pt}
\end{figure}
%%%%%%%%%%%%%%%%%%%%%%%%%%%%%%%%%%%%%%%%%%%%%%%%%%%%%%%%%%%




\begin{figure*}
\centering
\includegraphics[width=1\linewidth]{figures/answers.pdf}
%\vspace{-5pt}
\caption{Distribution of answers per question type for a random sample of 60K questions for real images when subjects provide answers when given the image (top) and when not given the image (bottom).}
%\vspace{-5pt}
\label{fig:AnsPerQues}
%\setlength{\belowcaptionskip}{-10pt}
\end{figure*}


%%%%%%%%%%%%%%%%%%%%%%%%%%%%%%%%%%%%%%%%%%%%%%%%%%%%%%%%%%%
%\vspace{\subsectionReduceTop}
\subsection{Answers}
%\vspace{\subsectionReduceBot}
%%%%%%%%%%%%%%%%%%%%%%%%%%%%%%%%%%%%%%%%%%%%%%%%%%%%%%%%%%%

%\textbf{Typical Answers for Different Question Types.}
\textbf{Typical Answers.}
%Next, we analyze the answers provided for different question types.
\figref{fig:AnsPerQues} (top) shows the distribution of answers for several question types.
We can see that a number of question types, such as ``Is the\ldots'', ``Are\ldots'', and ``Does\ldots'' are
typically answered using ``yes'' and ``no'' as answers.
%\textcolor{red}{Question types such as ``How many\ldots'' are answered using numbers. $12.31\%$ and $14.48\%$ of the questions are answered using numbers on real images and abstract scenes, respectively.}
Other questions such as ``What is\ldots'' and ``What type\ldots'' have a rich diversity
of responses. Other question types such as ``What color\ldots'' or ``Which\ldots'' have more specialized responses,
such as colors, or ``left'' and ``right''. 
See the appendix for a list of the most popular answers.

\textbf{Lengths.}
Most answers consist of a single word, with the distribution of answers containing one, two, or three words, respectively being $89.32\%$, $6.91\%$, and $2.74\%$ for real images and $90.51\%$, $5.89\%$, and $2.49\%$ for abstract scenes.
%$89.16\%$, $7.00\%$, and $2.77\%$ of answers containing one, two, or three words, respectively.
The brevity of answers is not surprising, since the questions tend to elicit specific
information from the images. This is in contrast with image captions that generically
describe the entire image and hence tend to be longer. The brevity of our answers makes
automatic evaluation feasible. While it may be tempting to believe the brevity of the answers
makes the problem easier, recall that they are human-provided open-ended answers to
open-ended questions. The questions typically require complex reasoning to arrive at these
deceptively simple answers (see \figref{fig:qualResults}).
There are currently 23,234 unique one-word answers in our dataset for real images and 3,770 for abstract scenes.
%There are currently 10,011 unique one-word answers in our dataset.

\textbf{`Yes/No' and `Number' Answers.}
Many questions are answered using either ``yes'' or ``no'' (or sometimes ``maybe'') -- 
$38.37\%$ and $40.66\%$ of the questions on real images and abstract scenes respectively. 
Among these `yes/no' questions, there is a bias towards %answering with 
``yes'' -- %with ``yes'' being preferred %$61.32\%$ and $58.46\%$ 
$58.83\%$ and $55.86\%$ of `yes/no' answers are ``yes'' for real images and abstract scenes. 
Question types such as ``How many\ldots'' are answered using numbers -- 
$12.31\%$ and $14.48\%$ of the questions on real images and abstract scenes are `number' questions. 
``2'' is the most popular answer among the `number' questions, making up 
$26.04\%$ of the `number' answers for real images and $39.85\%$ for abstract scenes. 

\textbf{Subject Confidence.}
When the subjects answered the questions, we asked
``Do you think you were able to answer the question correctly?''.
\figref{fig:ConfScores} shows the distribution of responses. A majority of the answers
were labeled as confident for both real images and abstract scenes. % respectively.

\textbf{Inter-human Agreement.}
Does the self-judgment of confidence correspond to the answer agreement between subjects?
\figref{fig:ConfScores} shows the percentage of questions in which 
(i) $7$ or more, 
(ii) $3-7$, or 
(iii) less than $3$ subjects agree on the answers given their average confidence score 
(0 = not confident, 1 = confident).
As expected, the agreement between subjects increases with confidence.
However, even if all of the subjects are confident the answers may still vary.
This is not surprising since some answers may vary, yet have very similar meaning, such as ``happy'' and ``joyful''.

\begin{figure}[t]
\centering
\includegraphics[width=1\linewidth]{figures/Confidence.pdf}
%\vspace{-5pt}
\caption{Number of questions per average confidence score (0 = not confident, 1 = confident) for real images and abstract scenes (black lines). Percentage of questions where 7 or more answers are same, 3-7 are same, less than 3 are same (color bars). }
%\vspace{-7pt}
\label{fig:ConfScores}
%\setlength{\belowcaptionskip}{-10pt}
\end{figure}

As shown in \tableref{table:commonsense_acc} (Question + Image), there is significant inter-human
agreement in the answers for both real images ($83.30\%$) and abstract scenes ($87.49\%$). 
%when humans are provided both the question and image while answering the question.
Note that on average each question has $2.70$ unique answers for real images and $2.39$ for abstract scenes. 
The agreement is significantly higher ($>95\%$) for \quotes{yes/no} questions and lower for other questions ($<76\%$), possibly due to the fact that we perform exact string matching and do not account for synonyms, plurality, \etc. Note that the automatic determination of synonyms is a difficult problem, since the level of answer granularity can vary across questions.




%%%%%%%%%%%%%%%%%%%%%%%%%%%%%%%%%%%%%%%%%%%%%%%%%%%%%%%%%%%
%\vspace{\subsectionReduceTop}
\subsection{Commonsense Knowledge}
\label{sec:cs}
%\vspace{\subsectionReduceBot}
%%%%%%%%%%%%%%%%%%%%%%%%%%%%%%%%%%%%%%%%%%%%%%%%%%%%%%%%%%%
\begin{figure*}[t]
 \includegraphics[width=\linewidth]{figures/age.pdf}
 \centering
\caption{\small Example questions judged by Mturk workers to be answerable by different age groups. The percentage of questions falling into each age group is shown in parentheses.}
 \label{fig:age}
 \end{figure*}
 	
\textbf{Is the Image Necessary?}
%Can the questions be answered using commonsense knowledge alone without the need for an image,
%\eg, ``What is the color of the sheep?''?
Clearly, some questions can sometimes be
answered correctly using commonsense knowledge alone without the need for an image,
\eg, ``What is the color of the fire hydrant?''.
We explore this issue by asking three subjects to answer
the questions \emph{without seeing the image} (see the examples in blue in \figref{fig:qualResults}).
In \tableref{table:commonsense_acc} (Question), we show the percentage of questions for which
the correct answer is provided over all questions, ``yes/no'' questions, and the other questions that
are not ``yes/no''. For ``yes/no'' questions, the human subjects respond better than chance.
For other questions, humans are only correct about $21\%$ of the time. This demonstrates that
understanding the visual information is critical to VQA and that commonsense information alone is not sufficient.

To show the qualitative difference in answers provided with and without images,
we show the distribution of answers for various question types in \figref{fig:AnsPerQues} (bottom).
The distribution of colors, numbers, and even ``yes/no'' responses is surprisingly different for answers
with and without images.
 
\textbf{Which Questions Require Common Sense?}
In order to identify questions that require commonsense reasoning to answer, we conducted 
two AMT studies (on a subset 10K questions from the real images of VQA trainval) asking subjects --
\begin{compactenum} 
\item Whether or not they believed a question required commonsense to answer the question, and 
\item The youngest age group that they believe a person must be in order to be able to correctly answer the question -- 
toddler (3-4), 
younger child (5-8), 
older child (9-12), 
teenager (13-17), 
adult (18+).
\end{compactenum}
Each question was shown to 10 subjects. We found that 
for $47.43\%$ of questions 3 or more subjects voted `yes' to commonsense, 
($18.14\%$: 6 or more).  
In the `perceived human age required to answer question' study, we found the following distribution of responses: 
toddler: $15.3\%$,
younger child: $39.7\%$, 
older child: $28.4\%$, 
teenager: $11.2\%$, 
adult: $5.5\%$.
In Figure \ref{fig:age} we show several questions for which a majority of subjects picked the specified age range. Surprisingly the perceived age needed to answer the questions is fairly well distributed across the different age ranges. As expected the questions that were judged answerable by an adult (18+) generally need specialized knowledge, whereas those answerable by a toddler (3-4) are more generic.
 
We measure the degree of commonsense required to answer a question as the percentage of subjects (out of 10) who voted ``yes'' in our ``whether or not a question requires commonsense'' study.
A fine-grained breakdown of average age and average degree of common sense (on a scale of $0-100$) required to answer a question is shown in \tableref{tab:typeacc}. The average age and the average degree of commonsense across all questions is $8.92$ and $31.01\%$ respectively. 

%\arxiv{To compute average age and average degree of commonsense across questions, we first compute the average age and average degree of commonsense (binary response scaled to $0-100$) per question (by taking average across 10 subjects for each question) and then take average across questions.} 

It is important to distinguish between:
\begin{compactenum}
\item How old someone needs to be to be able to answer a question correctly,  and
\item How old people \emph{think} someone needs to be to be able to answer a question correctly. 
\end{compactenum}

Our age annotations capture the latter -- perceptions of MTurk workers in an uncontrolled environment. As such, the relative ordering of question types in \tableref{tab:typeacc} is more important than absolute age numbers.
%The relative ordering of question types is more important than the absolute age numbers. It is important to note that the age annotations we have collected are just perceived ages: how old people -- untrained MTurk workers in an uncontrolled environment -- \emph{think} someone needs to be to be able to answer a question correctly.}
The two rankings of questions in terms of common sense required according to the two studies 
were largely correlated (Pearson's rank correlation: 0.58). 

%%%%%%%%%%%%%%%%%%%%%%%%%%%%%%%%%%%%%%%%%%%%%%%%%%%%%%%%%%%
\begin{table}[t]
\setlength{\tabcolsep}{3.2pt}
{\small
\begin{center}
%\begin{tabular}{@{}llccc@{}}
%\toprule
%Dataset & Input & All & Yes/No & Other \\
%%\hline
%\midrule
%    & Question & 40.81 & 67.60 & 21.22 \\
%Real   & Question + Caption* & 57.47 & 78.97 & 44.41 \\
%    & Question + Image & 83.30 & 95.77 & 72.67 \\
%%\hline
%\midrule
% & Question & 43.27 & 66.65 &  23.66 \\
%Abstract & Question + Caption* & 54.34 & 74.70 & 40.18 \\
% & Question + Image & 87.49 & 95.96 & 75.33 \\
%\bottomrule
%\end{tabular}
\begin{tabular}{@{}llcccc@{}}
\toprule
Dataset & Input & All & Yes/No & Number & Other \\
%\hline
\midrule
    & Question & 40.81 & 67.60 & 25.77 & 21.22 \\
Real   & Question + Caption* & 57.47 & 78.97 & 39.68 & 44.41 \\
    & Question + Image & 83.30 & 95.77 & 83.39 & 72.67 \\
%\hline
\midrule
 & Question & 43.27 & 66.65 & 28.52 & 23.66 \\
Abstract & Question + Caption* & 54.34 & 74.70 & 41.19 & 40.18 \\
 & Question + Image & 87.49 & 95.96 & 95.04 & 75.33 \\
\bottomrule
\end{tabular}
\end{center}
}
%\vspace{-7pt}
\caption {Test-standard accuracy of human subjects when asked to answer the 
question without seeing the image (Question), 
seeing just a caption of the image and not the image itself (Question + Caption), 
and seeing the image (Question + Image). 
Results are shown for all questions, ``yes/no'' \& ``number'' questions, and other questions 
that are neither answered ``yes/no'' nor number. 
All answers are free-form and not multiple-choice. 
*\hspace{1pt}These accuracies are evaluated on a subset of 3K train questions (1K images).}
% \textcolor{red}{and are not directly comparable to the corresponding numbers in older version.}}
\label{table:commonsense_acc}
%\vspace{\captionReduceBot}
%\vspace{-5pt}
\end{table}
%%%%%%%%%%%%%%%%%%%%%%%%%%%%%%%%%%%%%%%%%%%%%%%%%%%%%%%%%%%


%%%%%%%%%%%%%%%%%%%%%%%%%%%%%%%%%%%%%%%%%%%%%%%%%%%%%%%%%%%
%\vspace{\subsectionReduceTop}
\subsection{Captions \textbf{\vs} Questions}
%\vspace{\subsectionReduceBot}
%%%%%%%%%%%%%%%%%%%%%%%%%%%%%%%%%%%%%%%%%%%%%%%%%%%%%%%%%%%


Do generic image captions provide enough information to answer the questions?
\tableref{table:commonsense_acc} (Question + Caption) shows the percentage of questions answered
correctly when human subjects are given the question and a human-provided caption
describing the image, but not the image. As expected, the results are better than when humans are shown the questions alone.
However, the accuracies are significantly lower than when subjects are shown the actual image.
This demonstrates that in order to answer the questions correctly, deeper image understanding 
(beyond what image captions typically capture) is necessary. In fact, we find that the distributions of nouns, verbs, and adjectives mentioned in captions is statistically significantly different from those mentioned in our questions + answers (Kolmogorov-Smirnov test, $p<.001$) for both real images and abstract scenes. See the appendix for details. 
%This motivates the VQA task as a way to learn further information about visual scenes.
Though presented as an application paper, we have touched two fundamental problems: First, how to generate an orderless set of entities. Towards building generative models for more sophisticated combinatorial data structures such as graphs, knowing how to generate a set may be a good starting point. Second, how to capture the ambiguity of the groundtruth in a regression problem . Other than 3D reconstruction, many regression problems may have such inherent ambiguity. Our construction of the MoN loss by wrapping existing loss functions may be generalizable to these problems.

\section{Conclusion}\label{sec:conclusion}
%\vspace{-.1in}
In this work, we apply the attentional encoder-decoder for the task of abstractive summarization with very promising results, outperforming state-of-the-art results significantly on two different datasets. Each of our proposed novel models addresses a specific problem in abstractive summarization, yielding further improvement in performance. We also propose a new dataset for multi-sentence summarization and establish benchmark numbers on it. As part of our future work, we plan to focus our efforts on this data and build more robust models for summaries consisting of multiple sentences.


%Our results strongly demonstrate that sequence-to-sequence models are extremely promising for summarization. Some of the other lessons we learned from our experiments are: (i) the LVT-trick is very useful for summarization as it improves training speed while not sacrificing performance; (ii) traditional methods such as vocabulary expansion and syntax-based features can boost performance of deep learning based models as well. As part of our ongoing work, we are investigating on ways to effectively generate rare words in the summary, which appears to be a glaring weakness in the existing models.  



\section*{Acknowledgments}

We thank the anonymous reviewers for their thoughtful feedback. Stanford University gratefully acknowledges the support of the Defense Advanced Research Projects Agency (DARPA) Deep Exploration and Filtering of Text (DEFT) Program under Air Force Research Laboratory (AFRL) contract no. FA8750-13-2-0040. Any opinions, findings, and conclusion or recommendations expressed in this material are those of the authors and do not necessarily reflect the view of the DARPA, AFRL, or the US government.

\bibliography{refs}
\bibliographystyle{acl2016}

\appendix
\section{Model Architecture}
\label{sec:model_detail}

\subsection{Architectural Choices}

The proposed scheme in Section~\ref{sec:main} is a general framework where one
can freely define, for instance, the activation functions $f$ of recurrent
neural networks (RNN) and the alignment model $a$. Here, we describe the
choices we made for the experiments in this paper. 

\subsubsection{Recurrent Neural Network}
\label{sec:gatedrnn}

For the activation function $f$ of an RNN, we use the gated hidden unit
recently proposed by \citet{Cho2014}. The gated hidden unit is an alternative
to the conventional {\it simple} units such as an element-wise $\tanh$.  This
gated unit is similar to a long short-term memory (LSTM) unit proposed earlier
by \citet{Hochreiter+Schmidhuber-1997}, sharing with it the ability to better
model and learn long-term dependencies. This is made possible by having
computation paths in the unfolded RNN for which the product of derivatives is
close to 1.  These paths allow gradients to flow backward easily without
suffering too much from the vanishing
effect~\citep{Hochreiter91,Bengio-trnn93,Pascanu+al-ICML2013-small}. It is
therefore possible to use LSTM units instead of the gated hidden unit described
here, as was done in a similar context by \citet{Sutskever2014}.

The new state $s_i$ of the RNN employing $n$ gated hidden units\footnote{
    Here, we show the formula of the decoder. The same formula can be used in
    the encoder by simply ignoring the context vector $c_i$ and the related
    terms.
}
is computed by
\begin{align*}
    s_i = f(s_{i-1}, y_{i-1}, c_i) = (1 - z_i) \circ s_{i-1} + z_i \circ \tilde{s}_{i},
\end{align*}
where $\circ$ is an element-wise multiplication, and $z_i$ is the output of the
update gates (see below). The proposed updated state $\tilde{s}_{i}$ is computed
by
\begin{align*}
    \tilde{s}_{i} = \tanh \left( W e(y_{i - 1}) + U \left[ r_i \circ s_{i - 1} \right] +
    C c_i \right),
\end{align*}
where $e(y_{i-1}) \in \RR^{m}$ is an $m$-dimensional embedding of a word
$y_{i-1}$, and $r_i$ is the output of the reset gates (see below).  When $y_i$
is represented as a $1$-of-$K$ vector, $e(y_i)$ is simply a column of an
embedding matrix $E \in \RR^{m \times K}$. Whenever possible, we omit bias terms
to make the equations less cluttered.

The update gates $z_i$ allow each hidden unit to maintain its previous
activation, and the reset gates $r_i$ control how much and what information from
the previous state should be reset. We compute them by
\begin{align*}
    z_i &= \sigma \left( W_{z} e(y_{i - 1}) + U_{z} s_{i - 1} + C_{z} c_i\right), \\
    r_i &= \sigma \left( W_{r} e(y_{i - 1}) + U_{r} s_{i - 1} + C_{r} c_i\right),
\end{align*}
where $\sigma\left(\cdot\right)$ is a logistic sigmoid function.

At each step of the decoder, we compute the output probability
(Eq.~\eqref{eq:generate_y}) as a multi-layered function~\citep{Pascanu2014rec}.
We use a single hidden layer of maxout units~\citep{Goodfellow2013} and
normalize the output probabilities (one for each word) with a softmax function
(see Eq.~\eqref{eq:annotation_weight}).

\subsubsection{Alignment Model}

The alignment model should be designed considering that the model needs to be
evaluated $T_x \times T_y$ times for each sentence pair of lengths $T_x$ and
$T_y$. In order to reduce computation, we use a single-layer multilayer
perceptron such that 
\begin{align*}
    a(s_{i-1}, h_j) = v_a^{\top} \tanh\left( W_a s_{i-1} + U_a h_j \right),
\end{align*}
where $W_a \in \RR^{n\times n}, U_a \in \RR^{n\times 2n}$ and $v_a \in \RR^{n}$
are the weight matrices. Since $U_a h_j$ does not depend on $i$, we can
pre-compute it in advance to minimize the computational cost. 


\subsection{Detailed Description of the Model}
\subsubsection{Encoder}

In this section, we describe in detail the architecture of the proposed model
(RNNsearch) used in the experiments (see
Sec.~\ref{sec:exp_settings}--\ref{sec:exp_results}).  From here on, we omit all
bias terms in order to increase readability.

The model takes a source sentence of 1-of-K coded word vectors as input
\[
    \vx = (x_1, \ldots, x_{T_x}),\mbox{ }x_i \in \mathbb{R}^{K_x}
\]
and outputs a translated sentence of 1-of-K coded word vectors
\[
    \vy = (y_1, \ldots, y_{T_y}),\mbox{ }y_i \in \mathbb{R}^{K_y},
\]
where $K_x$ and $K_y$ are the vocabulary sizes of source and target languages,
respectively. $T_x$ and $T_y$ respectively denote the lengths of source and
target sentences.

First, the forward states of the bidirectional recurrent neural network (BiRNN)
are computed:
\begin{align*}
    \ora{h}_i =& 
    \begin{cases}
        (1 - \ora{z}_i) \circ \ora{h}_{i-1}  + \ora{z}_i \circ \ora{\underline{h}}_{i} &\mbox{, if }i > 0 \\
        0 &\mbox{, if }i = 0
    \end{cases}
\end{align*}
where
\begin{align*}
    \ora{\underline{h}}_i =& \tanh \left( \ora{W} \ov{E} x_i + \ora{U}\left[ \ora{r}_i \circ \ora{h}_{i-1} \right] \right) \\
    \ora{z}_i =& \sigma\left( \ora{W}_z \ov{E} x_i + \ora{U}_z \ora{h}_{i-1} \right) \\
    \ora{r}_i =& \sigma\left( \ora{W}_r \ov{E} x_i + \ora{U}_r \ora{h}_{i-1} \right).
\end{align*}
$\overline{E} \in \mathbb{R}^{m\times K_x}$ is the word embedding matrix.
$\ora{W}, \ora{W}_z, \ora{W}_r \in \mathbb{R}^{n\times m}$, $\ora{U}, \ora{U}_z,
\ora{U}_r \in \mathbb{R}^{n\times n}$ are weight matrices. $m$ and $n$ are the word
embedding dimensionality and the number of hidden units, respectively. 
$\sigma(\cdot)$ is as usual a logistic sigmoid function.

The backward states $(\ola{h}_1, \cdots, \ola{h}_{T_x})$ are computed similarly.
We share the word embedding matrix $\ov{E}$ between the forward and backward
RNNs, unlike the weight matrices.

We concatenate the forward and backward states to to obtain the annotations
$(h_1, h_2, \cdots, h_{T_x})$, where
\begin{align}
    \label{eq:annotation}
    h_i = \left[ 
        \begin{array}{c}
    \ora{h}_i \\
    \ola{h}_i 
\end{array}
\right]
    \end{align}

\subsubsection{Decoder}

The hidden state $s_i$ of the decoder given the annotations from the encoder is
computed by
\begin{align*}
    s_i =& (1 - z_i) \circ s_{i-1} + z_i \circ \tilde{s}_i,
\end{align*}
where
\begin{align*}
    \tilde{s}_{i} =& \tanh \left( W E y_{i - 1} + U \left[ r_i \circ s_{i - 1} \right] +
    C c_i \right) \\ 
    z_i =& \sigma\left( W_z E y_{i - 1} + U_z s_{i-1} 
    + C_z c_i \right)\\
    r_i =& \sigma\left( W_r E y_{i - 1} + U_r s_{i-1}
    + C_r c_i \right)
\end{align*}
$E$ is the word embedding matrix for the target language.
$W, W_z, W_r \in \mathbb{R}^{n\times m}$, 
$U, U_z, U_r \in \mathbb{R}^{n\times n}$, and
$C, C_z, C_r \in \mathbb{R}^{n\times 2n}$ are weights. Again, $m$ and $n$ are the word
embedding dimensionality and the number of hidden units, respectively.
The initial hidden state $s_0$ is computed by 
$
    s_{0} = \tanh\left( W_s \ola{h}_1 \right),
$
where $W_s \in \RR^{n \times n}$.

The context vector $c_i$ are recomputed at each step by the alignment model:
\begin{align*}
    c_i =& \sum_{j=1}^{T_x} \alpha_{ij} h_j,
\end{align*}
where
\begin{align*}
    \alpha_{ij} =& \frac{\exp\left(e_{ij}\right)}{\sum_{k=1}^{T_x}
    \exp\left(e_{ik}\right)}  \\
    e_{ij} =& v_a^{\top} \tanh\left( W_a s_{i-1} + U_a h_j \right),
\end{align*}
and $h_j$ is the $j$-th annotation in the source sentence (see
Eq.~\eqref{eq:annotation}).  $v_a \in \mathbb{R}^{n'}, W_a \in
\mathbb{R}^{n'\times n}$ and $U_a \in \mathbb{R}^{n'\times  2n}$ are weight
matrices.  Note that the model becomes
RNN Encoder--Decoder~\citep{Cho2014}, if we fix $c_i$ to $\ora{h}_{T_x}$.

With the decoder state $s_{i-1}$, the context $c_{i}$ and the last generated word
$y_{i-1}$, we define the probability of a target word $y_{i}$ as
\begin{align*}
    p(y_{i}|s_i,y_{i-1},c_{i}) \propto& \exp\left(y_{i}^{\top} W_o t_{i}\right),
\end{align*}
where
\begin{align*}
    t_i =&  \left[ \max\left\{\tilde{t}_{i, 2j-1}, \tilde{t}_{i,2j}\right\}
    \right]_{j=1,\ldots,l}^{\top}
\end{align*}
and $\tilde{t}_{i,k}$ is the $k$-th element of a vector $\tilde{t}_i$ which is
computed by
\begin{align*}
    \tilde{t}_{i} =& U_o s_{i - 1} + V_o E y_{i-1} + C_o c_i.
\end{align*}
$W_o \in \mathbb{R}^{K_y\times  l}$, $U_o \in \mathbb{R}^{2l\times n}$, $V_o \in
\mathbb{R}^{2l\times m}$ and $C_o \in \mathbb{R}^{2l\times 2n}$ are weight
matrices. This can be understood as having a deep output~\citep{Pascanu2014rec}
with a single maxout hidden layer~\citep{Goodfellow2013}.

\subsubsection{Model Size}

For all the models used in this paper, the size of a hidden layer $n$ is 1000,
the word embedding dimensionality $m$ is 620 and the size of the maxout hidden
layer in the deep output $l$ is 500. The number of hidden units in the alignment
model $n'$ is 1000.

\begin{table}[t]
    \centering                                                                                         
    \begin{tabular}{c|cccccc}                                                                           
        Model & Updates {\scriptsize ($\times 10^5$)} & Epochs 
        & Hours & GPU & Train NLL & Dev. NLL \\
        \hline                                                           
        \hline
        RNNenc-30 & 8.46 & 6.4 & 109 & {\small TITAN BLACK} & 28.1 & 53.0 \\                                                
        RNNenc-50 & 6.00 & 4.5 & 108 & {\small Quadro K-6000} & 44.0 & 43.6 \\
        \hline
        RNNsearch-30 & 4.71 & 3.6 & 113 & {\small TITAN BLACK} & 26.7 & 47.2 \\
        RNNsearch-50 & 2.88 & 2.2 & 111 & {\small Quadro K-6000} & 40.7 & 38.1 \\
        \hline
    RNNsearch-50$^\star$ & 6.67 & 5.0 & 252 & {\small Quadro K-6000} & 36.7 & 35.2 \\
    \end{tabular}                                                                                      
    \caption{Learning statistics and relevant information. Each update
        corresponds to updating the parameters once using a single minibatch. 
        One epoch is one pass through the training set.
        NLL is the average conditional log-probabilities of the
        sentences in either the training set or the development set. Note that
    the lengths of the sentences differ.}
    \label{tbl:stat}                                                                                   
\end{table}                                                                                            


\section{Training Procedure}
\label{sec:training_detail}

\subsection{Parameter Initialization}

We initialized the recurrent weight matrices $U, U_z, U_r, \ola{U}, \ola{U}_z,
\ola{U}_r, \ora{U}, \ora{U}_z$ and $\ora{U}_r$ as random orthogonal matrices.
For $W_a$ and $U_a$, we initialized them by sampling each element from the
Gaussian distribution of mean $0$ and variance $0.001^2$. All the elements of
$V_a$ and all the bias vectors were initialized to zero. Any other weight matrix
was initialized by sampling from the Gaussian distribution of mean $0$ and
variance $0.01^2$.

\subsection{Training}

We used the stochastic gradient descent (SGD) algorithm.
Adadelta~\citep{Zeiler2012} was used to automatically adapt the learning rate of
each parameter ($\epsilon=10^{-6}$ and $\rho=0.95$). We explicitly normalized
the $L_2$-norm of the gradient of the cost function each time to be at most a
predefined threshold of $1$, when the norm was larger than the
threshold~\citep{Pascanu2013}.  Each SGD update direction was computed with a
minibatch of 80 sentences. 

At each update our implementation requires time proportional to the length of
the longest sentence in a minibatch. Hence, to minimize the waste of
computation, before every 20-th update, we retrieved 1600 sentence pairs, sorted them
according to the lengths and split them into 20 minibatches. The training data
was shuffled once before training and was traversed sequentially in this manner.

In Tables~\ref{tbl:stat} we present the statistics related to training all the
models used in the experiments.


\section{Translations of Long Sentences}
\label{sec:long_translation}

\begin{table}[htp]
    \begin{minipage}{0.99\textwidth}
        \small
        \centering
        \begin{tabular}{p{1.9cm} | p{12cm}}
%Source & Everything can't be sorted out in one appointment, but you know you can count on the psychiatrist, call on him if needed, that you have not been abandoned to yourself.
%\\
%\hline Reference & Ce n'est pas une s\'eance qui fait tout mais on sait qu'on peut compter sur lui, le rappeler si besoin, qu'on n'est pas livr\'es à nous-m\^emes.
%\\
%\hline RNNenc-50 & Tout ne peut pas \^etre inscrit dans une nomination, mais vous savez que vous pouvez compter sur le psychiatre, si vous le d\'esirez, si vous n'avez pas \'et\'e abandonn\'es.
%\\
%\hline RNNsearch-50 & Tout ne peut pas \^etre r\'egl\'e par une nomination, mais vous savez que vous pouvez compter sur le psychiatre, qu'il vous demande si cela est n\'ecessaire, que vous n'avez pas \'et\'e abandonn\'e.
%\\
%\hline Google \mbox{Translate} & Tout ne peut pas \^etre rgl en un seul rendez-vous, mais
%vous savez que vous pouvez compter sur le psychiatre, faire appel à lui en cas
%de besoin, que vous n'avez pas t abandonn à vous-m\^eme.
%\\
%\hline
%\multicolumn{2}{c}{} \\
\hline Source & An admitting privilege is the right of a doctor to admit a patient to a hospital or a medical centre to carry out a diagnosis or a procedure, based on his status as a health care worker at a hospital.
\\
\hline Reference & Le privilège d'admission est le droit d'un m\'edecin, en vertu de son statut de membre soignant d'un h\^opital, d'admettre un patient dans un h\^opital ou un centre m\'edical afin d'y d\'elivrer un diagnostic ou un traitement.
\\
\hline RNNenc-50 & Un privilège d'admission est le droit d'un m\'edecin de reconnaître un patient à l'h\^opital ou un centre m\'edical d'un diagnostic ou de prendre un diagnostic en fonction de son \'etat de sant\'e.
\\
\hline RNNsearch-50 & Un privilège d'admission est le droit d'un m\'edecin d'admettre un patient à un h\^opital ou un centre m\'edical pour effectuer un diagnostic ou une proc\'edure, selon son statut de travailleur des soins de sant\'e à l'h\^opital.
\\
\hline Google \mbox{Translate} & Un privilège admettre est le droit d'un m\'edecin
d'admettre un patient dans un h\^opital ou un centre m\'edical pour effectuer un
diagnostic ou une proc\'edure, fond\'ee sur sa situation en tant que travailleur de
soins de sant\'e dans un h\^opital.
\\
\hline
\multicolumn{2}{c}{} \\
%\hline Source & For the third quarter ended September 30 , Bombardier 's net profit fell to \$ 147 million , or 8 cents per share , from \$ 172 million , or 9 cents per share a year earlier .
%\\
%\hline Reference & Au troisième trimestre clos le 30 septembre , le b\'en\'efice net de Bombardier a chut\'e à 147 M \$ , ou 8 cents par action , par rapport à 172 M \$ , ou 9 cents par action , un an plus t\^ot .
%\\
%\hline RNNenc-50 & Pour le troisième trimestre termin\'e le 30 septembre , le b\'en\'efice net de Bombardier a chut\'e de 147 milliards de dollars , soit 8 cents , soit de 172 millions de dollars , soit 9 millions de dollars par an .
%\\
%\hline RNNsearch-50 & Pour le troisième trimestre , le 30 septembre , le b\'en\'efice net de Bombardier est tomb\'e à 147 millions de dollars , soit 8 cents par action , soit de 172 millions de dollars , soit 9 cents par action par rapport à l' ann\'ee pr\'ec\'edente .
%\\
%\hline
%\multicolumn{2}{c}{} \\
\hline Source & 
This kind of experience is part of Disney's efforts to "extend
the lifetime of its series and build new relationships with audiences via
digital platforms that are becoming ever more important," he added.
\\
\hline Reference & Ce type d'exp\'erience entre dans le cadre des efforts de Disney pour "\'etendre la dur\'ee de vie de ses s\'eries et construire de nouvelles relations avec son public grâce à des plateformes num\'eriques qui sont de plus en plus importantes", a-t-il ajout\'e.
\\
\hline RNNenc-50 & Ce type d'exp\'erience fait partie des initiatives du Disney pour "prolonger la dur\'ee de vie de ses nouvelles et de d\'evelopper des liens avec les lecteurs num\'eriques qui deviennent plus complexes.
\\
\hline RNNsearch-50 & Ce genre d'exp\'erience fait partie des efforts de Disney pour "prolonger la dur\'ee de vie de ses s\'eries et cr\'eer de nouvelles relations avec des publics via des plateformes num\'eriques de plus en plus importantes", a-t-il ajout\'e.
\\
\hline Google \mbox{Translate} & Ce genre d'exp\'erience fait partie des efforts de Disney
à ``\'etendre la dur\'ee de vie de sa s\'erie et construire de nouvelles relations avec
le public par le biais des plates-formes num\'eriques qui deviennent de plus en
plus important'', at-il ajout\'e.
\\
\hline
\multicolumn{2}{c}{} \\
\hline Source & 
In a press conference on Thursday, Mr Blair stated that there was nothing in
this video that might constitute a "reasonable motive" that could lead to
criminal charges being brought against the mayor.
\\
\hline Reference & En conf\'erence de presse, jeudi, M. Blair a affirm\'e qu'il n'y avait rien dans cette vid\'eo qui puisse constituer des "motifs raisonnables" pouvant mener au d\'ep\^ot d'une accusation criminelle contre le maire.
\\
\hline RNNenc-50 & Lors de la conf\'erence de presse de jeudi, M. Blair a dit qu'il n'y avait rien dans cette vid\'eo qui pourrait constituer une "motivation raisonnable" pouvant entraîner des accusations criminelles port\'ees contre le maire.
\\
\hline RNNsearch-50 & Lors d'une conf\'erence de presse jeudi, M. Blair a d\'eclar\'e qu'il n'y avait rien dans cette vid\'eo qui pourrait constituer un "motif raisonnable" qui pourrait conduire à des accusations criminelles contre le maire.
\\
\hline Google \mbox{Translate} & 
Lors d'une conf\'erence de presse jeudi, M. Blair a d\'eclar\'e qu'il n'y avait rien
dans cette vido qui pourrait constituer un "motif raisonnable" qui pourrait
mener à des accusations criminelles portes contre le maire.
\\
\hline
        \end{tabular}
    \end{minipage}
    \caption{The translations generated by RNNenc-50 and RNNsearch-50 from long
        source sentences (30 words or more) selected from the test set. For each
        source sentence, we also show the gold-standard translation.  The
        translations by Google Translate were made on 27 August 2014.
}

\label{tab:translations}
\end{table}


\end{document}

\section{Conclusion and future work}
\label{diss}
In this paper we propose a Quality Aware Network (QAN) for set-to-set recognition. It automatically learns the concept of quality for each sample in a set without supervised signal and aggregates the most discriminative samples to generate set representation. We theoretically and experimentally demonstrate that the quality predicted by network is beneficial to set representation and better than human labelled. 

QAN can be seen as an attention model that pay attention to high quality elements in a image set. However, an image with poor quality may still has some discriminative regions. Considering this, our future work will explore a fine-grained quality aware network that pay attention to high quality regions instead of high quality images in a image set.

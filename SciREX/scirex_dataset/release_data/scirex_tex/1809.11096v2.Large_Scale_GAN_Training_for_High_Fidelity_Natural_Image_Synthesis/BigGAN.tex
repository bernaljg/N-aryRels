\documentclass{article} % For LaTeX2e
\usepackage{iclr2019_conference,times}

% Optional math commands from https://github.com/goodfeli/dlbook_notation.
\input{math_commands.tex}

\usepackage{hyperref}
\usepackage{listings}
\usepackage{url}
\usepackage{soul}
\usepackage{pifont} % Source of checkmark and red X
\usepackage{graphicx} % For figures
\usepackage{caption} % For subfigures
\usepackage{subcaption} % For the architectural tables
\usepackage{booktabs} % For the architecture tables
\usepackage{multirow}
\usepackage{comment} % For commenting out blocks
\usepackage{fmtcount} % For counting appendices
\usepackage[titletoc,title]{appendix}

\usepackage{algorithmic} % For algorithms
\usepackage{algorithm} % For algorithms


\title{Large Scale GAN Training for\\ High Fidelity Natural Image Synthesis}

\author{Andrew Brock\thanks{Work done at DeepMind} \ \thanks{Equal contribution} \\
Heriot-Watt University\\
\texttt{ajb5@hw.ac.uk} \\
\And
Jeff Donahue\footnotemark[2] \\
DeepMind \\
\texttt{jeffdonahue@google.com} \\
\And
Karen Simonyan\footnotemark[2] \\
DeepMind \\
\texttt{simonyan@google.com}
}

\newcommand{\fix}{\marginpar{FIX}}
\newcommand{\new}{\marginpar{NEW}}
\newcommand{\bbR}{\mathbb{R}}
\newcommand{\bmb}{{\bm b}}
\newcommand{\bms}{{\bm s}}
\newcommand{\bmu}{{\bm u}}
\newcommand{\bmv}{{\bm v}}
\newcommand{\bmw}{{\bm w}}
\newcommand{\bmx}{{\bm x}}
\newcommand{\bmh}{{\bm h}}
\newcommand{\bmy}{{\bm y}}
\newcommand{\bmz}{{\bm z}}
\newcommand{\bmm}{{\bm m}}
\newcommand{\bmSigma}{{\bm \Sigma}}
\newcommand{\bmLambda}{{\bm \Lambda}}
\newcommand{\bmxi}{{\bm \xi}}
\newcommand{\bmzero}{{\bm 0}}
\newcommand{\bmg}{{\bm g}}
\newcommand{\bmeta}{{\bm \eta}}
\newcommand{\bmdelta}{{\bm \delta}}
\newcommand{\bmgamma}{{\bm \gamma}}
\newcommand{\bmmu}{{\bm \mu}}
\newcommand{\bmphi}{{\bm \phi}}
\newcommand{\set}[1]{\{#1\}}
%\newcommand{\E}{\mathop{\mathrm{E}}}

% Subfigure
\newcommand{\subf}[2]{%
  {\small\begin{tabular}[b]{@{}c@{}}
  #1\\#2
  \end{tabular}}%
}

% Checkmark and red X
\definecolor{mygreen}{rgb}{0.032, 0.6392, 0.2039}
\newcommand{\cmark}{\textcolor{mygreen}{\ding{51}}}%
\newcommand{\xmark}{\textcolor{red}{\ding{55}}}%

% Generator and Discriminator Symbols
\newcommand{\gen}{\textbf{\texttt{G}}}
\newcommand{\discr}{\textbf{\texttt{D}}}

% "phantom" zero to align numbers in tables
\newcommand{\phz}{\phantom{0}}

\iclrfinalcopy % Uncomment for camera-ready version, but NOT for submission.
\begin{document}


\maketitle

\begin{abstract}
Despite recent progress in generative image modeling, successfully generating high-resolution, diverse samples from complex datasets such as ImageNet remains an elusive goal. To this end, we train Generative Adversarial Networks at the largest scale yet attempted, and study the instabilities specific to such scale. We find that applying orthogonal regularization to the generator renders it amenable to a simple ``truncation trick,'' allowing fine control over the trade-off between sample fidelity and variety by reducing the variance of the Generator's input. Our modifications lead to models which set the new state of the art in class-conditional image synthesis. When trained on ImageNet at 128$\times$128 resolution, our models (BigGANs) achieve an Inception Score (IS) of 166.5 and Fr\'echet Inception Distance (FID) of 7.4, improving over the previous best IS of 52.52 and FID of 18.65.


\end{abstract}


\section{Introduction} 
\label{intro}

\begin{figure}[htbp]
\centering
\setlength{\tabcolsep}{1pt}
\begin{tabular}{cccc}
\includegraphics[width=0.24\textwidth]{images/samples0/dog3.jpg} & 
\includegraphics[width=0.24\textwidth]{images/samples0/landscape0.jpg} &
\includegraphics[width=0.24\textwidth]{images/samples0/Monarch0.jpg} & 
\includegraphics[width=0.24\textwidth]{images/samples0/Burger0.jpg} 
\end{tabular}
\caption{Class-conditional samples generated by our model.}
\label{samples0}
\end{figure}


The state of generative image modeling has advanced dramatically in recent years, with Generative Adversarial Networks (GANs, \citet{goodfellow2014gans}) at the forefront of efforts to generate high-fidelity, diverse images with models learned directly from data. GAN training is dynamic, and sensitive to nearly every aspect of its setup (from optimization parameters to model architecture), but a torrent of research has yielded empirical and theoretical insights enabling stable training in a variety of settings. Despite this progress, the current state of the art in conditional ImageNet modeling \citep{zhang2018sagan} achieves an Inception Score \citep{salimans2016improved} of 52.5, compared to 233 for real data.

In this work, we set out to close the gap in fidelity and variety between images generated by GANs and real-world images from the ImageNet dataset. We make the following three contributions towards this goal:

\begin{itemize}
  \item We demonstrate that GANs benefit dramatically from scaling, and train models with two to four times as many parameters and eight times the batch size compared to prior art. We introduce two simple, general architectural changes that improve scalability, and modify a regularization scheme to improve conditioning, demonstrably boosting performance.

  \item As a side effect of our modifications, our models become amenable to the ``truncation trick,'' a simple sampling technique that allows explicit, fine-grained control of the trade-off between sample variety and fidelity.

  \item We discover instabilities specific to large scale GANs, and characterize them empirically. Leveraging insights from this analysis, we demonstrate that a combination of novel and existing techniques can reduce these instabilities, but complete training stability can only be achieved at a dramatic cost to performance.

  
  
\end{itemize}

Our modifications substantially improve class-conditional GANs. When trained on ImageNet at 128$\times$128 resolution, our models (BigGANs) improve the state-of-the-art Inception Score (IS) and Fr\'echet Inception Distance (FID) from 52.52 and 18.65 to 166.5 and 7.4 respectively.
We also successfully train BigGANs on ImageNet at 256$\times$256 and 512$\times$512 resolution, and achieve IS and FID of 232.5 and 8.1 at 256$\times$256 and IS and FID of 241.5 and 11.5 at 512$\times$512. Finally, we train our models on an even larger
dataset -- JFT-300M --
%internal dataset,
and demonstrate that our design choices transfer well from ImageNet. Code and weights for our pretrained generators are publicly available 
% \textcolor{blue}{\href{https://tfhub.dev/s?q=biggan}{here.}}
\footnote{\scriptsize \url{https://tfhub.dev/s?q=biggan}}.


\section{Background} 
\label{background}

A Generative Adversarial Network (GAN) involves Generator (\gen{}) and Discriminator (\discr{}) networks whose purpose, respectively, is to map random noise to samples and discriminate real and generated samples. Formally, the GAN objective, in its original form \citep{goodfellow2014gans} involves finding a Nash equilibrium to the following two player min-max problem:

\begin{align}
	\min_{G} \max_{D} \E_{x\sim q_{\rm data}({\bm x})} [ \log D({\bm x})] +  \E_{{\bm z}\sim p({\bm z})} [\log(1-D(G({\bm z})))], \label{eq:advloss}
\end{align}

where $\bmz \in \bbR^{d_z}$ is a latent variable drawn from distribution $p(\bmz)$ such as $\mathcal{N}(0, I)$ or $\mathcal{U}[-1, 1]$. When applied to images, \gen{} and \discr{} are usually convolutional neural networks \citep{radford2016dcgan}. Without auxiliary stabilization techniques, this training procedure is notoriously brittle, requiring finely-tuned hyperparameters and architectural choices to work at all. 

Much recent research has accordingly focused on modifications to the vanilla GAN procedure to impart stability, drawing on a growing body of empirical and theoretical insights \citep{nowozin2016fgan, sonderby2017map, fedus2018many}. One line of work is focused on changing the objective function \citep{arjovsky2017wgan, mao2016lsgan, lim2017geometric, bellemare2017cramergan, salimans2016otgan}  to encourage convergence. Another line is focused on constraining \discr{} through gradient penalties \citep{gulrajani2017improved, kodali2014dragan, mescheder2018r1gp} or normalization \citep{miyato2018spectral}, both to counteract the use of unbounded loss functions and ensure \discr{} provides gradients everywhere to \gen{}. 

Of particular relevance to our work is Spectral Normalization \citep{miyato2018spectral}, which enforces Lipschitz continuity on \discr{} by normalizing its parameters with running estimates of their first singular values, inducing backwards dynamics that adaptively regularize the top singular direction. Relatedly \citet{odena2018causal} analyze the condition number of the Jacobian of \gen{} and find that performance is dependent on \gen{}'s conditioning. \citet{zhang2018sagan} find that employing Spectral Normalization in \gen{} improves stability, allowing for fewer \discr{} steps per iteration. We extend on these analyses to gain further insight into the pathology of GAN training.




Other works focus on the choice of architecture, such as SA-GAN \citep{zhang2018sagan} which adds the self-attention block from \citep{wang2018nonlocal} to improve the ability of both \gen{} and \discr{} to model global structure. ProGAN \citep{karras2018progan} trains high-resolution GANs in the single-class setting by training a single model across a sequence of increasing resolutions.

In conditional GANs \citep{mirza2014conditional} class information can be fed into the model in various ways.
In \citep{odena2017acgan} it is provided to \gen{} by concatenating a 1-hot class vector to the noise vector, and the objective is modified to encourage conditional samples to maximize the corresponding class probability predicted by an auxiliary classifier. \citet{devries2017modulating} and \citet{dumoulin2017artistic} modify the way class conditioning is passed to \gen{} by supplying it with class-conditional gains and biases in BatchNorm \citep{ioffe2015batchnorm} layers. In \cite{miyato2018cgans}, \discr{} is conditioned by using the cosine similarity between its features and a set of learned class embeddings as additional evidence for distinguishing real and generated samples, effectively encouraging generation of samples whose features match a learned class prototype.


Objectively evaluating implicit generative models is difficult \citep{theis2015note}. A variety of works have proposed heuristics for measuring the sample quality of models without tractable likelihoods \citep{salimans2016improved, heusel2017ttur, bińkowski2018demystifying, wu2017ais}. Of these, the Inception Score (IS, \citet{salimans2016improved}) and Fr\'echet Inception Distance (FID, \citet{heusel2017ttur}) have become popular despite their 
notable flaws \citep{barratt2018note}. We employ them as approximate measures of sample quality, and to enable comparison against previous work.



\section{Scaling Up GANs}
\begin{table}[tbp]
\small
\begin{center}\begin{tabular}{c|c|c|c|c|c|c|c|c} 
\hline
 Batch & Ch. & Param (M)  & Shared & Skip-$z$ & Ortho. & Itr $\times 10^3$ & FID  & IS \\

\hline   256& 64 & 81.5  &\multicolumn{3}{c|}{SA-GAN Baseline} & $1000$ &  $18.65$  & $ 52.52 $\\ % Baseline, params:  G: 42017412, D: 39448258, total 81465670
\hline   512 & 64 & 81.5  & \xmark & \xmark & \xmark & $1000$ &  $15.30$  & $58.77 (\pm 1.18) $\\ % setting 1, params:  G: 42017412, D: 39448258, total 81465670

\hline   1024 & 64 & 81.5  & \xmark & \xmark &\xmark & $1000$ &  $14.88$ & $63.03 (\pm 1.42)$\\ % setting 2, params: G: 42017412, D: 39448258, total 81465670

\hline   2048 & 64 & 81.5  & \xmark & \xmark &\xmark & $732$ &  $12.39$ & $76.85 (\pm 3.83)$\\ % setting 3, params: G: 42017412, D: 39448258, total 81465670
  
\hline   2048 & 96 & 173.5  & \xmark & \xmark &\xmark & $295 (\pm 18)$ &  $9.54 (\pm 0.62)$ & $92.98 (\pm 4.27)$\\ % setting 2.1, params: G: 85556164, D: 87982370, total 173538534

\hline   2048 & 96 & 160.6  & \cmark & \xmark &\xmark & $185 (\pm 11) $ &  $9.18 (\pm 0.13)$ & $94.94 (\pm 1.32)$\\ % setting 2.2, params:  G: 72664516, D: 87982370, total 160646886

\hline   2048 & 96 & 158.3  & \cmark & \cmark &\xmark & $152 (\pm 7)$ & $8.73 (\pm 0.45)$ & $98.76 (\pm 2.84)$\\ % setting 2.3, params: G: 70305988, D: 87982370, total 158288358

\hline   2048 & 96 & 158.3  & \cmark & \cmark &\cmark & $165 (\pm 13)$ &  $8.51 (\pm 0.32)$ & $99.31 (\pm 2.10)$\\ %  setting 2.4, params: G: 70305988, D: 87982370, total 158288358

\hline   2048 & 64 & 71.3  & \cmark & \cmark &\cmark & $371 (\pm 7) $ &  $10.48 (\pm 0.10) $ & $ 86.90 (\pm 0.61)$\\ % setting 2.7, params: G: 31850628, D: 39448258, total 71298886

\hline

\end{tabular}
\end{center}

\caption{\label{ablation_table} Fr\'echet Inception Distance (FID, lower is better) and Inception Score (IS, higher is better) for ablations of our proposed modifications. \textit{Batch} is batch size, \textit{Param} is total number of parameters, \textit{Ch.} is the channel multiplier representing the number of units in each layer, \textit{Shared} is using shared embeddings, 
% \textit{Skip-$z$} is using a hierarchical latent space,
\textit{Skip-$z$} is using skip connections from the latent to multiple layers, 
\textit{Ortho.} is Orthogonal Regularization, and \textit{Itr} indicates if the setting is stable to $10^6$ iterations, or it collapses at the given iteration. Other than rows 1-4, results are computed across 8 random initializations.
}
\end{table}



In this section, we explore methods for scaling up GAN training to reap the performance benefits of larger models and larger batches. As a baseline, we employ the SA-GAN architecture of \citet{zhang2018sagan}, which uses the hinge loss \citep{lim2017geometric, tran2017hierarchical} GAN objective. We provide class information to \gen{} with class-conditional BatchNorm \citep{dumoulin2017artistic, devries2017modulating} and to \discr{} with projection \citep{miyato2018cgans}. The optimization settings follow \citet{zhang2018sagan} (notably employing Spectral Norm in \gen{}) with the modification that we halve the learning rates and take two \discr{} steps per \gen{} step.  For evaluation, we employ moving averages of \gen{}'s weights following \citet{karras2018progan, mescheder2018r1gp, yazici2018ema}, with a decay of $0.9999$. We use Orthogonal Initialization \citep{saxe2014ortho}, whereas previous works used $\mathcal{N}(0,0.02I)$ \citep{radford2016dcgan} or Xavier initialization \citep{glorot2010init}. Each model is trained on 128 to 512 cores of a Google TPUv3
Pod~\citep{tpu}, and computes BatchNorm statistics in \gen{} across all devices, rather than per-device as is typical. We find progressive growing \citep{karras2018progan} unnecessary even for our 512$\times$512 models. Additional details are in Appendix~\ref{appendix_experimental_details}.

We begin by increasing the batch size for the baseline model, and immediately find tremendous benefits in doing so. Rows 1-4 of Table~\ref{ablation_table} show that simply increasing the batch size by a factor of 8 improves the state-of-the-art IS by 46\%. We conjecture that this is a result of each batch covering more modes, providing better gradients for both networks. One notable side effect of this scaling is that our models reach better final performance in fewer iterations, but become unstable and undergo complete training collapse. We discuss the causes and ramifications of this in Section~\ref{sec:analysis}. For these experiments, we report scores from checkpoints saved just before collapse.

We then increase the width (number of channels) in each layer by 50\%, approximately doubling the number of parameters in both models. This leads to a further IS improvement of 21\%, which we posit is due to the increased capacity of the model relative to the complexity of the dataset. 
% Doubling the depth does not appear to have the same effect on ImageNet models, instead degrading performance.
Doubling the depth did not initially lead to improvement -- we addressed this later in the BigGAN-deep model, which uses a different residual block structure.


We note that class embeddings $c$ used for the conditional BatchNorm layers in \gen{} contain a large number of weights. Instead of having a separate layer for each embedding~\citep{miyato2018spectral,zhang2018sagan}, we opt to use a shared embedding, which is linearly projected to each layer's gains and biases~\citep{perez2018film}. This reduces computation and memory costs, and  improves training speed (in number of iterations required to reach a given performance) by 37\%.
Next, we 
% employ a variant of hierarchical latent spaces, where the noise vector $z$ is fed into multiple layers of \gen{} rather than just the initial layer.
add direct skip connections (skip-$z$) from the noise vector $z$ to multiple layers of \gen{} rather than just the initial layer.
The intuition behind this design is to allow \gen{} to use the latent space to directly influence features at different resolutions and levels of hierarchy. 
% For our architecture, this is easily accomplished by splitting $z$ into one chunk per resolution, and concatenating each chunk to the conditional vector $c$ which gets projected to the BatchNorm gains and biases.
In BigGAN, this is accomplished by splitting $z$ into one chunk per resolution, and concatenating each chunk to the conditional vector $c$ which gets projected to the BatchNorm gains and biases.
In BigGAN-deep, we use an even simpler design, concatenating the entire $z$ with the conditional vector without splitting it into chunks.
Previous works \citep{goodfellow2014gans, denton2015lapgan} have considered variants of this concept; our implementation is a minor modification of this design. 
% Hierarchical latents improve memory and compute costs (primarily by reducing the parametric budget of the first linear layer), 
Skip-$z$ provides a modest performance improvement of around 4\%, and improves training speed  by a further 18\%.  



\subsection{Trading off variety and fidelity with the Truncation Trick}
\label{subsec_truncation}


\begin{figure}[tbp]
  \centering
  \begin{tabular}{cc}
  \subf{
  \setlength{\tabcolsep}{1pt}
  \begin{tabular}{cccc}
  \includegraphics[width=0.18\textwidth]{images/trunc_figure2/Truncfigure1full5.jpg} & 
\includegraphics[width=0.18\textwidth]{images/trunc_figure2/Truncfigure1full3.jpg} &
\includegraphics[width=0.18\textwidth]{images/trunc_figure2/Truncfigure1full2.jpg} & 
\includegraphics[width=0.18\textwidth]{images/trunc_figure2/Truncfigure1full1.jpg}
 

 
\end{tabular}
}{(a)}

&
 \subf{\includegraphics[width=0.18\textwidth]{images/rainbowdogs0.jpg}}{(b)}
  
 \end{tabular}
\caption{(a) The effects of increasing truncation. From left to right, the threshold is set to 2, 1, 0.5, 0.04. (b) Saturation artifacts from applying truncation to a poorly conditioned model.}

\label{trunc_figure}
\end{figure}


Unlike models which need to backpropagate through their latents, GANs can employ an arbitrary prior $p(z)$, yet the vast majority of previous works have chosen to draw $z$ from either $\mathcal{N}(0, I)$ or $\mathcal{U}[-1, 1]$. We question the optimality of this choice and explore alternatives in Appendix~\ref{appendix_latents}. 

Remarkably, our best results come from using a different latent distribution for sampling than was used in training. Taking a model trained with $z\sim\mathcal{N}(0, I)$ and sampling $z$ from a \textit{truncated normal} (where values which fall outside a range are resampled to fall inside that range) immediately provides a boost to IS and FID. We call this the \textit{Truncation Trick}: truncating a $z$ vector by resampling the values with magnitude above a chosen threshold leads to improvement in individual sample quality at the cost of reduction in overall sample variety. Figure~\ref{trunc_figure}(a) demonstrates this: as the threshold is reduced, and elements of $z$ are truncated towards zero (the mode of the latent distribution), individual samples approach the mode of \gen{}'s output distribution. Related observations about this trade-off were made in \citep{marchesi2017megapixel, pieters2018bachelors}.

This technique allows fine-grained, post-hoc selection of the trade-off between sample quality and variety for a given \gen{}. Notably, we can compute FID and IS for a range of thresholds, obtaining the variety-fidelity curve reminiscent of the precision-recall curve (Figure~\ref{appendix_ISvFID128}). As IS does not penalize lack of variety in class-conditional models, reducing the truncation threshold leads to a direct increase in IS (analogous to precision). FID penalizes lack of variety (analogous to recall) but also rewards precision, so we initially see a moderate improvement in FID, but as truncation approaches zero and variety diminishes, the FID sharply drops. The distribution shift caused by sampling with different latents than those seen in training is problematic for many models. Some of our larger models are not amenable to truncation, producing saturation artifacts (Figure~\ref{trunc_figure}(b)) when fed truncated noise. To counteract this, we seek to enforce amenability to truncation by conditioning \gen{} to be smooth, so that the full space of $z$ will map to good output samples. For this, we turn to Orthogonal Regularization \citep{brock2017photo}, which directly enforces the orthogonality condition:

\begin{equation}
    R_\beta(W) = \beta\|W^\top W - I\|_{\mathrm{F}}^2,
\end{equation}




where $W$ is a weight matrix and $\beta$ a hyperparameter. This regularization is known to often be too limiting \citep{miyato2018spectral}, so we explore several variants designed to relax the constraint while still imparting the desired smoothness to our models. The version we find to work best removes the diagonal terms from the regularization, and aims to minimize the pairwise cosine similarity between filters but does not constrain their norm:


\begin{equation}
\label{eq:ortho3}
    R_\beta(W) = \beta\|W^\top W \odot (\mathbf{1} - I)\|_{\mathrm{F}}^2,
\end{equation}
where $\mathbf{1}$ denotes a matrix with all elements set to $1$.
We sweep $\beta$ values and select $10^{-4}$, finding this small added penalty sufficient to improve the likelihood that our models will be amenable to truncation. Across runs in Table~\ref{ablation_table}, we observe that without Orthogonal Regularization, only 16\% of models are amenable to truncation, compared to 60\% when trained with Orthogonal Regularization.

%Across runs in Table~\ref{ablation_table}, we observe that without Orthogonal Regularization, only 16\% of models are amenable to truncation, compared to 60\% when trained with Orthogonal Regularization.



\subsection{Summary}
We find that current GAN techniques are sufficient to enable scaling to large models and distributed, large-batch training. We find that we can  dramatically improve the state of the art and train models up to 512$\times$512 resolution without need for explicit multiscale methods like \cite{karras2018progan}. Despite these improvements, our models undergo training collapse, necessitating early stopping in practice. In the next two sections we investigate why settings which were stable in previous works become unstable when applied at scale. 


\section{Analysis}
\label{sec:analysis}

 
\begin{figure}[htbp]
\centering
\setlength{\tabcolsep}{1pt}
\begin{tabular}{cc}

\subf{\includegraphics[width=0.48\textwidth]{images/GSV0_newa.jpg}}{(a) \gen{}} &
\subf{\includegraphics[width=0.48\textwidth]{images/DSV0_newa.jpg}}{(b) \discr{}}
\end{tabular}
\caption{A typical plot of the first singular value $\sigma_0$ in the layers of \gen{} (a) and \discr{} (b) before Spectral Normalization. Most layers in \gen{} have well-behaved spectra, but without constraints a small subset grow throughout training and explode at collapse. \discr{}'s spectra are noisier but otherwise better-behaved. Colors from red to violet indicate increasing depth.}
\label{spectra}
\end{figure}


\subsection{Characterizing Instability: The Generator}
\label{subsec:gen_instability}
Much previous work has investigated GAN stability from a variety of analytical angles and on toy problems, but the instabilities we observe occur for settings which are stable at small scale, necessitating direct analysis at large scale. 
We monitor a range of weight, gradient, and loss statistics during training, in search of a metric which might presage the onset of training collapse, similar to \citep{odena2018causal}. We found the top three singular values
$\sigma_0, \sigma_1, \sigma_2$ of each weight matrix to be the most informative.
They can be efficiently computed using the Alrnoldi iteration method~\citep{golub2000eigenvalue}, which extends the power iteration method, used in~\citet{miyato2018spectral}, to estimation of additional singular vectors and values. A clear pattern emerges, as can be seen in Figure~\ref{spectra}(a) and Appendix~\ref{appendix_monitored_stats}: most \gen{} layers have well-behaved spectral norms, but some layers (typically the first layer in \gen{}, which is over-complete and not convolutional) are ill-behaved, with spectral norms that grow throughout training and explode at collapse.



To ascertain if this pathology is a cause of collapse or merely a symptom, we study the effects of imposing additional conditioning on \gen{} to explicitly counteract spectral explosion. First, we directly regularize the top singular values $\sigma_0$ of each weight, either towards a fixed value $\sigma_{reg}$ or towards some ratio $r$ of the second singular value, $r \cdot sg(\sigma_1)$ (with $sg$ the stop-gradient operation to prevent the regularization from increasing $\sigma_1$). Alternatively, we employ a partial singular value decomposition to instead clamp $\sigma_0$.  Given a weight $W$, its first singular vectors $u_0$ and $v_0$, and $\sigma_{clamp}$ the value to which the $\sigma_0$ will be clamped, our weights become:
\begin{equation}
    W = W - \max(0, \sigma_0 - \sigma_{clamp}) v_0 u_0^\top,
\end{equation}
where $\sigma_{clamp}$ is set to either $\sigma_{reg}$ or $r \cdot sg(\sigma_1)$. We observe that both with and without Spectral Normalization these techniques have the effect of preventing the gradual increase and explosion of either $\sigma_0$ or $\frac{\sigma_0}{\sigma_1}$, but even though in some cases they mildly improve performance, no combination prevents training collapse. This evidence suggests that while conditioning \gen{} might improve stability, it is insufficient to ensure stability. We accordingly turn our attention to \discr{}.



\subsection{Characterizing Instability: The Discriminator}
\label{subsec:discr_instability}

As with \gen{}, we analyze the spectra of \discr{}'s weights to gain insight into its behavior, then seek to stabilize training by imposing additional constraints. Figure~\ref{spectra}(b)  displays a typical plot of $\sigma_0$ for \discr{} (with further plots in Appendix~\ref{appendix_monitored_stats}). Unlike \gen{}, we see that the spectra are noisy, $\frac{\sigma_0}{\sigma_1}$ is well-behaved, and the singular values grow throughout training but only jump at collapse, instead of exploding.

The spikes in \discr{}'s spectra might suggest that it periodically receives very large gradients, but we observe that the Frobenius norms are smooth (Appendix~\ref{appendix_monitored_stats}), suggesting that this effect is primarily concentrated on the top few singular directions. We posit that this noise is a result of optimization through the adversarial training process, where \gen{} periodically produces batches which strongly perturb \discr{} . If this spectral noise is causally related to instability, a natural counter is to employ gradient penalties, which explicitly regularize changes in \discr{}'s Jacobian. We explore the $R_1$ zero-centered gradient penalty from \citet{mescheder2018r1gp}:
\begin{equation}
\label{eq:discriminator}
R_1:=
\frac{\gamma}{2} \E_{ p_{\mathcal D}(x)}
\left[
\|\nabla D(x)\|_F^2
\right].
\end{equation}

With the default suggested $\gamma$ strength of 10, training becomes stable and improves the smoothness and boundedness of spectra in both \gen{} and \discr{}, but performance severely degrades, resulting in a 45\% reduction in IS. Reducing the penalty partially alleviates this degradation, but results in increasingly ill-behaved spectra; even with the penalty strength reduced to $1$ (the lowest strength for which sudden collapse does not occur) the IS is reduced by 20\%.
Repeating this experiment with various strengths of Orthogonal Regularization, DropOut \citep{srivastava2014dropout}, and L2 (See Appendix~\ref{appendix_sweeps} for details), reveals similar behaviors for these regularization strategies: with high enough penalties on \discr{}, training stability can be achieved, but at a substantial cost to performance.



We also observe that \discr{}'s loss approaches zero during training, but undergoes a sharp upward jump at collapse (Appendix~\ref{appendix_monitored_stats}).
One possible explanation for this behavior is that \discr{} is overfitting to the training set, memorizing training examples rather than learning some meaningful boundary between real and generated images.  As a simple test for \discr{}'s memorization (related to \cite{gulrajani2017improved}), we evaluate uncollapsed discriminators on the ImageNet training and validation sets, and measure what percentage of samples are classified as real or generated. While the training accuracy is consistently above 98\%, the validation accuracy falls in the range of 50-55\%, no better than random guessing (regardless of regularization strategy). This confirms that \discr{} 
is indeed memorizing the training set;
we deem this in line with \discr{}'s role, which is not explicitly to generalize, but to distill the training data and provide a useful learning signal for \gen{}. Additional experiments and discussion are provided in Appendix~\ref{appendix_additional_discussion}.



\subsection{Summary}
We find that stability does not come solely from \gen{} or \discr{}, but from their interaction through the adversarial training process. While the symptoms of their poor conditioning can be used to track and identify instability, ensuring reasonable conditioning proves necessary for training but insufficient to prevent eventual training collapse. It is possible to enforce stability by strongly constraining \discr{}, but doing so incurs a dramatic cost in performance. With current techniques, better final performance can be achieved by relaxing this conditioning and allowing collapse to occur at the later stages of training, by which time a model is sufficiently trained to achieve good results.

\section{Experiments}
\begin{figure}[htbp]
\centering
\setlength{\tabcolsep}{1pt}
\begin{tabular}{cccc}
\subf{
\includegraphics[width=0.24\textwidth]{images/samples1/thatchroof0.jpg}}{(a) 128$\times$128} & 
\subf{
\includegraphics[width=0.24\textwidth]{images/samples1/1760464_bird0.jpg}}{(b) 256$\times$256} &
\subf{
\includegraphics[width=0.24\textwidth]{images/samples1/512fox0.jpg}
}{(c) 512$\times$512} &
\subf{\includegraphics[width=0.24\textwidth]{images/badsamples/TennisBallDog.png}}{(d)}



\end{tabular}
\caption{Samples from our BigGAN model with truncation threshold 0.5 (a-c) and an example of class leakage in a partially trained model (d).}
\label{results_samples}
\end{figure}

% Results table for Experiments section
% Note that for FIDs we might just report e.g. "6.34 (\pm .21)" instead of (\pm0.21) to save space (can't fit that tasty leading zero =(
\begin{table}[tbp]
\small
\begin{center}\begin{tabular}{c|c|c|c|c|c} 
\hline
Model & Res. &    FID/IS   & (min FID) / IS & FID / (valid IS) & FID / (max IS) \\

\hline \hline
SN-GAN & 128 & $27.62 / 36.80$ & N/A & N/A & N/A \\

\hline 
SA-GAN & 128 & $18.65 / 52.52$ & N/A & N/A & N/A \\

\hline\hline
BigGAN & 128 & $8.7\pm.6/98.8\pm3$ & $7.7\pm.2/126.5\pm 0$  & $9.6\pm.4/166.3\pm1$ & $25\pm2/206\pm2$ \\

\hline 
BigGAN & 256 & $8.7\pm.1/142.3\pm2$ & $7.7\pm.1/178.0\pm5$ & $9.3\pm.3/233.1\pm1$ & $25\pm5/291\pm4$ \\

\hline 
BigGAN & 512 & $8.1 / 144.2$  & $7.6 / 170.3$  & $11.8 / 241.4$ & $27.0 / 275$ \\

\hline \hline
BigGAN-deep & 128 & $5.7 \pm .3/124.5 \pm 2$ & $6.3 \pm .3/148.1 \pm 4$  & $7.4 \pm .6/166.5 \pm 1$ & $25 \pm 2/253 \pm 11$ \\

\hline
BigGAN-deep & 256 & $6.9 \pm .2/171.4 \pm 2$ & $7.0 \pm .1/202.6 \pm 2$ & $8.1 \pm .1/232.5 \pm 2$ & $27 \pm 8/317 \pm 6$ \\

\hline
BigGAN-deep & 512 & $7.5/152.8$  & $7.7/181.4$  & $11.5/241.5$ & $39.7/298$ \\

\hline
\end{tabular}
\end{center}

\caption{\label{results_table} Evaluation of models at different resolutions. We report scores without truncation (Column 3), scores at the best FID (Column 4), scores at the IS of validation data (Column 5), and scores at the max IS (Column 6). Standard deviations are computed over at least three random initializations.}
\end{table}
\subsection{Evaluation on ImageNet}
We evaluate our models on ImageNet ILSVRC 2012~\citep{ILSVRC2015} at 128$\times$128, 256$\times$256, and 512$\times$512 resolutions, employing the settings from Table~\ref{ablation_table}, row 8. 
The samples generated by our models are presented in Figure~\ref{results_samples}, with additional samples in Appendix~\ref{appendix_samples}, and online
\footnote{\scriptsize \url{https://drive.google.com/drive/folders/1lWC6XEPD0LT5KUnPXeve_kWeY-FxH002}}.
We report IS and FID in Table~\ref{results_table}. As our models are able to trade sample variety for quality, it is unclear how best to compare against prior art; we accordingly report values at three settings, with complete curves in Appendix~\ref{appendix_additional_plots}. First, we report the FID/IS values at the truncation setting which attains the best FID. Second, we report the FID at the truncation setting for which our model's IS is the same as that attained by the real validation data, reasoning that this is a passable measure of maximum sample variety achieved while still achieving a good level of ``objectness.''  Third, we report FID at the maximum IS achieved by each model, to demonstrate how much variety must be traded off to maximize quality. In all three cases, our models outperform the previous state-of-the-art IS and FID scores achieved by \citet{miyato2018spectral} and \cite{zhang2018sagan}.

In addition to the BigGAN model introduced in the first version of the paper and used in the majority of experiments (unless otherwise stated), we also present a 4x deeper model (BigGAN-deep) which uses a different configuration of residual blocks. As can be seen from Table~\ref{results_table}, BigGAN-deep substantially outperforms BigGAN across all resolutions and metrics. This confirms that our findings extend to other architectures, and that increased depth leads to improvement in sample quality.
Both BigGAN and BigGAN-deep architectures are described in Appendix~\ref{appendix_architecture}.

Our observation that \discr{} overfits to the training set, coupled with our model's sample quality, raises the obvious question of whether or not \gen{} simply memorizes training points. To test this, we perform class-wise nearest neighbors analysis in pixel space and the feature space of pre-trained classifier networks (Appendix~\ref{appendix_samples}). In addition, we present both interpolations between samples and class-wise interpolations (where $z$ is held constant) in Figures~\ref{appendix_ZCinterp} and \ref{appendix_Cinterp}. Our model convincingly interpolates between disparate samples, and the nearest neighbors for its samples are visually distinct, suggesting that our model does not simply memorize training data.

We note that some failure modes of our partially-trained models are distinct from those previously observed. Most previous failures involve local artifacts \citep{odena2016deconvolution}, images consisting of texture blobs instead of  objects \citep{salimans2016improved}, or the canonical mode collapse. We observe \textit{class leakage}, where images from one class contain properties of another, as exemplified by Figure~\ref{results_samples}(d).  We also find that many classes on ImageNet are more difficult than others for our model; our model is more successful at generating dogs (which make up a large portion of the dataset, and are mostly distinguished by their texture) than crowds  (which comprise a small portion of the dataset and have more large-scale structure). Further discussion is available in Appendix~\ref{appendix_samples}.

\begin{comment}  % ANONYMOUS VERSION
\subsection{Additional evaluation on an internal dataset}
To confirm that our design choices are effective for even larger and more complex and diverse datasets, we also present results of our system on an internal dataset.
The dataset contains 292M images labeled with 8.5K categories -- two orders of magnitude larger than ImageNet. For images with multiple labels, we sample a single label randomly and independently whenever an image is sampled.
To compute IS and FID for the GANs trained on this dataset, we use an Inception v2 classifier~\citep{szegedy2015rethinking} trained on this dataset.
Quantitative results are presented in Table~\ref{jft_table}.
All models are trained with batch size 2048.
We compare an ablated version of our model --
comparable to SA-GAN~\citep{zhang2018sagan} but with the larger batch size --
against a ``full'' version that makes uses of all of the techniques applied to obtain the best results on ImageNet (shared embedding, hierarchical latents, and orthogonal regularization).
Our results show that these techniques substantially improve performance even in the setting of this much larger dataset at the same model capacity (64 base channels).
We further show that for a dataset of this scale, we see significant additional improvements from expanding the capacity of our models to 128 base channels, while for ImageNet GANs that additional capacity was not beneficial.
 
In Figure~\ref{appendix_jft_trunc} (Appendix~\ref{appendix_additional_plots}), we present truncation plots for models trained on this dataset.
Unlike for ImageNet, where truncation limits of $\sigma\approx0$ tend to produce the highest fidelity scores, here IS is typically maximized when the truncation value $\sigma$ ranges from 0.5 to 1.
We suspect that this is at least partially due to the dataset's intra-class variability, as well as the relative complexity of the image distribution, which includes images with multiple objects at a variety of scales.
Interestingly, unlike models trained on ImageNet, where training tends to collapse without heavy regularization (Section~\ref{sec:analysis}), the models trained on the larger dataset remain stable over many hundreds of thousands of iterations.
This suggests that moving beyond ImageNet to larger datasets may partially alleviate GAN stability issues.

The improvement over the baseline GAN model that we achieve on this dataset
without changes to the underlying models or training and regularization techniques (beyond expanded capacity) demonstrates that
our findings extend from ImageNet
to datasets with scale and complexity thus far unprecedented for generative models of images.
\end{comment}

%%%% BEGIN NON-ANONYMOUS VERSION
%\begin{comment}
\begin{table}[tbp]
\small
\begin{center}\begin{tabular}{c|c|c|c|c|c|c|c|c} 
\hline
 %Res. & Batch & Ch. & Param (M) & Shared & Hier. & Ortho. & FID & IS \\
 Ch. & Param (M) & Shared & Skip-$z$ & Ortho. & FID & IS & (min FID) / IS & FID / (max IS) \\
\hline
%\multirow{4}{*}{256}
% G: 254953092, D: 62152898 = 317,105,990
 64 & 317.1 & \xmark & \xmark & \xmark & $48.38$ & $23.27$ & $48.6 / 23.1$ & $49.1 / 23.9$ \\
\hline
% G: 37197444, D: 62152898 = 99,350,342
 64 & 99.4 & \cmark & \cmark & \cmark & $23.48$ & $24.78$ & $22.4 / 21.0$ & $60.9 / 35.8$ \\
\hline
% G: 82107844, D: 125786402 = 207,894,246
 96 & 207.9 & \cmark & \cmark & \cmark & $18.84$ & $27.86$ & $17.1 / 23.3$ & $51.6 / 38.1$ \\
\hline
% G: 144559364, D: 211124610 = 355,683,974
128 & 355.7 & \cmark & \cmark & \cmark & $13.75$ & $30.61$ & $13.0 / 28.0$ & $46.2 / 47.8$ \\
\hline
\end{tabular}
\end{center}
\caption{
\label{jft_table}
%Results on the internal dataset at $256\times256$ resolution.
%The \textit{FID} and \textit{IS} columns report these scores given by the internal dataset-trained Inception v2 classifier with noise distributed as $z \sim \mathcal{N}(0, I)$ (non-truncated).
%The \textit{(min FID) / IS} and \textit{FID / (max IS)} columns report scores at the best FID and IS from a sweep across truncated noise distributions ranging from $\sigma = 0$ to $\sigma = 2$.
%Images from the validation set have an IS of 50.88 and FID of 1.94.
BigGAN results on JFT-300M at $256\times256$ resolution.
The \textit{FID} and \textit{IS} columns report these scores given by the JFT-300M-trained Inception v2 classifier with noise distributed as $z \sim \mathcal{N}(0, I)$ (non-truncated).
The \textit{(min FID) / IS} and \textit{FID / (max IS)} columns report scores at the best FID and IS from a sweep across truncated noise distributions ranging from $\sigma = 0$ to $\sigma = 2$.
Images from the JFT-300M validation set have an IS of 50.88 and FID of 1.94.
}
\end{table}


% train+val set:
% (JFT) inception_score = 50.659447,+/- 1.394818, acc: -1.000000, FID = 1.914317  duration = 139.051551 s
% 
% test set:
% (JFT) inception_score = 50.883789,+/- 4.031054, acc: -1.000000, FID = 1.935230  duration = 139.914789 s
\subsection{Additional evaluation on JFT-300M}
To confirm that our design choices are effective for even larger and more complex and diverse datasets, we also present results of our system on a subset of JFT-300M~\citep{sun17revisiting}.
The full JFT-300M dataset contains 300M real-world images labeled with 18K categories.
Since the category distribution is heavily long-tailed, we subsample the dataset to keep only images with the 8.5K most common labels.
The resulting dataset contains 292M images -- two orders of magnitude larger than ImageNet. For images with multiple labels, we sample a single label randomly and independently whenever an image is sampled.
To compute IS and FID for the GANs trained on this dataset, we use an Inception v2 classifier~\citep{szegedy2015rethinking} trained on this dataset.
Quantitative results are presented in Table~\ref{jft_table}.
All models are trained with batch size 2048.
We compare an ablated version of our model --
comparable to SA-GAN~\citep{zhang2018sagan} but with the larger batch size --
against a ``full'' BigGAN model that makes uses of all of the techniques applied to obtain the best results on ImageNet (shared embedding, skip-$z$, and orthogonal regularization).
Our results show that these techniques substantially improve performance even in the setting of this much larger dataset at the same model capacity (64 base channels).
We further show that for a dataset of this scale, we see significant additional improvements from expanding the capacity of our models to 128 base channels, while for ImageNet GANs that additional capacity was not beneficial.
 
In Figure~\ref{appendix_jft_trunc} (Appendix~\ref{appendix_additional_plots}), we present truncation plots for models trained on this dataset.
Unlike for ImageNet, where truncation limits of $\sigma\approx0$ tend to produce the highest fidelity scores, IS is typically maximized for our JFT-300M models when the truncation value $\sigma$ ranges from 0.5 to 1.
We suspect that this is at least partially due to the intra-class variability of JFT-300M labels, as well as the relative complexity of the image distribution, which includes images with multiple objects at a variety of scales.
Interestingly, unlike models trained on ImageNet, where training tends to collapse without heavy regularization (Section~\ref{sec:analysis}), the models trained on JFT-300M remain stable over many hundreds of thousands of iterations.
This suggests that moving beyond ImageNet to larger datasets may partially alleviate GAN stability issues.

The improvement over the baseline GAN model that we achieve on this dataset
without changes to the underlying models or training and regularization techniques (beyond expanded capacity) demonstrates that
our findings extend from ImageNet
to datasets with scale and complexity thus far unprecedented for generative models of images.
%\end{comment}
%%%% END NON-ANONYMOUS VERSION



\section{Conclusion}
We have demonstrated that Generative Adversarial Networks trained to model natural images of multiple categories highly benefit from scaling up, both in terms of fidelity and variety of the generated samples. As a result, our models set a new level of performance among ImageNet GAN models, improving on the state of the art by a large margin.
We have also presented an analysis of the training behavior of large scale GANs, characterized their stability in terms of the singular values of their weights, and discussed the interplay between stability and performance.


\subsubsection*{Acknowledgments}
We would like to thank 
Kai Arulkumaran, Matthias Bauer, Peter Buchlovsky, Jeffrey Defauw, Sander Dieleman, Ian Goodfellow, Ariel Gordon, Karol Gregor, Dominik Grewe, Chris Jones, Jacob Menick, Augustus Odena, Suman Ravuri, Ali Razavi, Mihaela Rosca, and Jeff Stanway.

\bibliography{iclr2019_conference}
\bibliographystyle{iclr2019_conference}

\begin{appendices}

\newpage
\section{Additional Samples, Interpolations, and Nearest Neighbors from ImageNet models}
\label{appendix_samples}
\begin{figure}[htbp]
\centering
\setlength{\tabcolsep}{1pt}
\begin{tabular}{cccc}
\includegraphics[width=0.24\textwidth]{images/samples0/Peacock0.jpg} & 
\includegraphics[width=0.24\textwidth]{images/samples0/dog0.jpg} &
\includegraphics[width=0.24\textwidth]{images/samples0/wok0.jpg} & 
\includegraphics[width=0.24\textwidth]{images/samples0/Cat0.jpg} \\
\includegraphics[width=0.24\textwidth]{images/samples0/dresser0.jpg} & 
\includegraphics[width=0.24\textwidth]{images/samples0/bird3.jpg} &
\includegraphics[width=0.24\textwidth]{images/samples1/256hamster0.jpg} & 
\includegraphics[width=0.24\textwidth]{images/samples1/256saltshaker0.jpg} \\
\end{tabular}
\caption{Samples generated by our BigGAN model at 256$\times$256 resolution.}
\label{appendix_samples256}
\end{figure}


\begin{figure}[htbp]
\centering
\setlength{\tabcolsep}{1pt}
\begin{tabular}{cccc}
\includegraphics[width=0.24\textwidth]{images/samples1/512wolf0.jpg} & 
\includegraphics[width=0.24\textwidth]{images/samples1/512mushroom0.jpg} &
\includegraphics[width=0.24\textwidth]{images/samples1/512bubble0.jpg} & 
\includegraphics[width=0.24\textwidth]{images/samples1/mocha0.jpg} \\
\includegraphics[width=0.24\textwidth]{images/samples1/512pasta0.jpg} & 
\includegraphics[width=0.24\textwidth]{images/samples1/512fountain0.jpg} &
\includegraphics[width=0.24\textwidth]{images/samples1/512dog0.jpg} & 
\includegraphics[width=0.24\textwidth]{images/samples1/512rocket0.jpg} 
\end{tabular}
\caption{Samples generated by our BigGAN model at 512$\times$512 resolution.}
\label{appendix_samples512}
\end{figure}

\begin{figure}[htbp]
\centering
\setlength{\tabcolsep}{1pt}
\begin{tabular}{cc}
\subf{\begin{tabular}{cc}
\includegraphics[width=0.24\textwidth]{images/samples1/512dog2.jpg} & 
\includegraphics[width=0.24\textwidth]{images/samples1/512crags1.jpg} \\
\includegraphics[width=0.24\textwidth]{images/samples1/512library0.jpg} & 
\includegraphics[width=0.24\textwidth]{images/samples1/512lemon0.jpg} 
\end{tabular}}{(a)} &
\subf{\begin{tabular}{cc}
\includegraphics[width=0.24\textwidth]{images/samples1/featherboa0.jpg} & 
\includegraphics[width=0.24\textwidth]{images/samples1/512baseball0.jpg} \\
\includegraphics[width=0.24\textwidth]{images/samples1/512drummers0.jpg} & 
\includegraphics[width=0.24\textwidth]{images/samples1/512frenchhorn0.jpg}
\end{tabular}}{(b)}
\end{tabular}
\caption{Comparing easy classes (a) with difficult classes (b) at 512$\times$512. Classes such as dogs which are largely textural, and common in the dataset, are far easier to model than classes involving unaligned human faces or crowds. Such classes are more dynamic and structured, and often have details to which  human observers are more sensitive. The difficulty of modeling global structure is further exacerbated when producing high-resolution images, even with non-local blocks.}
\label{appendix_samples_difficulty}
\end{figure}


\begin{figure}[htbp]
\centering
\includegraphics[width=0.98\textwidth]{images/interps0/256ZCInterp2.jpg} \\
\includegraphics[width=0.98\textwidth]{images/interps0/256ZCInterp4.jpg} \\
\includegraphics[width=0.98\textwidth]{images/interps0/256ZCInterp6.jpg} \\
\includegraphics[width=0.98\textwidth]{images/interps0/256ZCInterp9.jpg} \\
\includegraphics[width=0.98\textwidth]{images/interps0/256ZCInterp0.jpg}
\includegraphics[width=0.98\textwidth]{images/interps0/256ZCInterp8.jpg} 
\caption{Interpolations between $z,c$ pairs.}
\label{appendix_ZCinterp}
\end{figure}

\begin{figure}[htbp]
\centering
\includegraphics[width=0.98\textwidth]{images/interps0/256CInterp3.jpg} \\
\includegraphics[width=0.98\textwidth]{images/interps0/256CInterp1.jpg} \\
\includegraphics[width=0.98\textwidth]{images/interps0/256CInterp2.jpg} \\
\includegraphics[width=0.98\textwidth]{images/interps0/256CInterp6.jpg} \\
\includegraphics[width=0.98\textwidth]{images/interps0/256CInterp1a.jpg} \\
\includegraphics[width=0.98\textwidth]{images/interps0/256Cinterp7.jpg} 
\caption{Interpolations between $c$ with $z$ held constant. Pose semantics are frequently maintained between endpoints (particularly in the final row). Row 2 demonstrates that grayscale is encoded in the joint $z,c$ space, rather than in $z$.}
\label{appendix_Cinterp}
\end{figure}



\begin{figure}[htbp]
\centering
\includegraphics[width=0.98\textwidth]{images/neighbors/dog_vgg_fc7.jpg}
\caption{Nearest neighbors in VGG-16-fc7~\citep{simonyan15very} feature space. The generated image is in the top left.}
\label{appendix_nearest_dogVGG}
\end{figure}

\begin{figure}[htbp]
\centering
\includegraphics[width=0.98\textwidth]{images/neighbors/dog_resnet_flat.jpg}
\caption{Nearest neighbors in ResNet-50-avgpool~\citep{he2016resnets} feature space. The generated image is in the top left.}
\label{appendix_nearest_dogResNet}
\end{figure}


\begin{figure}[htbp]
\centering
\includegraphics[width=0.98\textwidth]{images/neighbors/dog_pixel.jpg}
\caption{Nearest neighbors in pixel space. The generated image is in the top left.}
\label{appendix_nearest_dogPixel}
\end{figure}

\begin{figure}[htbp]
\centering
\includegraphics[width=0.98\textwidth]{images/neighbors/burger_vgg_fc7.jpg} 
\caption{Nearest neighbors in VGG-16-fc7~\citep{simonyan15very} feature space. The generated image is in the top left.}
\label{appendix_nearest_burgerVGG}
\end{figure}

\begin{figure}[htbp]
\centering
\includegraphics[width=0.98\textwidth]{images/neighbors/butterfly_resnet_flat.jpg} 
\caption{Nearest neighbors in ResNet-50-avgpool~\citep{he2016resnets} feature space. The generated image is in the top left.}
\label{appendix_nearest_MonarchResNet}
\end{figure}


\clearpage


\newpage
\section{Architectural details}
\label{appendix_architecture}
% \subsection{BigGAN}
In the BigGAN model (Figure~\ref{appendix_resblock_figure}), we use the ResNet \citep{he2016resnets} GAN architecture of \citep{zhang2018sagan}, which is identical to that used by \citep{miyato2018spectral}, but with the channel pattern in \discr{} modified so that the number of filters in the first convolutional layer of each block is equal to the number of output filters (rather than the number of input filters, as in \citet{miyato2018spectral, gulrajani2017improved}). 
We use a single shared class embedding in \gen{}, and skip connections for the latent vector $z$ (skip-$z$). In particular, we employ hierarchical latent spaces, so that the latent vector $z$ is split along its channel dimension into chunks of equal size ($20$-D in our case), and each chunk is concatenated to the shared class embedding and passed to a corresponding residual block as a conditioning vector.
The conditioning of each block is linearly projected to produce per-sample gains and biases for the BatchNorm layers of the block. The bias projections are zero-centered, while the gain projections are centered at $1$. Since the number of residual blocks depends on the image resolution, the full dimensionality of $z$ is 120 for $128\times 128$, 140 for $256\times 256$, and 160 for $512\times 512$ images.

The BigGAN-deep model (Figure~\ref{appendix_resblock_figure_deep}) differs from BigGAN in several aspects.
It uses a simpler variant of skip-$z$ conditioning: instead of first splitting $z$ into chunks, we concatenate the entire $z$ with the class embedding, and pass the resulting vector to each residual block through skip connections.
BigGAN-deep is based on residual blocks with bottlenecks~\citep{he2016resnets}, which incorporate two additional $1 \times 1$ convolutions: the first reduces the number of channels by a factor of $4$ before the more expensive $3 \times 3$ convolutions; the second produces the required number of output channels. 
While BigGAN relies on $1\times 1$ convolutions in the skip connections whenever the number of channels needs to change, in BigGAN-deep we use a different strategy aimed at preserving identity throughout the skip connections. In $\gen{}$, where the number of channels needs to be reduced, we simply retain the first group of channels and drop the rest to produce the required number of channels. In $\discr{}$, where the number of channels should be increased, we pass the input channels unperturbed, and concatenate them with the remaining channels produced by a $1 \times 1$ convolution.
As far as the network configuration is concerned, the discriminator is an exact reflection of the generator. There are two blocks at each resolution (BigGAN uses one), and as a result BigGAN-deep is four times deeper than BigGAN. 
Despite their increased depth, the BigGAN-deep models have significantly fewer parameters mainly due to the bottleneck structure of their residual blocks.
For example, the $128\times128$ BigGAN-deep \gen{} and \discr{} have 50.4M and 34.6M parameters respectively, while the corresponding original BigGAN models have 70.4M and 88.0M parameters.
All BigGAN-deep models use attention at $64 \times 64$ resolution, channel width multiplier $ch=128$, and $z\in \bbR^{128}$.

 
\begin{figure}[htbp]
\centering
\begin{tabular}{c}
\subf{\includegraphics[width=0.25\textwidth]{images/ArchitectureMain-BigGAN.pdf}}{(a)}
\hspace{0.5cm}
\subf{\includegraphics[width=0.38\textwidth]{images/ResBlock-BigGAN-G.pdf}}{(b)}
\hspace{0.5cm}
\subf{\includegraphics[width=0.2\textwidth]{images/ResBlock-BigGAN-D.pdf}}{(c)}
\end{tabular}
\caption{
(a) A typical architectural layout for BigGAN's \gen{}; details are in the following tables. 
(b) A Residual Block (\textit{ResBlock up})  in BigGAN's \gen{}.
(c) A Residual Block (\textit{ResBlock down}) in BigGAN's \discr{}.
}
\label{appendix_resblock_figure}
\end{figure}

\begin{figure}[htbp]
\centering
\begin{tabular}{c}
\subf{\includegraphics[width=0.15\textwidth]{images/ArchitectureMain-BigGAN-deep.pdf}}{(a)} \\
\subf{\includegraphics[width=0.45\textwidth]{images/ResBlock-BigGAN-deep-G.pdf}}{(b)}
\hspace{2cm}
\subf{\includegraphics[width=0.3\textwidth]{images/ResBlock-BigGAN-deep-D.pdf}}{(c)}
\end{tabular}
\caption{
(a) A typical architectural layout for BigGAN-deep's \gen{}; details are in the following tables. 
(b) A Residual Block (\textit{ResBlock up}) in BigGAN-deep's \gen{}.
(c) A Residual Block (\textit{ResBlock down}) in BigGAN-deep's \discr{}.
A \textit{ResBlock} (without \textit{up} or \textit{down}) in BigGAN-deep does not include the \textit{Upsample} or \textit{Average Pooling} layers, and has identity skip connections.
}
\label{appendix_resblock_figure_deep}
\end{figure}

\begin{table}[ht]
         \caption{\label{tab:resnets_imagenet128} BigGAN architecture for $128\times 128$ images. $ch$ represents the channel width multiplier in each network from Table~\ref{ablation_table}.}
          \centering
          \small
          \begin{subtable}{.4\textwidth}
              \centering
              {\begin{tabular}{c}
                  \toprule
                  \midrule
                  $z\in \bbR^{120} \sim \mathcal{N}(0, I)$ \\
                  Embed($y$) $\in \bbR^{128}$ \\
                  \midrule
                  Linear $(20+128) \rightarrow 4 \times 4 \times 16 ch $ \\
                  \midrule
                  ResBlock up $16ch \rightarrow 16ch$ \\
                  \midrule
                  ResBlock up $16ch \rightarrow 8ch$\\
                  \midrule
                  ResBlock up $8ch \rightarrow 4ch$\\
                  \midrule
                  ResBlock up $4ch \rightarrow 2ch$\\
                  \midrule
                  Non-Local Block ($64\times 64$)\\
                  \midrule
                  ResBlock up $2ch \rightarrow ch$\\
                  \midrule
                  BN, ReLU, $3\times 3$ Conv $ch\rightarrow 3$ \\
                  \midrule
                  Tanh\\
                  \midrule
                  \bottomrule
              \end{tabular}}
              \caption{\label{tab:gen_resnet_imagenet_128} Generator}
          \end{subtable}
          \begin{subtable}{.4\textwidth}
              \centering
              {\begin{tabular}{c}
                  \toprule
                  \midrule
                  RGB image $x\in \bbR^{128 \times 128 \times 3}$ \\
                  \midrule
                  ResBlock down $ch \rightarrow 2ch$\\
                  \midrule
                  Non-Local Block ($64\times 64$) \\
                  \midrule
                  ResBlock down $2ch \rightarrow 4ch$\\
                  \midrule
                  ResBlock down $4ch \rightarrow 8ch$\\
                  \midrule
                  ResBlock down $8ch \rightarrow 16ch$\\
                  \midrule
                  ResBlock down $16ch \rightarrow 16ch$\\
                  \midrule
                  ResBlock $16ch \rightarrow 16ch$\\
                  \midrule
                  ReLU, Global sum pooling\\
                  \midrule
                  Embed($y$)$\cdot \bmh$ + (linear $\rightarrow$ 1) \\
                  \midrule
                  \bottomrule
              \end{tabular}}
              \caption{\label{tab:dis_resnet_imagenet_128} Discriminator}
          \end{subtable}
\end{table}

\begin{table}[ht]
         \caption{\label{tab:resnets_imagenet256} BigGAN architecture for $256\times 256$ images. 
         Relative to the $128\times 128$ architecture, we add an additional ResBlock in each network at 16$\times$16 resolution, and move the non-local block in \gen{} to $128\times 128$ resolution. Memory constraints prevent us from moving the non-local block in \discr{}.}
          \centering
          \small
          \begin{subtable}{.4\textwidth}
              \centering
              {\begin{tabular}{c}
                  \toprule
                  \midrule
                  $z\in \bbR^{140} \sim \mathcal{N}(0, I)$ \\
                  Embed($y$) $\in \bbR^{128}$ \\
                  \midrule
                  Linear $(20+128) \rightarrow 4 \times 4 \times 16 ch $ \\
                  \midrule
                  ResBlock up $16ch \rightarrow 16ch$ \\
                  \midrule
                  ResBlock up $16ch \rightarrow 8ch$\\
                  \midrule
                  ResBlock up $8ch \rightarrow 8ch$\\
                  \midrule
                  ResBlock up $8ch \rightarrow 4ch$\\
                  \midrule
                  ResBlock up $4ch \rightarrow 2ch$\\
                  \midrule
                  Non-Local Block ($128\times 128$) \\
                  \midrule
                  ResBlock up $2ch \rightarrow ch$\\
                  \midrule
                  BN, ReLU, $3\times 3$ Conv $ch\rightarrow 3$ \\
                  \midrule
                  Tanh\\
                  \midrule
                  \bottomrule
              \end{tabular}}
              \caption{\label{tab:gen_resnet_imagenet_256} Generator}
          \end{subtable}
          \begin{subtable}{.4\textwidth}
              \centering
              {\begin{tabular}{c}
                  \toprule
                  \midrule
                  RGB image $x\in \bbR^{256 \times 256 \times 3}$ \\
                  \midrule
                  ResBlock down $ch \rightarrow 2ch$\\
                  \midrule
                  ResBlock down $2ch \rightarrow 4ch$\\
                  \midrule
                  Non-Local Block ($64\times 64$) \\
                  \midrule
                  ResBlock down $4ch \rightarrow 8ch$\\
                  \midrule
                  ResBlock down $8ch \rightarrow 8ch$\\
                  \midrule
                  ResBlock down $8ch \rightarrow 16ch$\\
                  \midrule
                  ResBlock down $16ch \rightarrow 16ch$\\
                  \midrule
                  ResBlock $16ch \rightarrow 16ch$\\
                  \midrule
                  ReLU, Global sum pooling\\
                  \midrule
                  Embed($y$)$\cdot \bmh$ + (linear $\rightarrow$ 1) \\
                  \midrule
                  \bottomrule
              \end{tabular}}
              \caption{\label{tab:dis_resnet_imagenet_256} Discriminator}
          \end{subtable}
\end{table}

\begin{table}[ht]
         \caption{\label{tab:resnets_imagenet512} BigGAN architecture for $512\times 512$ images.
         Relative to the $256\times 256$ architecture, we add an additional ResBlock at the $512\times 512$ resolution. Memory constraints force us to move the non-local block in both networks back to  $64\times 64$ resolution as in the $128\times 128$ pixel setting.}
          \centering
          \small
          \begin{subtable}{.4\textwidth}
              \centering
              {\begin{tabular}{c}
                  \toprule
                  \midrule
                  $z\in \bbR^{160} \sim \mathcal{N}(0, I)$ \\
                  Embed($y$) $\in \bbR^{128}$ \\
                  \midrule
                  Linear $(20+128) \rightarrow 4 \times 4 \times 16 ch $ \\
                  \midrule
                  ResBlock up $16ch \rightarrow 16ch$ \\
                  \midrule
                  ResBlock up $16ch \rightarrow 8ch$\\
                  \midrule
                  ResBlock up $8ch \rightarrow 8ch$\\
                  \midrule
                  ResBlock up $8ch \rightarrow 4ch$\\
                   \midrule
                  Non-Local Block ($64\times 64$) \\
                  \midrule
                  ResBlock up $4ch \rightarrow 2ch$\\
                 \midrule
                  ResBlock up $2ch \rightarrow ch$\\
                  \midrule
                  ResBlock up $ch \rightarrow ch$\\
                  \midrule
                  BN, ReLU, $3\times 3$ Conv $ch\rightarrow 3$ \\
                  \midrule
                  Tanh\\
                  \midrule
                  \bottomrule
              \end{tabular}}
              \caption{\label{tab:gen_resnet_imagenet_512} Generator}
          \end{subtable}
          \begin{subtable}{.4\textwidth}
              \centering
              {\begin{tabular}{c}
                  \toprule
                  \midrule
                  RGB image $x\in \bbR^{512 \times 512 \times 3}$ \\
                  \midrule
                  ResBlock down $ch \rightarrow ch$\\
                  \midrule
                  ResBlock down $ch \rightarrow 2ch$\\
                  \midrule
                  ResBlock down $2ch \rightarrow 4ch$\\
                  \midrule
                  Non-Local Block ($64\times 64$) \\
                  \midrule
                  ResBlock down $4ch \rightarrow 8ch$\\
                  \midrule
                  ResBlock down $8ch \rightarrow 8ch$\\
                  \midrule
                  ResBlock down $8ch \rightarrow 16ch$\\
                  \midrule
                  ResBlock down $16ch \rightarrow 16ch$\\
                  \midrule
                  ResBlock $16ch \rightarrow 16ch$\\
                  \midrule
                  ReLU, Global sum pooling\\
                  \midrule
                  Embed($y$)$\cdot \bmh$ + (linear $\rightarrow$ 1) \\
                  \midrule
                  \bottomrule
              \end{tabular}}
              \caption{\label{tab:dis_resnet_imagenet_512} Discriminator}
          \end{subtable}
\end{table}

\begin{table}[ht]
         \caption{\label{tab:deep_resnets_imagenet128} BigGAN-deep architecture for $128\times 128$ images.}
          \centering
          \small
          \begin{subtable}{.4\textwidth}
              \centering
              {\begin{tabular}{c}
                  \toprule
                  \midrule
                  $z\in \bbR^{128} \sim \mathcal{N}(0, I)$ \\
                  Embed($y$) $\in \bbR^{128}$ \\
                  \midrule
                  Linear $(128+128) \rightarrow 4 \times 4 \times 16 ch $ \\
                  \midrule
                  ResBlock $16ch \rightarrow 16ch$ \\
                  \midrule
                  ResBlock up $16ch \rightarrow 16ch$ \\
                  \midrule
                  ResBlock $16ch \rightarrow 16ch$ \\
                  \midrule
                  ResBlock up $16ch \rightarrow 8ch$ \\
                  \midrule
                  ResBlock $8ch \rightarrow 8ch$ \\
                  \midrule
                  ResBlock up $8ch \rightarrow 4ch$ \\
                  \midrule
                  ResBlock $4ch \rightarrow 4ch$ \\
                  \midrule
                  ResBlock up $4ch \rightarrow 2ch$ \\
                  \midrule
                  Non-Local Block ($64\times 64$)\\
                  \midrule
                  ResBlock $2ch \rightarrow 2ch$ \\
                  \midrule
                  ResBlock up $2ch \rightarrow ch$ \\
                  \midrule
                  BN, ReLU, $3\times 3$ Conv $ch\rightarrow 3$ \\
                  \midrule
                  Tanh\\
                  \midrule
                  \bottomrule
              \end{tabular}}
              \caption{\label{tab:deep_gen_resnet_imagenet_128} Generator}
          \end{subtable}
          \begin{subtable}{.4\textwidth}
              \centering
              {\begin{tabular}{c}
                  \toprule
                  \midrule
                  RGB image $x\in \bbR^{128 \times 128 \times 3}$ \\
                  \midrule
                  $3\times 3$ Conv $3\rightarrow ch$\\
                  \midrule
                  ResBlock down $ch \rightarrow 2ch$\\
                  \midrule
                  ResBlock $2ch \rightarrow 2ch$\\
                  \midrule
                  Non-Local Block ($64\times 64$) \\
                  \midrule
                  ResBlock down $2ch \rightarrow 4ch$\\
                  \midrule
                  ResBlock $4ch \rightarrow 4ch$\\
                  \midrule
                  ResBlock down $4ch \rightarrow 8ch$\\
                  \midrule
                  ResBlock $8ch \rightarrow 8ch$\\
                  \midrule
                  ResBlock down $8ch \rightarrow 16ch$\\
                  \midrule
                  ResBlock $16ch \rightarrow 16ch$\\
                  \midrule
                  ResBlock down $16ch \rightarrow 16ch$\\
                  \midrule
                  ResBlock $16ch \rightarrow 16ch$\\
                  \midrule
                  ReLU, Global sum pooling\\
                  \midrule
                  Embed($y$)$\cdot \bmh$ + (linear $\rightarrow$ 1) \\
                  \midrule
                  \bottomrule
              \end{tabular}}
              \caption{\label{tab:deep_dis_resnet_imagenet_128} Discriminator}
          \end{subtable}
\end{table}

\begin{table}[ht]
         \caption{\label{tab:deep_resnets_imagenet256} BigGAN-deep architecture for $256\times 256$ images.}
          \centering
          \small
          \begin{subtable}{.4\textwidth}
              \centering
              {\begin{tabular}{c}
                  \toprule
                  \midrule
                  $z\in \bbR^{128} \sim \mathcal{N}(0, I)$ \\
                  Embed($y$) $\in \bbR^{128}$ \\
                  \midrule
                  Linear $(128+128) \rightarrow 4 \times 4 \times 16 ch $ \\
                  \midrule
                  ResBlock $16ch \rightarrow 16ch$ \\
                  \midrule
                  ResBlock up $16ch \rightarrow 16ch$ \\
                  \midrule
                  ResBlock $16ch \rightarrow 16ch$ \\
                  \midrule
                  ResBlock up $16ch \rightarrow 8ch$ \\
                  \midrule
                  ResBlock $8ch \rightarrow 8ch$ \\
                  \midrule
                  ResBlock up $8ch \rightarrow 8ch$ \\
                  \midrule
                  ResBlock $8ch \rightarrow 8ch$ \\
                  \midrule
                  ResBlock up $8ch \rightarrow 4ch$ \\
                  \midrule
                  Non-Local Block ($64\times 64$)\\
                  \midrule
                  ResBlock $4ch \rightarrow 4ch$ \\
                  \midrule
                  ResBlock up $4ch \rightarrow 2ch$ \\
                  \midrule
                  ResBlock $2ch \rightarrow 2ch$ \\
                  \midrule
                  ResBlock up $2ch \rightarrow ch$ \\
                  \midrule
                  BN, ReLU, $3\times 3$ Conv $ch\rightarrow 3$ \\
                  \midrule
                  Tanh\\
                  \midrule
                  \bottomrule
              \end{tabular}}
              \caption{\label{tab:deep_gen_resnet_imagenet_256} Generator}
          \end{subtable}
          \begin{subtable}{.4\textwidth}
              \centering
              {\begin{tabular}{c}
                  \toprule
                  \midrule
                  RGB image $x\in \bbR^{256 \times 256 \times 3}$ \\
                  \midrule
                  $3\times 3$ Conv $3\rightarrow ch$\\
                  \midrule
                  ResBlock down $ch \rightarrow 2ch$\\
                  \midrule
                  ResBlock $2ch \rightarrow 2ch$\\
                  \midrule
                  ResBlock down $2ch \rightarrow 4ch$\\
                  \midrule
                  ResBlock $4ch \rightarrow 4ch$\\
                  \midrule
                  Non-Local Block ($64\times 64$) \\
                  \midrule
                  ResBlock down $4ch \rightarrow 8ch$\\
                  \midrule
                  ResBlock $8ch \rightarrow 8ch$\\
                  \midrule
                  ResBlock down $8ch \rightarrow 8ch$\\
                  \midrule
                  ResBlock $8ch \rightarrow 8ch$\\
                  \midrule
                  ResBlock down $8ch \rightarrow 16ch$\\
                  \midrule
                  ResBlock $16ch \rightarrow 16ch$\\
                  \midrule
                  ResBlock down $16ch \rightarrow 16ch$\\
                  \midrule
                  ResBlock $16ch \rightarrow 16ch$\\
                  \midrule
                  ReLU, Global sum pooling\\
                  \midrule
                  Embed($y$)$\cdot \bmh$ + (linear $\rightarrow$ 1) \\
                  \midrule
                  \bottomrule
              \end{tabular}}
              \caption{\label{tab:deep_dis_resnet_imagenet_256} Discriminator}
          \end{subtable}
\end{table}

\begin{table}[ht]
         \caption{\label{tab:deep_resnets_imagenet512} BigGAN-deep architecture for $512\times 512$ images.}
          \centering
          \small
          \begin{subtable}{.4\textwidth}
              \centering
              {\begin{tabular}{c}
                  \toprule
                  \midrule
                  $z\in \bbR^{128} \sim \mathcal{N}(0, I)$ \\
                  Embed($y$) $\in \bbR^{128}$ \\
                  \midrule
                  Linear $(128+128) \rightarrow 4 \times 4 \times 16 ch $ \\
                  \midrule
                  ResBlock $16ch \rightarrow 16ch$ \\
                  \midrule
                  ResBlock up $16ch \rightarrow 16ch$ \\
                  \midrule
                  ResBlock $16ch \rightarrow 16ch$ \\
                  \midrule
                  ResBlock up $16ch \rightarrow 8ch$ \\
                  \midrule
                  ResBlock $8ch \rightarrow 8ch$ \\
                  \midrule
                  ResBlock up $8ch \rightarrow 8ch$ \\
                  \midrule
                  ResBlock $8ch \rightarrow 8ch$ \\
                  \midrule
                  ResBlock up $8ch \rightarrow 4ch$ \\
                  \midrule
                  Non-Local Block ($64\times 64$)\\
                  \midrule
                  ResBlock $4ch \rightarrow 4ch$ \\
                  \midrule
                  ResBlock up $4ch \rightarrow 2ch$ \\
                  \midrule
                  ResBlock $2ch \rightarrow 2ch$ \\
                  \midrule
                  ResBlock up $2ch \rightarrow ch$ \\
                  \midrule
                  ResBlock $ch \rightarrow ch$ \\
                  \midrule
                  ResBlock up $ch \rightarrow ch$ \\
                  \midrule
                  BN, ReLU, $3\times 3$ Conv $ch\rightarrow 3$ \\
                  \midrule
                  Tanh\\
                  \midrule
                  \bottomrule
              \end{tabular}}
              \caption{\label{tab:deep_gen_resnet_imagenet_512} Generator}
          \end{subtable}
          \begin{subtable}{.4\textwidth}
              \centering
              {\begin{tabular}{c}
                  \toprule
                  \midrule
                  RGB image $x\in \bbR^{512 \times 512 \times 3}$ \\
                  \midrule
                  $3\times 3$ Conv $3\rightarrow ch$\\
                  \midrule
                  ResBlock down $ch \rightarrow ch$\\
                  \midrule
                  ResBlock $ch \rightarrow ch$\\
                  \midrule
                  ResBlock down $ch \rightarrow 2ch$\\
                  \midrule
                  ResBlock $2ch \rightarrow 2ch$\\
                  \midrule
                  ResBlock down $2ch \rightarrow 4ch$\\
                  \midrule
                  ResBlock $4ch \rightarrow 4ch$\\
                  \midrule
                  Non-Local Block ($64\times 64$) \\
                  \midrule
                  ResBlock down $4ch \rightarrow 8ch$\\
                  \midrule
                  ResBlock $8ch \rightarrow 8ch$\\
                  \midrule
                  ResBlock down $8ch \rightarrow 8ch$\\
                  \midrule
                  ResBlock $8ch \rightarrow 8ch$\\
                  \midrule
                  ResBlock down $8ch \rightarrow 16ch$\\
                  \midrule
                  ResBlock $16ch \rightarrow 16ch$\\
                  \midrule
                  ResBlock down $16ch \rightarrow 16ch$\\
                  \midrule
                  ResBlock $16ch \rightarrow 16ch$\\
                  \midrule
                  ReLU, Global sum pooling\\
                  \midrule
                  Embed($y$)$\cdot \bmh$ + (linear $\rightarrow$ 1) \\
                  \midrule
                  \bottomrule
              \end{tabular}}
              \caption{\label{tab:deep_dis_resnet_imagenet_512} Discriminator}
          \end{subtable}
\end{table}


\clearpage

\newpage
\section{Experimental Details}
\label{appendix_experimental_details}
Our basic setup follows SA-GAN \citep{zhang2018sagan}, and is implemented in TensorFlow \citep{Abadi2016tf}. We employ the architectures detailed in Appendix~\ref{appendix_architecture}, with non-local blocks inserted at a single stage in each network. Both \gen{} and \discr{} networks are initialized with Orthogonal Initialization \citep{saxe2014ortho}.  
We use Adam optimizer \citep{kingma2014adam} with $\beta_1=0$ and $\beta_2=0.999$ and a constant learning rate. 
For BigGAN models at all resolutions, we use $2\cdot10^{-4}$ in \discr{} and $5\cdot10^{-5}$ in \gen{}.
For BigGAN-deep, we use the learning rate of $2\cdot10^{-4}$ in \discr{} and $5\cdot10^{-5}$ in \gen{} for $128\times 128$ models, and $2.5\cdot10^{-5}$ in both \discr{} and \gen{} for $256\times 256$ and $512\times 512$ models.
We experimented with the number of \discr{} steps per \gen{} step (varying it from $1$ to $6$) and found that two \discr{} steps per \gen{} step gave the best results. 

We use an exponential moving average of the weights of \gen{} at sampling time, with a decay rate set to 0.9999. We employ cross-replica BatchNorm \citep{ioffe2015batchnorm} in \gen{}, where batch statistics are aggregated across all devices, rather than a single device as in standard implementations. Spectral Normalization \citep{miyato2018spectral} is used in both \gen{} and \discr{}, following SA-GAN \citep{zhang2018sagan}. We train on a Google TPU v3
Pod, with the number of cores proportional to the resolution: 128 for 128$\times$128, 256 for 256$\times$256, and 512 for 512$\times$512. Training takes between 24 and 48 hours for most models. We increase $\epsilon$ from the default $10^{-8}$ to $10^{-4}$ in BatchNorm and Spectral Norm to mollify low-precision numerical issues.
We preprocess data by cropping along the long edge and rescaling to a given resolution with area resampling. 
% As the ImageNet dataset has many low-resolution images, directly training at 512$\times$512 produces aliased results, so we filter out all images with a short edge length less than 400 pixels. Similar to the CelebA-HQ dataset employed by \citet{karras2018progan}, this reduces the dataset size to around 200,000 instances.

\subsection{BatchNorm Statistics and Sampling}
The default behavior with batch normalized classifier networks is to use a running average of the activation moments at test time. Previous works \citep{radford2016dcgan} have instead used batch statistics when sampling images. While this is not technically an invalid way to sample, it means that results are dependent on the test batch size (and how many devices it is split across), and further complicates reproducibility.

We find that this detail is extremely important, with changes in test batch size producing drastic changes in performance. This is further exacerbated when one uses exponential moving averages of \gen{}'s weights for sampling, as the BatchNorm running averages are computed with non-averaged weights and are poor estimates of the activation statistics for the averaged weights.

To counteract both these issues, we employ ``standing statistics,'' where we compute activation statistics at sampling time by running the \gen{} through multiple forward passes (typically 100) each with different batches of random noise, and storing means and variances aggregated across all forward passes. Analogous to using running statistics, this results in \gen{}'s outputs becoming invariant to batch size and the number of devices, even when producing a single sample.

\subsection{CIFAR-10}
We run our networks on CIFAR-10 \citep{krizhevsky2009cifar} using the settings from Table~\ref{ablation_table}, row 8, and achieve an IS of 9.22 and an FID of 14.73 without truncation.

\subsection{Inception Scores of ImageNet Images}
We compute the IS for both the training and validation sets of ImageNet. At 128$\times$128 the training data has an IS of 233, and the validation data has an IS of 166. At 256$\times$256 the training data has an IS of 377, and the validation data has an IS of 234. At 512$\times$512 the training data has an IS of 348, and the validation data has an IS of 241. The discrepancy between training and validation scores is due to the Inception classifier having been trained on the training data, resulting in high-confidence outputs that are preferred by the Inception Score.

\newpage
\section{Additional Plots}
\label{appendix_additional_plots}
\begin{figure}[htbp]
\centering
% \includegraphics[width=0.85\textwidth]{images/ISvFID128a.png}
\includegraphics[width=0.85\textwidth]{images/BigGAN-deep_ISvFID_128.png} 
\caption{IS vs. FID at 128$\times$128. Scores are averaged across three random seeds.}
\label{appendix_ISvFID128}
\end{figure}


\begin{figure}[htbp]
\centering
\begin{tabular}{cc}
% \includegraphics[width=0.47\textwidth]{images/ISvFID256b.png} &
% \includegraphics[width=0.47\textwidth]{images/ISvFID512a.png}
\includegraphics[width=0.47\textwidth]{images/BigGAN-deep_ISvFID_256.png} &
\includegraphics[width=0.47\textwidth]{images/BigGAN-deep_ISvFID_512.png} 
\end{tabular}
\caption{IS vs. FID at 256 and 512 pixels. Scores are averaged across three random seeds for 256.}%$\times$256.}
\label{appendix_ISvFID256}
\end{figure}

\begin{figure}[htbp]
\centering
\begin{tabular}{c}
\includegraphics[width=0.65\textwidth]{images/jft_truncation_curves/jft_truncation_curve_256.pdf} \\
\includegraphics[width=0.65\textwidth]{images/jft_truncation_curves/jft_truncation_curve_256_0p5_1p5.pdf}
%\includegraphics[width=0.65\textwidth]{images/jft_truncation_curves/anon_jft_truncation_curve_256.pdf} \\
%\includegraphics[width=0.65\textwidth]{images/jft_truncation_curves/anon_jft_truncation_curve_256_0p5_1p5.pdf}
\end{tabular}
\caption{
JFT-300M IS vs. FID at 256$\times$256.
%IS vs. FID for the internal dataset at 256$\times$256.
We show truncation values from $\sigma = 0$ to $\sigma = 2$ (top) 
and from $\sigma = 0.5$ to $\sigma = 1.5$ (bottom).
Each curve corresponds to a row in Table~\ref{jft_table}.
The curve labeled with \textit{baseline} corresponds to the first row (with orthogonal regularization and other techniques disabled), while the rest correspond to rows 2-4 -- the same architecture at different capacities (\textit{Ch}). 
\label{appendix_jft_trunc}
}
\end{figure}
\clearpage

\newpage
\section{Choosing Latent Spaces}
\label{appendix_latents}
While most previous work has employed $\mathcal{N}(0, I)$ or $\mathcal{U}[-1, 1]$ as the prior for $z$ (the noise input to \gen{}), we are free to choose any latent distribution from which we can sample.  We explore the choice of latents by considering an array of possible designs, described below. For each latent, we provide the intuition behind its design and briefly describe how it performs when used as a drop-in replacement for  $z\sim\mathcal{N}(0, I)$ in an SA-GAN baseline. As the Truncation Trick proved more beneficial than switching to any of these latents, we do not perform a full ablation study, and employ $z\sim\mathcal{N}(0, I)$ for our main results to take full advantage of truncation. The two latents which we find to work best without truncation are  Bernoulli $\{0, 1\}$ and Censored Normal $\max\left(\mathcal{N}(0, I), 0\right)$, both of which improve speed of training and lightly improve final performance, but are less amenable to truncation. 

We also ablate the choice of latent space dimensonality (which by default is $z\in \bbR^{128}$), finding that we are able to successfully train with latent dimensions as low as $z\in \bbR^{8}$, and that with $z\in \bbR^{32}$ we see a minimal drop in performance. While this is substantially smaller than many previous works, direct comparison to single-class networks (such as those in \citet{karras2018progan}, which employ a $z\in \bbR^{512}$ latent space on a highly constrained dataset with 30,000 images) is improper, as our networks have additional class information provided as input.

\subsection*{Latents}
\begin{itemize}
\item $\mathcal{N}(0, I)$. A standard choice of the latent space which we use in the main experiments.

\item $\mathcal{U}[-1, 1]$. Another standard choice; we find that it performs similarly to $\mathcal{N}(0, I)$.

\item Bernoulli $\{0, 1\}$. A discrete latent might reflect our prior that underlying factors of variation in natural images are not continuous, but discrete (one feature is present, another is not). This latent outperforms $\mathcal{N}(0, I)$ (in terms of IS) by 8\% and requires 60\% fewer iterations.

\item $\max\left(\mathcal{N}(0, I), 0\right)$, also called Censored Normal. This latent is designed to introduce sparsity in the latent space (reflecting our prior that certain latent features are sometimes present and sometimes not), but also allow those latents to vary continuously, expressing different degrees of intensity for latents which are active. This latent outperforms $\mathcal{N}(0, I)$ (in terms of IS) by 15-20\% and tends to require fewer iterations.


\item Bernoulli $\{-1, 1\}$. This latent is designed to be discrete, but not sparse (as the network can learn to activate in response to negative inputs). This latent performs near-identically to $\mathcal{N}(0, I)$.

\item Independent Categorical in $\{-1, 0, 1\}$, with equal probability. This distribution is chosen to be discrete and have sparsity, but also to allow latents to take on both positive and negative values. This latent performs near-identically to $\mathcal{N}(0, I)$.

\item $\mathcal{N}(0, I)$  multiplied by Bernoulli $\{0, 1\}$. This distribution is chosen to have continuous latent factors which are also sparse (with a peak at zero), similar to Censored Normal but not constrained to be positive. This latent performs near-identically to $\mathcal{N}(0, I)$.

\item Concatenating  $\mathcal{N}(0, I)$ and Bernoulli $\{0, 1\}$, each taking half of the latent dimensions. This is inspired by \citet{chen2016infogan}, and is chosen to allow some factors of variation to be discrete, while others are continuous. This latent outperforms $\mathcal{N}(0, I)$ by around 5\%.

\item Variance annealing: we sample from $\mathcal{N}(0, \sigma I)$, where $\sigma$ is allowed to vary over training. We compared a variety of piecewise schedules and found that starting with $\sigma = 2$ and annealing towards $\sigma=1$ over the course of training mildly improved performance. The space of possible variance schedules is large, and we did not explore it in depth -- we suspect that a more principled or better-tuned schedule could more strongly impact performance.

\item Per-sample variable variance: $\mathcal{N}(0, \sigma_i I)$, where $\sigma_i\sim\mathcal{U}[\sigma_l, \sigma_h]$ independently for each sample $i$ in a batch, and $(\sigma_l, \sigma_h)$ are hyperparameters. This distribution was chosen to try and improve amenability to the Truncation Trick by feeding the network noise samples with non-constant variance. This did not appear to affect performance, but we did not explore it in depth. One might also consider scheduling $(\sigma_l, \sigma_h)$, similar to variance annealing.

\end{itemize}








\newpage
\section{Monitored Training Statistics}
\label{appendix_monitored_stats}
\begin{figure}[htbp]
\centering
\setlength{\tabcolsep}{1pt}
\begin{tabular}{cc}
\subf{\includegraphics[width=0.48\textwidth]{images/1535537/GSV0.jpg}}{(a) \gen{} $\sigma_0$ } & 
\subf{\includegraphics[width=0.48\textwidth]{images/1535537/GSVR.jpg}}{(b) \gen{} $\frac{\sigma_0}{\sigma_1}$ } \\
\subf{\includegraphics[width=0.48\textwidth]{images/1535537/GSV1.jpg}}{(c) \gen{} $\sigma_1$} & 
\subf{\includegraphics[width=0.48\textwidth]{images/1535537/GSV2.jpg}}{(d) \gen{} $\sigma_2$} \\
\subf{\includegraphics[width=0.48\textwidth]{images/1535537/DSV0.jpg}}{(e) \discr{} $\sigma_0$ } & 
\subf{\includegraphics[width=0.48\textwidth]{images/1535537/DSVR.jpg}}{(f) \discr{} $\frac{\sigma_0}{\sigma_1}$ } \\
\subf{\includegraphics[width=0.48\textwidth]{images/1535537/DSV1.jpg}}{(g) \discr{} $\sigma_1$} & 
\subf{\includegraphics[width=0.48\textwidth]{images/1535537/DSV2.jpg}}{(h) \discr{} $\sigma_2$} 
\end{tabular}
\caption{Training statistics for a typical model without special modifications. Collapse occurs after 200000 iterations.}
\label{GD_spectra_unreg}
\end{figure}


% Reg sigma bois
\begin{figure}[htbp]
\centering
\setlength{\tabcolsep}{1pt}
\begin{tabular}{cc}
\subf{\includegraphics[width=0.48\textwidth]{images/1506681_GSR2_1/GSV0.jpg}}{(a) $\sigma_0$ } & 
\subf{\includegraphics[width=0.48\textwidth]{images/1506681_GSR2_1/GSVR.jpg}}{(b) $\frac{\sigma_0}{\sigma_1}$ } \\
\subf{\includegraphics[width=0.48\textwidth]{images/1506681_GSR2_1/GSV1.jpg}}{(c) $\sigma_1$} & 
\subf{\includegraphics[width=0.48\textwidth]{images/1506681_GSR2_1/GSV2.jpg}}{(d) $\sigma_2$}\\
\end{tabular}
\caption{\gen{} training statistics with $\sigma_0$ in \gen{} regularized towards 1. Collapse occurs after 125000 iterations.}
\label{G_spectra_regsigma}
\end{figure}

\begin{figure}[htbp]
\centering
\setlength{\tabcolsep}{1pt}
\begin{tabular}{cc}
\subf{\includegraphics[width=0.48\textwidth]{images/1506681_GSR2_1/DSV0.jpg}}{(a) $\sigma_0$ } & 
\subf{\includegraphics[width=0.48\textwidth]{images/1506681_GSR2_1/DSVR.jpg}}{(b) $\frac{\sigma_0}{\sigma_1}$ } \\
\subf{\includegraphics[width=0.48\textwidth]{images/1506681_GSR2_1/DSV1.jpg}}{(c) $\sigma_1$} & 
\subf{\includegraphics[width=0.48\textwidth]{images/1506681_GSR2_1/DSV2.jpg}}{(d) $\sigma_2$} \\
\end{tabular}
\caption{\discr{} training statistics with $\sigma_0$ in \gen{} regularized towards 1. Collapse occurs after 125000 iterations.}
\label{D_spectra_regsigma}
\end{figure}



% R1GP bois
\begin{figure}[htbp]
\centering
\setlength{\tabcolsep}{1pt}
\begin{tabular}{cc}
\subf{\includegraphics[width=0.48\textwidth]{images/1519233_R1GP5/GSV0.jpg}}{(a) $\sigma_0$ } & 
\subf{\includegraphics[width=0.48\textwidth]{images/1519233_R1GP5/GSVR.jpg}}{(b) $\frac{\sigma_0}{\sigma_1}$ } \\
\subf{\includegraphics[width=0.48\textwidth]{images/1519233_R1GP5/GSV1.jpg}}{(c) $\sigma_1$} & 
\subf{\includegraphics[width=0.48\textwidth]{images/1519233_R1GP5/GSV2.jpg}}{(d) $\sigma_2$} \\
\end{tabular}
\caption{\gen{} training statistics with an R1 Gradient Penalty of strength 10 on \discr{}. This model does not collapse, but only reaches a maximum IS of 55.}
\label{G_spectra_R1GP5}
\end{figure}

\begin{figure}[htbp]
\centering
\setlength{\tabcolsep}{1pt}
\begin{tabular}{cc}
\subf{\includegraphics[width=0.48\textwidth]{images/1519233_R1GP5/DSV0.jpg}}{(a) $\sigma_0$ } & 
\subf{\includegraphics[width=0.48\textwidth]{images/1519233_R1GP5/DSVR.jpg}}{(b) $\frac{\sigma_0}{\sigma_1}$ } \\
\subf{\includegraphics[width=0.48\textwidth]{images/1519233_R1GP5/DSV1.jpg}}{(c) $\sigma_1$} & 
\subf{\includegraphics[width=0.48\textwidth]{images/1519233_R1GP5/DSV2.jpg}}{(d) $\sigma_2$} 
\end{tabular}
\caption{\discr{} training statistics with an R1 Gradient Penalty of strength 10 on \discr{}. This model does not collapse, but only reaches a maximum IS of 55.}
\label{D_spectra_R1GP5}
\end{figure}


% Dropout bois
\begin{figure}[htbp]
\centering
\setlength{\tabcolsep}{1pt}
\begin{tabular}{cc}
\subf{\includegraphics[width=0.48\textwidth]{images/1576139_dropout0.2/GSV0.jpg}}{(a) $\sigma_0$ } & 
\subf{\includegraphics[width=0.48\textwidth]{images/1576139_dropout0.2/GSVR.jpg}}{(b) $\frac{\sigma_0}{\sigma_1}$ } \\
\subf{\includegraphics[width=0.48\textwidth]{images/1576139_dropout0.2/GSV1.jpg}}{(c) $\sigma_1$} & 
\subf{\includegraphics[width=0.48\textwidth]{images/1576139_dropout0.2/GSV2.jpg}}{(d) $\sigma_2$}
\end{tabular}
\caption{\gen{} training statistics with Dropout (keep probability 0.8) applied to the last feature layer of \discr{}. This model does not collapse, but only reaches a maximum IS of 70.}
\label{G_spectra_dropout0.2}
\end{figure}

\begin{figure}[htbp]
\centering
\setlength{\tabcolsep}{1pt}
\begin{tabular}{cc}
\subf{\includegraphics[width=0.48\textwidth]{images/1576139_dropout0.2/DSV0.jpg}}{(a) $\sigma_0$ } & 
\subf{\includegraphics[width=0.48\textwidth]{images/1576139_dropout0.2/DSVR.jpg}}{(b) $\frac{\sigma_0}{\sigma_1}$ } \\
\subf{\includegraphics[width=0.48\textwidth]{images/1576139_dropout0.2/DSV1.jpg}}{(c) $\sigma_1$} & 
\subf{\includegraphics[width=0.48\textwidth]{images/1576139_dropout0.2/DSV2.jpg}}{(d) $\sigma_2$} \end{tabular}
\caption{\discr{} training statistics with Dropout (keep probability 0.8) applied to the last feature layer of \discr{}. This model does not collapse, but only reaches a maximum IS of 70.}
\label{D_spectra_dropout0.2}
\end{figure}



\begin{figure}[htbp]
\centering
\setlength{\tabcolsep}{1pt}
\begin{tabular}{cc}
\subf{\includegraphics[width=0.48\textwidth]{images/1535537/GFROB.jpg}}{(a) \gen{} $\|W\|_2$} &
\subf{\includegraphics[width=0.48\textwidth]{images/1535537/DFROB.jpg}}{(b) \discr{} $\|W\|_2$} \\
\subf{\includegraphics[width=0.48\textwidth]{images/1535537/losses.jpg}}{(c) losses} &
\subf{\includegraphics[width=0.48\textwidth]{images/1535537/gradnormvars.jpg}}{(d) Variance of all gradient norms in \gen{} and \discr{}}
\end{tabular}
\caption{Additional training statistics for a typical model without special modifications. Collapse occurs after 200000 iterations.}
\label{additional_stats_vanilla}
\end{figure}


\begin{figure}[htbp]
\centering
\setlength{\tabcolsep}{1pt}
\begin{tabular}{cc}
\subf{\includegraphics[width=0.48\textwidth]{images/1519233_R1GP5/GFROB.jpg}}{(a) \gen{} $\|W\|_2$} & 
\subf{\includegraphics[width=0.48\textwidth]{images/1519233_R1GP5/DFROB.jpg}}{(b) \discr{} $\|W\|_2$} \\ 
\subf{\includegraphics[width=0.48\textwidth]{images/1519233_R1GP5/losses.jpg}}{(c) losses} &
\subf{\includegraphics[width=0.48\textwidth]{images/1519233_R1GP5/gradnormvars.jpg}}{(d) Variance of all gradient norms in \gen{} and \discr{}}
\end{tabular}
\caption{Additional training statistics with an R1 Gradient Penalty of strength 10 on \discr{}. This model does not collapse, but only reaches a maximum IS of 55.}
\label{additional_stats_R1GP10}
\end{figure}

\clearpage

\newpage
\section{Additional Discussion: Stability and Collapse}
\label{appendix_additional_discussion}
In this section, we present and discuss additional investigations into the stability of our models, expanding upon the discussion in Section~\ref{sec:analysis}.

% \subsection{Is collapse unavoidable?}
\subsection{Intervening Before Collapse}
% \subsection{Can One Intervene to Prevent Collapse?}
The symptoms of collapse are sharp and sudden, with sample quality dropping from its peak to its lowest value over the course of a few hundred iterations. We can detect this collapse when the singular values in \gen{} explode, but while the (unnormalized) singular values grow throughout training, there is no consistent threshold at which collapse occurs. This raises the question of whether it is possible to prevent or delay collapse by taking a model checkpoint several thousand iterations before collapse, and continuing training with some hyperparameters modified (e.g., the learning rate).

We conducted a range of intervention experiments wherein we took checkpoints of a collapsed model ten or twenty thousand iterations before collapse, changed some aspect of the training setup, then observed whether collapse occurred, when it occurred relative to the original collapse, and the final performance attained at collapse.

We found that increasing the learning rates (relative to their initial values) in either \gen{} or \discr{}, or both \gen{} and \discr{},  led to immediate collapse. This occurred even when doubling the learning rates from $2\cdot10^{-4}$ in \discr{} and $5\cdot10^{-5}$ in \gen{}, to $4\cdot10^{-4}$ in \discr{} and $1\cdot10^{-4}$ in \gen{}, a setting which is not normally unstable when used as the initial learning rates. We also tried changing the momentum terms (Adam's $\beta_1$ and $\beta_2$), or resetting the momentum vectors to zero, but this tended to either make no difference or, when increasing the momentum, cause immediate collapse.


We found that decreasing the learning rate in \gen{}, but keeping the learning rate in \discr{} unchanged could delay collapse (in some cases by over one hundred thousand iterations), but also crippled training---once the learning rate in \gen{} was decayed, performance either stayed constant or slowly decayed. Conversely, reducing the learning rate in \discr{} while keeping \gen{}'s learning rate led to immediate collapse. We hypothesize that this is because of the need for \discr{} to remain optimal throughout training---if its learning rate is reduced, it can no longer ``keep up" with \gen{}, and training collapses. With this in mind, we also tried increasing the number of \discr{} steps per \gen{} step, but this either had no effect, or delayed collapse at the cost of crippling training (similar to decaying \gen{}'s learning rate).


%...Suggests the training dynamics of our models are in some sense attuned to their optimization hyperparameters, and changing those hyperparameters partway through training encourages unfavorable dynamics...

%...One possible explanation is that as the singular values grow, the model draws close to the point of collapse and as they get larger the probability of total collapse approaches 1, and all it takes is a small perturbation to knock everything over...

To further illuminate these dynamics, we construct two additional intervention experiments, one where we freeze \gen{} before collapse (by ceasing all parameter updates) and observe whether \discr{} remains stable, and the reverse, where we freeze \discr{} before collapse and observe whether \gen{} remains stable. We find that when \gen{} is frozen, \discr{} remains stable, and slowly reduces both components of its loss towards zero. However, when \discr{} is frozen, \gen{} immediately and dramatically collapses, maxing out \discr{}'s loss to values upwards of 300, compared to the normal range of 0 to 3. 

%Regardless of \gen{}'s conditioning or optimization settings, the consequence of \gen{} being allowed to win the game is a complete breakdown of the training process.


This leads to two conclusions: first, as has been noted in previous works \citep{miyato2018spectral, gulrajani2017improved, zhang2018sagan}, \discr{} must remain optimal with respect to \gen{} both for stability and to provide useful gradient information. The consequence of \gen{} being allowed to win the game is a complete breakdown of the training process, regardless of \gen{}'s conditioning or optimization settings. Second, favoring \discr{} over \gen{} (either by training it with a larger learning rate, or for more steps) is insufficient to ensure stability even if \discr{} is well-conditioned. This suggests either that in practice, an optimal \discr{} is necessary but insufficient for training stability, or that some aspect of the system results in \discr{} not being trained towards optimality. With the latter possibility in mind, we take a closer look at the noise in \discr{}'s spectra in the following section.

%Second, if collapse occurs because of a failure of \discr{}, providing \discr{} with a larger learning rate or more steps per \gen{} step does not mollify this collapse, suggesting that collapse does not \textit{solely} occur because \discr{} is not optimal with respect to its instantaneous \gen{}. So, while it is necessary for \discr{} to ``keep up" with \gen{}, there is likely a separate source of instability. 

%Many previous works (SNGAN, SAGAN) have noted that it is necessary for D to stay optimal wrt G in order to provide good gradients and remain stable.



% This suggests two things: first, that the training dynamics of our models are in some sense attuned to their optimization hyperparameters, and changing those hyperparameters partway through training encourages unfavorable dynamics. 

%\andycom{separating this section for a moment though it probably fits in with the above paragraph}



%-This leads to the following conclusion/posit/supposition/hypothesis: Regardless of G's conditioning, the consequence of G winning the game is immediate and dramatic training collapse. It follows that D must remain optimal throughout training to prevent G from doing this, and that the gradients that flow from D should be constrained to prevent G from being able to adversarially annihilate the whole kit and kaboodle.
%Previous works have argued for conditioning and keeping D optimal based on the need to provide good gradients everywhere (non saturation) but have not... -this is in contrast to previous works which argue that the optimality of the discriminator is necessary in order for the gradients to G to be good? I guess that's still what we're seeing, if D is not optimal then its gradients are incorrect and G will follow a bad path and go unstable

%-Paragraph about freezing G and finding that training remains stable, but freezing D and finding that training SUPER explodes and G bottoms out D's loss. This suggests that the gradients provided by \discr{} are only valid for the instantaneous \gen{} it's paired with. If \discr{} is not [trained to stay on top of G], \gen{} will immediately learn highly specific adversarial 


\newpage
\subsection{Spikes in the Discriminator's Spectra}

\begin{figure}[htbp]
\centering
\setlength{\tabcolsep}{1pt}
\begin{tabular}{cc}

\subf{\includegraphics[width=0.48\textwidth]{images/DSpectraCloseup0aCleaned.jpg}}{(a) \discr{} $\sigma_0$}  &
\subf{\includegraphics[width=0.48\textwidth]{images/DSpectraCloseup0bCleaned.jpg}}{(b) \discr{} $\frac{\sigma_0}{\sigma_1}$}
\end{tabular}
\caption{A closeup of \discr{}'s spectra at a noise spike.}
\label{D_spectra_closeup}
\end{figure}

If some element of \discr{}'s training process results in undesirable dynamics, it follows that the behavior of \discr{}'s spectra may hold clues as to what that element is. The top three singular values of \discr{}  differ from \gen{}'s in that they have a large noise component, tend to grow throughout training but only show a small response to collapse, and the ratio of the first two singular values tends to be centered around one, suggesting that the spectra of \discr{} have a slow decay. When viewed up close (Figure~\ref{D_spectra_closeup}), the noise spikes resemble an impulse response: at each spike, the spectra jump upwards, then slowly decrease, with some oscillation.

One possible explanation is that this behavior is a consequence of \discr{} memorizing the training data, as suggested by experiments in Section~\ref{subsec:discr_instability}.
 As it approaches perfect memorization, it receives less and less signal from real data, as both the original GAN loss and the hinge loss provide zero gradients when \discr{} outputs a confident and correct prediction for a given example. If the gradient signal from real data attenuates to zero, this can result in \discr{} eventually becoming biased due to exclusively received gradients that encourage its outputs to be negative. If this bias passes a certain threshold, \discr{} will eventually misclassify a large number of real examples and receive a large gradient encouraging positive outputs, resulting in the observed impulse responses.
 
 %...The noise in the losses of \discr{} and \gen{} (Figure~\ref{additional_stats_vanilla}(c)) could suggest another, related explanation. Spikes in the loss coincide with spikes in the spectra of \discr{}, but the magnitude of those loss spikes for \discr{} is dominated by the loss from the fake data; the upward spikes in \discr{}'s loss coincide with the downward spikes in \gen{}'s loss and have similar magnitude. This would mean that the spikes are a result of \gen{} producing a batch of images which \discr{} largely mistakes for real, resulting in a spike towards a negative bias... \andycom{I think this argument isn't invalid but it's kind of weak and it doesn't really add anything}
 
 
 %This can be seen where the upward spikes in \discr{}'s loss coinciding with the downward spikes in \gen{}'s loss and having 
 
% appears to partially agree with this interpretation-- However, the magnitude of the loss spikes is dominated by the loss from the fake data (the upward spikes in \discr{}'s loss coincide with and have similar magnitude to the downward spikes in \gen{}'s loss), which would mean that it's not a large spike towards a positive bias, but instead a large spike towards a negative bias...

 
This argument suggests several fixes. First, one might consider an unbounded loss (such as the Wasserstein loss \citep{arjovsky2017wgan}) which would not suffer this gradient attentuation. We found that even with gradient penalties and brief re-tuning of optimizer hyperparameters, our models did not stably train for more than a few thousand iterations with this loss. We instead explored changing the margin of the hinge loss as a partial compromise: for a given model and minibatch of data, increasing the margin will result in more examples falling within the margin, and thus contributing to the loss.\footnote{Unconstrained models could easily learn a different output scale to account for this margin, but the use of Spectral Normalization constrains our models and makes the specific selection of the margin meaningful.}. Training with a smaller margin (by a factor of 2) measurably reduces performance, but training with a larger margin (by up to a factor of 3) does not prevent collapse or reduce the noise in \discr{}'s spectra. Increasing the margin beyond 3 results in unstable training similar to using the Wasserstein loss. Finally, the memorization argument might suggest that using a smaller \discr{} or using dropout in \discr{} would improve training by reducing its capacity to memorize, but in practice this degrades training.




%.

%Upward spikes in D's loss also coincide with downward spikes in G's loss, and the loss term from fake data seems to dominate these spikes--could this be G overfitting to D's adversarial examples?

%-removing the bias doesn't fix this.





%One possible explanation for this behavior spikes is that \discr{} is memorizing the training data, as suggested by experiments in Section 4.2....

%-the memorization argument might suggest that using a smaller \discr{} would improve training by reducing its capacity to memorize, but in practice this degrades training.

%eventually becoming biased due to solely receiving gradients that encourage its outputs to be negative. Once this bias passes a certain threshold, \discr{} 

%-note that an unbounded loss like WGAN might not have this issue, but with our current hyperparameter settings we found WGAN unstable, even with very high R1GP values. Regardless, we believe that the tendency for \discr{} to memorize the training data is fairly general. Many previous works speculate whether or not GAN *generators* simply memorize individual samples, but this is the first time anyone has noted that \discr{} *almost certainly* memorizes samples. This may also have implications for G memorizing, since it is never directly exposed to samples so it has a hard time doing memorization, but D is directly given samples so it could memorize more easily.

%Expanding on this, we examine the spikes in \discr{}'s spectra in }. We note that, when viewed up close, these spikes

%-While this observed behavior may be partially due to the power method estimates slightly lagging the true singular values (especially if they undergo a large change), 


%-also note that sigma0/sigma1 is approximately 1--D's weights tend to be near-orthogonal? or at least, D tends to have a slow decay in svs

%

% This might partially explain why regularizing D improves stability but reduces performance, as a constrained \discr{} would be less able to memorize the training set. However, we still observe (in Section~\ref{subsec:discr_instability}) that regularized \discr{} achieve very high training accuracy and near-random test accuracy. This does not necessarily mean that regularzation fails to mollify overfitting, but suggests that this is not the primary phenomenon of interest here.


%-$^$This might also partially explain why regularizing D imparts stability but reduces performance: a constrained \discr{} is less able to overfit and memorize the training set, mollifying the effect of memorization on an undesirable output bias...yet, even with regulari

%-spikes also correlate with large loss values

%Catastrophic forgetting?

%-one other explanation would be a super-adversarial example knocks D's socks off, when G gets it just right for a batch or two and confuses the heck outta D, then it rights itself?

%-Note that even if we reset training a ton and roll the dice repeatedly so that we trundle along and avoid the huge loss spikes, collapse still occurs, so while the big spikes may play a part, they're not the only thing and avoiding them does not fix the problem.


%Noting


%if we take a model checkpoint several thousand iterations before collapse and change some hyperparameters (e.g. decay the learning rate) can


%models with different random seeds have different final singular values.

%. This begs the question of 






%-Discriminator memorization angle

%In additon 



%-applying orthogonal regularization and strongly constraining G also does not prevent collapse
%-one can go back several thousand iterations but the crash still inevitably happens
%-changing the learning rate does not prevent the crash, although if you kick the learning rate too high crash will immediately happen (even if you're switching from a learning rate which is stable to another one which is stable if you start with it)
%-If you use hierarchical latents, then the layers closest to the Z inputs are the ones that explode and many of them will explode


% We experiment with the following regularizers on D







%but that the spectra are noisy 

%We seek to impose stability through additional constraints on \discr{}.

%-Note the pathology of D's singular values: they are far noisier, with many tiny perturbations followed by oscillation. The oscillations may be artifacts of the power iterations rather than oscillations in the underlying singular values. We hypothesize this is a direct result of the adversarial training procedure: as D approaches optimality, it perfectly memorizes the real data, 

% alt: (less and less signal is provided by the real samples, resulting in D becoming biased)

%-figure showing D's singular values

%-note that imposing R1GP on D also has the effect of stabilizing G--we see that in one case the top singular value goes up and stops at a certain value. Also note that the R1GP has the odd effect of making one layer in D explodey
%-Also note that the frobenius norms grow throughout training but no matter what value we set L2 to we could not prevent collapse--maybe an adaptive L2 or a scheduled L2 is necessary.


%-Most recent papers on gan stability have focused on changing D 
%-ways we stabilize D: R1GP of various strengths, L2 on D, ortho reg (various varieties and strengths), dropout in the final feature layer (various strengths, also scheduled), output magnitude reg [cite progan]

%-we find that doing this can impart stability, but dramatically reduces performance
%-we leave our GANs unregularized and simply let them die at high performance
%-consider including the experiment where we freeze G and let D roll on, finding it to be stable, but if we freeze D and let D train on, it blows up the losses and its quality goes down the drain, highlighting the importance of discriminator optimality throughout training. Note however that we found increasing the number of D steps beyond 3 reduced performance, as did increasing D's learning rate or momentum, so there's another factor at play making it so that we can't just train D super hard and expect it to beat G constantly. \andycom{Maybe this makes memorization worse? Not sure, but let's keep speculation to a minimum if we can't back it up with evidence}

%-we note that D is dramatically overfitting to the training data--this suggests that D does not learn any sort of meaningful decision boundary between real and fake data, but instead memorizes the training set and learns a decision boundary that guides G samples to lie inside that manifold. We note that regularizing D (through data augmentation, gradient penalties, or reducing its capacity) does not change this behavior--no \discr{} we train ever has above 55\% accuracy on the ImageNet test set. \karencom{but we should note that actually that's not a problem, since D's role is to distill the data into itself and provide useful learning signal for G. We can also hypothesize that recent attempts to weight-share D with an inference net (or a classifier?) are not hugely successful because of this discrepancy in objectives.}



%\karencom{Do we need this paragraph? Maybe remove or move to Appendix}
%We seek further insight into this pathology by considering two experiments with uncollapsed models. First, we freeze \gen{} partway through training and continue to train \discr{}. In this setting, \discr{} is stable, suggesting either that \discr{} is capable of additionally memorizing the space of \gen{}'s samples, or that \discr{}'s memorization of the training set is not problematic for stability. 
%We repeat this experiment with the roles reversed, freezing \discr{} and training \gen{}. This results in immediate training collapse, with \gen{} bottoming out its loss, maxing out \discr{}'s loss, and sample quality diverging. From this, we draw two conclusions. First, as often noted in the literature, it is absolutely necessary for \discr{} to be somewhat optimal with respect to any given \gen{} in order for training to proceed. Second, training collapse does not solely stem from poor behavior in either \gen{} or \discr{}, but from their interactions in the adversarial training process. The success of methods which target \discr{}'s conditioning can arguably be understood from this point of view: so long as \discr{} is constrained and near-optimal, it provides gradients which keep \gen{} well-behaved, and render \gen{} unable to induce collapse.


%-This leads to the following conclusion/posit/supposition/hypothesis: Regardless of G's conditioning, the consequence of G winning the game is immediate and dramatic training collapse. It follows that D must remain optimal throughout training to prevent G from doing this, and that the gradients that flow from D should be constrained to prevent G from being able to adversarially annihilate the whole kit and kaboodle.

%Previous works have argued for conditioning and keeping D optimal based on the need to provide good gradients everywhere (non saturation) but have not... -this is in contrast to previous works which argue that the optimality of the discriminator is necessary in order for the gradients to G to be good? I guess that's still what we're seeing, if D is not optimal then its gradients are incorrect and G will follow a bad path and go unstable


%-stability requires optimal D, and G will break the game given the chance. it follows that it is necessary to ensure the conditioning of D is good in some sense--since regardless of g's conditioning, g will always 

%We posit that the poor conditioning we observe is a symptom



% suggesting that even if \discr{} is memorizing the training set, this 

% If this behavior is the source of instability, we might expect that training \discr{} alone with a frozen \gen{}

% given the capacity of our models, it is reasonable to suspect overfitting, but what does that actually mean here?

% We illuminate, investigate, study the effects, consequences
% We test this hypothesis by considering two experiments with uncollapsed models. First, we freeze \gen{} partway through training and continue to train \discr{}. In this setting, \discr{} is stable


% -however in the other test that's not the case--D is perfectly stable when G is not trained

% -experiment: Freeze G, D is stable, freeze D, G is stable
% -Conclusion: instability comes not from G or from D alone, but explicitly from their interaction

%The consequences of this behavior are further illuminated by the following experiment


%To further illuminate this behavior, we





%---describe experiments where we freeze G and where we freeze D






%  damping the oscillations and [enforcing stability] \andycom{awk, fix}



% This behavior indicates that while the primary symptoms of training collapse appear suddenly and sharply, other early indicators are present long before. We find that even if a training run is reset to a checkpoint thousands of iterations before collapse (with a new random seed governing noise samples and data sampling), collapse still occurs at approximately the same iteration. Freezing the first linear layer or modifying learning rates does not prevent this--another early layer will explode instead.

% This is also consistent with our observation that even if a training run is reset to a checkpoint thousands of iterations before collapse, collapse will still occur at approximately the same iteration and level of performance.

%\andycom{we suspect it's either the first singular value or the ratio of the first singular value which explode}. This may also explain why even if one resets to a checkpoint thousands of iterations before collapse, collapse will occur at approximately the same performance and iteration. % The last sentence may need to be rearranged and stuck elsewhere

%-enforcing very strong conditioning (summed ortho2) on G results in all singular values being 1, but doesn't prevent a crash; we conclude that while \gen{} conditioning is *necessary* for GAN stability, it is not *sufficient* and turn our attention to \discr{}


% -we also tried regularizing these quantities instead of clamping them, with similar results

% Even with very strong orthogonal regularization applied to \gen{} (setting all singular values to $1$ throughout training and enforcing a fixed Frobenius norm), collapse still occurs.



% junk paragraphs rewrite bro
%One possible explanation for these spikes is that \discr{} is memorizing the training data, and as it approaches optimality it receives less and less signal from real data, eventually becoming biased due to solely receiving gradients that encourage its outputs to be negative. Once this bias passes a certain threshold, \discr{} 

%-one other explanation would be a super-adversarial example knocks D's socks off, when G gets it just right for a batch or two and confuses the heck outta D, then it rights itself?

%-Note that even if we reset training a ton and roll the dice repeatedly so that we trundle along and avoid the huge loss spikes, collapse still occurs, so while the big spikes may play a part, they're not the only thing and avoiding them does not fix the problem.


\newpage
\section{Negative Results}
\label{appendix_negative_results}
We explored a range of novel and existing techniques which ended up degrading or otherwise not affecting performance in our setting. We report them here; our evaluations for this section are not as thorough as those for the main architectural choices. 

Our intention in reporting these results is to save time for future work, and to give a more complete picture of our attempts to improve performance or stability. We note, however, that these results must be understood to be specific to the particular setup we used. A pitfall of reporting negative results is that one might report that a particular technique doesn't work, when the reality is that this technique did not have the desired effect when applied in a particular way to a particular problem. Drawing overly general conclusions might close off potentially fruitful avenues of research.

\begin{itemize}
\item We found that doubling the depth (by inserting an additional Residual block after every up- or down-sampling block) hampered performance. 

\item We experimented with sharing class embeddings between both \gen{} and \discr{} (as opposed to just within \gen{}). This is accomplished by replacing \discr{}'s class embedding with a projection from \gen{}'s embeddings, as is done in \gen{}'s BatchNorm layers. In our initial experiments this seemed to help and accelerate training, but we found this trick scaled poorly and was sensitive to optimization hyperparameters, particularly the choice of number of \discr{} steps per \gen{} step.

\item We tried replacing BatchNorm in \gen{} with WeightNorm \citep{salimans2016weightnorm}, but this crippled training. We also tried removing BatchNorm and only having Spectral Normalization, but this also crippled training.

\item We tried adding BatchNorm to \discr{} (both class-conditional and unconditional) in addition to Spectral Normalization, but this crippled training.

\item We tried varying the choice of location of the attention block in \gen{} and \discr{} (and inserting multiple attention blocks at different resolutions) but found that at 128$\times$128 there was no noticeable benefit to doing so, and compute and memory costs increased substantially. We found a benefit to moving the attention block up one stage when moving to 256$\times$256, which is in line with our expectations given the increased resolution.

\item We tried using filter sizes of 5 or 7 instead of 3 in either \gen{} or \discr{} or both. We found that having a filter size of 5 in \gen{} only provided a small improvement over the baseline but came at an unjustifiable compute cost. All other settings degraded performance. 
\item We tried varying the dilation for convolutional filters in both \gen{} and \discr{} at 128$\times$128, but found that even a small amount of dilation in either network degraded performance.

\item We tried bilinear upsampling in \gen{} in place of nearest-neighbors upsampling, but this degraded performance.
\item In some of our models, we observed class-conditional mode collapse, where the model would only output one or two samples for a subset of classes but was still able to generate samples for all other classes. We noticed that the collapsed classes had embedings which had become very large relative to the other embeddings, and attempted to ameliorate this issue by applying weight decay to the shared embedding only. We found that small amounts of weight decay ($10^{-6}$) instead degraded performance, and that only even smaller values ($10^{-8}$) did not degrade performance, but these values were also too small to prevent the class vectors from exploding. Higher-resolution models appear to be more resilient to this problem, and none of our final models appear to suffer from this type of collapse.

\item We experimented with using MLPs instead of linear projections from \gen{}'s class embeddings to its BatchNorm gains and biases, but did not find any benefit to doing so. We also experimented with Spectrally Normalizing these MLPs, and with providing these (and the linear projections) with a bias at their output, but did not notice any benefit.

\item We tried gradient norm clipping (both the global variant typically used in recurrent networks, and a local version where the clipping value is determined on a per-parameter basis) but found this did not alleviate instability.


\end{itemize}




\newpage
\section{Hyperparameters}
\label{appendix_sweeps}
We performed various hyperparameter sweeps in this work:
\begin{itemize}
\item We swept the Cartesian product of the learning rates for each network through [$10^{-5}$, $5\cdot10^{-5}$, $10^{-4}$, $2\cdot10^{-4}$, $4\cdot10^{-4}$, $8\cdot10^{-4}$, $10^{-3}$], and initially found that the SA-GAN settings (\gen{}'s learning rate $10^{-4}$, \discr{}'s learning rate $4\cdot10^{-4}$) were optimal at lower batch sizes; we did not repeat this sweep at higher batch sizes but did try halving and doubling the learning rate, arriving at the halved settings used for our experiments.

\item We swept the R1 gradient penalty strength through [$10^{-3}$, $10^{-2}$, $10^{-1}$, $0.5$, $1$, $2$, $3$, $5$, $10$]. We find that the strength of the penalty correlates negatively with performance, but that settings above $0.5$ impart training stability.

\item We swept the keep probabilities for DropOut in the final layer of \discr{} through [$0.5$, $0.6$, $0.7$, $0.8$, $0.9$, $0.95$]. We find that DropOut has a similar stabilizing effect to R1 but also degrades performance.

\item We swept \discr{}'s Adam $\beta_1$ parameter through [$0.1$, $0.2$, $0.3$, $0.4$, $0.5$] and found it to have a light regularization effect similar to DropOut, but not to significantly improve results. Higher $\beta_1$ terms in either network crippled training.

\item We swept the strength of the modified Orthogonal Regularization penalty in \gen{} through [$10^{-5}$, $5\cdot10^{-5}$, $10^{-4}$, $5\cdot10^{-4}$, $10^{-3}$, $10^{-2}$], and selected $10^{-4}$.

\end{itemize}





\end{appendices}

\end{document}

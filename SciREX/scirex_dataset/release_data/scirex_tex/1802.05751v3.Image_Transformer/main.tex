\documentclass{article} % For LaTeX2e

% Recommended, but optional, packages for figures and better typesetting:
\usepackage{microtype}
\usepackage{subfigure}
\usepackage{booktabs} % for professional tables

\usepackage[colorlinks=true, linkcolor=black, citecolor=black, filecolor=black, urlcolor=black]{hyperref}
\usepackage{enumerate}
\usepackage{url}
\usepackage{amsmath}
\usepackage[draft]{fixme}
\usepackage{graphicx}
\usepackage{xcolor}
\usepackage{tikz}
\usepackage{multicol}
\usepackage[utf8]{inputenc}
% \usepackage{bbm}

% \usepackage{longtable}
% Helpful commands.
\newcommand\sigmoid{\sigma}
\newcommand\argmax{\mathrm{argmax}}
\newcommand\gru{\ensuremath{\mathrm{GRU}}}
\newcommand\cgru{\ensuremath{\mathrm{CGRU}}}
\newcommand\dcgru{\ensuremath{\mathrm{CGRU}^d}}
\newcommand\sfin{s_\mathrm{fin}}
\newcommand\floor[1]{\left \lfloor{#1}\right \rfloor}
\newcommand{\calM}{\mathcal{M}}
\newcommand{\norm}[1]{\left\lVert#1\right\rVert}
\newcommand{\height}{h}
\newcommand{\width}{w}
\newcommand{\modeldim}{d}
\newcommand{\query}{q}
\newcommand{\memory}{m}
\newcommand{\val}{v}
\newcommand{\key}{k}
\newcommand{\vect}[1]{\boldsymbol{\mathbf{#1}}}
% Custom Commands:
\newcommand\blfootnote[1]{%
  \begingroup
  \renewcommand\thefootnote{}\footnote{#1}%
  \addtocounter{footnote}{-1}%
  \endgroup
}
% USE THIS FOR COMMENTS ON THE PAPER'S MARGIN.
% \sidenote{trandustin: my comments here which don't block text}
% \usepackage[usenames,dvipsnames]{xcolor}
\usepackage{ragged2e}
\DeclareRobustCommand{\sidenote}[1]{\marginpar{
                                    \RaggedRight
                                    \textcolor{red}{\textsf{#1}}}}
\setlength{\marginparwidth}{0.5in} % For sidenotes on two-column papers
% \title{Image Transformer}


% Attempt to make hyperref and algorithmic work together better:
% \newcommand{\theHalgorithm}{\arabic{algorithm}}

% Use the following line for the initial blind version submitted for review:
% \usepackage{icml2018}
% If accepted, instead use the following line for the camera-ready submission:
\usepackage[accepted]{icml2018}
% The \icmltitle you define below is probably too long as a header.
% Therefore, a short form for the running title is supplied here:
% \icmltitlerunning{Submission and Formatting Instructions for ICML 2018}

\begin{document}

\twocolumn[
\icmltitle{Image Transformer}

% It is OKAY to include author information, even for blind
% submissions: the style file will automatically remove it for you
% unless you've provided the [accepted] option to the icml2018
% package.

% List of affiliations: The first argument should be a (short)
% identifier you will use later to specify author affiliations
% Academic affiliations should list Department, University, City, Region, Country
% Industry affiliations should list Company, City, Region, Country

% You can specify symbols, otherwise they are numbered in order.
% Ideally, you should not use this facility. Affiliations will be numbered
% in order of appearance and this is the preferred way.
\icmlsetsymbol{equal}{*}

\begin{icmlauthorlist}
\icmlauthor{Niki Parmar *}{g}
\icmlauthor{Ashish Vaswani *}{g}
\icmlauthor{Jakob Uszkoreit}{g}



\icmlauthor{\L{}ukasz Kaiser}{g}
\icmlauthor{Noam Shazeer}{g}
\icmlauthor{Alexander Ku }{b,i}
\icmlauthor{Dustin Tran}{a}
\icmlaffiliation{g}{Google Brain, Mountain View, USA}
\icmlaffiliation{a}{Google AI, Mountain View, USA}
\icmlaffiliation{b}{Department of Electrical Engineering and Computer Sciences, University of California, Berkeley}
\icmlaffiliation{i}{Work done during an internship at Google Brain}
\end{icmlauthorlist}


\icmlcorrespondingauthor{Ashish Vaswani, Niki Parmar, Jakob Uszkoreit}{avaswani@google.com, nikip@google.com, usz@google.com}

% You may provide any keywords that you
% find helpful for describing your paper; these are used to populate
% the "keywords" metadata in the PDF but will not be shown in the document
\icmlkeywords{Machine Learning, ICML}

\vskip 0.3in
]
%\icmlEqualContribution

% this must go after the closing bracket ] following \twocolumn[ ...

% This command actually creates the footnote in the first column
% listing the affiliations and the copyright notice.
% The command takes one argument, which is text to display at the start of the footnote.
% The \icmlEqualContribution command is standard text for equal contribution.
% Remove it (just {}) if you do not need this facility.
%\footnote{\icmlEqualContribution}
%\footnote{\icmlcorrespondingauthor}
%\printAffiliationsAndNotice{}  % leave blank if no need to mention equal contribution
\printAffiliationsAndNotice{\icmlEqualContribution} % otherwise use the standard text.

% \begin{document}
% \maketitle
% \vspace{2cm}
\begin{abstract}
Image generation has been successfully cast as an autoregressive sequence generation or transformation problem. Recent work has shown that self-attention is an effective way of modeling textual sequences.
In this work, we generalize a recently proposed model architecture based on self-attention, the Transformer, to a sequence modeling formulation of image generation with a tractable likelihood. By restricting the self-attention mechanism to attend to local neighborhoods we significantly increase the size of images the model can process in practice, despite maintaining significantly larger receptive fields per layer than typical convolutional neural networks. While conceptually simple, our generative models significantly outperform the current state of the art in image generation on ImageNet, improving the best published negative log-likelihood on ImageNet from 3.83 to 3.77.
We also present results on image super-resolution with a large magnification ratio, applying an encoder-decoder configuration of our architecture. In a human evaluation study, we find that images generated by our super-resolution model fool human observers three times more often than the previous state of the art.

% TODO: Add back after open sourcing
%\blfootnote{Code available at \url{anonymized}}
\end{abstract}

% \vspace{-5mm}
% \begin{center}
% \begin{longtable}{@{\hspace{.05cm}}c@{\hspace{.05cm}}c@{\hspace{.05cm}}c@{\hspace{.05cm}}c@{\hspace{.05cm}}c@{\hspace{.05cm}}c@{\hspace{.05cm}}c@{\hspace{.05cm}}c@{\hspace{.05cm}}c} \\ 
% % Input & Bicubic & \regression & $\tau=1.0$ & $\tau=0.9$ & $\tau=0.8$ & Truth & \NN{} & GAN~\cite{srez} & GAN~\cite{srez} \\ 
%  \endhead 
%  \hspace{-8mm}\includegraphics[width=.12\linewidth]{front_page_images/5_input.png}} 
% & {\includegraphics[width=.12\linewidth]{front_page_images/5_output.png}}
% & {\includegraphics[width=.12\linewidth]{front_page_images/5_target.png}}
% & {\includegraphics[width=.12\linewidth]{front_page_images/588_input.png}}
% & {\includegraphics[width=.12\linewidth]{front_page_images/588_output.png}}
% & {\includegraphics[width=.12\linewidth]{front_page_images/588_target.png}}
% & {\includegraphics[width=.12\linewidth]{front_page_images/787_input.png}}
% & {\includegraphics[width=.12\linewidth]{front_page_images/787_output.png}}
% & {\includegraphics[width=.12\linewidth]{front_page_images/787_target.png}}
%  \\ [-0.75mm]
% {\hspace{-8mm}\includegraphics[width=.12\linewidth]{front_page_images/labelwhited_5.png}}
% & {\includegraphics[width=.12\linewidth]{front_page_images/labeloutputs_cifar10_completed_1_0_rs8_739.png}}
% & {\includegraphics[width=.12\linewidth]{front_page_images/labeltargets_cifar10_completed_1_0_rs8_736.png}}
% & {\includegraphics[width=.12\linewidth]{front_page_images/labelwhited_0.png}}
% & {\includegraphics[width=.12\linewidth]{front_page_images/labeloutputs_cifar10_completed_1_0_rs8_3244.png}}
% & {\includegraphics[width=.12\linewidth]{front_page_images/labeltargets_cifar10_completed_1_0_rs8_3240.png}}
% & {\includegraphics[width=.12\linewidth]{front_page_images/labelwhited_7.png}}
% & {\includegraphics[width=.12\linewidth]{front_page_images/labeloutputs_cifar10_completed_1_0_rs8_3416.png}}
% & {\includegraphics[width=.12\linewidth]{front_page_images/labeltargets_cifar10_completed_1_0_rs8_3422.png}}
% % & {\includegraphics[width=.12\linewidth]{front_page_images/labelwhited_6.png}}
% % & {\includegraphics[width=.12\linewidth]{front_page_images/labeloutputs_cifar10_completed_1_0_rs8_832.png}}
% % & {\includegraphics[width=.12\linewidth]{front_page_images/labeltargets_cifar10_completed_1_0_rs8_833.png}}

% \\
 
% \end{longtable} 
% \end{center}
    
\vspace{-5mm}
\begin{table}[h!]
\begin{center} 
\begin{tabular}{@{\hspace{.05cm}}c@{\hspace{.05cm}}c@{\hspace{.05cm}}c@{\hspace{.5cm}}c@{\hspace{.05cm}}c@{\hspace{.05cm}}c@{\hspace{.05cm}}c} \\ 
% Input & Bicubic & \regression & $\tau=1.0$ & $\tau=0.9$ & $\tau=0.8$ & Truth & \NN{} & GAN~\cite{srez} & GAN~\cite{srez} \\ 
%  \endhead 
 {\includegraphics[width=.15\linewidth]{front_page_images/5_input.png}}
& {\includegraphics[width=.15\linewidth]{front_page_images/5_output.png}}
& {\includegraphics[width=.15\linewidth]{front_page_images/5_target.png}}
& & {\includegraphics[width=.15\linewidth]{front_page_images/labelwhited_5.png}}
& {\includegraphics[width=.15\linewidth]{front_page_images/labeloutputs_cifar10_completed_1_0_rs8_739.png}}
& {\includegraphics[width=.15\linewidth]{front_page_images/labeltargets_cifar10_completed_1_0_rs8_736.png}}
\\ [-0.75mm]
{\includegraphics[width=.15\linewidth]{front_page_images/588_input.png}}
& {\includegraphics[width=.15\linewidth]{front_page_images/588_output.png}}
& {\includegraphics[width=.15\linewidth]{front_page_images/588_target.png}}
& & {\includegraphics[width=.15\linewidth]{front_page_images/labelwhited_0.png}}
& {\includegraphics[width=.15\linewidth]{front_page_images/labeloutputs_cifar10_completed_1_0_rs8_3244.png}}
& {\includegraphics[width=.15\linewidth]{front_page_images/labeltargets_cifar10_completed_1_0_rs8_3240.png}}
\\ [-0.75mm]
 {\includegraphics[width=.15\linewidth]{front_page_images/787_input.png}}
& {\includegraphics[width=.15\linewidth]{front_page_images/787_output.png}}
& {\includegraphics[width=.15\linewidth]{front_page_images/787_target.png}}
& & {\includegraphics[width=.15\linewidth]{front_page_images/labelwhited_7.png}}
& {\includegraphics[width=.15\linewidth]{front_page_images/labeloutputs_cifar10_completed_1_0_rs8_3416.png}}
& {\includegraphics[width=.15\linewidth]{front_page_images/labeltargets_cifar10_completed_1_0_rs8_3422.png}}
% & {\includegraphics[width=.105\linewidth]{front_page_images/labelwhited_6.png}}
% & {\includegraphics[width=.105\linewidth]{front_page_images/labeloutputs_cifar10_completed_1_0_rs8_832.png}}
% & {\includegraphics[width=.105\linewidth]{front_page_images/labeltargets_cifar10_completed_1_0_rs8_833.png}}
\\
% \label{fig:front-page}
% \caption{Three outputs of a CelebA super-resolution model followed by three image completions by a conditional CIFAR-10 model, with input, model output and the original from left to right}

\end{tabular}
% \label{fig:front-page}
\caption{Three outputs of a CelebA super-resolution model followed by three image completions by a conditional CIFAR-10 model, with input, model output and the original from left to right}
\end{center}
\end{table}

\section{Introduction}
\section{Introduction}

Humans use different forms of communications such as speech, hand gestures and emotions. Being able to understand one's emotions and the encoded feelings is an important factor for an appropriate and correct understanding.


With the ongoing research in the field of robotics, especially in the field of humanoid robots, it becomes interesting to integrate these capabilities into machines allowing for a more diverse and natural way of communication. One example is the Software called EmotiChat~\cite{Anderson06areal-time}. This is a chat application with emotion recognition. The user is monitored and whenever an emotion is detected (smile, etc.), an emoticon is inserted into the chat window. Besides Human Computer Interaction other fields like surveillance or driver safety could also profit from it. Being able to detect the mood of the driver could help to detect the level of attention, so that automatic systems can adapt.\\
\let\thefootnote\relax\footnote{*F. Trier and P. Burkert contributed equally to this work.}


Many methods rely on extraction of the facial region. This can be realized through manual inference~\cite{4032815} or an automatic detection approach~\cite{Anderson06areal-time}.
Methods often involve the Facial Action Coding System (FACS) which describes the facial expression using Action Units (AU). An Action Unit is a facial action like "raising the Inner Brow". Multiple activations of AUs describe the facial expression~\cite{kumar2009face}. Being able to correctly detect AUs is a helpful step, since it allows making a statement about the activation level of the corresponding emotion. \\
Handcrafted facial landmarks can be used such as done by Kotsia et al.~\cite{4032815}. Detecting such landmarks can be hard, as the distance between them differs depending on the person~\cite{6998925}. Not only AUs can be used to detect emotions, but also texture. When a face shows an emotion the structure changes and different filters can be applied to detect this~\cite{6998925}.\\


\begin{figure}
   \centering
        \includegraphics[width=\columnwidth]{Fig1}
   \caption{Example images from the MMI (top) and CKP (bottom). The emotions from left to right are: \textit{Anger}, \textit{Sadness}, \textit{Disgust}, \textit{Happiness}, \textit{Fear}, \textit{Surprise}. The emotion \textit{Contempt} of the CKP set is not displayed.}\label{fig:example_images}
\end{figure}




The presented approach uses Artificial Neural Networks (ANN). ANNs differ, as they are trained on the data with less need for manual interference. 
Convolutional Neural Networks are a special kind of ANN and have been shown to work well as feature extractor when using images as input~\cite{donahue2013decaf} and are real-time capable. This allows for the usage of the raw input images without any pre- or postprocessing.\\
GoogleNet~\cite{DBLP:journals/corr/SzegedyLJSRAEVR14} is a deep neural network architecture that relies on CNNs. It has been introduced during the Image Net Large Scale Visual Recognition Challenge(ILSVRC) 2014. This challenge analyses the quality of different image classification approaches submitted by different groups. The images are separated into 1000 different classes organized by the WordNet hierarchy. In the challenge "object detection with additional training data" GoogleNet has achieved about 44\% precision~\cite{LSVRC-results}. These results have demonstrated the potential which lies in this kind of architecture. Therefore it has been used as inspiration for the proposed architecture.\\
The proposed network has been evaluated on the Extended Cohn-Kanade Dataset (Section~\ref{sec:ckp}) and on the MMI Dataset (Section~\ref{sec:mmi}). Typical pictures of persons showing emotions can be seen in Fig.~\ref{fig:example_images}.
The emotion \textit{Contempt} of the CKP set is not shown as no subject with consent for publication and an annotated emotion is part of the dataset. Results of experiments on these datasets demonstrate the success of using a deep layered neural network structure. With a 10-fold cross-validation a recognition accuracy of 99.6\% has been achieved. \\

The paper is arranged as follows: After this introduction, Related Work (Section~\ref{sec:related}) is presented which focuses on Emotion/Expression recognition and the various approaches scientists have taken. Next is Section~\ref{sec:background}, Background, which focuses on the main components of the architecture proposed in this article. Section~\ref{sec:datasets} contains a summary of the used Datasets. In Section~\ref{sec:architecture} the architecture is presented. This is followed by the experiments and its results (Section~\ref{sec:experiments}) . Finally, Section~\ref{sec:conclusion} summarizes the article and concludes the article.

\section{Background}
In the simplest seq2seq scenario, we are given a collection of source-target
sequence pairs and tasked with learning to generate
target sequences from source sequences. For instance, we might view machine translation in this way, where in particular we attempt to generate English sentences from (corresponding) French sentences. Seq2seq models are part of the broader class of ``encoder-decoder'' models~\cite{cho14on}, which first use an encoding model to transform a source object into an encoded representation $\boldx$. Many different sequential
(and non-sequential) encoders have proven to be effective for
different source domains. In this work we are agnostic to the
form of the encoding model, and simply assume an abstract source
representation $\boldx$. %In experiments we utilize an attention-based LSTM encoder \cite{} which has shown to be effective for many tasks \cite{}.

Once the input sequence is encoded, seq2seq models generate a target
sequence using a \textit{decoder}. The decoder is tasked with
generating a target sequence of words from a target vocabulary $\mcV$. In particular, words are generated sequentially by conditioning on the input representation $\boldx$ and on the previously generated words or \textit{history}. We use the notation $\pfx{T}$ to refer to an arbitrary word sequence of length $T$, and the notation $\goldpfx{T}$ to refer to the \textit{gold} (i.e., correct) target word sequence for an input $\boldx$. 

Most seq2seq systems utilize a recurrent neural network (RNN) for the decoder model. Formally, a recurrent neural network is a parameterized non-linear
function $\RNN$ that recursively maps a sequence of vectors to a
sequence of hidden states. Let $\boldm_1, \ldots, \boldm_T$ be a
sequence of $T$ vectors, and let $\boldh_0$ be some initial state
vector. Applying an RNN to any such sequence yields hidden states
$\boldh_t$ at each time-step $t$, as follows:
\begin{align*}
\boldh_t \gets \RNN(\boldm_t, \boldh_{t-1}; \btheta),
\end{align*}
where $\btheta$ is the set of model parameters, which are shared over time. In this work, the vectors $\boldm_t$ will always correspond to the embeddings of a target word sequence $\pfx{T}$, and so we will also write $\boldh_t \gets \RNN(w_t, \boldh_{t-1}; \btheta)$, with $w_t$ standing in for its embedding.
 
%To back-propagate errors through a recurrent neural network, we accumulate the 
%gradients of each state with respect to subsequent states by running a backward procedure we will refer to as $\BRNN$ at each time-step (starting at the penultimate step): 
%\begin{align*}
%\nabla_{\boldh_t} \mcL \gets \BRNN(y_{t+1}, \boldh_{t},\nabla_{\boldh_{t+1}} \mcL),
%\end{align*}
%$\BRNN$ takes into account $\boldh_t$'s contribution to any loss incurred from its next-step prediction, as well as to any loss incurred through $\boldh_{t+1}$. In what follows, we will often abbreviate $\nabla_{\boldh_t} \mcL$ as $\nabla_{\boldh_t}$.  
%%\begin{align*}
%%\nabla_{\boldh_t} \mcL \gets \nabla_{\boldh_t} \mcL + \BRNN(\nabla_{\boldh_{t+1}} \mcL, \boldm_t, \boldh_{t}).
%%\end{align*}
%%Note that $\boldm_t$ is the embedding corresponding to output word $w_t$. 
%Running this $\BRNN$ procedure from $t \niceq T$ to $t \niceq 1$ is known as back-propagation through time (BPTT).

%\textbf{something about BPTT}

%  which takes the form of a recurrent
% neural network (RNN). 

% where a
% decoder RNN generates a target sequence of T
% words w1 · · · wT (such as a translation or summary),
% from an

% As RNN decoding is the main focus of this work,
% we now describe this process in greater detail.  

RNN decoders are typically trained to act as conditional language
models. That is, one attempts to model the probability of the $t$'th target
word conditioned on $\boldx$ and the target history by stipulating that $p(w_{t} | \pfx{t-1}, \boldx) \niceq g(w_{t},
\boldh_{t-1}, \boldx)$, for some parameterized function $g$ typically computed with an affine layer followed by a softmax. In computing these probabilities, the state $\boldh_{t-1}$ represents the target history, and $\boldh_0$ is typically set to be some function of $\boldx$. The complete model (including encoder) is trained,
analogously to a neural language model, to minimize the cross-entropy
loss at each time-step while conditioning on the gold history in the
training data. That is, the model is trained to minimize $-\ln \prod_{t=1}^{T} p(y_{t} |\goldpfx{t-1}, \boldx)$.

Once the decoder is trained, discrete sequence generation can be
performed by approximately maximizing the probability of the target
sequence under the conditional distribution,
$\hat{y}_{1:T} \niceq \mathrm{argbeam}_{w_{1:T}} \prod_{t=1}^{T} p(w_t |\pfx{t-1}, \boldx)$, where we use the notation $\mathrm{argbeam}$ to emphasize that the decoding process requires heuristic search, since the RNN model is non-Markovian. In practice, a simple beam search
procedure that explores $K$ prospective histories at each time-step
has proven to be an effective decoding approach. However, as noted above,
decoding in this manner after conditional language-model style training \textit{potentially} suffers from the issues of exposure bias and label bias, which motivates the work of this paper.

% However we note that this procedure potentially
% suffers from the issues 


% and we will often omit the
% $\boldx$ argument to $f$ when there is only a single $\boldx$ in
% question.


  

% , which often takes the form of a recurrent
% neural network. 



% For the sake of this work the sequential form of the input sequence is
% actually 


% Seq2seq is highly related to the corresponding 
% \textit{encoder-decoder} approached  



%  $w_{1:s}$ 
% $w_{1:t}$


% It has become popular in recent years to use RNNs within an
% ``encoder-decoder'' framework, where a decoder RNN generates a target
% sequence of $T$ words $\longpfx{T}$ (such as a translation or
% summary), from an 


%  The methods we describe below are designed
% specifically for encoder-decoder scenarios where the decoder is an
% RNN; we make no assumption about the encoder.



% \noindent \textbf{RNNs:} A recurrent neural network is a parameterized
% non-linear function $\RNN$ that recursively maps a sequence of vectors
% to a sequence of hidden states. Let $\boldm_1, \ldots, \boldm_t$ be a
% sequence of $t$ vectors, and let $\boldh_0$ be some initial state
% vector. Applying an RNN to any such sequence yields hidden states
% $\boldh_t$ at each time-step, as follows:
% %{\small
% \begin{align*}
% \boldh_t \gets \RNN(\boldm_t, \boldh_{t-1}; \btheta),
% \end{align*}
% %}
% \noindent where $\btheta$ is the set of model parameters, which are shared over time. 


%Accordingly, we consider the generation of target word sequences $\longpfx{T}$ of length $T$, where we have used $\cdot$ as the concatenation operator, and where each word token $w_j$ comes from our target vocabulary $\mcV$. We denote by $\boldx$ the input representation on which the target generation conditions. We refer to the \textit{gold} (i.e., correct) output word sequence for an input $\boldx$ as $\longgoldpfx{T}$. We will often abbreviate sequences  $\longpfx{T}$ as $\pfx{T}$. % (and similarly for $\longgoldpfx{T}$ and $\goldpfx{T}$).\\ %, and we refer to set of all possible $\boldx$'s as $\mcX$.  \\

% When using an RNN decoder, it is typical to model the probability of
% the $t\,{+}\,1$'st target word's type being $w$ given the preceding
% words and the input as a function of $\boldh_t$. That is, one
% stipulates that $p(w_{t+1} \niceq w|\pfx{t}, \boldx) \propto g(w,
% \boldh_t, \boldx)$, for some function $g$ that examines the hidden
% state at time $t$ and $\boldx$. It is then natural to train such a
% model with a cross-entropy loss at each time-step. In this paper we
% will instead be interested in modeling non-probabilistic scores of
% arbitrary \textit{sequences} formed from the target vocabulary
% $\mcV$. We will accordingly define the score of an entire
% \textit{prefix} $\pfx{t}$ followed by a single word $w$ as
% \begin{align} \label{eq:score}
% \score(\pfx{t} \cdot w) \triangleq f(w, \boldh_t, \boldx),
% \end{align} 
% where, analogously, $f$ is some function examining the current hidden-state of the relevant RNN at time $t$ as well as the input representation $\boldx$. Note that we use $\cdot$ as the concatenation operator.  


\begin{table*}[h]
\begin{center}
\begin{tabular}{@{\hspace{.05cm}}c@{\hspace{.05cm}}c@{\hspace{.05cm}}c@{\hspace{.05cm}}c@{\hspace{.05cm}}c@{\hspace{.3cm}}c@{\hspace{.05cm}}c@{\hspace{.05cm}}c} \\ 
% Input & Bicubic & \regression & $\tau=1.0$ & $\tau=0.9$ & $\tau=0.8$ & Truth & \NN{} & GAN~\cite{srez} & GAN~\cite{srez} \\ 
 %\endhead 
 Input & \multicolumn{3}{c}{Gen} & Truth & Input & Gen & Truth \\
  {\includegraphics[width=.1\linewidth]{cifar_img_superres_completion/labelwhited_0.png}} 
 & {\includegraphics[width=.1\linewidth]{cifar_img_superres_completion/labeloutputs_cifar10_completed_1_0_rs8_3243.png}} 
 & {\includegraphics[width=.1\linewidth]{cifar_img_superres_completion/labeloutputs_cifar10_completed_1_0_rs8_3244.png}} 
 & {\includegraphics[width=.1\linewidth]{cifar_img_superres_completion/labeloutputs_cifar10_completed_1_0_rs8_3245.png}} 
 & {\includegraphics[width=.1\linewidth]{cifar_img_superres_completion/labeltargets_cifar10_completed_1_0_rs8_3240.png}} 
 & {\includegraphics[width=.1\linewidth]{cifar_img_superres_completion/258input.png}} 
 & {\includegraphics[width=.1\linewidth]{cifar_img_superres_completion/258output.png}}
 & {\includegraphics[width=.1\linewidth]{cifar_img_superres_completion/258target.png}} 
 \\ [-0.75mm]
  {\includegraphics[width=.1\linewidth]{cifar_img_superres_completion/labelwhited_9.png}} 
 & {\includegraphics[width=.1\linewidth]{cifar_img_superres_completion/labeloutputs_cifar10_completed_1_0_rs8_1036.png}} 
 & {\includegraphics[width=.1\linewidth]{cifar_img_superres_completion/labeloutputs_cifar10_completed_1_0_rs8_1038.png}} 
 & {\includegraphics[width=.1\linewidth]{cifar_img_superres_completion/labeloutputs_cifar10_completed_1_0_rs8_1037.png}} 
 & {\includegraphics[width=.1\linewidth]{cifar_img_superres_completion/labeltargets_cifar10_completed_1_0_rs8_1036.png}} 
  & {\includegraphics[width=.1\linewidth]{cifar_img_superres_completion/352input.png}} 
 & {\includegraphics[width=.1\linewidth]{cifar_img_superres_completion/352output.png}}
 & {\includegraphics[width=.1\linewidth]{cifar_img_superres_completion/352target.png}} 
 \\ [-0.75mm]
   {\includegraphics[width=.1\linewidth]{cifar_img_superres_completion/labelwhited_5.png}} 
 & {\includegraphics[width=.1\linewidth]{cifar_img_superres_completion/labeloutputs_cifar10_completed_1_0_rs8_736.png}} 
 & {\includegraphics[width=.1\linewidth]{cifar_img_superres_completion/labeloutputs_cifar10_completed_1_0_rs8_737.png}} 
 & {\includegraphics[width=.1\linewidth]{cifar_img_superres_completion/labeloutputs_cifar10_completed_1_0_rs8_739.png}} 
 & {\includegraphics[width=.1\linewidth]{cifar_img_superres_completion/labeltargets_cifar10_completed_1_0_rs8_736.png}} 
 & {\includegraphics[width=.1\linewidth]{cifar_img_superres_completion/30input.png}} 
 & {\includegraphics[width=.1\linewidth]{cifar_img_superres_completion/30output.png}}
 & {\includegraphics[width=.1\linewidth]{cifar_img_superres_completion/30target.png}} 
 \\ [-0.75mm]
   {\includegraphics[width=.1\linewidth]{cifar_img_superres_completion/labelwhited_8.png}} 
 & {\includegraphics[width=.1\linewidth]{cifar_img_superres_completion/labeloutputs_cifar10_completed_1_0_rs8_205.png}} 
 & {\includegraphics[width=.1\linewidth]{cifar_img_superres_completion/labeloutputs_cifar10_completed_1_0_rs8_200.png}} 
 & {\includegraphics[width=.1\linewidth]{cifar_img_superres_completion/labeloutputs_cifar10_completed_1_0_rs8_203.png}} 
 & {\includegraphics[width=.1\linewidth]{cifar_img_superres_completion/labeltargets_cifar10_completed_1_0_rs8_203.png}} 
 & {\includegraphics[width=.1\linewidth]{cifar_img_superres_completion/140input.png}} 
 & {\includegraphics[width=.1\linewidth]{cifar_img_superres_completion/140output.png}}
 & {\includegraphics[width=.1\linewidth]{cifar_img_superres_completion/140target.png}} 
 \\
 \label{tab:completion_and_superres}
\end{tabular} 
\caption{On the left are image completions from our best conditional generation model, where we sample the second half. On the right are samples from our four-fold super-resolution model trained on CIFAR-10. Our images look realistic and plausible, show good diversity among the completion samples and observe the outputs carry surprising details for coarse inputs in super-resolution.}
\end{center}
\end{table*}
\section{Model Architecture}
\section{MT-DNN-1}
\label{sec:mt-dnn-1}

\subsection{Preliminaries}
\label{subsec:prelim}
In this work, our multi-task model combines classification, regression and pair-wise ranking tasks, which are summerised in Table~\ref{tab:task}. We briefly introduce the definition of each task as follows: 
\begin{table}[htb!]
	\begin{center}
		\begin{tabular}{@{\hskip1pt}l@{\hskip1pt}|@{\hskip1pt}c@{\hskip1pt}|@{\hskip1pt}c@{\hskip1pt}|@{\hskip1pt}c}
			\hline \bf Input &Classification&Regression &Ranking\\ \hline \hline
			single sentence &$\checkmark$&& \\
			pairwise text &$\checkmark$&$\checkmark$&$\checkmark$ \\ \hline
		\end{tabular}
	\end{center}
	\lgspace
	\caption{Summary of tasks in our multi-task framework.
	}
	\label{tab:task}
\lgspace
\end{table}
\begin{figure}[!t]
\centering
\adjustbox{trim={.065\width} {.01\height} {.05\width} {.01\height},clip}
{\includegraphics[scale=0.7]{mtl_model}}
\caption{Model architecture.}
\label{fig:mtl_model} 
\end{figure}

\begin{figure}[!t]
\centering
\adjustbox{trim={.05\width} {.01\height} {.05\width} {.01\height},clip}
{\includegraphics[scale=0.7]{mtl_model_v2}}
\caption{Model architecture version 2.}
\label{fig:mtl_model_v2} 
\end{figure}

\textbf{Task definition}

\textbf{Objective}

\textbf{Single classification}
\xiaodl{Need to cluster different tasks..}

\textbf{Sentence-pair classification}: given a pair of sentence, $(S_1, S_2)$, the model predicts a label indicating the relation of this pair of sentences: $P(C|S_1, S_2)$. For example, natural language inference is a typical instance of the sentence-pair classification task: a premise and a hypothesis are denoted by $S_1$ and $S_2$, respectively; the label, $C$, belongs one of three relations (\textit{contradiction}, \textit{neutral} and \textit{entailment}). 

\textbf{Regression}


\textbf{Pair-wise Ranking}
\begin{algorithm}[ht!]
 \SetAlgoLined
Initialize model parameters $\Theta$ randomly  \\
Set M \quad\textit{//the number of updates for the shared layer} \\
%\textit{Counter} = 0\\
 \For{$iteration$ in $0 ... \infty$}{
 	 %1. \textit{Counter} += 1\\
     1. Pick a task $t$ randomly \\
     2. Pick sample(s) from task $t$, i.e., \\
     \hspace{0.4cm}$(Q,C=\{0,1\})$ for classification \\
     \hspace{0.4cm}$(Q, D)$ for ranking\\
     3. Compute loss: $L(\Theta)$, i.e.,\\
     \hspace{0.4cm} the \textit{cross-entropy} for classification \\
     \hspace{0.4cm} the ranking loss for ranking\cite{learning-to-rank2005burges}\\

     4. Compute gradient: $\nabla(\Theta)$ \\
     5. Update model: $\Theta = \Theta - \epsilon \nabla(\Theta)$ \quad\textit{}
     % \eIf{Counter $<$ M}{
  	 %5. Update model: $\Theta = \Theta - \epsilon \nabla(\Theta)$ \quad\textit{//update both $\Theta^s$ and $\Theta^t$} \\
   %}{
   	% 6. Update model: $\Theta^t = \Theta^t - \epsilon \nabla(\Theta^t)$ 
  %}
 }
 \caption{\label{algo:mtdnn} Training a Multi-task model.}
 \algorithmfootnote{Note that $\Theta$ denotes the model parameters. \textcolor{red}{TODO: update alg based on task defination.}}
\end{algorithm}

\section{Inference}
Across all of the presented experiments, we use categorical sampling during decoding with a tempered $\mathrm{softmax}$ \citep{PixelRecursiveSuperResolution}. We adjust the concentration of the distribution we sample from with a temperature $\tau > 0$ by which we divide the logits for the channel intensities.

We tuned $\tau$ between $0.8$ and $1.0$, observing the highest perceptual quality in unconditioned and class-conditional image generation with  $\tau=1.0$.
For super-resolution we present results for different temperatures in Table~\ref{tab:CelebASuperResolution}.


\section{Experiments}
% !TEX root = ../multi_task.tex

We evaluate the presented MTL method on a number of problems. First, we use MultiMNIST \citep{multi_mnist}, an MTL adaptation of MNIST \citep{mnist}. Next, we tackle multi-label classification on the CelebA dataset \citep{celeba} by considering each label as a distinct binary classification task. These problems include both classification and regression, with the number of tasks ranging from 2 to 40. Finally, we experiment with scene understanding, jointly tackling the tasks of semantic segmentation, instance segmentation, and depth estimation on the Cityscapes dataset \citep{cityscapes}. We discuss each experiment separately in the following subsections.

The baselines we consider are (i) \textbf{uniform scaling:} minimizing a uniformly weighted sum of loss functions \mbox{$\frac{1}{T}\sum_t \lL^t$}, \mbox{(ii) \textbf{single task:}} solving tasks independently, \mbox{(iii) \textbf{grid search:}} exhaustively trying various values from $\{ c^t \in [0,1] | \sum_t c^t = 1\}$ and optimizing for $\frac{1}{T}\sum_t c^t \lL^t$, \mbox{(iv) \textbf{\citet{Kendall2018}:}} using the uncertainty weighting proposed by \citet{Kendall2018}, and \mbox{(v) \textbf{GradNorm:}} using the normalization proposed by \citet{Chen2018}.



\subsection{MultiMNIST}
\label{sec:multi_mnist_exp}

Our initial experiments are on MultiMNIST, an MTL version of the MNIST dataset \citep{multi_mnist}. In order to convert digit classification into a multi-task problem, \citet{multi_mnist} overlaid multiple images together. We use a similar construction. For each image, a different one is chosen uniformly in random. Then one of these images is put at the top-left and the other one is at the bottom-right. The resulting tasks are: classifying the digit on the top-left (task-L) and classifying the digit on the bottom-right (task-R). We use 60K examples and directly apply existing single-task MNIST models. The MultiMNIST dataset is illustrated in the supplement.

We use the LeNet architecture \citep{mnist}. We treat all layers except the last as the representation function $g$ and put two fully-connected layers as task-specific functions (see the supplement for details). We visualize the performance profile as a scatter plot of accuracies on task-L and task-R in Figure~\ref{fig:multi_mnist_performance_curve}, and list the results in Table~\ref{tab:multi_mnist}.

In this setup, any static scaling results in lower accuracy than solving each task separately (the single-task baseline). The two tasks appear to compete for model capacity, since increase in the accuracy of one task results in decrease in the accuracy of the other. Uncertainty weighting \citep{Kendall2018} and GradNorm \citep{Chen2018} find solutions that are slightly better than grid search but distinctly worse than the single-task baseline. In contrast, our method finds a solution that efficiently utilizes the model capacity and yields accuracies that are as good as the single-task solutions. This experiment demonstrates the effectiveness of our method as well as the necessity of treating MTL as multi-objective optimization. Even after a large hyper-parameter search, \emph{any} scaling of tasks does not approach the effectiveness of our method.



\subsection{Multi-Label Classification}

\begin{figure}[t]
\includegraphics[width=\textwidth]{radar_full_new}
\vspace{1mm}
\caption{Radar charts of percentage error per attribute on CelebA \citep{celeba}. Lower is better. We divide attributes into two sets for legibility: easy on the left, hard on the right. Zoom in for details.}
\label{fig:multi_label_radar}
\end{figure}


\begin{wraptable}{r}{0.3\textwidth}
%\vspace{-4mm}
\captionof{table}{Mean of error per category of MTL algorithms in multi-label classification on CelebA \citep{celeba}.}
\begin{tabular}{r@{\hspace{2mm}}c@{}}
\toprule
& Average  \\
&  error \\
\midrule
Single task & $8.77$ \\
Uniform scaling & $9.62$ \\
\citealt{Kendall2018} & $9.53$ \\
GradNorm & $8.44$ \\
Ours & $\mathbf{8.25}$  \\
\bottomrule
\end{tabular}
\label{table:multi_label_bar}
%\vspace{-5mm}
\end{wraptable}

Next, we tackle multi-label classification. Given a set of attributes, multi-label classification calls for deciding whether each attribute holds for the input. We use the CelebA dataset \citep{celeba}, which includes 200K face images annotated with 40 attributes. Each attribute gives rise to a binary classification task and we cast this as a 40-way MTL problem. We use ResNet-18 \citep{resnet} without the final layer as a shared representation function, and attach a linear layer for each attribute (see the supplement for further details).


We plot the resulting error for each binary classification task as a radar chart in Figure~\ref{fig:multi_label_radar}. The average over them is listed in Table~\ref{table:multi_label_bar}. We skip grid search since it is not feasible over 40 tasks. Although uniform scaling is the norm in the multi-label classification literature, single-task performance is significantly better. Our method outperforms baselines for significant majority of tasks and achieves comparable performance in rest. This experiment also shows that our method remains effective when the number of tasks is high.


\subsection{Scene Understanding}

To evaluate our method in a more realistic setting, we use scene understanding. Given an RGB image, we solve three tasks: semantic segmentation (assigning pixel-level class labels), instance segmentation (assigning pixel-level instance labels), and monocular depth estimation (estimating continuous disparity per pixel). We follow the experimental procedure of \citet{Kendall2018} and use an encoder-decoder architecture. The encoder is based on ResNet-50 \citep{resnet} and is shared by all three tasks. The decoders are task-specific and are based on the pyramid pooling module \citep{pspnet} (see the supplement for further implementation details).

Since the output space of instance segmentation is unconstrained (the number of instances is not known in advance), we use a proxy problem as in \citet{Kendall2018}. For each pixel, we estimate the location of the center of mass of the instance that encompasses the pixel. These center votes can then be clustered to extract the instances. In our experiments, we directly report the MSE in the proxy task. Figure~\ref{fig:cityscapes_performance_profile} shows the performance profile for each pair of tasks, although we perform all experiments on all three tasks jointly. The pairwise performance profiles shown in Figure~\ref{fig:cityscapes_performance_profile} are simply 2D projections of the three-dimensional profile, presented this way for legibility. The results are also listed in Table~\ref{tab:cityscapes_results}.

MTL outperforms single-task accuracy, indicating that the tasks cooperate and help each other. Our method outperforms all baselines on all tasks.


\subsection{Role of the Approximation}

In order to understand the role of the approximation proposed in Section~\ref{sec:approximation}, we compare the final performance and training time of our algorithm with and without the presented approximation in Table~\ref{tab:approximation_tradeoff} (runtime measured on a single Titan Xp GPU). For a small number of tasks (3 for scene understanding), training time is reduced by 40\%. For the multi-label classification experiment (40 tasks), the presented approximation accelerates learning by a factor of 25.

On the accuracy side, we expect both methods to perform similarly as long as the full-rank assumption is satisfied. As expected, the accuracy of both methods is very similar. Somewhat surprisingly, our approximation results in slightly improved accuracy in all experiments. While counter-intuitive at first, we hypothesize that this is related to the use of SGD in the learning algorithm. Stability analysis in convex optimization suggests that if gradients are computed with an error $\hat{\nabla}_\btheta \mathcal{L}^t = \nabla_\btheta \mathcal{L}^t + \mathbf{e}^t$ ($\btheta$ corresponds to $\btheta^{sh}$ in (\ref{eq:kkt_opt})), as opposed to $\mathbf{Z}$ in the approximate problem in \ref{eq:approx}, the error in the solution is bounded as $\|\hat{\mathbf{\alpha}} - \mathbf{\alpha} \|_2 \leq \mathcal{O}(\max_t \|\mathbf{e}^t\|_2)$. Considering the fact that the gradients are computed over the full parameter set (millions of dimensions) for the original problem and over a smaller space for the approximation (batch size times representation which is in the thousands), the dimension of the error vector is significantly higher in the original problem. We expect the $l_2$ norm of such a random vector to depend on the dimension.

In summary, our quantitative analysis of the approximation suggests that (i) the approximation does not cause an accuracy drop and (ii) by solving an equivalent problem in a lower-dimensional space, our method achieves both better computational efficiency and higher stability.

  {\small
  \begin{table}[t]
%  \vspace{-4mm}
  \caption{Effect of the MGDA-UB approximation. We report the final accuracies as well as training times for our method with and without the approximation.}
  %\vspace{1mm}
  \centering
  \begin{tabular}{@{}r@{\hspace{3mm}}c@{\hspace{3mm}}c@{\hspace{2mm}}c@{\hspace{2mm}}c@{}c@{\hspace{5mm}}c@{\hspace{2mm}}c@{}}
  \toprule
  & \multicolumn{4}{c}{Scene understanding (3 tasks)} &  & \multicolumn{2}{c}{Multi-label (40 tasks)}  \\
  \cmidrule(r){2-5} \cmidrule(lr){7-8}
                  & Training & Segmentation & Instance  & Disparity      & & Training & Average \\
                 & time     &  mIoU [\%]       & error [px] & error [px] & & time (hour)      & error \\
  \midrule
  Ours (w/o approx.) & $38.6$ & $66.13$ & $10.28$ & $2.59$ & & $429.9$ & $8.33$ \\
  Ours & $\mathbf{23.3}$ & $\mathbf{66.63}$ & $\mathbf{10.25}$ & $\mathbf{2.54}$  & & $\mathbf{16.1}$ & $\mathbf{8.25}$ \\
  \bottomrule
  \end{tabular}
  %\vspace{-2mm}
  \label{tab:approximation_tradeoff}
  \end{table}}


% % This is a table 

% \begin{center}
% \begin{longtable}{@{\hspace{.05cm}}c@{\hspace{.05cm}}c@{\hspace{.05cm}}c@{\hspace{.05cm}}c@{\hspace{.05cm}}c@{\hspace{.05cm}}c@{\hspace{.05cm}}c@{\hspace{.05cm}}c@{\hspace{.05cm}}c@{\hspace{.05cm}}c} \\ 
% % Input & Bicubic & \regression & $\tau=1.0$ & $\tau=0.9$ & $\tau=0.8$ & Truth & \NN{} & GAN~\cite{srez} & GAN~\cite{srez} \\ 
%  \endhead 
%   {\includegraphics[width=.09\linewidth]{cifar10_303_cond_images/0/labeloutputs_cifar10_perp_2097_1070.png}} 
%  & {\includegraphics[width=.09\linewidth]{cifar10_303_cond_images/0/labeloutputs_cifar10_perp_2097_807.png}} 
%  & {\includegraphics[width=.09\linewidth]{cifar10_303_cond_images/0/labeloutputs_cifar10_perp_2097_4116.png}} 
%  & {\includegraphics[width=.09\linewidth]{cifar10_303_cond_images/0/labeloutputs_cifar10_perp_2097_4095.png}} 
%  & {\includegraphics[width=.09\linewidth]{cifar10_303_cond_images/0/labeloutputs_cifar10_perp_2097_4006.png}} 
%  & {\includegraphics[width=.09\linewidth]{cifar10_303_cond_images/0/labeloutputs_cifar10_perp_2097_3910.png}} 
%  & {\includegraphics[width=.09\linewidth]{cifar10_303_cond_images/0/labeloutputs_cifar10_perp_2097_326.png}} 
%  & {\includegraphics[width=.09\linewidth]{cifar10_303_cond_images/0/labeloutputs_cifar10_perp_2097_1492.png}} 
%  & {\includegraphics[width=.09\linewidth]{cifar10_303_cond_images/0/labeloutputs_cifar10_perp_2097_1445.png}}
%  & {\includegraphics[width=.09\linewidth]{cifar10_303_cond_images/0/labeloutputs_cifar10_perp_2097_1255.png}}
%  \\ [-0.75mm]
%  {\includegraphics[width=.09\linewidth]{cifar10_303_cond_images/1/labeloutputs_cifar10_perp_2097_1755.png}} 
%  & {\includegraphics[width=.09\linewidth]{cifar10_303_cond_images/1/labeloutputs_cifar10_perp_2097_1853.png}} 
%  & {\includegraphics[width=.09\linewidth]{cifar10_303_cond_images/1/labeloutputs_cifar10_perp_2097_2368.png}} 
%  & {\includegraphics[width=.09\linewidth]{cifar10_303_cond_images/1/labeloutputs_cifar10_perp_2097_2580.png}} 
%  & {\includegraphics[width=.09\linewidth]{cifar10_303_cond_images/1/labeloutputs_cifar10_perp_2097_3093.png}} 
%  & {\includegraphics[width=.09\linewidth]{cifar10_303_cond_images/1/labeloutputs_cifar10_perp_2097_3094.png}} 
%  & {\includegraphics[width=.09\linewidth]{cifar10_303_cond_images/1/labeloutputs_cifar10_perp_2097_3336.png}} 
%  & {\includegraphics[width=.09\linewidth]{cifar10_303_cond_images/1/labeloutputs_cifar10_perp_2097_4215.png}} 
%  & {\includegraphics[width=.09\linewidth]{cifar10_303_cond_images/1/labeloutputs_cifar10_perp_2097_674.png}}
%  & {\includegraphics[width=.09\linewidth]{cifar10_303_cond_images/1/labeloutputs_cifar10_perp_2097_831.png}}
% \\ [-0.75mm]
% {\includegraphics[width=.09\linewidth]{cifar10_303_cond_images/2/labeloutputs_cifar10_perp_2097_1072.png}}
% & {\includegraphics[width=.09\linewidth]{cifar10_303_cond_images/2/labeloutputs_cifar10_perp_2097_2191.png}}
% & {\includegraphics[width=.09\linewidth]{cifar10_303_cond_images/2/labeloutputs_cifar10_perp_2097_2258.png}}
% & {\includegraphics[width=.09\linewidth]{cifar10_303_cond_images/2/labeloutputs_cifar10_perp_2097_2495.png}}
% & {\includegraphics[width=.09\linewidth]{cifar10_303_cond_images/2/labeloutputs_cifar10_perp_2097_3050.png}}
% & {\includegraphics[width=.09\linewidth]{cifar10_303_cond_images/2/labeloutputs_cifar10_perp_2097_3995.png}}
% & {\includegraphics[width=.09\linewidth]{cifar10_303_cond_images/2/labeloutputs_cifar10_perp_2097_4147.png}}
% & {\includegraphics[width=.09\linewidth]{cifar10_303_cond_images/2/labeloutputs_cifar10_perp_2097_6.png}}
% & {\includegraphics[width=.09\linewidth]{cifar10_303_cond_images/2/labeloutputs_cifar10_perp_2097_732.png}}
% & {\includegraphics[width=.09\linewidth]{cifar10_303_cond_images/2/labeloutputs_cifar10_perp_2097_994.png}}
% \\  [-0.75mm]
% {\includegraphics[width=.09\linewidth]{cifar10_303_cond_images/3/labeloutputs_cifar10_perp_2097_1003.png}}
% & {\includegraphics[width=.09\linewidth]{cifar10_303_cond_images/3/labeloutputs_cifar10_perp_2097_1611.png}}
% & {\includegraphics[width=.09\linewidth]{cifar10_303_cond_images/3/labeloutputs_cifar10_perp_2097_1618.png}}
% & {\includegraphics[width=.09\linewidth]{cifar10_303_cond_images/3/labeloutputs_cifar10_perp_2097_2362.png}}
% & {\includegraphics[width=.09\linewidth]{cifar10_303_cond_images/3/labeloutputs_cifar10_perp_2097_3040.png}}
% & {\includegraphics[width=.09\linewidth]{cifar10_303_cond_images/3/labeloutputs_cifar10_perp_2097_3256.png}}
% & {\includegraphics[width=.09\linewidth]{cifar10_303_cond_images/3/labeloutputs_cifar10_perp_2097_3257.png}}
% & {\includegraphics[width=.09\linewidth]{cifar10_303_cond_images/3/labeloutputs_cifar10_perp_2097_3552.png}}
% & {\includegraphics[width=.09\linewidth]{cifar10_303_cond_images/3/labeloutputs_cifar10_perp_2097_3608.png}}
% & {\includegraphics[width=.09\linewidth]{cifar10_303_cond_images/3/labeloutputs_cifar10_perp_2097_3610.png}}
% \\  [-0.75mm]
% {\includegraphics[width=.09\linewidth]{cifar10_303_cond_images/4/labeloutputs_cifar10_perp_2097_2756.png}}
% & {\includegraphics[width=.09\linewidth]{cifar10_303_cond_images/4/labeloutputs_cifar10_perp_2097_2859.png}}
% & {\includegraphics[width=.09\linewidth]{cifar10_303_cond_images/4/labeloutputs_cifar10_perp_2097_29.png}}
% & {\includegraphics[width=.09\linewidth]{cifar10_303_cond_images/4/labeloutputs_cifar10_perp_2097_2923.png}}
% & {\includegraphics[width=.09\linewidth]{cifar10_303_cond_images/4/labeloutputs_cifar10_perp_2097_3831.png}}
% & {\includegraphics[width=.09\linewidth]{cifar10_303_cond_images/4/labeloutputs_cifar10_perp_2097_3876.png}}
% & {\includegraphics[width=.09\linewidth]{cifar10_303_cond_images/4/labeloutputs_cifar10_perp_2097_396.png}}
% & {\includegraphics[width=.09\linewidth]{cifar10_303_cond_images/4/labeloutputs_cifar10_perp_2097_3963.png}}
% & {\includegraphics[width=.09\linewidth]{cifar10_303_cond_images/4/labeloutputs_cifar10_perp_2097_4086.png}}
% & {\includegraphics[width=.09\linewidth]{cifar10_303_cond_images/4/labeloutputs_cifar10_perp_2097_536.png}}
% \\ [-0.75mm]
% {\includegraphics[width=.09\linewidth]{cifar10_303_cond_images/5/labeloutputs_cifar10_perp_2097_1322.png}}
% & {\includegraphics[width=.09\linewidth]{cifar10_303_cond_images/5/labeloutputs_cifar10_perp_2097_1705.png}}
% & {\includegraphics[width=.09\linewidth]{cifar10_303_cond_images/5/labeloutputs_cifar10_perp_2097_213.png}}
% & {\includegraphics[width=.09\linewidth]{cifar10_303_cond_images/5/labeloutputs_cifar10_perp_2097_2183.png}}
% & {\includegraphics[width=.09\linewidth]{cifar10_303_cond_images/5/labeloutputs_cifar10_perp_2097_2233.png}}
% & {\includegraphics[width=.09\linewidth]{cifar10_303_cond_images/5/labeloutputs_cifar10_perp_2097_2234.png}}
% & {\includegraphics[width=.09\linewidth]{cifar10_303_cond_images/5/labeloutputs_cifar10_perp_2097_4038.png}}
% & {\includegraphics[width=.09\linewidth]{cifar10_303_cond_images/5/labeloutputs_cifar10_perp_2097_4204.png}}
% & {\includegraphics[width=.09\linewidth]{cifar10_303_cond_images/5/labeloutputs_cifar10_perp_2097_4207.png}}
% & {\includegraphics[width=.09\linewidth]{cifar10_303_cond_images/5/labeloutputs_cifar10_perp_2097_727.png}}
% \\ [-0.75mm]
% {\includegraphics[width=.09\linewidth]{cifar10_303_cond_images/6/labeloutputs_cifar10_perp_2097_127.png}}
% & {\includegraphics[width=.09\linewidth]{cifar10_303_cond_images/6/labeloutputs_cifar10_perp_2097_1548.png}}
% & {\includegraphics[width=.09\linewidth]{cifar10_303_cond_images/6/labeloutputs_cifar10_perp_2097_1696.png}}
% & {\includegraphics[width=.09\linewidth]{cifar10_303_cond_images/6/labeloutputs_cifar10_perp_2097_1907.png}}
% & {\includegraphics[width=.09\linewidth]{cifar10_303_cond_images/6/labeloutputs_cifar10_perp_2097_2147.png}}
% & {\includegraphics[width=.09\linewidth]{cifar10_303_cond_images/6/labeloutputs_cifar10_perp_2097_2729.png}}
% & {\includegraphics[width=.09\linewidth]{cifar10_303_cond_images/6/labeloutputs_cifar10_perp_2097_3787.png}}
% & {\includegraphics[width=.09\linewidth]{cifar10_303_cond_images/6/labeloutputs_cifar10_perp_2097_3792.png}}
% & {\includegraphics[width=.09\linewidth]{cifar10_303_cond_images/6/labeloutputs_cifar10_perp_2097_3895.png}}
% & {\includegraphics[width=.09\linewidth]{cifar10_303_cond_images/6/labeloutputs_cifar10_perp_2097_4111.png}}
% \\ [-0.75mm]
% {\includegraphics[width=.09\linewidth]{cifar10_303_cond_images/7/labeloutputs_cifar10_perp_2097_1113.png}}
% & {\includegraphics[width=.09\linewidth]{cifar10_303_cond_images/7/labeloutputs_cifar10_perp_2097_149.png}}
% & {\includegraphics[width=.09\linewidth]{cifar10_303_cond_images/7/labeloutputs_cifar10_perp_2097_2114.png}}
% & {\includegraphics[width=.09\linewidth]{cifar10_303_cond_images/7/labeloutputs_cifar10_perp_2097_2604.png}}
% & {\includegraphics[width=.09\linewidth]{cifar10_303_cond_images/7/labeloutputs_cifar10_perp_2097_3009.png}}
% & {\includegraphics[width=.09\linewidth]{cifar10_303_cond_images/7/labeloutputs_cifar10_perp_2097_3139.png}}
% & {\includegraphics[width=.09\linewidth]{cifar10_303_cond_images/7/labeloutputs_cifar10_perp_2097_3143.png}}
% & {\includegraphics[width=.09\linewidth]{cifar10_303_cond_images/7/labeloutputs_cifar10_perp_2097_3515.png}}
% & {\includegraphics[width=.09\linewidth]{cifar10_303_cond_images/7/labeloutputs_cifar10_perp_2097_3525.png}}
% & {\includegraphics[width=.09\linewidth]{cifar10_303_cond_images/7/labeloutputs_cifar10_perp_2097_4068.png}}
% \\ [-0.75mm]
% {\includegraphics[width=.09\linewidth]{cifar10_303_cond_images/8/labeloutputs_cifar10_perp_2097_1578.png}}
% & {\includegraphics[width=.09\linewidth]{cifar10_303_cond_images/8/labeloutputs_cifar10_perp_2097_1635.png}}
% & {\includegraphics[width=.09\linewidth]{cifar10_303_cond_images/8/labeloutputs_cifar10_perp_2097_1927.png}}
% & {\includegraphics[width=.09\linewidth]{cifar10_303_cond_images/8/labeloutputs_cifar10_perp_2097_230.png}}
% & {\includegraphics[width=.09\linewidth]{cifar10_303_cond_images/8/labeloutputs_cifar10_perp_2097_2432.png}}
% & {\includegraphics[width=.09\linewidth]{cifar10_303_cond_images/8/labeloutputs_cifar10_perp_2097_2748.png}}
% & {\includegraphics[width=.09\linewidth]{cifar10_303_cond_images/8/labeloutputs_cifar10_perp_2097_2990.png}}
% & {\includegraphics[width=.09\linewidth]{cifar10_303_cond_images/8/labeloutputs_cifar10_perp_2097_3482.png}}
% & {\includegraphics[width=.09\linewidth]{cifar10_303_cond_images/8/labeloutputs_cifar10_perp_2097_3625.png}}
% & {\includegraphics[width=.09\linewidth]{cifar10_303_cond_images/8/labeloutputs_cifar10_perp_2097_560.png}}
% \\ [-0.75mm]
% {\includegraphics[width=.09\linewidth]{cifar10_303_cond_images/9/labeloutputs_cifar10_perp_2097_107.png}}
% & {\includegraphics[width=.09\linewidth]{cifar10_303_cond_images/9/labeloutputs_cifar10_perp_2097_1348.png}}
% & {\includegraphics[width=.09\linewidth]{cifar10_303_cond_images/9/labeloutputs_cifar10_perp_2097_2377.png}}
% & {\includegraphics[width=.09\linewidth]{cifar10_303_cond_images/9/labeloutputs_cifar10_perp_2097_2380.png}}
% & {\includegraphics[width=.09\linewidth]{cifar10_303_cond_images/9/labeloutputs_cifar10_perp_2097_3601.png}}
% & {\includegraphics[width=.09\linewidth]{cifar10_303_cond_images/9/labeloutputs_cifar10_perp_2097_3641.png}}
% & {\includegraphics[width=.09\linewidth]{cifar10_303_cond_images/9/labeloutputs_cifar10_perp_2097_946.png}}
% & {\includegraphics[width=.09\linewidth]{cifar10_303_cond_images/9/labeloutputs_cifar10_perp_2097_951.png}}
% & {\includegraphics[width=.09\linewidth]{cifar10_303_cond_images/9/labeloutputs_cifar10_perp_2097_96.png}}
% & {\includegraphics[width=.09\linewidth]{cifar10_303_cond_images/9/labeloutputs_cifar10_perp_2097_978.png}}
% \\
% \end{longtable} 
% \end{center}


% % % This is a table 

% \begin{center}
% \begin{longtable}{@{\hspace{.05cm}}c@{\hspace{.05cm}}c@{\hspace{.05cm}}c@{\hspace{.05cm}}c@{\hspace{.05cm}}c@{\hspace{.05cm}}c@{\hspace{.05cm}}c@{\hspace{.05cm}}c@{\hspace{.05cm}}c@{\hspace{.05cm}}c} \\ 
% % Input & Bicubic & \regression & $\tau=1.0$ & $\tau=0.9$ & $\tau=0.8$ & Truth & \NN{} & GAN~\cite{srez} & GAN~\cite{srez} \\ 
%  \endhead 
%   {\includegraphics[width=.09\linewidth]{cifar10_299_cond_images/0/labeloutputs_cifar10_perp_2047_1069.png}}
% & {\includegraphics[width=.09\linewidth]{cifar10_299_cond_images/0/labeloutputs_cifar10_perp_2047_1213.png}}
% & {\includegraphics[width=.09\linewidth]{cifar10_299_cond_images/0/labeloutputs_cifar10_perp_2047_1250.png}}
% & {\includegraphics[width=.09\linewidth]{cifar10_299_cond_images/0/labeloutputs_cifar10_perp_2047_1739.png}}
% & {\includegraphics[width=.09\linewidth]{cifar10_299_cond_images/0/labeloutputs_cifar10_perp_2047_2025.png}}
% & {\includegraphics[width=.09\linewidth]{cifar10_299_cond_images/0/labeloutputs_cifar10_perp_2047_2077.png}}
% & {\includegraphics[width=.09\linewidth]{cifar10_299_cond_images/0/labeloutputs_cifar10_perp_2047_2160.png}}
% & {\includegraphics[width=.09\linewidth]{cifar10_299_cond_images/0/labeloutputs_cifar10_perp_2047_2222.png}}
% & {\includegraphics[width=.09\linewidth]{cifar10_299_cond_images/0/labeloutputs_cifar10_perp_2047_325.png}}
% & {\includegraphics[width=.09\linewidth]{cifar10_299_cond_images/0/labeloutputs_cifar10_perp_2047_403.png}}
%  \\ [-0.75mm]
%  {\includegraphics[width=.09\linewidth]{cifar10_299_cond_images/1/labeloutputs_cifar10_perp_2047_1331.png}}
% & {\includegraphics[width=.09\linewidth]{cifar10_299_cond_images/1/labeloutputs_cifar10_perp_2047_1367.png}}
% & {\includegraphics[width=.09\linewidth]{cifar10_299_cond_images/1/labeloutputs_cifar10_perp_2047_1397.png}}
% & {\includegraphics[width=.09\linewidth]{cifar10_299_cond_images/1/labeloutputs_cifar10_perp_2047_1754.png}}
% & {\includegraphics[width=.09\linewidth]{cifar10_299_cond_images/1/labeloutputs_cifar10_perp_2047_1849.png}}
% & {\includegraphics[width=.09\linewidth]{cifar10_299_cond_images/1/labeloutputs_cifar10_perp_2047_1850.png}}
% & {\includegraphics[width=.09\linewidth]{cifar10_299_cond_images/1/labeloutputs_cifar10_perp_2047_265.png}}
% & {\includegraphics[width=.09\linewidth]{cifar10_299_cond_images/1/labeloutputs_cifar10_perp_2047_575.png}}
% & {\includegraphics[width=.09\linewidth]{cifar10_299_cond_images/1/labeloutputs_cifar10_perp_2047_673.png}}
% & {\includegraphics[width=.09\linewidth]{cifar10_299_cond_images/1/labeloutputs_cifar10_perp_2047_770.png}}
% \\ [-0.75mm]
% {\includegraphics[width=.09\linewidth]{cifar10_299_cond_images/2/labeloutputs_cifar10_perp_2047_1244.png}}
% & {\includegraphics[width=.09\linewidth]{cifar10_299_cond_images/2/labeloutputs_cifar10_perp_2047_1391.png}}
% & {\includegraphics[width=.09\linewidth]{cifar10_299_cond_images/2/labeloutputs_cifar10_perp_2047_153.png}}
% & {\includegraphics[width=.09\linewidth]{cifar10_299_cond_images/2/labeloutputs_cifar10_perp_2047_2120.png}}
% & {\includegraphics[width=.09\linewidth]{cifar10_299_cond_images/2/labeloutputs_cifar10_perp_2047_2188.png}}
% & {\includegraphics[width=.09\linewidth]{cifar10_299_cond_images/2/labeloutputs_cifar10_perp_2047_2298.png}}
% & {\includegraphics[width=.09\linewidth]{cifar10_299_cond_images/2/labeloutputs_cifar10_perp_2047_2303.png}}
% & {\includegraphics[width=.09\linewidth]{cifar10_299_cond_images/2/labeloutputs_cifar10_perp_2047_2586.png}}
% & {\includegraphics[width=.09\linewidth]{cifar10_299_cond_images/2/labeloutputs_cifar10_perp_2047_482.png}}
% & {\includegraphics[width=.09\linewidth]{cifar10_299_cond_images/2/labeloutputs_cifar10_perp_2047_993.png}}
% \\  [-0.75mm]
% {\includegraphics[width=.09\linewidth]{cifar10_299_cond_images/3/labeloutputs_cifar10_perp_2047_1647.png}}
% & {\includegraphics[width=.09\linewidth]{cifar10_299_cond_images/3/labeloutputs_cifar10_perp_2047_1672.png}}
% & {\includegraphics[width=.09\linewidth]{cifar10_299_cond_images/3/labeloutputs_cifar10_perp_2047_1673.png}}
% & {\includegraphics[width=.09\linewidth]{cifar10_299_cond_images/3/labeloutputs_cifar10_perp_2047_1987.png}}
% & {\includegraphics[width=.09\linewidth]{cifar10_299_cond_images/3/labeloutputs_cifar10_perp_2047_2548.png}}
% & {\includegraphics[width=.09\linewidth]{cifar10_299_cond_images/3/labeloutputs_cifar10_perp_2047_2554.png}}
% & {\includegraphics[width=.09\linewidth]{cifar10_299_cond_images/3/labeloutputs_cifar10_perp_2047_422.png}}
% & {\includegraphics[width=.09\linewidth]{cifar10_299_cond_images/3/labeloutputs_cifar10_perp_2047_445.png}}
% & {\includegraphics[width=.09\linewidth]{cifar10_299_cond_images/3/labeloutputs_cifar10_perp_2047_68.png}}
% & {\includegraphics[width=.09\linewidth]{cifar10_299_cond_images/3/labeloutputs_cifar10_perp_2047_877.png}}
% \\  [-0.75mm]
% {\includegraphics[width=.09\linewidth]{cifar10_299_cond_images/4/labeloutputs_cifar10_perp_2047_1058.png}}
% & {\includegraphics[width=.09\linewidth]{cifar10_299_cond_images/4/labeloutputs_cifar10_perp_2047_13.png}}
% & {\includegraphics[width=.09\linewidth]{cifar10_299_cond_images/4/labeloutputs_cifar10_perp_2047_1600.png}}
% & {\includegraphics[width=.09\linewidth]{cifar10_299_cond_images/4/labeloutputs_cifar10_perp_2047_22.png}}
% & {\includegraphics[width=.09\linewidth]{cifar10_299_cond_images/4/labeloutputs_cifar10_perp_2047_2623.png}}
% & {\includegraphics[width=.09\linewidth]{cifar10_299_cond_images/4/labeloutputs_cifar10_perp_2047_28.png}}
% & {\includegraphics[width=.09\linewidth]{cifar10_299_cond_images/4/labeloutputs_cifar10_perp_2047_411.png}}
% & {\includegraphics[width=.09\linewidth]{cifar10_299_cond_images/4/labeloutputs_cifar10_perp_2047_433.png}}
% & {\includegraphics[width=.09\linewidth]{cifar10_299_cond_images/4/labeloutputs_cifar10_perp_2047_448.png}}
% & {\includegraphics[width=.09\linewidth]{cifar10_299_cond_images/4/labeloutputs_cifar10_perp_2047_862.png}}
% \\ [-0.75mm]
% {\includegraphics[width=.09\linewidth]{cifar10_299_cond_images/5/labeloutputs_cifar10_perp_2047_1357.png}}
% & {\includegraphics[width=.09\linewidth]{cifar10_299_cond_images/5/labeloutputs_cifar10_perp_2047_139.png}}
% & {\includegraphics[width=.09\linewidth]{cifar10_299_cond_images/5/labeloutputs_cifar10_perp_2047_142.png}}
% & {\includegraphics[width=.09\linewidth]{cifar10_299_cond_images/5/labeloutputs_cifar10_perp_2047_204.png}}
% & {\includegraphics[width=.09\linewidth]{cifar10_299_cond_images/5/labeloutputs_cifar10_perp_2047_2239.png}}
% & {\includegraphics[width=.09\linewidth]{cifar10_299_cond_images/5/labeloutputs_cifar10_perp_2047_2450.png}}
% & {\includegraphics[width=.09\linewidth]{cifar10_299_cond_images/5/labeloutputs_cifar10_perp_2047_2453.png}}
% & {\includegraphics[width=.09\linewidth]{cifar10_299_cond_images/5/labeloutputs_cifar10_perp_2047_2472.png}}
% & {\includegraphics[width=.09\linewidth]{cifar10_299_cond_images/5/labeloutputs_cifar10_perp_2047_2572.png}}
% & {\includegraphics[width=.09\linewidth]{cifar10_299_cond_images/5/labeloutputs_cifar10_perp_2047_506.png}}
% \\ [-0.75mm]
% {\includegraphics[width=.09\linewidth]{cifar10_299_cond_images/6/labeloutputs_cifar10_perp_2047_126.png}}
% & {\includegraphics[width=.09\linewidth]{cifar10_299_cond_images/6/labeloutputs_cifar10_perp_2047_1431.png}}
% & {\includegraphics[width=.09\linewidth]{cifar10_299_cond_images/6/labeloutputs_cifar10_perp_2047_2145.png}}
% & {\includegraphics[width=.09\linewidth]{cifar10_299_cond_images/6/labeloutputs_cifar10_perp_2047_2194.png}}
% & {\includegraphics[width=.09\linewidth]{cifar10_299_cond_images/6/labeloutputs_cifar10_perp_2047_2196.png}}
% & {\includegraphics[width=.09\linewidth]{cifar10_299_cond_images/6/labeloutputs_cifar10_perp_2047_2228.png}}
% & {\includegraphics[width=.09\linewidth]{cifar10_299_cond_images/6/labeloutputs_cifar10_perp_2047_2540.png}}
% & {\includegraphics[width=.09\linewidth]{cifar10_299_cond_images/6/labeloutputs_cifar10_perp_2047_2543.png}}
% & {\includegraphics[width=.09\linewidth]{cifar10_299_cond_images/6/labeloutputs_cifar10_perp_2047_719.png}}
% & {\includegraphics[width=.09\linewidth]{cifar10_299_cond_images/6/labeloutputs_cifar10_perp_2047_884.png}}
% \\ [-0.75mm]
% {\includegraphics[width=.09\linewidth]{cifar10_299_cond_images/7/labeloutputs_cifar10_perp_2047_1036.png}}
% & {\includegraphics[width=.09\linewidth]{cifar10_299_cond_images/7/labeloutputs_cifar10_perp_2047_1038.png}}
% & {\includegraphics[width=.09\linewidth]{cifar10_299_cond_images/7/labeloutputs_cifar10_perp_2047_1226.png}}
% & {\includegraphics[width=.09\linewidth]{cifar10_299_cond_images/7/labeloutputs_cifar10_perp_2047_1416.png}}
% & {\includegraphics[width=.09\linewidth]{cifar10_299_cond_images/7/labeloutputs_cifar10_perp_2047_1499.png}}
% & {\includegraphics[width=.09\linewidth]{cifar10_299_cond_images/7/labeloutputs_cifar10_perp_2047_1626.png}}
% & {\includegraphics[width=.09\linewidth]{cifar10_299_cond_images/7/labeloutputs_cifar10_perp_2047_1930.png}}
% & {\includegraphics[width=.09\linewidth]{cifar10_299_cond_images/7/labeloutputs_cifar10_perp_2047_263.png}}
% & {\includegraphics[width=.09\linewidth]{cifar10_299_cond_images/7/labeloutputs_cifar10_perp_2047_32.png}}
% & {\includegraphics[width=.09\linewidth]{cifar10_299_cond_images/7/labeloutputs_cifar10_perp_2047_906.png}}

% \\ [-0.75mm]
% {\includegraphics[width=.09\linewidth]{cifar10_299_cond_images/8/labeloutputs_cifar10_perp_2047_135.png}}
% & {\includegraphics[width=.09\linewidth]{cifar10_299_cond_images/8/labeloutputs_cifar10_perp_2047_1583.png}}
% & {\includegraphics[width=.09\linewidth]{cifar10_299_cond_images/8/labeloutputs_cifar10_perp_2047_1632.png}}
% & {\includegraphics[width=.09\linewidth]{cifar10_299_cond_images/8/labeloutputs_cifar10_perp_2047_1635.png}}
% & {\includegraphics[width=.09\linewidth]{cifar10_299_cond_images/8/labeloutputs_cifar10_perp_2047_2111.png}}
% & {\includegraphics[width=.09\linewidth]{cifar10_299_cond_images/8/labeloutputs_cifar10_perp_2047_2266.png}}
% & {\includegraphics[width=.09\linewidth]{cifar10_299_cond_images/8/labeloutputs_cifar10_perp_2047_2271.png}}
% & {\includegraphics[width=.09\linewidth]{cifar10_299_cond_images/8/labeloutputs_cifar10_perp_2047_2516.png}}
% & {\includegraphics[width=.09\linewidth]{cifar10_299_cond_images/8/labeloutputs_cifar10_perp_2047_2518.png}}
% & {\includegraphics[width=.09\linewidth]{cifar10_299_cond_images/8/labeloutputs_cifar10_perp_2047_764.png}}
% \\ [-0.75mm]
% {\includegraphics[width=.09\linewidth]{cifar10_299_cond_images/9/labeloutputs_cifar10_perp_2047_1206.png}}
% & {\includegraphics[width=.09\linewidth]{cifar10_299_cond_images/9/labeloutputs_cifar10_perp_2047_1955.png}}
% & {\includegraphics[width=.09\linewidth]{cifar10_299_cond_images/9/labeloutputs_cifar10_perp_2047_2103.png}}
% & {\includegraphics[width=.09\linewidth]{cifar10_299_cond_images/9/labeloutputs_cifar10_perp_2047_2142.png}}
% & {\includegraphics[width=.09\linewidth]{cifar10_299_cond_images/9/labeloutputs_cifar10_perp_2047_2311.png}}
% & {\includegraphics[width=.09\linewidth]{cifar10_299_cond_images/9/labeloutputs_cifar10_perp_2047_2526.png}}
% & {\includegraphics[width=.09\linewidth]{cifar10_299_cond_images/9/labeloutputs_cifar10_perp_2047_2564.png}}
% & {\includegraphics[width=.09\linewidth]{cifar10_299_cond_images/9/labeloutputs_cifar10_perp_2047_845.png}}
% & {\includegraphics[width=.09\linewidth]{cifar10_299_cond_images/9/labeloutputs_cifar10_perp_2047_846.png}}
% & {\includegraphics[width=.09\linewidth]{cifar10_299_cond_images/9/labeloutputs_cifar10_perp_2047_898.png}}
% \\
% \end{longtable} 
% \end{center}


% This is a table 

\begin{center}
\begin{longtable}{@{\hspace{.05cm}}c@{\hspace{.05cm}}c@{\hspace{.05cm}}c@{\hspace{.05cm}}c@{\hspace{.05cm}}c@{\hspace{.05cm}}c@{\hspace{.05cm}}c@{\hspace{.05cm}}c@{\hspace{.05cm}}c@{\hspace{.05cm}}c} \\ 
% Input & Bicubic & \regression & $\tau=1.0$ & $\tau=0.9$ & $\tau=0.8$ & Truth & \NN{} & GAN~\cite{srez} & GAN~\cite{srez} \\ 
 \endhead 
&  {\includegraphics[width=.09\linewidth]{cifar10_299_cond_images/0/labeloutputs_cifar10_perp_2047_1069.png}}
& {\includegraphics[width=.09\linewidth]{cifar10_299_cond_images/0/labeloutputs_cifar10_perp_2047_1213.png}}
& {\includegraphics[width=.09\linewidth]{cifar10_299_cond_images/0/labeloutputs_cifar10_perp_2047_2077.png}}
& {\includegraphics[width=.09\linewidth]{cifar10_299_cond_images/0/labeloutputs_cifar10_perp_2047_2222.png}}
& {\includegraphics[width=.09\linewidth]{cifar10_299_cond_images/0/labeloutputs_cifar10_perp_2047_325.png}}
 \\ [-0.75mm]
 {\includegraphics[width=.09\linewidth]{cifar10_299_cond_images/1/labeloutputs_cifar10_perp_2047_1331.png}}
& {\includegraphics[width=.09\linewidth]{cifar10_299_cond_images/1/labeloutputs_cifar10_perp_2047_1754.png}}
& {\includegraphics[width=.09\linewidth]{cifar10_299_cond_images/1/labeloutputs_cifar10_perp_2047_1849.png}}
& {\includegraphics[width=.09\linewidth]{cifar10_299_cond_images/1/labeloutputs_cifar10_perp_2047_265.png}}
& {\includegraphics[width=.09\linewidth]{cifar10_299_cond_images/1/labeloutputs_cifar10_perp_2047_575.png}}
\\ [-0.75mm]
& {\includegraphics[width=.09\linewidth]{cifar10_299_cond_images/2/labeloutputs_cifar10_perp_2047_1391.png}}
& {\includegraphics[width=.09\linewidth]{cifar10_299_cond_images/2/labeloutputs_cifar10_perp_2047_153.png}}
& {\includegraphics[width=.09\linewidth]{cifar10_299_cond_images/2/labeloutputs_cifar10_perp_2047_2188.png}}
& {\includegraphics[width=.09\linewidth]{cifar10_299_cond_images/2/labeloutputs_cifar10_perp_2047_2298.png}}
& {\includegraphics[width=.09\linewidth]{cifar10_299_cond_images/2/labeloutputs_cifar10_perp_2047_2303.png}}
\\  [-0.75mm]
{\includegraphics[width=.09\linewidth]{cifar10_299_cond_images/3/labeloutputs_cifar10_perp_2047_1647.png}}
& {\includegraphics[width=.09\linewidth]{cifar10_299_cond_images/3/labeloutputs_cifar10_perp_2047_1672.png}}
& {\includegraphics[width=.09\linewidth]{cifar10_299_cond_images/3/labeloutputs_cifar10_perp_2047_1673.png}}
& {\includegraphics[width=.09\linewidth]{cifar10_299_cond_images/3/labeloutputs_cifar10_perp_2047_1987.png}}
& {\includegraphics[width=.09\linewidth]{cifar10_299_cond_images/3/labeloutputs_cifar10_perp_2047_68.png}}
\\  [-0.75mm]
& {\includegraphics[width=.09\linewidth]{cifar10_299_cond_images/4/labeloutputs_cifar10_perp_2047_22.png}}
& {\includegraphics[width=.09\linewidth]{cifar10_299_cond_images/4/labeloutputs_cifar10_perp_2047_28.png}}
& {\includegraphics[width=.09\linewidth]{cifar10_299_cond_images/4/labeloutputs_cifar10_perp_2047_411.png}}
& {\includegraphics[width=.09\linewidth]{cifar10_299_cond_images/4/labeloutputs_cifar10_perp_2047_433.png}}
& {\includegraphics[width=.09\linewidth]{cifar10_299_cond_images/4/labeloutputs_cifar10_perp_2047_862.png}}
\\ [-0.75mm]
& {\includegraphics[width=.09\linewidth]{cifar10_299_cond_images/5/labeloutputs_cifar10_perp_2047_139.png}}
& {\includegraphics[width=.09\linewidth]{cifar10_299_cond_images/5/labeloutputs_cifar10_perp_2047_142.png}}
& {\includegraphics[width=.09\linewidth]{cifar10_299_cond_images/5/labeloutputs_cifar10_perp_2047_2453.png}}
& {\includegraphics[width=.09\linewidth]{cifar10_299_cond_images/5/labeloutputs_cifar10_perp_2047_2472.png}}
& {\includegraphics[width=.09\linewidth]{cifar10_299_cond_images/5/labeloutputs_cifar10_perp_2047_2572.png}}
\\ [-0.75mm]
{\includegraphics[width=.09\linewidth]{cifar10_299_cond_images/6/labeloutputs_cifar10_perp_2047_126.png}}
& {\includegraphics[width=.09\linewidth]{cifar10_299_cond_images/6/labeloutputs_cifar10_perp_2047_1431.png}}
& {\includegraphics[width=.09\linewidth]{cifar10_299_cond_images/6/labeloutputs_cifar10_perp_2047_2228.png}}
& {\includegraphics[width=.09\linewidth]{cifar10_299_cond_images/6/labeloutputs_cifar10_perp_2047_2543.png}}
& {\includegraphics[width=.09\linewidth]{cifar10_299_cond_images/6/labeloutputs_cifar10_perp_2047_719.png}}
\\ [-0.75mm]
& {\includegraphics[width=.09\linewidth]{cifar10_299_cond_images/7/labeloutputs_cifar10_perp_2047_1038.png}}
& {\includegraphics[width=.09\linewidth]{cifar10_299_cond_images/7/labeloutputs_cifar10_perp_2047_1226.png}}
& {\includegraphics[width=.09\linewidth]{cifar10_299_cond_images/7/labeloutputs_cifar10_perp_2047_1626.png}}
& {\includegraphics[width=.09\linewidth]{cifar10_299_cond_images/7/labeloutputs_cifar10_perp_2047_1930.png}}
& {\includegraphics[width=.09\linewidth]{cifar10_299_cond_images/7/labeloutputs_cifar10_perp_2047_263.png}}
\\ [-0.75mm]
& {\includegraphics[width=.09\linewidth]{cifar10_299_cond_images/8/labeloutputs_cifar10_perp_2047_1632.png}}
& {\includegraphics[width=.09\linewidth]{cifar10_299_cond_images/8/labeloutputs_cifar10_perp_2047_2266.png}}
& {\includegraphics[width=.09\linewidth]{cifar10_299_cond_images/8/labeloutputs_cifar10_perp_2047_2271.png}}
& {\includegraphics[width=.09\linewidth]{cifar10_299_cond_images/8/labeloutputs_cifar10_perp_2047_2518.png}}
& {\includegraphics[width=.09\linewidth]{cifar10_299_cond_images/8/labeloutputs_cifar10_perp_2047_764.png}}
\\ [-0.75mm]
{\includegraphics[width=.09\linewidth]{cifar10_299_cond_images/9/labeloutputs_cifar10_perp_2047_1206.png}}
& {\includegraphics[width=.09\linewidth]{cifar10_299_cond_images/9/labeloutputs_cifar10_perp_2047_1955.png}}
& {\includegraphics[width=.09\linewidth]{cifar10_299_cond_images/9/labeloutputs_cifar10_perp_2047_2564.png}}
& {\includegraphics[width=.09\linewidth]{cifar10_299_cond_images/9/labeloutputs_cifar10_perp_2047_845.png}}
& {\includegraphics[width=.09\linewidth]{cifar10_299_cond_images/9/labeloutputs_cifar10_perp_2047_846.png}}
\\
\end{longtable} 
\end{center}



% \begin{table*}[h!]
\label{tab:celeba_images}
\centering
\begin{tabular}{@{\hspace{.05cm}}c@{\hspace{.05cm}}c@{\hspace{.05cm}}c@{\hspace{.05cm}}c@{\hspace{.05cm}}c@{\hspace{.05cm}}c@{\hspace{.05cm}}c@{\hspace{.05cm}}c} \\ 
% Input & Bicubic & \regression & $\tau=1.0$ & $\tau=0.9$ & $\tau=0.8$ & Truth & \NN{} & GAN~\cite{srez} & GAN~\cite{srez} \\ 
% \endhead 
  Input &  \multicolumn{3}{c} {1D Local Attention} &  \multicolumn{3}{c} {2D Local Attention} & Original \\
  & $\tau=0.8$ & $\tau=0.9$ & $\tau=1.0$ & $\tau=0.8$ & $\tau=0.9$ & $\tau=1.0$ & \\
{\includegraphics[width=.1\linewidth]{celeba_images/inputs/101_128x.png}}
& {\includegraphics[width=.1\linewidth]{celeba_images/base_1d_0.8/65_128x.png}}
& {\includegraphics[width=.1\linewidth]{celeba_images/base_1d_0.9/65_128x.png}}
& {\includegraphics[width=.1\linewidth]{celeba_images/base_1d_1.0/65_128x.png}}
& {\includegraphics[width=.1\linewidth]{celeba_images/base_2d_0.8/65_128x.png}}
& {\includegraphics[width=.1\linewidth]{celeba_images/base_2d_0.9/65_128x.png}}
& {\includegraphics[width=.1\linewidth]{celeba_images/base_2d_1.0/65_128x.png}}
& {\includegraphics[width=.1\linewidth]{celeba_images/targets/101_128x.png}}
 %\\ [-0.75mm]
 %{\includegraphics[width=.1\linewidth]{celeba_images/inputs/33_128x.png}}
%& {\includegraphics[width=.1\linewidth]{celeba_images/base_1d_0.8/33_128x.png}}
%& {\includegraphics[width=.1\linewidth]{celeba_images/base_1d_0.9/33_128x.png}}
%& {\includegraphics[width=.1\linewidth]{celeba_images/base_1d_1.0/33_128x.png}}
%& {\includegraphics[width=.1\linewidth]{celeba_images/base_2d_0.8/33_128x.png}}
%& {\includegraphics[width=.1\linewidth]{celeba_images/base_2d_0.9/33_128x.png}}
%& {\includegraphics[width=.1\linewidth]{celeba_images/base_2d_1.0/33_128x.png}}
%& {\includegraphics[width=.1\linewidth]{celeba_images/targets/33_128x.png}}
 \\ [-0.75mm]
 {\includegraphics[width=.1\linewidth]{celeba_images/inputs/155_128x.png}}
& {\includegraphics[width=.1\linewidth]{celeba_images/base_1d_0.8/101_128x.png}}
& {\includegraphics[width=.1\linewidth]{celeba_images/base_1d_0.9/101_128x.png}}
& {\includegraphics[width=.1\linewidth]{celeba_images/base_1d_1.0/101_128x.png}}
& {\includegraphics[width=.1\linewidth]{celeba_images/base_2d_0.8/101_128x.png}}
& {\includegraphics[width=.1\linewidth]{celeba_images/base_2d_0.9/101_128x.png}}
& {\includegraphics[width=.1\linewidth]{celeba_images/base_2d_1.0/101_128x.png}}
& {\includegraphics[width=.1\linewidth]{celeba_images/targets/155_128x.png}}
 \\ [-0.75mm]
 {\includegraphics[width=.1\linewidth]{celeba_images/inputs/119_128x.png}}
& {\includegraphics[width=.1\linewidth]{celeba_images/base_1d_0.8/83_128x.png}}
& {\includegraphics[width=.1\linewidth]{celeba_images/base_1d_0.9/83_128x.png}}
& {\includegraphics[width=.1\linewidth]{celeba_images/base_1d_1.0/83_128x.png}}
& {\includegraphics[width=.1\linewidth]{celeba_images/base_2d_0.8/83_128x.png}}
& {\includegraphics[width=.1\linewidth]{celeba_images/base_2d_0.9/83_128x.png}}
& {\includegraphics[width=.1\linewidth]{celeba_images/base_2d_1.0/83_128x.png}}
& {\includegraphics[width=.1\linewidth]{celeba_images/targets/119_128x.png}}
 \\ [-0.75mm]
 {\includegraphics[width=.1\linewidth]{celeba_images/inputs/24_128x.png}}
& {\includegraphics[width=.1\linewidth]{celeba_images/base_1d_0.8/24_128x.png}}
& {\includegraphics[width=.1\linewidth]{celeba_images/base_1d_0.9/24_128x.png}}
& {\includegraphics[width=.1\linewidth]{celeba_images/base_1d_1.0/24_128x.png}}
& {\includegraphics[width=.1\linewidth]{celeba_images/base_2d_0.8/24_128x.png}}
& {\includegraphics[width=.1\linewidth]{celeba_images/base_2d_0.9/24_128x.png}}
& {\includegraphics[width=.1\linewidth]{celeba_images/base_2d_1.0/24_128x.png}}
& {\includegraphics[width=.1\linewidth]{celeba_images/targets/24_128x.png}}
 \\ [-0.75mm]
 {\includegraphics[width=.1\linewidth]{celeba_images/inputs/21_128x.png}}
& {\includegraphics[width=.1\linewidth]{celeba_images/base_1d_0.8/21_128x.png}}
& {\includegraphics[width=.1\linewidth]{celeba_images/base_1d_0.9/21_128x.png}}
& {\includegraphics[width=.1\linewidth]{celeba_images/base_1d_1.0/21_128x.png}}
& {\includegraphics[width=.1\linewidth]{celeba_images/base_2d_0.8/21_128x.png}}
& {\includegraphics[width=.1\linewidth]{celeba_images/base_2d_0.9/21_128x.png}}
& {\includegraphics[width=.1\linewidth]{celeba_images/base_2d_1.0/21_128x.png}}
& {\includegraphics[width=.1\linewidth]{celeba_images/targets/21_128x.png}}
% \\ [-0.75mm]
% {\includegraphics[width=.1\linewidth]{celeba_images/inputs/17_128x.png}}
%& {\includegraphics[width=.1\linewidth]{celeba_images/base_1d_0.8/17_128x.png}}
%& {\includegraphics[width=.1\linewidth]{celeba_images/base_1d_0.9/17_128x.png}}
%& {\includegraphics[width=.1\linewidth]{celeba_images/base_1d_1.0/17_128x.png}}
%& {\includegraphics[width=.1\linewidth]{celeba_images/base_2d_0.8/17_128x.png}}
%& {\includegraphics[width=.1\linewidth]{celeba_images/base_2d_0.9/17_128x.png}}
%& {\includegraphics[width=.1\linewidth]{celeba_images/base_2d_1.0/17_128x.png}}
%& {\includegraphics[width=.1\linewidth]{celeba_images/targets/17_128x.png}}
\end{tabular} 
\caption{Images from our 1D and 2D local attention super-resolution models trained on CelebA, sampled with different temperatures. 2D local attention with $\tau=0.9$ scored highest in our human evaluation study.}
\end{table*}
% \input{cifar_superres_completion}

\section{Conclusion}

In this work we demonstrate that models based on self-attention can operate effectively on modalities other than text, and through local self-attention scale to significantly larger structures than sentences. With fewer layers, its larger receptive fields allow the Image Transformer to significantly improve over the state of the art in unconditional, probabilistic image modeling of comparatively complex images from ImageNet as well as super-resolution.

We further hope to have provided additional evidence that even in the light of generative adversarial networks, likelihood-based models of images is very much a promising area for further research - as is using network architectures such as the Image Transformer in GANs.

In future work we would like to explore a broader variety of conditioning information including free-form text, as previously proposed \citep{Mansimov15}, and tasks combining modalities such as language-driven editing of images.

Fundamentally, we aim to move beyond still images to video \citep{Kalchbrenner16} and towards applications in model-based reinforcement learning.

%who has done superresolution with seq2seq?


%- Density modeling for images with and without conditioning

%RNNs and CNNs have been shown to work
%- Slow: TFLOP estimates for PixelCNN/PixelRNN
%- Complicated

%Comparison with GANs?
%- with GANs unclear (still?) how to do conditioning

\bibliography{deeplearn}
\bibliographystyle{icml2018}

% \appendix
% \newpage
% \section{Positive Definiteness of~$K$}\label{sec:appendixA}
To show that the kernel~$K$ defined in~(\ref{eq:kernel}) is positive definite
(p.d.), we simply use elementary rules from the kernel literature described in
Sections 2.3.2 and 3.4.1 of~\cite{shawe2004}.  A linear combination of p.d. kernels with non-negative weights is also p.d. (see Proposition 3.22
of\cite{shawe2004}), and thus it is sufficient to show that for all $\z,\z'$
in~$\Omega$, the following kernel on $\Omega \to \HH$ is p.d.:
\begin{displaymath}
   (\varphi,\varphi') \mapsto \big\|\varphi(\z)\big\|_\HH  \normH{\varphi'(\z')} e^{-\frac{1}{2\sigma^2} \normH{\tildephi(\z)-\tildephi'(\z')}^2}.
\end{displaymath}
Specifically, it is also sufficient to
show that the following kernel on $\HH$ is p.d.:
\begin{displaymath}
   (\phi,\phi') \mapsto \big\|{\phi}\big\|_\HH  \normH{\phi'} e^{-\frac{1}{2\sigma^2} \normH{\frac{\phi}{\|\phi\|_\HH}-\frac{\phi'}{\|\phi'\|_\HH}}^2}.
\end{displaymath}
with the convention $\phi/\|\phi\|_\HH=0$ if~$\phi=0$.
This is a pointwise product of two kernels and is p.d. when each of the two
kernels is p.d. The first one is obviously p.d.: $(\phi,\phi') \mapsto
\|{\phi}\|_\HH  \normH{\phi'}$. The second one is a composition of the Gaussian
kernel---which is p.d.---, with feature maps $\phi/\|\phi\|_\HH$ of a
normalized linear kernel in~$\HH$.  This composition is p.d. according to
Proposition 3.22, item (v) of~\cite{shawe2004} since the normalization does
not remove the positive-definiteness property.

\section{List of Architectures Reported in the Experiments}\label{appendix:arch}
We present in details the architectures used in the paper in Table~\ref{table:arch}.
\begin{table}[hbtp]
   \centering
   \begin{tabular}{|*{9}{c|}}
      \hline
      Arch. & $N$ & $m_1$  & $p_1$  &  $\gamma_1$ & $m_2$ &  $p_2$ & $S$  &  $\sharp$ param\\
      \hline
      \hline
      \multicolumn{9}{|c|}{MNIST} \\
      \hline
      CKN-GM1 & 2 &  $1 \times 1$  &  12  & 2 &  $3 \times 3$ &  50 &  $4 \times 4$ & $5\,400$\\
      \hline
      CKN-GM2 & 2 &  $1 \times 1$  &  12  & 2 &  $3 \times 3$ &  400 &  $3 \times 3$& $43\,200$ \\
      \hline
      CKN-PM1 & 1 &  $5 \times 5$  &  200  & 2 &  - &  - &  $4 \times 4$  & $5\,000$ \\
      \hline
      CKN-PM2 & 2 &  $5 \times 5$  &  50  & 2 &  $2 \times 2$ &  200 &  $6 \times 6$ & $41\,250$ \\
      \hline
      \hline
      \multicolumn{9}{|c|}{CIFAR-10} \\
      \hline
      CKN-GM & 2 &  $1 \times 1$  &  12  & 2 &  $2 \times 2$ & 800 &  $4 \times 4$ & $38\,400$\\
      \hline
      CKN-PM & 2 &  $2 \times 2$  &  100  & 2 &  $2 \times 2$ &  800 &  $4 \times 4$ & $321\,200$\\
      \hline
      \hline
      \multicolumn{9}{|c|}{STL-10} \\
      \hline
      CKN-GM & 2 &  $1 \times 1$  &  12  & 2 &  $3 \times 3$ & 800 &  $4 \times 4$ & $86\,400$\\
      \hline
      CKN-PM & 2 &  $3 \times 3$  &  50  & 2 &  $3 \times 3$ &  800 &  $3 \times 3$ & $361\,350$\\
      \hline

   \end{tabular}
   \caption{List of architectures reported in the paper. $N$ is the number of layers; $p_1$ and~$p_2$ represent the number of filters are each layer; $m_1$ and~$m_2$ represent the size of the patches~$\NN_1$ and~$\NN_2$ that are of size~$m_1 \times m_1$ and~$m_2 \times m_2$ on their respective feature maps~$\zeta_1$ and~$\zeta_2$; $\gamma_1$ is the subsampling factor between layer 1 and layer 2; $S$ is the size of the output feature map, and the last column indicates the number of parameters that the network has to learn.}
   \label{table:arch}
\end{table}


\end{document}

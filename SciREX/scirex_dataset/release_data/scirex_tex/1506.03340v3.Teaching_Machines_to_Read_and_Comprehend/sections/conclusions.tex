\section{Conclusion}
\label{conclusions}
The supervised paradigm for training machine reading and comprehension models
provides a promising avenue for making progress on the path to building full
natural language understanding systems. We have demonstrated a methodology for
obtaining a large number of document-query-answer triples and shown that
recurrent and attention based neural networks provide an effective modelling
framework for this task.
Our analysis indicates that the Attentive and Impatient Readers are able to
propagate and integrate semantic information over long distances. In particular
we believe that the incorporation of an attention mechanism is the key
contributor to these results.


The attention mechanism
that we have employed is just one instantiation of a very general idea which
can be further exploited. However, the incorporation of world knowledge and
multi-document queries will also require the development of attention and
embedding mechanisms whose complexity to query does not scale linearly with the
data set size.
There are still many queries requiring complex inference and long range reference resolution that our models are not yet able to answer. As such our data provides a scalable challenge that should support NLP research into the future. Further, significantly bigger training data sets can be acquired using the techniques we have described, undoubtedly allowing us to train more expressive and accurate models.

\documentclass[10pt,twocolumn,letterpaper]{article}
\usepackage[T1]{fontenc}
\usepackage{iccv}
\usepackage{times}

\usepackage[font={footnotesize}]{caption}
\usepackage[linesnumbered, ruled]{algorithm2e}
\usepackage[percent]{overpic}
\usepackage{amsmath}
\usepackage{amssymb}
\usepackage{booktabs}
\usepackage{color}
\usepackage{comment}
\usepackage{contour}
\usepackage{dsfont}
\usepackage{floatrow}
\usepackage{graphicx}
\usepackage{layouts}
\usepackage{listings}
\usepackage{multirow}
\usepackage{subcaption}
\usepackage{tabularx}
\usepackage{tcolorbox}
\usepackage{transparent}
\usepackage{url}
\usepackage{verbatim}
\usepackage{xspace}

\usepackage[pagebackref=true,breaklinks=true,letterpaper=true,colorlinks,bookmarks=false]{hyperref}
%\usepackage[capitalise]{cleveref}
\usepackage{cleveref}

\iccvfinalcopy
\def\iccvPaperID{3193}
\ificcvfinal\pagestyle{empty}\fi

\newfloatcommand{capbtabbox}{table}[][\FBwidth]

\newcommand{\mytodo}[1]{{\noindent\textcolor{red}{TODO: #1}}}
\newcommand{\todo}[1]{\mytodo{#1}}
\newcommand{\bbf}{\mathbf{f}}
\newcommand{\bx}{\mathbf{x}}
\newcommand{\by}{\mathbf{y}}
\newcommand{\bz}{\mathbf{z}}
\newcommand{\bw}{\mathbf{w}}
\newcommand{\be}{\mathbf{e}}
\newcommand{\bv}{\mathbf{v}}
\newcommand{\methodname}{Invariant Information Clustering\xspace}
\newcommand{\methodnameshort}{IIC\xspace}
\newcommand{\bdash}{\mathbf{'}}
\newcommand{\bc}{\mathbf{c}}
\newcommand{\defeq}{\overset{\Delta}{=}}
\newcommand{\cmt}[1]{\ignorespaces}

\newcommand{\matP}{\mathbf{P}}  %\boldsymbol{P}

% Default fixed font does not support bold face
\DeclareFixedFont{\ttb}{T1}{txtt}{bx}{n}{6.5} % for bold
\DeclareFixedFont{\ttm}{T1}{txtt}{m}{n}{6.5}  % for normal

% Suppress annoying badness warnings (disable for final version).
\hfuzz=10000pt
\vfuzz=10000pt
\hbadness=2000
\vbadness=\maxdimen

% No space paragraph.
\makeatletter
\renewcommand{\paragraph}{%
  \@startsection{paragraph}{4}%
  %{\z@}{3.25ex \@plus 1ex \@minus .2ex}{-1em}%
  {\z@}{0.5em}{-1em}%
  {\normalfont\normalsize\bfseries}%
}
\makeatother

\DeclareMathOperator*{\argmax}{argmax}
\DeclareMathOperator*{\argmin}{argmin}

% General parameters, for ALL pages:
\renewcommand{\topfraction}{0.9}    % max fraction of floats at top
\renewcommand{\bottomfraction}{0.8} % max fraction of floats at bottom

% Parameters for TEXT pages (not float pages):
\setcounter{topnumber}{2}
\setcounter{bottomnumber}{2}
\setcounter{totalnumber}{4} % 2 may work better
\setcounter{dbltopnumber}{2} % for 2-column pages
\renewcommand{\dbltopfraction}{0.9} % fit big float above 2-col. text
\renewcommand{\textfraction}{0.07} % allow minimal text w. figs

% Parameters for FLOAT pages (not text pages):
\renewcommand{\floatpagefraction}{0.7} % require fuller float pages

% N.B.: floatpagefraction MUST be less than topfraction !!
\renewcommand{\dblfloatpagefraction}{0.7} % require fuller float pages

% Separation between text and figure etc
\setlength{\textfloatsep}{5.0pt plus 2.0pt minus 4.0pt}
\setlength{\floatsep}{5.0pt plus 2.0pt minus 2.0pt}
\setlength{\intextsep}{5.0pt plus 2.0pt minus 2.0pt}
\setlength{\dbltextfloatsep}{5.0pt plus 2.0pt minus 2.0pt}
\setlength{\dblfloatsep}{5.0pt plus 2.0pt minus 2.0pt}

\floatsetup[table]{capposition=bottom,captionskip=0.2em}
\floatsetup[figure]{capposition=bottom,captionskip=0.2em}

\title{\methodname for \\ Unsupervised Image Classification and Segmentation}
\author{Xu Ji\\
University of Oxford\\
{\tt\small xuji@robots.ox.ac.uk}
\and
Jo\~ao F. Henriques\\
University of Oxford\\
{\tt\small joao@robots.ox.ac.uk}
\and
Andrea Vedaldi\\
University of Oxford\\
{\tt\small vedaldi@robots.ox.ac.uk}}

\begin{document}
\maketitle
\begin{abstract}
%We present a novel clustering objective that learns a classifier and an image representation from scratch given only unlabelled data samples.
%We present a novel clustering objective that learns a classifier given only unlabelled data samples, learning an image representation from scratch.

We present a novel clustering objective that learns a neural network classifier from scratch, given only unlabelled data samples.
The model discovers clusters that accurately match semantic classes, achieving state-of-the-art results in eight unsupervised clustering benchmarks spanning image classification and segmentation.
These include STL10, an unsupervised variant of ImageNet, and CIFAR10, where we significantly beat the accuracy of our closest competitors by 6.6 and 9.5 absolute percentage points respectively.
%The method is applicable to any data source that produces pairs of correlated data samples and, while we obtain such pairs by randomly transforming images, it is not restricted to computer vision data.
%The objective is to maximise the mutual information between the class assignments of each pair, optimising the cluster assignments end-to-end.
The method is not specialised to computer vision and operates on any paired dataset samples; in our experiments we use random transforms to obtain a pair from each image. 
The trained network directly outputs semantic labels, rather than high dimensional representations that need external processing to be usable for semantic clustering.
The objective is simply to maximise mutual information between the class assignments of each pair.
It is easy to implement and rigorously grounded in information theory, meaning we effortlessly avoid degenerate solutions that other clustering methods are susceptible to.
%This allows it to effortlessly avoids certain degenerate solutions that afflict other methods that output instead high dimensional representations that need post-processing for clustering.
In addition to the fully unsupervised mode, we also test two semi-supervised settings. The first achieves 88.8\% accuracy on STL10 classification, setting a new global state-of-the-art over all existing methods (whether supervised, semi-supervised or unsupervised).
The second shows robustness to 90\% reductions in label coverage, of relevance to applications that wish to make use of small amounts of labels. \lstinline[basicstyle=\ttfamily]{github.com/xu-ji/IIC} 
%The first achieves a new global state-of-the-art accuracy of 88.8\% on STL10 classification, which is better than all existing methods (whether supervised, semi-supervised or unsupervised). The second shows robustness to 90\% reductions in label coverage, of relevance to applications that wish to make use of small amounts of labels.
%when one can only afford to collect small amounts of labels.

\begin{comment}
We present a novel clustering objective that trains a randomly-initialised network into a classifier without any example labels provided in training or testing.
The discovered clusters correspond to semantic classes with high accuracy, setting new state-of-the-art records on 8 unsupervised datasets across image classification and segmentation, including STL10, an unsupervised variant of ImageNet on which we beat our closest competitor by 8\%. 
The method is not specialised to computer vision and can use any data with a pairwise distance metric; in our experiments we use random transforms to obtain a pair from each image.
The objective is simply to maximise mutual information between the class assignments of each pair.
It is easy to implement and rigorously grounded in information theory, meaning unlike other methods, we are able to avoid degenerate solutions with no effort. 
The objective optimises for the final clusters and therefore the trained network directly outputs semantic labels rather than high dimensional representations that need external processing to be usable.
In addition to the fully unsupervised mode, we also test two semi-supervised settings, setting a new global state of the art of 88.8\% accuracy on STL10 classification out of all known methods (whether supervised, semi-supervised or unsupervised) and demonstrating robustness to 90\% reductions in label coverage, of relevance to applications that wish to make use of small amounts of labels.
\end{comment}
\end{abstract}

\section{Introduction}
\label{sec:intro}

Language modeling is among the important problems that require modeling long-term dependency, with successful applications such as unsupervised pretraining~\citep{dai2015semi,peters2018deep,radford2018improving,devlin2018bert}.
However, it has been a challenge to equip neural networks with the capability to model long-term dependency in sequential data.
Recurrent neural networks (RNNs), in particular Long Short-Term Memory (LSTM) networks~\citep{hochreiter1997long}, have been a standard solution to language modeling and obtained strong results on multiple benchmarks.
Despite the wide adaption, RNNs are difficult to optimize due to gradient vanishing and explosion~\citep{hochreiter2001gradient}, and the introduction of gating in LSTMs and the gradient clipping technique~\citep{graves2013generating} might not be sufficient to fully address this issue.
% ,pascanu2012understanding
Empirically, previous work has found that LSTM language models use 200 context words on average~\citep{khandelwal2018sharp}, indicating room for further improvement.

On the other hand, the direct connections between long-distance word pairs baked in attention mechanisms might ease optimization and enable the learning of long-term dependency~\citep{bahdanau2014neural,vaswani2017attention}.
Recently, \citet{al2018character} designed a set of auxiliary losses to train deep Transformer networks for character-level language modeling, which outperform LSTMs by a large margin.
Despite the success, the LM training in~\citet{al2018character} is performed on separated fixed-length segments of a few hundred characters, without any information flow across segments.
As a consequence of the fixed context length, the model cannot capture any longer-term dependency beyond the predefined context length.
In addition, the fixed-length segments are created by selecting a consecutive chunk of symbols without respecting the sentence or any other semantic boundary.
Hence, the model lacks necessary contextual information needed to well predict the first few symbols, leading to inefficient optimization and inferior performance.
We refer to this problem as \textit{context fragmentation}.

%However, the context length is fixed to hundreds of characters and thus it is not possible to model longer-term dependency. Moreover, it is not clear how the model performs on word-level language modeling data, as the granularity changes.

% Moreover, using auxiliary losses brings additional challenges such as properly tuning the mixture weights and the loss decay schedule.

To address the aforementioned limitations of fixed-length contexts, we propose a new architecture called Transformer-XL (meaning extra long).
We introduce the notion of recurrence into our deep self-attention network. In particular, instead of computing the hidden states from scratch for each new segment, we reuse the hidden states obtained in previous segments.
The reused hidden states serve as memory for the current segment, which builds up a recurrent connection between the segments.
As a result, modeling very long-term dependency becomes possible because information can be propagated through the recurrent connections.
Meanwhile, passing information from the previous segment can also resolve the problem of context fragmentation.
More importantly, we show the necessity of using relative positional encodings rather than absolute ones, in order to enable state reuse without causing temporal confusion.
Hence, as an additional technical contribution, we introduce a simple but more effective relative positional encoding formulation that generalizes to attention lengths longer than the one observed during training.

Transformer-XL obtained strong results on five datasets, varying from word-level to character-level language modeling.
Transformer-XL is also able to generate relatively coherent long text articles with \textit{thousands of} tokens (see Appendix \ref{sec:gen}), trained on only 100M tokens.
% Transformer-XL improves the previous state-of-the-art (SoTA) results from 1.06 to 0.99 in bpc on enwiki8, from 1.13 to 1.08 in bpc on text8, from 20.5 to 18.3 in perplexity on WikiText-103, and from 23.7 to 21.8 in perplexity on One Billion Word.
% Transformer-XL improves the previous state-of-the-art (SoTA) results to 0.99 in bpc on enwiki8, 1.08 in bpc on text8, 18.3 in perplexity on WikiText-103, and 21.8 in perplexity on One Billion Word.
% On small data, Transformer-XL also achieves a perplexity of 54.5 on Penn Treebank without finetuning, which is SoTA when comparable settings are considered.

Our main technical contributions include introducing the notion of recurrence in a purely self-attentive model and deriving a novel positional encoding scheme. These two techniques form a complete set of solutions, as any one of them alone does not address the issue of fixed-length contexts. Transformer-XL is the first self-attention model that achieves substantially better results than RNNs on both character-level and word-level language modeling.

% On WikiText-103, Transformer-XL improves the previous state-of-the-art (SoTA) results from 33 perplexity to 24, with a relative reduction of 27\%. On enwiki8 character-level language modeling, Transformer-XL achieves a SoTA bpc of 1.03, which outperforms \cite{al2018character} by 0.03 with 60+\% fewer parameters. Given a more common model size with 40+M parameters, Transformer-XL achieves a bpc of 1.06, compared to 1.11 by \cite{al2018character}. Transformer-XL also achieves perplexities of 54.5 on Penn Treebank and 29.4 on One Billion Word, which are SoTA when comparable settings are considered.

% Due to the ability of modeling long-range context, our best model uses attention lengths of 1,600 and 3,800 on WikiText-103 and enwiki8 respectively. We also devise a metric called \textit{Relative Effective Context Length} (RECL) that aims to fairly compare the ability of long-range dependency modeling.
% % perform a fair comparison of the gains brought by increasing the context lengths for different models.
% In this setting, Transformer-XL learns a RECL of 900 words on WikiText-103, while the numbers for recurrent networks and Transformer are only 500 and 128.

% We use two methods to quantitatively study the effective lengths of Transformer-XL and the baselines. Similar to \cite{khandelwal2018sharp}, we gradually increase the attention length at test time until no further noticeable improvement ($\sim$0.1\% relative gains) can be observed. Our best model in this settings use attention lengths of 1,600 and 3,800 on WikiText-103 and enwiki8 respectively.
% %In addition, since the effective context length of Transformer-XL can be longer than the attention length due to our recurrent formulation, we devise a metric called \textit{Relative Effective Context Length} (RECL) that aims to perform a fair comparison of the gains brought by increasing the context lengths for different models.
% In addition, we devise a metric called \textit{Relative Effective Context Length} (RECL) that aims to perform a fair comparison of the gains brought by increasing the context lengths for different models.
% In this setting, Transformer-XL learns a RECL of 900 words on WikiText-103, while the numbers for recurrent networks and Transformer are only 500 and 128.

\paragraph{3D Object Detection from RGB-D Data} Researchers have approached the 3D detection problem by taking various ways to represent RGB-D data.

\emph{Front view image based methods:} ~\cite{chen2016monocular, mousavian20163d, xiang2015data} take monocular RGB images and shape priors or occlusion patterns to infer 3D bounding boxes. ~\cite{li2016vehicle, deng2017amodal} represent depth data as 2D maps and apply CNNs to localize objects in 2D image. In comparison we represent depth as a point cloud and use advanced 3D deep networks (PointNets) that can exploit 3D geometry more effectively.

\emph{Bird's eye view based methods:} MV3D~\cite{cvpr17chen} projects LiDAR point cloud to bird's eye view and trains a region proposal network (RPN~\cite{ren2015faster}) for 3D bounding box proposal. However, the method lags behind in detecting small objects, such as pedestrians and cyclists and cannot easily adapt to scenes with multiple objects in vertical direction.
%Our method shares the idea with~\cite{cvpr17chen} in reducing 3D search cost by 2D search first. What differentiates our method from \cite{cvpr17chen} is that, \hao{???} instead of projecting point cloud to images costing loss in 3D geometry, we directly apply PointNet to point clouds that correspond to the 2D regions. % Besides, our method and MV3D can potentially be combined in the bird's eye setting. 3D proposals from our frustum-based PointNet and MV3D can be combined and our 3D network can also be used for bounding box estimation for point cloud in the bird's eye 2D region.

\emph{3D based methods:} ~\cite{wang2015voting, song2014sliding} train 3D object classifiers by SVMs on hand-designed geometry features extracted from point cloud and then localize objects using sliding-window search. \cite{engelcke2017vote3deep} extends ~\cite{wang2015voting} by replacing SVM with 3D CNN on voxelized 3D grids. \cite{ren2016three} designs new geometric features for 3D object detection in a point cloud. \cite{song2016deep, li20163d} convert a point cloud of the entire scene into a volumetric grid and use 3D volumetric CNN for object proposal and classification. Computation cost for those method is usually quite high due to the expensive cost of 3D convolutions and large 3D search space.
%In comparison, we use 2D region proposals from RGB images to reduce the search space from the entire 3D scenes into 3D frustums. Since the points cloud in the frustums have largely varying depth ranges and can be very sparse, it's not applicable to apply CNN on bird's eye view or apply 3D CNN in grids. Our frustum-based PointNet, on the other hand, suits well for this type of data and is able to accurately estimate 3D bounding box with good efficiency.
Recently, \cite{lahoud20172d} proposes a 2D-driven 3D object detection method that is similar to ours in spirit. However, they use hand-crafted features (based on histogram of point coordinates) with simple fully connected networks to regress 3D box location and pose, which is sub-optimal in both speed and performance. In contrast, we propose a more flexible and effective solution with deep 3D feature learning (PointNets).
%In addition we also get 3D instance segmentation as intermediate outputs. Evaluated on SUN-RGBD we show our method is \emph{8.9\%} better than theirs in mAP and \emph{34x} faster at the same time.


% \begin{enumerate}
%     \item ZOOX~\cite{mousavian20163d} image based
%     \item Vote3Deep~\cite{engelcke2017vote3deep} 3d cnn. Recent LIDAR-based methods place 3D windows in 3D voxel grids to score the point cloud
%     \item Voting for Voting~\cite{wang2015voting} Recent LIDAR-based methods place 3D windows in 3D voxel grids to score the point cloud. apply SVM classifers on 3D grids encoded with geometry features
%     \item MV3D~\cite{cvpr17chen}
%     \item VeloFCN~\cite{li2016vehicle} apply convolutional networks to the front view point map in a dense box prediction scheme
%     \item 3DOP~\cite{chen20153d} image based. reconstructs depth from stereo images and uses an energy minimization approach to generate 3D box proposals, which are fed to an R-CNN [10] pipeline for object recognition
%     \item Mono3D~\cite{chen2016monocular} image based. shares the same pipeline with 3DOP, it generates 3D proposals from monocular images.
%     \item 3DFCN~\cite{li20163d} 3d cnn.
%     \item 3DVP~\cite{xiang2015data} introduces 3D voxel patterns and employ a set of ACF detectors to do 2D detection and 3D pose estimation
%     \item Are Cars just 3D Box?~\cite{zeeshan2014cars} fit model to image patch
%     \item ~\cite{zia2013detailed} fit model to image patch
% \end{enumerate}
% \begin{enumerate}
%     \item SlidingShapes~\cite{song2014sliding} apply SVM classifers on 3D grids encoded with geometry features
%     \item DeepSlidingShapes~\cite{song2015sun} 3d cnn.
%     \item 2D-driven~\cite{lahoud20172d}
%     \item ~\cite{deng2017amodal} rgb-d images
%     \item COG feature~\cite{ren2016three}
%     \item Align 3D model in RGB-D~\cite{gupta2015aligning}
% \end{enumerate}

\paragraph{Deep Learning on Point Clouds}
Most existing works convert point clouds to images or volumetric forms before feature learning. \cite{wu20153d, maturana2015voxnet, qi2016volumetric} voxelize point clouds into volumetric grids and generalize image CNNs to 3D CNNs. ~\cite{li2016fpnn, riegler2016octnet, wang2017cnn, engelcke2017vote3deep} design more efficient 3D CNN or neural network architectures that exploit sparsity in point cloud.
However, these CNN based methods still require quantitization of point clouds with certain voxel resolution.
Recently, a few works~\cite{qi2017pointnet,qi2017pointnetplusplus} propose a novel type of network architectures (PointNets) that directly consumes raw point clouds without converting them to other formats. While PointNets have been applied to single object classification and semantic segmentation, our work explores how to extend the architecture for the purpose of 3D object detection.
\begin{figure*}
\centering
\includegraphics[width=0.85\textwidth]{architecture}
\caption{The overall architecture of our proposed network. The network
contains layers of symmetric convolution (encoder) and deconvolution (decoder).
Skip shortcuts are connected every a few (in our experiments, two) layers from
convolutional feature maps to their mirrored deconvolutional feature maps.
The response from a convolutional layer is directly propagated to the corresponding
mirrored deconvolutional layer, both forwardly and backwardly.}
\label{fig1}
\end{figure*}

\section{Very deep convolutional auto-encoder for image restoration}
\label{sec:main}

The proposed framework mainly contains a chain of convolutional layers and symmetric
deconvolutional layers, as shown in Figure \ref{fig1}. Skip connections are connected
symmetrically from convolutional layers to deconvolutional layers. We term our method
``RED-Net''---very deep Residual Encoder-Decoder Networks.


\subsection{Architecture}

The framework is fully convolutional (and deconvolutional.  Deconvolution is essentially unsampling convolution). Rectification layers are added
after each convolution and deconvolution. For low-level image restoration problems, we
use neither pooling nor unpooling in the network as usually pooling discards useful image
details that are essential for these tasks. It is worth mentioning that since the convolutional
and deconvolutional layers are symmetric, the network is essentially pixel-wise prediction,
thus the size of input image can be arbitrary. The input and output of the network are images
of the same size $w\times h\times c$, where $w$, $h$ and $c$ are width, height and number of channels.

Our main idea is that the convolutional layers act as a feature extractor, which preserve the
primary components of objects in the image and meanwhile eliminating the corruptions.
After forwarding through the convolutional layers, the corrupted input  image is converted into
a ``clean" one. The subtle details of the image contents may be lost during this process.
The deconvolutional layers are then combined to recover the details of image contents.
The output of the deconvolutional layers is the recovered clean version of the input image.
Moreover, we add skip connections  from a convolutional layer to its corresponding
mirrored deconvolutional layer. The passed convolutional feature maps are summed to the
deconvolutional feature maps element-wise, and passed to the next layer after rectification.
Deriving from the above architecture, we have used two networksvin our experiments, which are of 20 layers
 and 30 layers
respectively, for image denoising, image super-resolution, JPEG deblocking and image inpainting.



\subsection{Deconvolution decoder}

Architectures combining layers of convolution and deconvolution~\cite{DBLP:conf/iccv/NohHH15,
hong2015decoupled} have been proposed for semantic segmentation recently. In contrast to
convolutional layers, in which multiple input activations within a filter window are fused
to output a single activation, deconvolutional layers associate a single input activation with
multiple outputs. Deconvolution is usually used as {\em learnable up-sampling layers}.

 In our network,
the convolutional layers successively down-sample the input image content into a  small
size abstraction. Deconvolutional layers then up-sample the abstraction back into its original resolution.

Besides the use of skip connections, a main difference between our model and
~\cite{DBLP:conf/iccv/NohHH15,hong2015decoupled} is that our network is fully convolutional and
deconvolutional, i.e., without pooling and un-pooling. The reason is that for low-level image restoration,
the aim is to eliminate low level corruption while preserving image details instead of learning
image abstractions. Different from high-level applications such as segmentation or recognition,
pooling typically eliminates the abundant image details and can deteriorate restoration performance.



One can simply replace deconvolution with convolution, which results in an architecture that is
very similar to recently proposed very deep fully convolutional neural networks
~\cite{DBLP:conf/cvpr/LongSD15,DBLP:journals/pami/DongLHT16}. However, there exist essential
differences between a fully convolution model and our model. Take image denoising as an example.
We compare the 5-layer and 10-layer fully convolutional network with our network
(combining convolution and deconvolution, but without skip connection). For fully convolutional
networks, we use padding or up-sampling the input to make the input and output be of the same size.
For our network, the first 5 layers are convolutional and the second 5 layers are deconvolutional.
All the other parameters for training are identical, i.e., trained with SGD and learning rate of
$10^{-6}$, noise level $\sigma=70$. The Peak Signal-to-Noise Ratio (PSNR) on the validation set
is reported, which shows that using deconvolution works better than the fully convolutional
counterpart, as shown in Figure \ref{fig2}.


Furthermore, in Figure \ref{fig3}, we visualize some results that are outputs of layer 2, 5, 8 and 10
from the 10-layer fully convolutional network and ours. In the fully convolution case, the noise
is eliminated step by step, i.e., the noise level is reduced after each layer. During this process,
the details of the image content may be lost. Nevertheless, in our network, convolution  preserves
the primary image content. Then deconvolution is used to compensate the details.


\begin{figure}[htb!]
\centering
\includegraphics[width=0.48\textwidth]{conv-vs-decv}
\caption{ PSNR  values  on the validation set during training. Our model  exhibits better PSNR
than the compared ones upon convergence.}
\label{fig2}
\end{figure}



\begin{figure}[htb!]
\centering
\subfigure[]{ \includegraphics[width=0.48\textwidth]{show-denoising-conv} }
\subfigure[]{ \includegraphics[width=0.48\textwidth]{show-denoising-decv} }
\caption{ (a) Visualization of the 10-layer fully convolutional network. The images from
top-left to bottom-right are: clean image, noisy image, output of conv-2, output of conv-5,
output of conv-8 and output of conv-10, where ``conv-$i$" stands for the $i$-th convolutional layer;
(b) Visualization of the 10-layer convolutional and deconvolutional network. The images from
top-left to bottom-right are: clean image, noisy image, output of conv-2, output of conv-5,
output of deconv-3 and output of deconv-5, where ``deconv-$i$" stands for the $i$-th deconvolutional layer.}
\label{fig3}
\end{figure}




\subsection{Skip connections}

An intuitive question is that, is a network with deconvolution able to recover image details from
the image abstraction only? We find that in shallow networks with only a few layers
of convolution layers, deconvolution is able to recover the details. However, when the
network goes deeper or using operations such as max pooling, even with deconvolution layers, it does not work
that well, possibly because too much details are already lost in the convolution and pooling.


The second question is that, when our network goes deeper, does it achieve performance gain?
We observe that deeper networks in image restoration tasks tend to easily suffer from
performance degradation. The reason may be two folds. First of all, with more layers of
convolution, a significant amount of image details could be lost or corrupted. Given only the image abstraction,
recovering its details is an under-determined problem. Secondly, in terms of optimization,
deep networks often suffer from gradients vanishing and become much harder to train---a problem
that is well addressed in the literature of neural networks.


To address the above two problems, inspired by highway networks \cite{DBLP:journals/corr/SrivastavaGS15}
and deep residual networks \cite{DBLP:journals/corr/HeZRS15}, we add skip connections between
two corresponding convolutional and deconvolutional layers as shown in Figure \ref{fig1}.
A building block is shown in Figure \ref{fig4}. There are two reasons for using such connections.
First, when the network goes deeper, as mentioned above, image details can be lost, making deconvolution
weaker in recovering them. However, the feature maps passed by skip connections carry much image detail,
which helps deconvolution to recover an improved clean version of the image. Second, the skip connections also achieve
benefits on back-propagating the gradient to bottom layers, which makes training deeper network much
easier as observed in \cite{DBLP:journals/corr/SrivastavaGS15} and \cite{DBLP:journals/corr/HeZRS15}.

Note that our skip layer connections are very different from the ones proposed in
\cite{DBLP:journals/corr/SrivastavaGS15} and \cite{DBLP:journals/corr/HeZRS15}, where the only concern
is on the optimization side. In our case, we want to pass information of the convolutional feature maps
to the corresponding deconvolutional layers. The very deep highway networks
\cite{DBLP:journals/corr/SrivastavaGS15} are essentially feedforward long short-term memory (LSTMs)
with forget gates, and the CNN layers of deep residual network \cite{DBLP:journals/corr/HeZRS15}
are feedforward LSTMs without gates. Note that our networks are in general not in the format of
standard feedforward LSTMs.

\begin{figure}[htb!]
\centering
\includegraphics[width=0.48\textwidth]{block}
\caption{An example of a building block in the proposed framework. The rectangle in solid and
dotted lines denote convolution and deconvolution respectively. $\oplus$ denotes element-wise sum of feature maps.}
\label{fig4}
\end{figure}

Instead of directly learning the mappings from the input $X$ to the output $Y$, we would like the network
to fit the residual~\cite{DBLP:journals/corr/HeZRS15} of the problem, which is denoted as $\mathcal{F}(X)=Y-X$.
Such a learning strategy is applied to inner blocks of the encoding-decoding network to make training more
effective. Skip connections are passed every two convolutional layers to their mirrored deconvolutional
layers. Other configurations are possible and our experiments show that this configuration already works
very well. Using such shortcuts makes the network easier to be trained and gains restoration performance
by increasing the network depth.




\subsection{Training}

In general, there are three types of layers in our network: convolution, deconvolution
and element-wise sum. Each layer is followed by a Rectified Linear Unit (ReLU)
~\cite{DBLP:conf/icml/NairH10}. Let $X$ be the input, the convolutional and
deconvolutional layers are expressed as:
\begin{equation}
F(X) = \max(0,W_k * X + B_k),
\end{equation}
where $W_k$ and $B_k$ represent the filters and biases, and $*$ denotes either
convolution or deconvolution operation for the convenience of formulation.
For element-wise sum layer, the output is the element-wise sum of two inputs
of the same size, followed by the ReLU activation:
\begin{equation}
F(X_1,X_2) = \max(0, X_1 + X_2)
\end{equation}

Learning the end-to-end mapping from corrupted images to clean images needs to
estimate the weights $\Theta$ represented by the convolutional and deconvolutional
kernels. Specifically, given a collection of $N$ training sample pairs $\{X^i,Y^i\}$,
where $X^i$ is a noisy image and $Y^i$ is the clean version as the groundtruth.
We minimize the following Mean Squared Error (MSE):
\begin{equation}
  \mathcal{L}(\Theta) = \frac{1}{N}\sum_{i=1}^{N}\|\mathcal{F}(X^i;\Theta)-Y^i\|_F^2.
\label{eq1}
\end{equation}

Traditionally, a  network can learn the mapping from the corrupted image to the clean version
directly. However, our network learns for the additive corruption from the input since there
is a skip connection between the input and the output of the network.
%
%
%
We found that optimizing for the corruption converges better than
optimizing for the clean image. In the extreme case, if the input is a clean image, it would be easier
to push the network to be zero mapping (learning the corruption) than to fit an identity
mapping (learning the clean image) with a stack of nonlinear layers.

We implement and train our network using Caffe~\cite{jia2014caffe}. Empirically, we find
that using Adam~\cite{DBLP:journals/corr/KingmaB14} with base learning rate of $10^{-4}$ for
training converges faster than traditional stochastic gradient descent (SGD). The base
learning rate for all layers are the same, different from ~\cite{DBLP:journals/pami/DongLHT16,
DBLP:conf/nips/JainS08}, in which a smaller learning rate is set for the last layer.
This  is not necessary in our network. Specifically, gradients with respect to the
parameters of $i$th layer is firstly computed as:
\begin{equation}
g = \nabla_{\theta_i}\mathcal{L}(\theta_i).
\end{equation}
Then, the two momentum vectors are computed as:
\begin{equation}
m = \beta_1m + (1 - \beta_1)g,\quad v = \beta_2v + (1-\beta_2)g^2.
\end{equation}
The update rule is:
\begin{equation}
\alpha = \alpha\sqrt{1-\beta_2^t}/(1-\beta_1^t), \quad \theta_i=\theta_i-\alpha m/(\sqrt{v}+\epsilon).
\end{equation}
$\beta_1$, $\beta_2$ and $\epsilon$ are set as the recommended values in~\cite{DBLP:journals/corr/KingmaB14}.

300 images from the Berkeley Segmentation Dataset (BSD)~\cite{MartinFTM01} are used to
generate image patches as the training set for each image restoration task.
%
%
%




\subsection{Testing}

Although trained on local patches, our network can perform restoration on images of arbitrary sizes.
Given a testing image, one can simply go forward through the network, which is already able to
 outperform existing methods. To achieve even better results, we propose
to process a corrupted image on multiple orientations. Different from segmentation, the
filter kernels in our network only eliminate the corruptions, which is usually not sensitive
to the orientation of image contents in low level restoration tasks. Therefore, we can rotate
and mirror flip the kernels and perform forward multiple times, and then average the output to
achieve an ensemble of multiple tests. We see that this can lead to slightly better performance.

% !TEX root = ../multi_task.tex

We evaluate the presented MTL method on a number of problems. First, we use MultiMNIST \citep{multi_mnist}, an MTL adaptation of MNIST \citep{mnist}. Next, we tackle multi-label classification on the CelebA dataset \citep{celeba} by considering each label as a distinct binary classification task. These problems include both classification and regression, with the number of tasks ranging from 2 to 40. Finally, we experiment with scene understanding, jointly tackling the tasks of semantic segmentation, instance segmentation, and depth estimation on the Cityscapes dataset \citep{cityscapes}. We discuss each experiment separately in the following subsections.

The baselines we consider are (i) \textbf{uniform scaling:} minimizing a uniformly weighted sum of loss functions \mbox{$\frac{1}{T}\sum_t \lL^t$}, \mbox{(ii) \textbf{single task:}} solving tasks independently, \mbox{(iii) \textbf{grid search:}} exhaustively trying various values from $\{ c^t \in [0,1] | \sum_t c^t = 1\}$ and optimizing for $\frac{1}{T}\sum_t c^t \lL^t$, \mbox{(iv) \textbf{\citet{Kendall2018}:}} using the uncertainty weighting proposed by \citet{Kendall2018}, and \mbox{(v) \textbf{GradNorm:}} using the normalization proposed by \citet{Chen2018}.



\subsection{MultiMNIST}
\label{sec:multi_mnist_exp}

Our initial experiments are on MultiMNIST, an MTL version of the MNIST dataset \citep{multi_mnist}. In order to convert digit classification into a multi-task problem, \citet{multi_mnist} overlaid multiple images together. We use a similar construction. For each image, a different one is chosen uniformly in random. Then one of these images is put at the top-left and the other one is at the bottom-right. The resulting tasks are: classifying the digit on the top-left (task-L) and classifying the digit on the bottom-right (task-R). We use 60K examples and directly apply existing single-task MNIST models. The MultiMNIST dataset is illustrated in the supplement.

We use the LeNet architecture \citep{mnist}. We treat all layers except the last as the representation function $g$ and put two fully-connected layers as task-specific functions (see the supplement for details). We visualize the performance profile as a scatter plot of accuracies on task-L and task-R in Figure~\ref{fig:multi_mnist_performance_curve}, and list the results in Table~\ref{tab:multi_mnist}.

In this setup, any static scaling results in lower accuracy than solving each task separately (the single-task baseline). The two tasks appear to compete for model capacity, since increase in the accuracy of one task results in decrease in the accuracy of the other. Uncertainty weighting \citep{Kendall2018} and GradNorm \citep{Chen2018} find solutions that are slightly better than grid search but distinctly worse than the single-task baseline. In contrast, our method finds a solution that efficiently utilizes the model capacity and yields accuracies that are as good as the single-task solutions. This experiment demonstrates the effectiveness of our method as well as the necessity of treating MTL as multi-objective optimization. Even after a large hyper-parameter search, \emph{any} scaling of tasks does not approach the effectiveness of our method.



\subsection{Multi-Label Classification}

\begin{figure}[t]
\includegraphics[width=\textwidth]{radar_full_new}
\vspace{1mm}
\caption{Radar charts of percentage error per attribute on CelebA \citep{celeba}. Lower is better. We divide attributes into two sets for legibility: easy on the left, hard on the right. Zoom in for details.}
\label{fig:multi_label_radar}
\end{figure}


\begin{wraptable}{r}{0.3\textwidth}
%\vspace{-4mm}
\captionof{table}{Mean of error per category of MTL algorithms in multi-label classification on CelebA \citep{celeba}.}
\begin{tabular}{r@{\hspace{2mm}}c@{}}
\toprule
& Average  \\
&  error \\
\midrule
Single task & $8.77$ \\
Uniform scaling & $9.62$ \\
\citealt{Kendall2018} & $9.53$ \\
GradNorm & $8.44$ \\
Ours & $\mathbf{8.25}$  \\
\bottomrule
\end{tabular}
\label{table:multi_label_bar}
%\vspace{-5mm}
\end{wraptable}

Next, we tackle multi-label classification. Given a set of attributes, multi-label classification calls for deciding whether each attribute holds for the input. We use the CelebA dataset \citep{celeba}, which includes 200K face images annotated with 40 attributes. Each attribute gives rise to a binary classification task and we cast this as a 40-way MTL problem. We use ResNet-18 \citep{resnet} without the final layer as a shared representation function, and attach a linear layer for each attribute (see the supplement for further details).


We plot the resulting error for each binary classification task as a radar chart in Figure~\ref{fig:multi_label_radar}. The average over them is listed in Table~\ref{table:multi_label_bar}. We skip grid search since it is not feasible over 40 tasks. Although uniform scaling is the norm in the multi-label classification literature, single-task performance is significantly better. Our method outperforms baselines for significant majority of tasks and achieves comparable performance in rest. This experiment also shows that our method remains effective when the number of tasks is high.


\subsection{Scene Understanding}

To evaluate our method in a more realistic setting, we use scene understanding. Given an RGB image, we solve three tasks: semantic segmentation (assigning pixel-level class labels), instance segmentation (assigning pixel-level instance labels), and monocular depth estimation (estimating continuous disparity per pixel). We follow the experimental procedure of \citet{Kendall2018} and use an encoder-decoder architecture. The encoder is based on ResNet-50 \citep{resnet} and is shared by all three tasks. The decoders are task-specific and are based on the pyramid pooling module \citep{pspnet} (see the supplement for further implementation details).

Since the output space of instance segmentation is unconstrained (the number of instances is not known in advance), we use a proxy problem as in \citet{Kendall2018}. For each pixel, we estimate the location of the center of mass of the instance that encompasses the pixel. These center votes can then be clustered to extract the instances. In our experiments, we directly report the MSE in the proxy task. Figure~\ref{fig:cityscapes_performance_profile} shows the performance profile for each pair of tasks, although we perform all experiments on all three tasks jointly. The pairwise performance profiles shown in Figure~\ref{fig:cityscapes_performance_profile} are simply 2D projections of the three-dimensional profile, presented this way for legibility. The results are also listed in Table~\ref{tab:cityscapes_results}.

MTL outperforms single-task accuracy, indicating that the tasks cooperate and help each other. Our method outperforms all baselines on all tasks.


\subsection{Role of the Approximation}

In order to understand the role of the approximation proposed in Section~\ref{sec:approximation}, we compare the final performance and training time of our algorithm with and without the presented approximation in Table~\ref{tab:approximation_tradeoff} (runtime measured on a single Titan Xp GPU). For a small number of tasks (3 for scene understanding), training time is reduced by 40\%. For the multi-label classification experiment (40 tasks), the presented approximation accelerates learning by a factor of 25.

On the accuracy side, we expect both methods to perform similarly as long as the full-rank assumption is satisfied. As expected, the accuracy of both methods is very similar. Somewhat surprisingly, our approximation results in slightly improved accuracy in all experiments. While counter-intuitive at first, we hypothesize that this is related to the use of SGD in the learning algorithm. Stability analysis in convex optimization suggests that if gradients are computed with an error $\hat{\nabla}_\btheta \mathcal{L}^t = \nabla_\btheta \mathcal{L}^t + \mathbf{e}^t$ ($\btheta$ corresponds to $\btheta^{sh}$ in (\ref{eq:kkt_opt})), as opposed to $\mathbf{Z}$ in the approximate problem in \ref{eq:approx}, the error in the solution is bounded as $\|\hat{\mathbf{\alpha}} - \mathbf{\alpha} \|_2 \leq \mathcal{O}(\max_t \|\mathbf{e}^t\|_2)$. Considering the fact that the gradients are computed over the full parameter set (millions of dimensions) for the original problem and over a smaller space for the approximation (batch size times representation which is in the thousands), the dimension of the error vector is significantly higher in the original problem. We expect the $l_2$ norm of such a random vector to depend on the dimension.

In summary, our quantitative analysis of the approximation suggests that (i) the approximation does not cause an accuracy drop and (ii) by solving an equivalent problem in a lower-dimensional space, our method achieves both better computational efficiency and higher stability.

  {\small
  \begin{table}[t]
%  \vspace{-4mm}
  \caption{Effect of the MGDA-UB approximation. We report the final accuracies as well as training times for our method with and without the approximation.}
  %\vspace{1mm}
  \centering
  \begin{tabular}{@{}r@{\hspace{3mm}}c@{\hspace{3mm}}c@{\hspace{2mm}}c@{\hspace{2mm}}c@{}c@{\hspace{5mm}}c@{\hspace{2mm}}c@{}}
  \toprule
  & \multicolumn{4}{c}{Scene understanding (3 tasks)} &  & \multicolumn{2}{c}{Multi-label (40 tasks)}  \\
  \cmidrule(r){2-5} \cmidrule(lr){7-8}
                  & Training & Segmentation & Instance  & Disparity      & & Training & Average \\
                 & time     &  mIoU [\%]       & error [px] & error [px] & & time (hour)      & error \\
  \midrule
  Ours (w/o approx.) & $38.6$ & $66.13$ & $10.28$ & $2.59$ & & $429.9$ & $8.33$ \\
  Ours & $\mathbf{23.3}$ & $\mathbf{66.63}$ & $\mathbf{10.25}$ & $\mathbf{2.54}$  & & $\mathbf{16.1}$ & $\mathbf{8.25}$ \\
  \bottomrule
  \end{tabular}
  %\vspace{-2mm}
  \label{tab:approximation_tradeoff}
  \end{table}}

\section{Conclusions}

Our work is motivated by two major deficiencies in training the current generative models for text generation: exposure bias and a loss which does not operate at the sequence level.
While Reinforcement learning can potentially address these issues, it struggles in settings when 
there are very large action spaces, such as in text generation. Towards that end, 
we propose the MIXER algorithm, which deals with these issues and enables successful training of reinforcement learning models for text generation. 
We achieve this by replacing the initial random policy with the optimal policy of a cross-entropy trained model and by gradually exposing the model more and more to its own predictions in an incremental learning framework.




%. First, the exposure bias affecting the commonly used cross-entropy loss. 
%While the model sees only ground truth inputs at training time, at test time model predictions are fed back as input to generate a full sequence. 
%Second, current text generation systems are often trained to predict the next word in the sequence without taking into account the quality of the % overall sequence. 
% These discrepancies make the generation process brittle.
%Reinforcement learning is a framework that can address these issues. 
%First, at training time the model is used to generate an entire sequence of actions. 
%Second, the reward does not need to factor over individual words nor does it need to be differentiable. 
%Therefore, we can easily and directly operate at the sequence level, generate at training time and optimize our model towards any desired metric, such as BLEU and ROUGE. 
%One challenge with reinforcement learning is that it struggles with very large action spaces such as for text generation.

% Mixed Incremental Cross-Entropy Reinforce (MIXER) 
%The algorithm we propose, MIXER, 
%deals with this issue and enables successful training of reinforcement learning models for text generation. 
%We achieve this by replacing the initial random policy with the optimal policy of a cross-entropy trained model and by gradually exposing the model more and more to its own predictions in an incremental learning framework.

Our results show that MIXER outperforms three strong baselines for greedy generation and it is very competitive with beam search. 
The approach we propose is agnostic to the underlying model or the form of the reward function. 
% We are free to use any other metric as reward such as ROUGE or METEOR instead of BLEU. 
% Similarly, we may use a different parametric model such as a feed- forward network or an LSTM \citep{lstm}.
In future work we would like to design better estimation techniques for the average reward $\bar{r}_t$, because poor estimates can lead to slow convergence of both REINFORCE and MIXER. 
Finally, our training algorithm relies on a single sample while it would be interesting to investigate the effect of more comprehensive search methods at training time.


% Our work addresses two major deficiencies in training the current generative models for text generation. First, it addresses the {\it exposure bias} affecting the commonly used cross-entropy loss. 
% %While the model sees only ground truth inputs at training 
% %time, at test time model predictions are fed back as input to generate a full sequence. 
% Second, it directly tries to optimize for the final evaluation metric, namely, BLEU. 
% %current text generation systems are often trained to predict the next word in the sequence without taking into account the quality of the overall sequence. These discrepancies make the generation process brittle. 
% Both these objectives are accomplished by the proposed Mixed Incremental Cross-Entropy Reinforce (MIXER) algorithm. 
% %Reinforcement learning is a framework that can address these issues. First, at training time the model is used to generate an entire sequence of actions. Second, the reward does not need to factor over individual words nor does it need to be differentiable. Therefore, we can easily and directly operate at the sequence level, generate at training time and optimize our model towards BLEU, our test time evaluation metric. One challenge with reinforcement learning is that it struggles with very large action spaces such as for text generation.
% MIXER is an extension of the REINFORCE algorithm applied to text generation, which 
% %Mixed Incremental Cross-Entropy Reinforce (MIXER) deals with this issue and enables
% %successful training of reinforcement learning models for text generation.
% replaces the initial random policy with the optimal policy of
% a cross-entropy trained model and it gradually exposes the model more and more to its own predictions in an incremental learning framework.

% Our results show that MIXER outperforms three strong baselines for greedy generation and it is very competitive with beam search. 
% The approach we propose is agnostic to the underlying model or the form of the reward function. 
% We are free to use any other metric as reward such as ROUGE or METEOR instead of BLEU. 
% Similarly, we may use a different parametric model such as a feed-forward network or an LSTM~\citep{lstm}.

% For future we would like to design better estimation techniques for the average reward $\bar{r}_t$, because poor estimates can lead to slow convergence of both REINFORCE and MIXER.
% Finally, our training algorithm relies on a single sample while it would be interesting to investigate the effect of more comprehensive search methods at training time.

% In this study, we investigated sequence level training algorithms for RNNs with the goal to improve text generation.
% Today, the dominant training protocol is cross-entropy loss, which optimizes the prediction of the next word in the sequence. However, at test time the model is asked to predict several words in the future by re-circulating its own prediction back to the input. 
% The problem of predicting several steps in the future while obtaining delayed feedback, and to perform prediction via a discrete sequence of actions inspired us to apply reinforcement learning techniques. Unfortunately, reinforcement learning techniques do not usually handle well large action spaces, like those we encounter in typical language modeling applications.

% MIXER addresses these limitations through pre-training and incremental learning. 

% MIXER addresses these limitations by leveraging both the fact that we have access to the optimal policy and by using incremental learning.
% Since we have examples of ground truth generation, we can "pre-train" the model for next step prediction via cross-entropy. This drastically reduces the actual search space. By using incremental learning, the model is then able to gradually produce stable sequences and to make effective use of its own predictions.

% Our empirical validation shows that the model we propose achieves the best BLEU score compared to three strong baselines. Moreover, generations can be further improved by using beam search. Note that the approach we proposed is agnostic of the particular underlying model and metric. We can easily replace BLEU with ROUGE, METEOR, \etc by simply swapping the function that computes rewards within the training loop. Similarly, the training algorithm applies to any type of model and RNN, LSTM~\citep{lstm} included.



% There are several avenues of future investigation. First, REINFORCE upon which we build, requires careful estimation of the average reward. Poor estimation of this value can yield very slow convergence. More generally, searching at training time is still an unsolved problem. In particular, it would be very powerful to include beam search also at training time. 


{\noindent\textbf{Acknowledgments.} We are grateful to ERC StG IDIU-638009 and EPSRC AIMS CDT for support.}
{\small\bibliographystyle{ieee_fullname}\bibliography{main}}
%\clearpage
%\section{Positive Definiteness of~$K$}\label{sec:appendixA}
To show that the kernel~$K$ defined in~(\ref{eq:kernel}) is positive definite
(p.d.), we simply use elementary rules from the kernel literature described in
Sections 2.3.2 and 3.4.1 of~\cite{shawe2004}.  A linear combination of p.d. kernels with non-negative weights is also p.d. (see Proposition 3.22
of\cite{shawe2004}), and thus it is sufficient to show that for all $\z,\z'$
in~$\Omega$, the following kernel on $\Omega \to \HH$ is p.d.:
\begin{displaymath}
   (\varphi,\varphi') \mapsto \big\|\varphi(\z)\big\|_\HH  \normH{\varphi'(\z')} e^{-\frac{1}{2\sigma^2} \normH{\tildephi(\z)-\tildephi'(\z')}^2}.
\end{displaymath}
Specifically, it is also sufficient to
show that the following kernel on $\HH$ is p.d.:
\begin{displaymath}
   (\phi,\phi') \mapsto \big\|{\phi}\big\|_\HH  \normH{\phi'} e^{-\frac{1}{2\sigma^2} \normH{\frac{\phi}{\|\phi\|_\HH}-\frac{\phi'}{\|\phi'\|_\HH}}^2}.
\end{displaymath}
with the convention $\phi/\|\phi\|_\HH=0$ if~$\phi=0$.
This is a pointwise product of two kernels and is p.d. when each of the two
kernels is p.d. The first one is obviously p.d.: $(\phi,\phi') \mapsto
\|{\phi}\|_\HH  \normH{\phi'}$. The second one is a composition of the Gaussian
kernel---which is p.d.---, with feature maps $\phi/\|\phi\|_\HH$ of a
normalized linear kernel in~$\HH$.  This composition is p.d. according to
Proposition 3.22, item (v) of~\cite{shawe2004} since the normalization does
not remove the positive-definiteness property.

\section{List of Architectures Reported in the Experiments}\label{appendix:arch}
We present in details the architectures used in the paper in Table~\ref{table:arch}.
\begin{table}[hbtp]
   \centering
   \begin{tabular}{|*{9}{c|}}
      \hline
      Arch. & $N$ & $m_1$  & $p_1$  &  $\gamma_1$ & $m_2$ &  $p_2$ & $S$  &  $\sharp$ param\\
      \hline
      \hline
      \multicolumn{9}{|c|}{MNIST} \\
      \hline
      CKN-GM1 & 2 &  $1 \times 1$  &  12  & 2 &  $3 \times 3$ &  50 &  $4 \times 4$ & $5\,400$\\
      \hline
      CKN-GM2 & 2 &  $1 \times 1$  &  12  & 2 &  $3 \times 3$ &  400 &  $3 \times 3$& $43\,200$ \\
      \hline
      CKN-PM1 & 1 &  $5 \times 5$  &  200  & 2 &  - &  - &  $4 \times 4$  & $5\,000$ \\
      \hline
      CKN-PM2 & 2 &  $5 \times 5$  &  50  & 2 &  $2 \times 2$ &  200 &  $6 \times 6$ & $41\,250$ \\
      \hline
      \hline
      \multicolumn{9}{|c|}{CIFAR-10} \\
      \hline
      CKN-GM & 2 &  $1 \times 1$  &  12  & 2 &  $2 \times 2$ & 800 &  $4 \times 4$ & $38\,400$\\
      \hline
      CKN-PM & 2 &  $2 \times 2$  &  100  & 2 &  $2 \times 2$ &  800 &  $4 \times 4$ & $321\,200$\\
      \hline
      \hline
      \multicolumn{9}{|c|}{STL-10} \\
      \hline
      CKN-GM & 2 &  $1 \times 1$  &  12  & 2 &  $3 \times 3$ & 800 &  $4 \times 4$ & $86\,400$\\
      \hline
      CKN-PM & 2 &  $3 \times 3$  &  50  & 2 &  $3 \times 3$ &  800 &  $3 \times 3$ & $361\,350$\\
      \hline

   \end{tabular}
   \caption{List of architectures reported in the paper. $N$ is the number of layers; $p_1$ and~$p_2$ represent the number of filters are each layer; $m_1$ and~$m_2$ represent the size of the patches~$\NN_1$ and~$\NN_2$ that are of size~$m_1 \times m_1$ and~$m_2 \times m_2$ on their respective feature maps~$\zeta_1$ and~$\zeta_2$; $\gamma_1$ is the subsampling factor between layer 1 and layer 2; $S$ is the size of the output feature map, and the last column indicates the number of parameters that the network has to learn.}
   \label{table:arch}
\end{table}


%% !TEX root = ../multi_task.tex

We evaluate the presented MTL method on a number of problems. First, we use MultiMNIST \citep{multi_mnist}, an MTL adaptation of MNIST \citep{mnist}. Next, we tackle multi-label classification on the CelebA dataset \citep{celeba} by considering each label as a distinct binary classification task. These problems include both classification and regression, with the number of tasks ranging from 2 to 40. Finally, we experiment with scene understanding, jointly tackling the tasks of semantic segmentation, instance segmentation, and depth estimation on the Cityscapes dataset \citep{cityscapes}. We discuss each experiment separately in the following subsections.

The baselines we consider are (i) \textbf{uniform scaling:} minimizing a uniformly weighted sum of loss functions \mbox{$\frac{1}{T}\sum_t \lL^t$}, \mbox{(ii) \textbf{single task:}} solving tasks independently, \mbox{(iii) \textbf{grid search:}} exhaustively trying various values from $\{ c^t \in [0,1] | \sum_t c^t = 1\}$ and optimizing for $\frac{1}{T}\sum_t c^t \lL^t$, \mbox{(iv) \textbf{\citet{Kendall2018}:}} using the uncertainty weighting proposed by \citet{Kendall2018}, and \mbox{(v) \textbf{GradNorm:}} using the normalization proposed by \citet{Chen2018}.



\subsection{MultiMNIST}
\label{sec:multi_mnist_exp}

Our initial experiments are on MultiMNIST, an MTL version of the MNIST dataset \citep{multi_mnist}. In order to convert digit classification into a multi-task problem, \citet{multi_mnist} overlaid multiple images together. We use a similar construction. For each image, a different one is chosen uniformly in random. Then one of these images is put at the top-left and the other one is at the bottom-right. The resulting tasks are: classifying the digit on the top-left (task-L) and classifying the digit on the bottom-right (task-R). We use 60K examples and directly apply existing single-task MNIST models. The MultiMNIST dataset is illustrated in the supplement.

We use the LeNet architecture \citep{mnist}. We treat all layers except the last as the representation function $g$ and put two fully-connected layers as task-specific functions (see the supplement for details). We visualize the performance profile as a scatter plot of accuracies on task-L and task-R in Figure~\ref{fig:multi_mnist_performance_curve}, and list the results in Table~\ref{tab:multi_mnist}.

In this setup, any static scaling results in lower accuracy than solving each task separately (the single-task baseline). The two tasks appear to compete for model capacity, since increase in the accuracy of one task results in decrease in the accuracy of the other. Uncertainty weighting \citep{Kendall2018} and GradNorm \citep{Chen2018} find solutions that are slightly better than grid search but distinctly worse than the single-task baseline. In contrast, our method finds a solution that efficiently utilizes the model capacity and yields accuracies that are as good as the single-task solutions. This experiment demonstrates the effectiveness of our method as well as the necessity of treating MTL as multi-objective optimization. Even after a large hyper-parameter search, \emph{any} scaling of tasks does not approach the effectiveness of our method.



\subsection{Multi-Label Classification}

\begin{figure}[t]
\includegraphics[width=\textwidth]{radar_full_new}
\vspace{1mm}
\caption{Radar charts of percentage error per attribute on CelebA \citep{celeba}. Lower is better. We divide attributes into two sets for legibility: easy on the left, hard on the right. Zoom in for details.}
\label{fig:multi_label_radar}
\end{figure}


\begin{wraptable}{r}{0.3\textwidth}
%\vspace{-4mm}
\captionof{table}{Mean of error per category of MTL algorithms in multi-label classification on CelebA \citep{celeba}.}
\begin{tabular}{r@{\hspace{2mm}}c@{}}
\toprule
& Average  \\
&  error \\
\midrule
Single task & $8.77$ \\
Uniform scaling & $9.62$ \\
\citealt{Kendall2018} & $9.53$ \\
GradNorm & $8.44$ \\
Ours & $\mathbf{8.25}$  \\
\bottomrule
\end{tabular}
\label{table:multi_label_bar}
%\vspace{-5mm}
\end{wraptable}

Next, we tackle multi-label classification. Given a set of attributes, multi-label classification calls for deciding whether each attribute holds for the input. We use the CelebA dataset \citep{celeba}, which includes 200K face images annotated with 40 attributes. Each attribute gives rise to a binary classification task and we cast this as a 40-way MTL problem. We use ResNet-18 \citep{resnet} without the final layer as a shared representation function, and attach a linear layer for each attribute (see the supplement for further details).


We plot the resulting error for each binary classification task as a radar chart in Figure~\ref{fig:multi_label_radar}. The average over them is listed in Table~\ref{table:multi_label_bar}. We skip grid search since it is not feasible over 40 tasks. Although uniform scaling is the norm in the multi-label classification literature, single-task performance is significantly better. Our method outperforms baselines for significant majority of tasks and achieves comparable performance in rest. This experiment also shows that our method remains effective when the number of tasks is high.


\subsection{Scene Understanding}

To evaluate our method in a more realistic setting, we use scene understanding. Given an RGB image, we solve three tasks: semantic segmentation (assigning pixel-level class labels), instance segmentation (assigning pixel-level instance labels), and monocular depth estimation (estimating continuous disparity per pixel). We follow the experimental procedure of \citet{Kendall2018} and use an encoder-decoder architecture. The encoder is based on ResNet-50 \citep{resnet} and is shared by all three tasks. The decoders are task-specific and are based on the pyramid pooling module \citep{pspnet} (see the supplement for further implementation details).

Since the output space of instance segmentation is unconstrained (the number of instances is not known in advance), we use a proxy problem as in \citet{Kendall2018}. For each pixel, we estimate the location of the center of mass of the instance that encompasses the pixel. These center votes can then be clustered to extract the instances. In our experiments, we directly report the MSE in the proxy task. Figure~\ref{fig:cityscapes_performance_profile} shows the performance profile for each pair of tasks, although we perform all experiments on all three tasks jointly. The pairwise performance profiles shown in Figure~\ref{fig:cityscapes_performance_profile} are simply 2D projections of the three-dimensional profile, presented this way for legibility. The results are also listed in Table~\ref{tab:cityscapes_results}.

MTL outperforms single-task accuracy, indicating that the tasks cooperate and help each other. Our method outperforms all baselines on all tasks.


\subsection{Role of the Approximation}

In order to understand the role of the approximation proposed in Section~\ref{sec:approximation}, we compare the final performance and training time of our algorithm with and without the presented approximation in Table~\ref{tab:approximation_tradeoff} (runtime measured on a single Titan Xp GPU). For a small number of tasks (3 for scene understanding), training time is reduced by 40\%. For the multi-label classification experiment (40 tasks), the presented approximation accelerates learning by a factor of 25.

On the accuracy side, we expect both methods to perform similarly as long as the full-rank assumption is satisfied. As expected, the accuracy of both methods is very similar. Somewhat surprisingly, our approximation results in slightly improved accuracy in all experiments. While counter-intuitive at first, we hypothesize that this is related to the use of SGD in the learning algorithm. Stability analysis in convex optimization suggests that if gradients are computed with an error $\hat{\nabla}_\btheta \mathcal{L}^t = \nabla_\btheta \mathcal{L}^t + \mathbf{e}^t$ ($\btheta$ corresponds to $\btheta^{sh}$ in (\ref{eq:kkt_opt})), as opposed to $\mathbf{Z}$ in the approximate problem in \ref{eq:approx}, the error in the solution is bounded as $\|\hat{\mathbf{\alpha}} - \mathbf{\alpha} \|_2 \leq \mathcal{O}(\max_t \|\mathbf{e}^t\|_2)$. Considering the fact that the gradients are computed over the full parameter set (millions of dimensions) for the original problem and over a smaller space for the approximation (batch size times representation which is in the thousands), the dimension of the error vector is significantly higher in the original problem. We expect the $l_2$ norm of such a random vector to depend on the dimension.

In summary, our quantitative analysis of the approximation suggests that (i) the approximation does not cause an accuracy drop and (ii) by solving an equivalent problem in a lower-dimensional space, our method achieves both better computational efficiency and higher stability.

  {\small
  \begin{table}[t]
%  \vspace{-4mm}
  \caption{Effect of the MGDA-UB approximation. We report the final accuracies as well as training times for our method with and without the approximation.}
  %\vspace{1mm}
  \centering
  \begin{tabular}{@{}r@{\hspace{3mm}}c@{\hspace{3mm}}c@{\hspace{2mm}}c@{\hspace{2mm}}c@{}c@{\hspace{5mm}}c@{\hspace{2mm}}c@{}}
  \toprule
  & \multicolumn{4}{c}{Scene understanding (3 tasks)} &  & \multicolumn{2}{c}{Multi-label (40 tasks)}  \\
  \cmidrule(r){2-5} \cmidrule(lr){7-8}
                  & Training & Segmentation & Instance  & Disparity      & & Training & Average \\
                 & time     &  mIoU [\%]       & error [px] & error [px] & & time (hour)      & error \\
  \midrule
  Ours (w/o approx.) & $38.6$ & $66.13$ & $10.28$ & $2.59$ & & $429.9$ & $8.33$ \\
  Ours & $\mathbf{23.3}$ & $\mathbf{66.63}$ & $\mathbf{10.25}$ & $\mathbf{2.54}$  & & $\mathbf{16.1}$ & $\mathbf{8.25}$ \\
  \bottomrule
  \end{tabular}
  %\vspace{-2mm}
  \label{tab:approximation_tradeoff}
  \end{table}}

%\clearpage\section{Summary of possible figures}
Pre-results section
\begin{itemize}
\item Splash image
\item How-it-works pipeline overview, with spatial proximity as convolution and twohead architecture.
\end{itemize}

\noindent Results section 
\begin{itemize}

\item (Table) Accuracy for IID on image clustering, vs baselines - sota
\item (Table) Accuracy for IID on segmentation datasets, vs baselines - sota
\item (Table) semisupervised learning on STL10, vs baselines - sota

\item (Table or image) Variation study for number of output clusters for IID+
\item (Table) Ablation study for IID
\item (Table) Variation study for effect of lambda.
\item *(Image) Cluster mass equalisation speed for varying lambda.
\item *(Table) Per class IOUs - some are harder than others.

\item *(Table or image) runtime for spatial proximity as convolution, vs brute force - e.g.graph for cumulative runtime.

\item *(Image) Example input images with their rendered output predictions, with failure cases, for both IID+ and IID, and compare between them (IID+ better, boundaries). Also, some cases where our preds are better than ground truth (513: 86), (which would not be captured in acc stats, evaluated against ground truth)
\item *(Image) Before-and-after-training projections (of point clouds and/or images)
\item *(Image) Show STL interpolates images belonging to unknown classes between known GT ones
\item (Table) experiment dataset sizes for train, mapping assignment, mapping test, for each dataset.


\item *(Image) Progression of segmentation getting more and more accurate.
\end{itemize}

\section{Points to be included or emphasized in the writing}
\begin{itemize}

\item \textbf{Benefits compared to other methods:}
\begin{enumerate}
\item Simplicity: a single network loss with no moving parts, unlike multi-part losses that need weights.
\item Network directly outputs the clustering function, with no use of k-means at any stage in any form. (Unlike all unsupervised baselines except DAC.)
\item Training from random initialisation i.e. no pre-training (unlike DEC), no warmup stages (unlike Hauesser), no explicit feature normalisation or PCA or whitening (unlike Deepcluster), random data sampling for each batch so no data resampling schemes (unlike DAC). 
\item Use of mutual information means resistance to degenerate solutions by design, with no need for separate measures to prevent cluster masses shrinking to 0. (Unlike Deepcluster)
\item Equipartitioning is robust enough to deal with datasets that have extra unknown classes that are not in test set (STL10). This is further helped by using 2 heads. (Unlike all other baselines, unsupervised or not.)
\end{enumerate}

\item \textbf{Data splits for integrity of the study.} IID+ is the version of IID what produces an overclustering (so the difference is output\_k > gt\_k), so network training proceeds completely unsupervised just like IID, but at evaluation time a many-to-one mapping needs to be found using labels instead, which constitutes a minimal amount of supervision. So we had to be careful the training and test sets had to be separate, unlike for IID, where output\_k = gt\_k so the assignment is just a permutation. (For IID, training set being the same or overlap with testing set is something done by a couple of other works.)

\item \textbf{IID is purely unsupervised (exact clustering) but IID+ (overclustering) sometimes makes more sense.} Only by increasing the expressivity can you hope to distinguish between apples and tennis balls; otherwise unreasonable. (Some supervision is usually needed in practice, as the fundamental problem with unsupervised learning is often detached from the task at hand - indeed the more “unsupervised” the setup, the more this is be true.) Also, in true unsupervised context you don't know the number of classes (like in STL10) and overestimation would make sense.

\item \textbf{Two output heads for IID allows us to have our cake and eat it: attain full comparability with other fully unsupervised baselines, whilst benefiting from the increased expressivity of IID+}. The final output of the network is taken from a gt\_k sized output layer that still benefits from the increased expressiveness in the learned representation that results from having more fine grained outputs on the auxiliary head. This is especially important for STL10 which has unknown classes in the unlabelled segment; we are able to learn an overestimate of the number of potential classes on the auxiliary head, learning from the full dataset, while the official output head picks up on the 10 of interest, being trained on the part of the dataset known to contain only them.

\item Moving the average over different local shifts outside of the computation of MI improves performance \textbf{(so we're not exactly doing eq. 8)}.

\item \textbf{Explain what STL10 is} (as it's not very famous) and that it's specifically designed for semi/unsupervised learning, and nobody's method is robust enough to use the unlabelled segment except us.

\item \textbf{Explain why we do MI between labels not image and label} (conditional entropy and tractability).

\item \textbf{Include 5 line Pytorch code snippet} showing how easy it is to write the image clustering loss. Say the code will be made available. \\ \url{https://gist.github.com/xu-ji/6a0e7e469bcb1986ece2ed916b22cf70}

\item Supp mat: all baseline implementation details, two forms of loss don't produce massively different results but uncollapsed one inches ahead (Coco: 456 463 464 465, Satellite: 511 497), output vector sparsity getting more sparse.

\item and look at notes doc

\end{itemize}


%\clearpageUsing CNNs that have been pre-trained in image classification task as blackbox
feature extractor and learning linear classifiers on these features has been
reported to yield remarkable results in different domains
\cite{donahue2013decaf}. Moreover a number of authors
\cite{girshick2014rich,chatfield2014bmvc} have later reported that fine-tuning
the CNN weights on the target labels leads to better performance in the target
tasks, particularly when data is scarce. However, most of these methods are
supervised and thus requires annotations to fine-tune networks.
\end{document}

\begin{table}[t]
\footnotesize    
\begin{tabular}{lc}
\toprule
& STL10 \\
\midrule
No auxiliary overclustering & 43.8 \cmt{692}\\
Single sub-head ($h=1$) & 57.6\cmt{693}\\
No sample repeats ($r=1$) & 47.0\cmt{694}\\
Unlabelled data segment ignored & 49.9\cmt{695}\\
\midrule
Full setting & \textbf{59.6} \cmt{569}\\
\bottomrule
\end{tabular}
\caption{\textbf{Ablations of \methodnameshort (unsupervised setting).} Each row shows a single change from the full setting. The full setting has auxiliary overclustering, 5 initialisation heads, 5 sample repeats, and uses the unlabelled data subset of STL10.}
\label{t:iid_imgclus_ablation}
\end{table}

% \begin{table}[h]
% \footnotesize 
% \begin{center}
% \begin{tabular}{lc}
% \toprule
% & STL10 \\
% \midrule
% No auxiliary overclustering & 39.9 \cmt{418}\\
% Single sub-head ($h=1$) & 50.7\cmt{419}\\
% No sample repeats ($r=1$) & 51.4\cmt{420}\\
% Unlabelled data segment ignored & 50.5\cmt{421}\\
% \midrule
% Full setting & \textbf{60.4} \cmt{247}\\
% \bottomrule
% \end{tabular}
% \end{center}
% \vspace{-1em}
% \caption{Ablation for IID. Each row contains one change from the full setting with all else equal. Full setting: with auxiliary overclustering, 5 initialisation heads, 5 sample repeats, unlabelled data segment used.}
% \label{t:iid_imgclus_ablation}
% \end{table}
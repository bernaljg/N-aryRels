\begin{abstract}
We propose coupled generative adversarial network (CoGAN) for learning a joint distribution of multi-domain images. In contrast to the existing approaches, which require tuples of corresponding images in different domains in the training set, CoGAN can learn a joint distribution without any tuple of corresponding images. It can learn a joint distribution with just samples drawn from the marginal distributions. This is achieved by enforcing a weight-sharing constraint that limits the network capacity and favors a joint distribution solution over a product of marginal distributions one. We apply CoGAN to several joint distribution learning tasks, including learning a joint distribution of color and depth images, and learning a joint distribution of face images with different attributes. For each task it successfully learns the joint distribution without any tuple of corresponding images. We also demonstrate its applications to domain adaptation and image transformation.
\end{abstract}

\section{Introduction}

The paper concerns the problem of learning a joint distribution of multi-domain images from data. A joint distribution of multi-domain images is a probability density function that gives a density value to each joint occurrence of images in different domains such as images of the same scene in different modalities (color and depth images) or images of the same face with different attributes (smiling and non-smiling). Once a joint distribution of multi-domain images is learned, it can be used to generate novel tuples of images. In addition to movie and game production, joint image distribution learning finds applications in image transformation and domain adaptation. When training data are given as tuples of corresponding images in different domains, several existing approaches~\cite{srivastava2012multimodal,wang2012semi,ngiam2011multimodal,yang2010image} can be applied. However, building a dataset with tuples of corresponding images is often a challenging task. This correspondence dependency greatly limits the applicability of the existing approaches.

To overcome the limitation, we propose the coupled generative adversarial networks (CoGAN) framework. It can learn a joint distribution of multi-domain images without existence of corresponding images in different domains in the training set. Only a set of images drawn separately from the marginal distributions of the individual domains is required. CoGAN is based on the generative adversarial networks (GAN) framework~\cite{goodfellow2014generative}, which has been established as a viable solution for image distribution learning tasks. CoGAN extends GAN for joint image distribution learning tasks.

CoGAN consists of a tuple of GANs, each for one image domain. When trained naively, the CoGAN learns a product of marginal distributions rather than a joint distribution. We show that by enforcing a weight-sharing constraint the CoGAN can learn a joint distribution without existence of corresponding images in different domains. The CoGAN framework is inspired by the idea that deep neural networks learn a hierarchical feature representation. By enforcing the layers that decode high-level semantics in the GANs to share the weights, it forces the GANs to decode the high-level semantics in the same way. The layers that decode low-level details then map the shared representation to images in individual domains for confusing the respective discriminative models. CoGAN is for multi-image domains but, for ease of presentation, we focused on the case of two image domains in the paper. However, the discussions and analyses can be easily generalized to multiple image domains.

We apply CoGAN to several joint image distribution learning tasks. Through convincing visualization results and quantitative evaluations, we verify its effectiveness. We also show its applications to unsupervised domain adaptation and image transformation.
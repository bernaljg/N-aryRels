





%
%
%
%
%
%
%
%
%
%
%
%
%
%
%
%
%
%
%





%
%
%
%
%
%
%
%
%
%
%
%
%
%
%
%
%
%
%









\section{Experiments}

We evaluate our method on $4$ popular semantic segmentation datasets: PASCAL VOC 2012, NYUDv2, PASCAL-Context and SIFT-flow.
The segmentation performance is measured by the intersection-over-union (IoU) score \cite{everingham2010pascal}, the pixel accuracy and the mean accuracy \cite{LongSD14}.

The first $5$ convolution blocks and the first convolution layer in the $6$th convolution block are initialized from the VGG-16 network \cite{simonyan2014very}.
All remaining layers are randomly initialized. All layers are trained using back-propagation/SGD.
As illustrated in Fig.~\ref{fig:pairwise},
we use $2$ types of pairwise potential functions.
In total, we have 1 type of unary potential function and 2 types of pairwise potential functions.
We formulate one specific FeatMap-Net and potential network (Unary-Net or Pairwise-Net) for one type of potential
function. 
We apply simple data augmentation in the training stage;
specifically, we perform random scaling (from $0.7$ to $1.2$) and flipping of the images for training.
Our system is built on MatConvNet \cite{matconvnet}.






\subsection{Results on NYUDv2}

We first evaluate our method on the dataset NYUDv2 \cite{silberman2012indoor}.
NYUDv2 dataset has 1449 RGB-D images. We use the segmentation labels provided in \cite{gupta2013perceptual} in which labels are processed into $40$ classes.
We use the standard training set which contains $795$ images and the test set which contains $654$ images.
We train our models only on RGB images without using the depth information.

%
%
%
%
%
%

%
%


Results are shown in Table \ref{tab:nyud}.
%
Unless otherwise specified, our models are initialized using the VGG-16 network.
VGG-16 is also used in the competing method FCN \cite{LongSD14}.
Our contextual model with CNN pairwise potentials achieves the best performance, which sets a new state-of-the-art result on the NYUDv2 dataset.
Note that we do not use any depth information in our model.





%
%

\paragraph{Component Evaluation}
We evaluate the performance contribution of different components of the FeatMap-Net for capturing patch-background context on the NYUDv2 dataset.
We present the results of adding different components of FeatMap-Net in Table \ref{tab:featmapnet}.
We start from a baseline setting of our FeatMap-Net (``FullyConvNet Baseline" in the result table), for which multi-scale and sliding pooling is removed.
This baseline setting is the conventional fully convolution network
for segmentation, which can be considered as our implementation of the FCN method in \cite{LongSD14}.
The result shows that our CNN baseline implementation (``FullyConvNet") achieves very similar performance (slightly better) than the FCN method.
Applying multi-scale network design and sliding pyramid pooling significantly improve the performance,
which clearly shows the benefits of encoding rich background context in our approach.
Applying the dense CRF method \cite{krahenbuhl2012efficient} for boundary refinement gains further improvement.
Finally, adding our contextual CNN pairwise potentials brings significant further improvement, for which we achieve the best performance in this dataset.





\begin{table}[t]
\caption{Segmentation results on NYUDv2 dataset (40 classes).
We compare to a number of recent methods.
Our method significantly outperforms the existing methods.}
\centering
\resizebox{1\linewidth}{!}
  {
  \begin{tabular}{ r | c | c c c }
method  &training data  &pixel accuracy &mean accuracy  &IoU\\ \hline \hline
Gupta et al. \cite{gupta2014learning}   &RGB-D  &60.3 &-  &28.6\\
FCN-32s \cite{LongSD14} &RGB  &60.0 &42.2 &29.2\\
FCN-HHA \cite{LongSD14} &RGB-D  &65.4 &46.1 &34.0\\ \hline
%
ours  &RGB  &\bf 70.0 &\bf 53.6 &\bf 40.6\\
%
 \end{tabular}
  }
\label{tab:nyud}
\end{table}



\begin{table}[t]
\caption{Ablation Experiments. The table shows the value added
by the different system components of our method on the NYUDv2 dataset (40 classes).
}
\centering
\resizebox{1\linewidth}{!}
  {
  \begin{tabular}{ r | c c c }
method  &pixel accuracy &mean accuracy  &IoU\\ \hline \hline
FCN-32s \cite{LongSD14} &60.0 &42.2 &29.2\\ \hline
FullyConvNet Baseline  &61.5 &43.2 &30.5\\
$+$ sliding pyramid pooling  &63.5 &45.3 &32.4\\
$+$ multi-scales  &67.0 &50.1 &37.0\\
$+$ boundary refinement &68.5 &50.9 &38.3\\
$+$ CNN contextual pairwise &70.0 &53.6 &40.6\\
 \end{tabular}
  }
\label{tab:featmapnet}
\end{table}









\begin{table*}[t]
\caption{Individual category results on the PASCAL VOC 2012 test set (IoU scores). Our method performs the best}
\centering
\resizebox{1\linewidth}{!}
  {
  \begin{tabular}{ r | c c c c c c c c c c c c c c c c c c c c | c }

method & \rot{aero}  &\rot{bike} &\rot{bird} &\rot{boat} &\rot{bottle}   &\rot{bus}  &\rot{car}  &\rot{cat}  &\rot{chair}    &\rot{cow}  &\rot{table}    &\rot{dog}  &\rot{horse}    &\rot{mbike}    &\rot{person}   &\rot{potted}   &\rot{sheep}    &\rot{sofa} &\rot{train}    &\rot{tv}  & mean \\ \hline \hline
\multicolumn{22}{c}{\bf Only using VOC training data} \\ \hline
FCN-8s \cite{LongSD14}    &76.8   &34.2   &68.9   &49.4   &60.3   &75.3   &74.7   &77.6   &21.4   &62.5   &46.8   &71.8   &63.9   &76.5   &73.9   &45.2   &72.4   &37.4   &70.9 & 55.1 & 62.2 \\
Zoom-out \cite{MostajabiYS14}  &85.6 &37.3 &\bf 83.2 &62.5 &66.0 &85.1 &80.7 &84.9 &27.2 &73.2 &57.5 &78.1 &79.2 &81.1 &77.1 &53.6 &74.0 &49.2 &71.7 &63.3 &69.6\\
DeepLab \cite{ChenPKMY14} &84.4 &54.5 &81.5 &63.6 &65.9 &85.1 &79.1 &83.4 &30.7 &74.1 &59.8 &79.0 &76.1 &83.2 &80.8 &59.7 &82.2 &50.4 &73.1 &63.7 &71.6 \\
CRF-RNN \cite{zheng2015conditional} &87.5 &39.0 &79.7 &64.2 &68.3 &87.6 &80.8 &84.4 &30.4 &78.2 &60.4 &80.5 &77.8 &83.1 &80.6 &59.5 &82.8 &47.8 &78.3 &67.1 &72.0 \\
DeconvNet \cite{noh2015learning} &89.9 &39.3 &79.7 &63.9 &68.2 &87.4 &81.2 &86.1 &28.5 &77.0 &62.0 &79.0 &80.3 &83.6 &80.2 &58.8 &\bf 83.4 &54.3 &80.7 &65.0 &72.5\\
DPN \cite{LiuDPN}  &87.7  &\bf 59.4 &78.4 &64.9 &70.3 &89.3 &83.5 &86.1 &31.7 &79.9 &\bf 62.6 &81.9 &80.0 &83.5 &82.3 &60.5 &83.2 &53.4 &77.9 &65.0 &74.1\\
ours &\bf 90.6  &37.6 & 80.0 &\bf 67.8 &\bf 74.4 &\bf 92.0 &\bf 85.2 &\bf 86.2 &\bf 39.1 &\bf 81.2 & 58.9 &\bf 83.8 &\bf 83.9 &\bf 84.3 &\bf 84.8 &\bf 62.1 & 83.2 &\bf 58.2 &\bf 80.8 &\bf 72.3 & \best 75.3 \\ \hline \hline
\multicolumn{22}{c}{\bf Using VOC+COCO training data} \\ \hline
DeepLab \cite{ChenPKMY14} &89.1 &38.3 &88.1 &63.3 &69.7 &87.1 &83.1 &85.0 &29.3 &76.5 &56.5 &79.8 &77.9 &85.8 &82.4 &57.4 &84.3 &54.9 &80.5 &64.1 &72.7 \\
CRF-RNN \cite{zheng2015conditional} &90.4 &55.3 &88.7 &68.4 &69.8 &88.3 &82.4 &85.1 &32.6 &78.5 &64.4 &79.6 &81.9 &\bf 86.4 &81.8 &58.6 &82.4 &53.5 &77.4 &70.1 &74.7\\
BoxSup \cite{Dai2015arXiv} &89.8  &38.0 &\bf 89.2 &\bf 68.9 &68.0 &89.6 &83.0 &87.7 &34.4 &83.6 &\bf 67.1 &81.5 &83.7 &85.2 &83.5 &58.6 &84.9 &55.8 &\bf 81.2 &70.7 &75.2\\
DPN \cite{LiuDPN} &89.0 & \bf 61.6 &87.7 &66.8 &74.7 &91.2 &\bf 84.3 &87.6 &36.5 & 86.3 &66.1 &84.4 &87.8 &85.6 &85.4 &63.6 &87.3 &61.3 &79.4 &66.4 &77.5 \\
%
ours+ &\bf 94.1	&40.7	&84.1	&67.8	&\bf 75.9	&\bf 93.4	&\bf 84.3	&\bf 88.4	&\bf 42.5	&\bf 86.4	&64.7	&\bf 85.4	&\bf 89.0	& 85.8	&\bf 86.0	&\bf 67.5	&\bf 90.2	&\bf 63.8	&80.9	&\bf 73.0	&\best 78.0 \\

\end{tabular}
  }
\label{tab:voc12_test_details}
\end{table*}















\newcounter{img_idx}







\newcounter{cntone}
\newcommand\settextone[2]{%
  \csdef{textone#1}{#2}}
\newcommand\addtextone[1]{%
  \stepcounter{cntone}%
  \csdef{textone\thecntone}{#1}}
\newcommand\gettextone[1]{%
  \csuse{textone#1}}


\newcounter{cnttwo}
\newcommand\settexttwo[2]{%
  \csdef{texttwo#1}{#2}}
\newcommand\addtexttwo[1]{%
  \stepcounter{cnttwo}%
  \csdef{texttwo\thecnttwo}{#1}}
\newcommand\gettexttwo[1]{%
  \csuse{texttwo#1}}





\addtextone{2007_001311}
\addtextone{2007_001284}
\addtextone{2007_001430}
\addtextone{2008_000149}
\addtextone{2007_007470}
\addtextone{2007_000762}
\addtextone{2008_000533}



\addtexttwo{2010_000666}
\addtexttwo{2007_000830}
\addtexttwo{2007_009346}
\addtexttwo{2009_003666}
\addtexttwo{2007_002624}
\addtexttwo{2010_005860}
\addtexttwo{2008_003333}




\newcounter{img_total_one}
\setcounter{img_total_one}{\arabic{cntone}}
\stepcounter{img_total_one}

\newcounter{img_total_two}
\setcounter{img_total_two}{\arabic{cnttwo}}
\stepcounter{img_total_two}





\begin{figure}[t]
\centering
\resizebox{1\linewidth}{!} {
    \begin{subfigure}{0.7in}
    \centering
    \forloop{img_idx}{1}{\value{img_idx} < \value{img_total_one}}{
        \includegraphics[width=0.7in,height=.8in,keepaspectratio]
        {examples/voc2012_val/img/{\gettextone{\arabic{img_idx}}}.jpg}\vspace{2pt}    
    }
    \vskip -7pt
    \caption{Testing}
    \end{subfigure}\hspace{1pt}
    \begin{subfigure}{0.7in}
    \centering
    \forloop{img_idx}{1}{\value{img_idx} < \value{img_total_one}}{
        \includegraphics[width=0.7in,height=.8in,keepaspectratio]
        {examples/voc2012_val/gt/{\gettextone{\arabic{img_idx}}}.png}\vspace{2pt}    
    }
    \vskip -7pt
    \caption{Truth}
    \end{subfigure}\hspace{1pt}
    \begin{subfigure}{0.7in}
    \centering
    \forloop{img_idx}{1}{\value{img_idx} < \value{img_total_one}}{
        \includegraphics[width=0.7in,height=.8in,keepaspectratio]
        {examples/voc2012_val/predict/{\gettextone{\arabic{img_idx}}}.png}\vspace{2pt}    
    }
    \vskip -7pt
    \caption{Predict}
    \end{subfigure}\hspace{2pt}

    \hfill\vrule\hfill\hspace{2pt}

    \begin{subfigure}{0.7in}
    \centering
    \forloop{img_idx}{1}{\value{img_idx} < \value{img_total_two}}{
        \includegraphics[width=0.7in,height=.8in,keepaspectratio]
        {examples/voc2012_val/img/{\gettexttwo{\arabic{img_idx}}}.jpg}\vspace{2pt}    
    }
    \vskip -7pt
    \caption{Testing}
    \end{subfigure}\hspace{1pt}
    \begin{subfigure}{0.7in}
    \centering
    \forloop{img_idx}{1}{\value{img_idx} < \value{img_total_two}}{
        \includegraphics[width=0.7in,height=.8in,keepaspectratio]
        {examples/voc2012_val/gt/{\gettexttwo{\arabic{img_idx}}}.png}\vspace{2pt}    
     }
     \vskip -7pt
     \caption{Truth}
     \end{subfigure}\hspace{1pt}
     \begin{subfigure}{0.7in}
     \centering
     \forloop{img_idx}{1}{\value{img_idx} < \value{img_total_two}}{
        \includegraphics[width=0.7in,height=.8in,keepaspectratio]
        {examples/voc2012_val/predict/{\gettexttwo{\arabic{img_idx}}}.png}\vspace{2pt}    
     }
     \vskip -7pt
     \caption{Predict}
     \end{subfigure} 
}
\vspace{2pt}
    \caption{Some prediction examples of our method.}
    \label{fig:example_voc}
\end{figure}











































\subsection{Results on PASCAL VOC 2012}

PASCAL VOC 2012 \cite{everingham2010pascal} is a well-known segmentation evaluation dataset which consists of 20 object categories and one background category.
This dataset is split into a training set, a validation set and a test set,
which respectively contain $1464$, $1449$ and $1456$ images.
Following a conventional setting in \cite{BharathECCV2014,ChenPKMY14}, the training set is augmented by extra annotated VOC images provided in \cite{HariharanABMM11}, which results in $10582$ training images.
We verify our performance on the PASCAL VOC 2012 test set.
We compare with a number of recent methods with competitive performance.
Since the ground truth labels are not available for the test set,
we report the result through the VOC evaluation server.


The results of IoU scores are shown in the last column of Table \ref{tab:voc12_test_details}.
We first train our model only using the VOC images.
We achieve $75.3$ 
%
%
IoU score which is the best result amongst methods that only use the VOC training data.


To improve the performance, following the setting in recent work \cite{ChenPKMY14,Dai2015arXiv},
we train our model with the extra images from the COCO dataset \cite{lin2014microsoft}.
With these extra training images, we achieve an IoU score of $77.2$.


For further improvement, we also exploit the the middle-layer features 
as in the recent methods \cite{ChenPKMY14,LongSD14,hariharan2014hypercolumns}.
We learn extra refinement layers on the feature maps from middle layers to refine the coarse prediction.
The feature maps from the middle layers encode lower level visual information which
helps to predict details in the object boundaries.
Specifically, we add $3$ refinement convolution layers 
on top of the feature maps from the first $5$ max-pooling layers and the input image.
The resulting feature maps and the coarse prediction score map
 are then concatenated and go through another $3$ refinement convolution layers to output the refined prediction.
The resolution of the prediction is increased from $1/16$ (coarse prediction) to $1/4$ of the input image.
With this refined prediction, we further perform boundary refinement \cite{krahenbuhl2012efficient} to generate the final prediction.
Finally,  we achieve an IoU score of $78.0$, {\em which is best reported result on this challenging dataset.}
\footnote{The result link at the VOC evaluation server: \url{http://host.robots.ox.ac.uk:8080/anonymous/XTTRFF.html}}




The results for each category are shown in Table \ref{tab:voc12_test_details}.
We outperform competing methods in most categories.
For only using the VOC training set, our method outperforms the second best method, DPN \cite{LiuDPN}, 
on $18$ categories out of $20$.
Using VOC+COCO training set, our method outperforms DPN \cite{LiuDPN}
on $15$ categories out of $20$.
Some prediction examples of our method are shown in Fig. \ref{fig:example_voc}.






\begin{table}[t]
\caption{Segmentation results on PASCAL-Context dataset (60 classes).
Our method performs the best.}
\centering
\resizebox{.8\linewidth}{!}
  {
  \begin{tabular}{ r | c c c}
  method	&pixel accuracy	&mean accuracy	&IoU\\ \hline \hline
O2P \cite{carreira2012semantic}	&-	&-	&18.1\\
CFM \cite{dai2014convolutional}	&-	&-	&34.4\\
FCN-8s \cite{LongSD14}	&65.9	&46.5	&35.1\\
BoxSup \cite{Dai2015arXiv}	&-	&-	&40.5\\ \hline
ours	&\bf 71.5	&\bf 53.9	&\bf 43.3\\
 \end{tabular}
  }
\label{tab:pascalcontext}
\end{table}







\begin{table}[t]
\caption{Segmentation results on SIFT-flow dataset (33 classes).
Our method performs the best.}
\centering
\resizebox{1\linewidth}{!}
  {
  \begin{tabular}{ r | c c c}
  method	&pixel accuracy	&mean accuracy	&IoU\\ \hline \hline
Liu et al. \cite{liu2011sift}	&76.7	&-	&-\\
Tighe et al. \cite{tighe2013finding} 	&75.6	&41.1	&-\\
Tighe et al. (MRF) \cite{tighe2013finding}	&78.6	&39.2	&-\\
Farabet et al. (balance) \cite{farabet2013learning}	&72.3	&50.8	&-\\
Farabet et al. \cite{farabet2013learning} 	&78.5	&29.6	&-\\
Pinheiro et al. \cite{pinheiro2013recurrent} &77.7 &29.8 &-\\
FCN-16s \cite{LongSD14}	&85.2	&51.7	&39.5\\ \hline
ours	&\bf 88.1	&\bf 53.4	&\bf 44.9\\
 \end{tabular}
  }
\label{tab:siftflow}
\end{table}






\subsection{Results on PASCAL-Context}
The
PASCAL-Context \cite{mottaghi2014role} dataset provides the segmentation labels of the whole scene (including the ``stuff" labels) for the PASCAL VOC images.
We use the segmentation labels which contain $60$ classes ($59$ classes plus the `` background" class ) for evaluation.
We use the provided training/test splits.
The training set contains $4998$ images and the test set has $5105$ images.

Results are shown in Table \ref{tab:pascalcontext}. Our method significantly outperforms the competing methods. To our knowledge, {\em ours is the best reported result on this dataset}.




\subsection{Results on SIFT-flow}

We further evaluate our method on the SIFT-flow dataset.
This dataset contains $2688$ images and provide the segmentation labels for $33$ classes.
We use the standard split for training and evaluation.
The training set has $2488$ images and the rest $200$ images are for testing.
Since images are in small sizes, we upscale the image by a factor of $2$ for training.
Results are shown in Table \ref{tab:siftflow}. We achieve the best performance for this dataset.





%
% File acl2016.tex
%
%% Based on the style files for ACL-2015, with some improvements
%%  taken from the NAACL-2016 style
%% Based on the style files for ACL-2014, which were, in turn,
%% Based on the style files for ACL-2013, which were, in turn,
%% Based on the style files for ACL-2012, which were, in turn,
%% based on the style files for ACL-2011, which were, in turn, 
%% based on the style files for ACL-2010, which were, in turn, 
%% based on the style files for ACL-IJCNLP-2009, which were, in turn,
%% based on the style files for EACL-2009 and IJCNLP-2008...

%% Based on the style files for EACL 2006 by 
%%e.agirre@ehu.es or Sergi.Balari@uab.es
%% and that of ACL 08 by Joakim Nivre and Noah Smith

\documentclass[11pt]{article}
\usepackage{acl2016}
\usepackage{times}
\usepackage{url}
\usepackage{latexsym}
\usepackage{graphicx} 
\usepackage{color}
\usepackage[T1]{fontenc}

\newcommand{\ramesh}[1]{\textcolor{red}{#1}}
\newcommand{\bing}[1]{\textcolor{green}{#1}}
\newcommand{\cicero}[1]{\textcolor{blue}{#1}}
\newcommand{\bowen}[1]{\textcolor{cyan}{#1}}

\aclfinalcopy % Uncomment this line for the final submission
%\def\aclpaperid{***} %  Enter the acl Paper ID here

%\setlength\titlebox{5cm}
% You can expand the titlebox if you need extra space
% to show all the authors. Please do not make the titlebox
% smaller than 5cm (the original size); we will check this
% in the camera-ready version and ask you to change it back.

\newcommand\BibTeX{B{\sc ib}\TeX}

\title{Abstractive Text Summarization using Sequence-to-sequence RNNs and Beyond}

 \author{Ramesh Nallapati \\
   IBM Watson \\
   {\tt nallapati@us.ibm.com} \\\And
   Bowen Zhou \\
   IBM Watson \\
   {\tt zhou@us.ibm.com} \\\And
   Cicero  dos Santos \\
   IBM Watson \\
   {\tt cicerons@us.ibm.com} \\\AND
   \c{C}a\u{g}lar G\.{u}l\c{c}ehre \\
   Universit\'e de Montr\'eal \\
   {\tt gulcehrc@iro.umontreal.ca} \\\And
   Bing Xiang \\
   IBM Watson \\
   {\tt bingxia@us.ibm.com} \\
  }
   

% \author{First Author \\
%   Affiliation / Address line 1 \\
%   Affiliation / Address line 2 \\
%   Affiliation / Address line 3 \\
%   {\tt email@domain} \\\And
%   Second Author \\
%   Affiliation / Address line 1 \\
%   Affiliation / Address line 2 \\
%   Affiliation / Address line 3 \\
%   {\tt email@domain} \\}

\date{}

\begin{document}
\maketitle
\begin{abstract}
In this work, we model abstractive text summarization using Attentional Encoder-Decoder Recurrent Neural Networks, and show that they achieve state-of-the-art performance on two different corpora. We propose several novel models that address critical problems in summarization that are not adequately modeled by the basic architecture, such as modeling key-words, capturing the hierarchy of sentence-to-word structure, and emitting words that are rare or unseen at training time. Our work shows that many of our proposed models contribute to further improvement in performance. We also propose a new dataset consisting of multi-sentence summaries, and establish performance benchmarks for further research.
\end{abstract}
  %such as (i) using features such as parts-of-speech and named-entity tags as additional input, (ii) using a hierarchical architecture to model sentences in the source document, and (iii) using the {\it large vocabulary trick} (\cite{lvt}) to speed up the computation.
\section{Introduction}

Humans use different forms of communications such as speech, hand gestures and emotions. Being able to understand one's emotions and the encoded feelings is an important factor for an appropriate and correct understanding.


With the ongoing research in the field of robotics, especially in the field of humanoid robots, it becomes interesting to integrate these capabilities into machines allowing for a more diverse and natural way of communication. One example is the Software called EmotiChat~\cite{Anderson06areal-time}. This is a chat application with emotion recognition. The user is monitored and whenever an emotion is detected (smile, etc.), an emoticon is inserted into the chat window. Besides Human Computer Interaction other fields like surveillance or driver safety could also profit from it. Being able to detect the mood of the driver could help to detect the level of attention, so that automatic systems can adapt.\\
\let\thefootnote\relax\footnote{*F. Trier and P. Burkert contributed equally to this work.}


Many methods rely on extraction of the facial region. This can be realized through manual inference~\cite{4032815} or an automatic detection approach~\cite{Anderson06areal-time}.
Methods often involve the Facial Action Coding System (FACS) which describes the facial expression using Action Units (AU). An Action Unit is a facial action like "raising the Inner Brow". Multiple activations of AUs describe the facial expression~\cite{kumar2009face}. Being able to correctly detect AUs is a helpful step, since it allows making a statement about the activation level of the corresponding emotion. \\
Handcrafted facial landmarks can be used such as done by Kotsia et al.~\cite{4032815}. Detecting such landmarks can be hard, as the distance between them differs depending on the person~\cite{6998925}. Not only AUs can be used to detect emotions, but also texture. When a face shows an emotion the structure changes and different filters can be applied to detect this~\cite{6998925}.\\


\begin{figure}
   \centering
        \includegraphics[width=\columnwidth]{Fig1}
   \caption{Example images from the MMI (top) and CKP (bottom). The emotions from left to right are: \textit{Anger}, \textit{Sadness}, \textit{Disgust}, \textit{Happiness}, \textit{Fear}, \textit{Surprise}. The emotion \textit{Contempt} of the CKP set is not displayed.}\label{fig:example_images}
\end{figure}




The presented approach uses Artificial Neural Networks (ANN). ANNs differ, as they are trained on the data with less need for manual interference. 
Convolutional Neural Networks are a special kind of ANN and have been shown to work well as feature extractor when using images as input~\cite{donahue2013decaf} and are real-time capable. This allows for the usage of the raw input images without any pre- or postprocessing.\\
GoogleNet~\cite{DBLP:journals/corr/SzegedyLJSRAEVR14} is a deep neural network architecture that relies on CNNs. It has been introduced during the Image Net Large Scale Visual Recognition Challenge(ILSVRC) 2014. This challenge analyses the quality of different image classification approaches submitted by different groups. The images are separated into 1000 different classes organized by the WordNet hierarchy. In the challenge "object detection with additional training data" GoogleNet has achieved about 44\% precision~\cite{LSVRC-results}. These results have demonstrated the potential which lies in this kind of architecture. Therefore it has been used as inspiration for the proposed architecture.\\
The proposed network has been evaluated on the Extended Cohn-Kanade Dataset (Section~\ref{sec:ckp}) and on the MMI Dataset (Section~\ref{sec:mmi}). Typical pictures of persons showing emotions can be seen in Fig.~\ref{fig:example_images}.
The emotion \textit{Contempt} of the CKP set is not shown as no subject with consent for publication and an annotated emotion is part of the dataset. Results of experiments on these datasets demonstrate the success of using a deep layered neural network structure. With a 10-fold cross-validation a recognition accuracy of 99.6\% has been achieved. \\

The paper is arranged as follows: After this introduction, Related Work (Section~\ref{sec:related}) is presented which focuses on Emotion/Expression recognition and the various approaches scientists have taken. Next is Section~\ref{sec:background}, Background, which focuses on the main components of the architecture proposed in this article. Section~\ref{sec:datasets} contains a summary of the used Datasets. In Section~\ref{sec:architecture} the architecture is presented. This is followed by the experiments and its results (Section~\ref{sec:experiments}) . Finally, Section~\ref{sec:conclusion} summarizes the article and concludes the article.
\begin{figure*}[t]
    \centering
    \begin{subfigure}[t]{0.5\textwidth}
        \centering
        \includegraphics[height=3.5cm]{Images/O_1_Models.png}
        \caption{\label{fig:o_1_models}Constant time late combining models}
    \end{subfigure}%
    ~ 
    \begin{subfigure}[t]{0.25\textwidth}
        \centering
        \includegraphics[height=3cm]{Images/O_n_Models.png}
        \caption{\label{fig:o_n_models}Linear time early combining models}
    \end{subfigure}%
    ~
    \begin{subfigure}[t]{0.20\textwidth}
        \centering
        \includegraphics[height=2.7cm]{Images/O_n2_Models.png}
        \caption{\label{fig:o_n2_models}Quadratic time early combining model}
    \end{subfigure}%
\end{figure*}


\section{Related work}

\subsection{Part-based models for detection and pose localization}
Previous work has proposed explicit modeling of object part appearances and locations for more accurate recognition and localization.
Starting with pictorial structures~\cite{pedro2000,pictorial},
% which were revisited with HOG features and discriminative training in DPM,
and continuing through poselets~\cite{BourdevMalikICCV09} and related work, many methods have jointly localized a set of geometrically related parts.
% \todo{Poselets, Ramanan, and others have explored models 
The deformable parts model (DPM)~\cite{dpm}, until recently the state-of-the-art PASCAL object detection method, models parts with additional learned filters in positions anchored with respect to the whole object bounding box, allowing parts to be displaced from this anchor with learned deformation costs.
The ``strong'' DPM~\cite{Hossein_ECCV12} adapted this method for the strongly supervised setting in which part locations are annotated at training time.
A limitation of these methods is their use of weak features (usually HOG~\cite{hog}). 

\subsection{Fine-grained categorization}
Recently, a large body of computer vision research has focused on the fine-grained classification problem
in a number of domains, such as animal breeds or species~\cite{BirdletsFarrellICCV11,KhoslaYaoJayadevaprakashFeiFei_FGVC2011,Liu_Dogs_2012,MartinezMunozEtalCVPR2009,ParkhiEtalCVPR12,YaoEtalCVPR11}, plant species~\cite{AngelovaCVPR13,Anelia13,leafsnap,nilsback_visual_2006,nilsback_automated_2008,VantageFramesCVPR13}, and man-made objects~\cite{maji13fine-grained,StarkKrauseBMVC12}. 

Several approaches are based on detecting and extracting features from certain parts of objects. Farrell et al.~\cite{BirdletsFarrellICCV11} proposed a pose-normalized representation using poselets \cite{BourdevMalikICCV09}. Deformable part models \cite{dpm} were used in \cite{ParkhiEtalCVPR12,dpd} for part localization. Based on the work of localizing fiducial landmarks on faces~\cite{Belhumeur_Localizing_2011}, Liu et al.~\cite{Liu_Dogs_2012} proposed an exemplar-based geometric method to detect dog faces and extract highly localized image features from keypoints to differentiate dog breeds. Furthermore, Berg et al.~\cite{poof} learned a set of highly discriminative intermediate features by learning a descriptor for each pair of keypoints. Moreover, in \cite{iccv13_keypoint}, the authors extend the non-parametric exemplar-based method of \cite{Belhumeur_Localizing_2011} by enforcing pose and subcategory consistency. Yao et al.~\cite{YaoEtalCVPR12} and Yang et al.~\cite{UW_NIPS12} have investigated template matching methods to reduce the cost of sliding window approaches. Recent work by G\"{o}ring et al.~\cite{Goering14:NPT} transfers part annotations from objects with similar global shape as non-parametric part detections. 
All these part-based methods, however, require the ground truth bounding box at test time for part localization or keypoint prediction. 

Human-in-the-loop methods \cite{BransonEtalECCV10,Deng_FineCrowd_2013,DuanEtalCVPR12} ask a human to name attributes of the object, click on certain parts or mark the most discriminative regions to improve classification accuracy.
Segmentation-based approaches are also very effective for fine-grained recognition.
Approaches such as \cite{Tricos_Chai_ECCV12,iccv13_alignment,ParkhiEtalICCV11,ParkhiEtalCVPR12,iccv13_partmatching} used region-level cues to infer the foreground segmentation mask and to discard the noisy visual information in the background.
Chai et al.~\cite{iccv13_symbiotic} showed that jointly learning part localization and foreground segmentation together can be beneficial for fine-grained categorization.
Similar to most previous part-based approaches, these efforts require the ground truth bounding box to initialize the segmentation seed.
In contrast, the aim of our work is to perform end-to-end fine-grained categorization with no knowledge at test time of the ground truth bounding box.
Our part detectors use convolutional features on bottom-up region proposals, together with learned non-parametric geometric constraints to more accurately localize object parts, thus enabling strong fine-grained categorization.


% \subsection{Part-based model for detection and pose localization}

%\cite{pictorial}

% \todo{maybe incorporate with the deep learning subsection}

\subsection{Convolutional networks}
In recent years, convolutional neural networks (CNNs) have been incorporated into a number of visual recognition systems in a wide variety of domains.
At least some of the strength of these models lies in their ability to \textit{learn} discriminative features from raw data inputs (e.g., image pixels), in contrast to more traditional object recognition pipelines which compute hand-engineered features on images as an initial preprocessing step.
CNNs were popularized by LeCun and colleagues who initially applied such models to digit recognition~\cite{Lecun89} and OCR~\cite{Lecun98OCR} and later to generic object recognition tasks~\cite{jarrett-iccv2009}.
With the introduction of large labeled image databases~\cite{ILSVRC} and GPU implementations used to efficiently perform the massive parallel computations required for learning and inference in large CNNs,
these networks have become the most accurate method for generic object classification~\cite{krizhevsky}.

Most recently, generic object detection methods have begun to leverage deep CNNs and outperformed any competing approaches based on traditional features.
OverFeat~\cite{overfeat} uses a CNN to regress to object locations in a coarse sliding-window detection framework.
Of particular inspiration to our work is the R-CNN method~\cite{rcnn} which leverages features from a deep CNN in a region proposal framework to achieve unprecedented object detection results on the PASCAL VOC dataset.
Our method generalizes R-CNN by applying it to model object parts in addition to whole objects, which our empirical results will demonstrate is essential for accurate fine-grained recognition.

%The R-CNN pipeline begins by preprocessing images using a region proposal method to generate a relatively small set of candidate windows which may be used as detection.
%The selective search method~\cite{selsearch} generates around 2000 candidate windows per image and was chosen for this piece of the pipeline due to its high recall in practice on the PASCAL VOC~\cite{PASCALVOC} dataset.
%Independent of the region proposal step, the deep CNN  architecture of~\cite{krizhevsky} is trained for a large-scale object recognition task on the 1000-category ILSVRC-2012~\cite{ILSVRC} dataset and then fine-tuned on PASCAL object proposals.
%Given the set of region proposals and trained CNN, R-CNN warps each of these regions to the fixed input size ($227 \times 227$) of the CNN and takes the features from a particular layer of the network as a descriptor representing a global descriptor for the region.
%We use the Caffe GPU CNN implementation~\cite{caffe} both to train the CNN proposed by~\cite{krizhevsky} and for fast feature extraction.
%Finally, a linear SVM is trained on these descriptors for each object category, with descriptors extracted from regions with high ground truth overlap taken as positives for the ground truth category, and regions with low ground truth overlap taken as negatives.

% !TEX root = ../multi_task.tex

We evaluate the presented MTL method on a number of problems. First, we use MultiMNIST \citep{multi_mnist}, an MTL adaptation of MNIST \citep{mnist}. Next, we tackle multi-label classification on the CelebA dataset \citep{celeba} by considering each label as a distinct binary classification task. These problems include both classification and regression, with the number of tasks ranging from 2 to 40. Finally, we experiment with scene understanding, jointly tackling the tasks of semantic segmentation, instance segmentation, and depth estimation on the Cityscapes dataset \citep{cityscapes}. We discuss each experiment separately in the following subsections.

The baselines we consider are (i) \textbf{uniform scaling:} minimizing a uniformly weighted sum of loss functions \mbox{$\frac{1}{T}\sum_t \lL^t$}, \mbox{(ii) \textbf{single task:}} solving tasks independently, \mbox{(iii) \textbf{grid search:}} exhaustively trying various values from $\{ c^t \in [0,1] | \sum_t c^t = 1\}$ and optimizing for $\frac{1}{T}\sum_t c^t \lL^t$, \mbox{(iv) \textbf{\citet{Kendall2018}:}} using the uncertainty weighting proposed by \citet{Kendall2018}, and \mbox{(v) \textbf{GradNorm:}} using the normalization proposed by \citet{Chen2018}.



\subsection{MultiMNIST}
\label{sec:multi_mnist_exp}

Our initial experiments are on MultiMNIST, an MTL version of the MNIST dataset \citep{multi_mnist}. In order to convert digit classification into a multi-task problem, \citet{multi_mnist} overlaid multiple images together. We use a similar construction. For each image, a different one is chosen uniformly in random. Then one of these images is put at the top-left and the other one is at the bottom-right. The resulting tasks are: classifying the digit on the top-left (task-L) and classifying the digit on the bottom-right (task-R). We use 60K examples and directly apply existing single-task MNIST models. The MultiMNIST dataset is illustrated in the supplement.

We use the LeNet architecture \citep{mnist}. We treat all layers except the last as the representation function $g$ and put two fully-connected layers as task-specific functions (see the supplement for details). We visualize the performance profile as a scatter plot of accuracies on task-L and task-R in Figure~\ref{fig:multi_mnist_performance_curve}, and list the results in Table~\ref{tab:multi_mnist}.

In this setup, any static scaling results in lower accuracy than solving each task separately (the single-task baseline). The two tasks appear to compete for model capacity, since increase in the accuracy of one task results in decrease in the accuracy of the other. Uncertainty weighting \citep{Kendall2018} and GradNorm \citep{Chen2018} find solutions that are slightly better than grid search but distinctly worse than the single-task baseline. In contrast, our method finds a solution that efficiently utilizes the model capacity and yields accuracies that are as good as the single-task solutions. This experiment demonstrates the effectiveness of our method as well as the necessity of treating MTL as multi-objective optimization. Even after a large hyper-parameter search, \emph{any} scaling of tasks does not approach the effectiveness of our method.



\subsection{Multi-Label Classification}

\begin{figure}[t]
\includegraphics[width=\textwidth]{radar_full_new}
\vspace{1mm}
\caption{Radar charts of percentage error per attribute on CelebA \citep{celeba}. Lower is better. We divide attributes into two sets for legibility: easy on the left, hard on the right. Zoom in for details.}
\label{fig:multi_label_radar}
\end{figure}


\begin{wraptable}{r}{0.3\textwidth}
%\vspace{-4mm}
\captionof{table}{Mean of error per category of MTL algorithms in multi-label classification on CelebA \citep{celeba}.}
\begin{tabular}{r@{\hspace{2mm}}c@{}}
\toprule
& Average  \\
&  error \\
\midrule
Single task & $8.77$ \\
Uniform scaling & $9.62$ \\
\citealt{Kendall2018} & $9.53$ \\
GradNorm & $8.44$ \\
Ours & $\mathbf{8.25}$  \\
\bottomrule
\end{tabular}
\label{table:multi_label_bar}
%\vspace{-5mm}
\end{wraptable}

Next, we tackle multi-label classification. Given a set of attributes, multi-label classification calls for deciding whether each attribute holds for the input. We use the CelebA dataset \citep{celeba}, which includes 200K face images annotated with 40 attributes. Each attribute gives rise to a binary classification task and we cast this as a 40-way MTL problem. We use ResNet-18 \citep{resnet} without the final layer as a shared representation function, and attach a linear layer for each attribute (see the supplement for further details).


We plot the resulting error for each binary classification task as a radar chart in Figure~\ref{fig:multi_label_radar}. The average over them is listed in Table~\ref{table:multi_label_bar}. We skip grid search since it is not feasible over 40 tasks. Although uniform scaling is the norm in the multi-label classification literature, single-task performance is significantly better. Our method outperforms baselines for significant majority of tasks and achieves comparable performance in rest. This experiment also shows that our method remains effective when the number of tasks is high.


\subsection{Scene Understanding}

To evaluate our method in a more realistic setting, we use scene understanding. Given an RGB image, we solve three tasks: semantic segmentation (assigning pixel-level class labels), instance segmentation (assigning pixel-level instance labels), and monocular depth estimation (estimating continuous disparity per pixel). We follow the experimental procedure of \citet{Kendall2018} and use an encoder-decoder architecture. The encoder is based on ResNet-50 \citep{resnet} and is shared by all three tasks. The decoders are task-specific and are based on the pyramid pooling module \citep{pspnet} (see the supplement for further implementation details).

Since the output space of instance segmentation is unconstrained (the number of instances is not known in advance), we use a proxy problem as in \citet{Kendall2018}. For each pixel, we estimate the location of the center of mass of the instance that encompasses the pixel. These center votes can then be clustered to extract the instances. In our experiments, we directly report the MSE in the proxy task. Figure~\ref{fig:cityscapes_performance_profile} shows the performance profile for each pair of tasks, although we perform all experiments on all three tasks jointly. The pairwise performance profiles shown in Figure~\ref{fig:cityscapes_performance_profile} are simply 2D projections of the three-dimensional profile, presented this way for legibility. The results are also listed in Table~\ref{tab:cityscapes_results}.

MTL outperforms single-task accuracy, indicating that the tasks cooperate and help each other. Our method outperforms all baselines on all tasks.


\subsection{Role of the Approximation}

In order to understand the role of the approximation proposed in Section~\ref{sec:approximation}, we compare the final performance and training time of our algorithm with and without the presented approximation in Table~\ref{tab:approximation_tradeoff} (runtime measured on a single Titan Xp GPU). For a small number of tasks (3 for scene understanding), training time is reduced by 40\%. For the multi-label classification experiment (40 tasks), the presented approximation accelerates learning by a factor of 25.

On the accuracy side, we expect both methods to perform similarly as long as the full-rank assumption is satisfied. As expected, the accuracy of both methods is very similar. Somewhat surprisingly, our approximation results in slightly improved accuracy in all experiments. While counter-intuitive at first, we hypothesize that this is related to the use of SGD in the learning algorithm. Stability analysis in convex optimization suggests that if gradients are computed with an error $\hat{\nabla}_\btheta \mathcal{L}^t = \nabla_\btheta \mathcal{L}^t + \mathbf{e}^t$ ($\btheta$ corresponds to $\btheta^{sh}$ in (\ref{eq:kkt_opt})), as opposed to $\mathbf{Z}$ in the approximate problem in \ref{eq:approx}, the error in the solution is bounded as $\|\hat{\mathbf{\alpha}} - \mathbf{\alpha} \|_2 \leq \mathcal{O}(\max_t \|\mathbf{e}^t\|_2)$. Considering the fact that the gradients are computed over the full parameter set (millions of dimensions) for the original problem and over a smaller space for the approximation (batch size times representation which is in the thousands), the dimension of the error vector is significantly higher in the original problem. We expect the $l_2$ norm of such a random vector to depend on the dimension.

In summary, our quantitative analysis of the approximation suggests that (i) the approximation does not cause an accuracy drop and (ii) by solving an equivalent problem in a lower-dimensional space, our method achieves both better computational efficiency and higher stability.

  {\small
  \begin{table}[t]
%  \vspace{-4mm}
  \caption{Effect of the MGDA-UB approximation. We report the final accuracies as well as training times for our method with and without the approximation.}
  %\vspace{1mm}
  \centering
  \begin{tabular}{@{}r@{\hspace{3mm}}c@{\hspace{3mm}}c@{\hspace{2mm}}c@{\hspace{2mm}}c@{}c@{\hspace{5mm}}c@{\hspace{2mm}}c@{}}
  \toprule
  & \multicolumn{4}{c}{Scene understanding (3 tasks)} &  & \multicolumn{2}{c}{Multi-label (40 tasks)}  \\
  \cmidrule(r){2-5} \cmidrule(lr){7-8}
                  & Training & Segmentation & Instance  & Disparity      & & Training & Average \\
                 & time     &  mIoU [\%]       & error [px] & error [px] & & time (hour)      & error \\
  \midrule
  Ours (w/o approx.) & $38.6$ & $66.13$ & $10.28$ & $2.59$ & & $429.9$ & $8.33$ \\
  Ours & $\mathbf{23.3}$ & $\mathbf{66.63}$ & $\mathbf{10.25}$ & $\mathbf{2.54}$  & & $\mathbf{16.1}$ & $\mathbf{8.25}$ \\
  \bottomrule
  \end{tabular}
  %\vspace{-2mm}
  \label{tab:approximation_tradeoff}
  \end{table}}

\section{VQA Dataset Analysis}
\label{sec:analysis}
%\vspace{\sectionReduceBot}
%%%%%%%%%%%%%%%%%%%%%%%%%%%%%%%%%%%%%%%%%%%%%%%%%%%%%%%%%%%
%%%%%%%%%%%%%%%%%%%%%%%%%%%%%%%%%%%%%%%%%%%%%%%%%%%%%%%%%%%
\begin{figure*}[t]
\centering
\includegraphics[width=1\linewidth]{figures/QuestionTypes3.pdf}
\caption{Distribution of questions by their first four words for a random sample of 60K questions for real images (left) and all questions for abstract scenes (right). The ordering of the words starts towards the center and radiates outwards. The arc length is proportional to the number of questions containing the word. White areas are words with contributions too small to show. }
%\vspace{-5pt}
\label{fig:QuesCluster}
%\setlength{\belowcaptionskip}{-10pt}
\end{figure*}
%%%%%%%%%%%%%%%%%%%%%%%%%%%%%%%%%%%%%%%%%%%%%%%%%%%%%%%%%%%

In this section, we provide an analysis of the questions and answers in the VQA train dataset.
To gain an understanding of the types of questions asked and answers provided, we visualize
the distribution of question types and answers. We also explore how often the questions may
be answered without the image using just commonsense information. Finally, we analyze whether
the information contained in an image caption is sufficient to answer the questions.

The dataset includes 614,163 questions 
%and a total of 
and 7,984,119 answers (including answers provided by workers with and without 
looking at the image) 
%and without looking at the image) 
for 204,721 images from the MS COCO dataset~\cite{coco} and 150,000 questions with 1,950,000 answers for $50,000$ abstract scenes.

%\textcolor{red}{
%We emphasize that the creation of a dataset of this scale and richness
%is a time consuming process, taking months to complete.
%While the entirety of the dataset has been collected,} at the time of original submission,
%120,520 questions with 270,210 answers for 50,000 MS COCO
%images and 30,000 questions with 79,740 answers for 10,000 abstract scenes had been collected.
%Please refer to the appendix for further details.
%\textcolor{red}{The results in this section still reflect that subset of the final dataset.}
%We emphasize that the creation of a dataset of this scale and richness
%is a time consuming process, taking months to complete.
%By our current estimates,
%approximately 5,000 questions and 40,000 answers are collected per day
%using Amazon Mechanical Turk (AMT).
%The entire dataset will take approximately three months to complete. At the time of submission,
%120,520 questions with 270,210 answers for 50,000 MS COCO
%images and 30,000 questions with 79,740 answers for 10,000 abstract scenes had been collected.
%Please refer to the appendix for further details.


%%%%%%%%%%%%%%%%%%%%%%%%%%%%%%%%%%%%%%%%%%%%%%%%%%%%%%%%%%%
%\vspace{\subsectionReduceTop}
\subsection{Questions}
%\vspace{\subsectionReduceBot}
%%%%%%%%%%%%%%%%%%%%%%%%%%%%%%%%%%%%%%%%%%%%%%%%%%%%%%%%%%%

\textbf{Types of Question.}
Given the structure of questions generated in the English language,
we can cluster questions into different types based on the words that start the question.
\figref{fig:QuesCluster} shows the distribution of questions based on the first four
words of the questions for both the real images (left) and abstract scenes (right).
Interestingly, the distribution of questions is quite similar for both real images and abstract scenes.
This helps demonstrate that the type of questions elicited by the abstract scenes is similar to
those elicited by the real images. There exists a surprising variety of question types,
including ``What is$\ldots$'', ``Is there$\ldots$'', ``How many$\ldots$'', and ``Does the$\ldots$''.
Quantitatively, the percentage of questions for different types is shown in \tableref{tab:typeacc}. Several example questions and answers are shown in \figref{fig:qualResults}.
%\textbf{Sub-Types.}
A particularly interesting type of question is the ``What is$\ldots$'' questions, since they have a
diverse set of possible answers. See the appendix for visualizations for ``What is$\ldots$'' questions.

\textbf{Lengths.}
\figref{fig:QuesLen} shows the distribution of question lengths.
We see that most questions range from four to ten words.


\begin{comment}\begin{table}[h]
{\small
\begin{tabular}{@{\extracolsep{\fill}}p{2cm}|ccccc@{\extracolsep{\fill}}}
%\toprule
Dataset  & Yes & No\\
%\midrule
Real   & 18.21 & 14.06 \\
Abstract & 26.54 & 16.70 \\
\end{tabular}
}
\vspace{5pt}
\caption{Percentage of ``yes'' and ``no'' questions in the real and abstract datasets.}
\label{table:yesno}
%\vspace{\captionReduceBot}
\end{table}
\end{comment}

%%%%%%%%%%%%%%%%%%%%%%%%%%%%%%%%%%%%%%%%%%%%%%%%%%%%%%%%%%%
\begin{figure}[t]
\centering
\includegraphics[width=1\linewidth]{figures/Lengths.pdf}
%\vspace{-9pt}
\caption{Percentage of questions with different word lengths for real images and abstract scenes.}
%\vspace{-5pt}
\label{fig:QuesLen}
%\setlength{\belowcaptionskip}{-10pt}
\end{figure}
%%%%%%%%%%%%%%%%%%%%%%%%%%%%%%%%%%%%%%%%%%%%%%%%%%%%%%%%%%%




\begin{figure*}
\centering
\includegraphics[width=1\linewidth]{figures/answers.pdf}
%\vspace{-5pt}
\caption{Distribution of answers per question type for a random sample of 60K questions for real images when subjects provide answers when given the image (top) and when not given the image (bottom).}
%\vspace{-5pt}
\label{fig:AnsPerQues}
%\setlength{\belowcaptionskip}{-10pt}
\end{figure*}


%%%%%%%%%%%%%%%%%%%%%%%%%%%%%%%%%%%%%%%%%%%%%%%%%%%%%%%%%%%
%\vspace{\subsectionReduceTop}
\subsection{Answers}
%\vspace{\subsectionReduceBot}
%%%%%%%%%%%%%%%%%%%%%%%%%%%%%%%%%%%%%%%%%%%%%%%%%%%%%%%%%%%

%\textbf{Typical Answers for Different Question Types.}
\textbf{Typical Answers.}
%Next, we analyze the answers provided for different question types.
\figref{fig:AnsPerQues} (top) shows the distribution of answers for several question types.
We can see that a number of question types, such as ``Is the\ldots'', ``Are\ldots'', and ``Does\ldots'' are
typically answered using ``yes'' and ``no'' as answers.
%\textcolor{red}{Question types such as ``How many\ldots'' are answered using numbers. $12.31\%$ and $14.48\%$ of the questions are answered using numbers on real images and abstract scenes, respectively.}
Other questions such as ``What is\ldots'' and ``What type\ldots'' have a rich diversity
of responses. Other question types such as ``What color\ldots'' or ``Which\ldots'' have more specialized responses,
such as colors, or ``left'' and ``right''. 
See the appendix for a list of the most popular answers.

\textbf{Lengths.}
Most answers consist of a single word, with the distribution of answers containing one, two, or three words, respectively being $89.32\%$, $6.91\%$, and $2.74\%$ for real images and $90.51\%$, $5.89\%$, and $2.49\%$ for abstract scenes.
%$89.16\%$, $7.00\%$, and $2.77\%$ of answers containing one, two, or three words, respectively.
The brevity of answers is not surprising, since the questions tend to elicit specific
information from the images. This is in contrast with image captions that generically
describe the entire image and hence tend to be longer. The brevity of our answers makes
automatic evaluation feasible. While it may be tempting to believe the brevity of the answers
makes the problem easier, recall that they are human-provided open-ended answers to
open-ended questions. The questions typically require complex reasoning to arrive at these
deceptively simple answers (see \figref{fig:qualResults}).
There are currently 23,234 unique one-word answers in our dataset for real images and 3,770 for abstract scenes.
%There are currently 10,011 unique one-word answers in our dataset.

\textbf{`Yes/No' and `Number' Answers.}
Many questions are answered using either ``yes'' or ``no'' (or sometimes ``maybe'') -- 
$38.37\%$ and $40.66\%$ of the questions on real images and abstract scenes respectively. 
Among these `yes/no' questions, there is a bias towards %answering with 
``yes'' -- %with ``yes'' being preferred %$61.32\%$ and $58.46\%$ 
$58.83\%$ and $55.86\%$ of `yes/no' answers are ``yes'' for real images and abstract scenes. 
Question types such as ``How many\ldots'' are answered using numbers -- 
$12.31\%$ and $14.48\%$ of the questions on real images and abstract scenes are `number' questions. 
``2'' is the most popular answer among the `number' questions, making up 
$26.04\%$ of the `number' answers for real images and $39.85\%$ for abstract scenes. 

\textbf{Subject Confidence.}
When the subjects answered the questions, we asked
``Do you think you were able to answer the question correctly?''.
\figref{fig:ConfScores} shows the distribution of responses. A majority of the answers
were labeled as confident for both real images and abstract scenes. % respectively.

\textbf{Inter-human Agreement.}
Does the self-judgment of confidence correspond to the answer agreement between subjects?
\figref{fig:ConfScores} shows the percentage of questions in which 
(i) $7$ or more, 
(ii) $3-7$, or 
(iii) less than $3$ subjects agree on the answers given their average confidence score 
(0 = not confident, 1 = confident).
As expected, the agreement between subjects increases with confidence.
However, even if all of the subjects are confident the answers may still vary.
This is not surprising since some answers may vary, yet have very similar meaning, such as ``happy'' and ``joyful''.

\begin{figure}[t]
\centering
\includegraphics[width=1\linewidth]{figures/Confidence.pdf}
%\vspace{-5pt}
\caption{Number of questions per average confidence score (0 = not confident, 1 = confident) for real images and abstract scenes (black lines). Percentage of questions where 7 or more answers are same, 3-7 are same, less than 3 are same (color bars). }
%\vspace{-7pt}
\label{fig:ConfScores}
%\setlength{\belowcaptionskip}{-10pt}
\end{figure}

As shown in \tableref{table:commonsense_acc} (Question + Image), there is significant inter-human
agreement in the answers for both real images ($83.30\%$) and abstract scenes ($87.49\%$). 
%when humans are provided both the question and image while answering the question.
Note that on average each question has $2.70$ unique answers for real images and $2.39$ for abstract scenes. 
The agreement is significantly higher ($>95\%$) for \quotes{yes/no} questions and lower for other questions ($<76\%$), possibly due to the fact that we perform exact string matching and do not account for synonyms, plurality, \etc. Note that the automatic determination of synonyms is a difficult problem, since the level of answer granularity can vary across questions.




%%%%%%%%%%%%%%%%%%%%%%%%%%%%%%%%%%%%%%%%%%%%%%%%%%%%%%%%%%%
%\vspace{\subsectionReduceTop}
\subsection{Commonsense Knowledge}
\label{sec:cs}
%\vspace{\subsectionReduceBot}
%%%%%%%%%%%%%%%%%%%%%%%%%%%%%%%%%%%%%%%%%%%%%%%%%%%%%%%%%%%
\begin{figure*}[t]
 \includegraphics[width=\linewidth]{figures/age.pdf}
 \centering
\caption{\small Example questions judged by Mturk workers to be answerable by different age groups. The percentage of questions falling into each age group is shown in parentheses.}
 \label{fig:age}
 \end{figure*}
 	
\textbf{Is the Image Necessary?}
%Can the questions be answered using commonsense knowledge alone without the need for an image,
%\eg, ``What is the color of the sheep?''?
Clearly, some questions can sometimes be
answered correctly using commonsense knowledge alone without the need for an image,
\eg, ``What is the color of the fire hydrant?''.
We explore this issue by asking three subjects to answer
the questions \emph{without seeing the image} (see the examples in blue in \figref{fig:qualResults}).
In \tableref{table:commonsense_acc} (Question), we show the percentage of questions for which
the correct answer is provided over all questions, ``yes/no'' questions, and the other questions that
are not ``yes/no''. For ``yes/no'' questions, the human subjects respond better than chance.
For other questions, humans are only correct about $21\%$ of the time. This demonstrates that
understanding the visual information is critical to VQA and that commonsense information alone is not sufficient.

To show the qualitative difference in answers provided with and without images,
we show the distribution of answers for various question types in \figref{fig:AnsPerQues} (bottom).
The distribution of colors, numbers, and even ``yes/no'' responses is surprisingly different for answers
with and without images.
 
\textbf{Which Questions Require Common Sense?}
In order to identify questions that require commonsense reasoning to answer, we conducted 
two AMT studies (on a subset 10K questions from the real images of VQA trainval) asking subjects --
\begin{compactenum} 
\item Whether or not they believed a question required commonsense to answer the question, and 
\item The youngest age group that they believe a person must be in order to be able to correctly answer the question -- 
toddler (3-4), 
younger child (5-8), 
older child (9-12), 
teenager (13-17), 
adult (18+).
\end{compactenum}
Each question was shown to 10 subjects. We found that 
for $47.43\%$ of questions 3 or more subjects voted `yes' to commonsense, 
($18.14\%$: 6 or more).  
In the `perceived human age required to answer question' study, we found the following distribution of responses: 
toddler: $15.3\%$,
younger child: $39.7\%$, 
older child: $28.4\%$, 
teenager: $11.2\%$, 
adult: $5.5\%$.
In Figure \ref{fig:age} we show several questions for which a majority of subjects picked the specified age range. Surprisingly the perceived age needed to answer the questions is fairly well distributed across the different age ranges. As expected the questions that were judged answerable by an adult (18+) generally need specialized knowledge, whereas those answerable by a toddler (3-4) are more generic.
 
We measure the degree of commonsense required to answer a question as the percentage of subjects (out of 10) who voted ``yes'' in our ``whether or not a question requires commonsense'' study.
A fine-grained breakdown of average age and average degree of common sense (on a scale of $0-100$) required to answer a question is shown in \tableref{tab:typeacc}. The average age and the average degree of commonsense across all questions is $8.92$ and $31.01\%$ respectively. 

%\arxiv{To compute average age and average degree of commonsense across questions, we first compute the average age and average degree of commonsense (binary response scaled to $0-100$) per question (by taking average across 10 subjects for each question) and then take average across questions.} 

It is important to distinguish between:
\begin{compactenum}
\item How old someone needs to be to be able to answer a question correctly,  and
\item How old people \emph{think} someone needs to be to be able to answer a question correctly. 
\end{compactenum}

Our age annotations capture the latter -- perceptions of MTurk workers in an uncontrolled environment. As such, the relative ordering of question types in \tableref{tab:typeacc} is more important than absolute age numbers.
%The relative ordering of question types is more important than the absolute age numbers. It is important to note that the age annotations we have collected are just perceived ages: how old people -- untrained MTurk workers in an uncontrolled environment -- \emph{think} someone needs to be to be able to answer a question correctly.}
The two rankings of questions in terms of common sense required according to the two studies 
were largely correlated (Pearson's rank correlation: 0.58). 

%%%%%%%%%%%%%%%%%%%%%%%%%%%%%%%%%%%%%%%%%%%%%%%%%%%%%%%%%%%
\begin{table}[t]
\setlength{\tabcolsep}{3.2pt}
{\small
\begin{center}
%\begin{tabular}{@{}llccc@{}}
%\toprule
%Dataset & Input & All & Yes/No & Other \\
%%\hline
%\midrule
%    & Question & 40.81 & 67.60 & 21.22 \\
%Real   & Question + Caption* & 57.47 & 78.97 & 44.41 \\
%    & Question + Image & 83.30 & 95.77 & 72.67 \\
%%\hline
%\midrule
% & Question & 43.27 & 66.65 &  23.66 \\
%Abstract & Question + Caption* & 54.34 & 74.70 & 40.18 \\
% & Question + Image & 87.49 & 95.96 & 75.33 \\
%\bottomrule
%\end{tabular}
\begin{tabular}{@{}llcccc@{}}
\toprule
Dataset & Input & All & Yes/No & Number & Other \\
%\hline
\midrule
    & Question & 40.81 & 67.60 & 25.77 & 21.22 \\
Real   & Question + Caption* & 57.47 & 78.97 & 39.68 & 44.41 \\
    & Question + Image & 83.30 & 95.77 & 83.39 & 72.67 \\
%\hline
\midrule
 & Question & 43.27 & 66.65 & 28.52 & 23.66 \\
Abstract & Question + Caption* & 54.34 & 74.70 & 41.19 & 40.18 \\
 & Question + Image & 87.49 & 95.96 & 95.04 & 75.33 \\
\bottomrule
\end{tabular}
\end{center}
}
%\vspace{-7pt}
\caption {Test-standard accuracy of human subjects when asked to answer the 
question without seeing the image (Question), 
seeing just a caption of the image and not the image itself (Question + Caption), 
and seeing the image (Question + Image). 
Results are shown for all questions, ``yes/no'' \& ``number'' questions, and other questions 
that are neither answered ``yes/no'' nor number. 
All answers are free-form and not multiple-choice. 
*\hspace{1pt}These accuracies are evaluated on a subset of 3K train questions (1K images).}
% \textcolor{red}{and are not directly comparable to the corresponding numbers in older version.}}
\label{table:commonsense_acc}
%\vspace{\captionReduceBot}
%\vspace{-5pt}
\end{table}
%%%%%%%%%%%%%%%%%%%%%%%%%%%%%%%%%%%%%%%%%%%%%%%%%%%%%%%%%%%


%%%%%%%%%%%%%%%%%%%%%%%%%%%%%%%%%%%%%%%%%%%%%%%%%%%%%%%%%%%
%\vspace{\subsectionReduceTop}
\subsection{Captions \textbf{\vs} Questions}
%\vspace{\subsectionReduceBot}
%%%%%%%%%%%%%%%%%%%%%%%%%%%%%%%%%%%%%%%%%%%%%%%%%%%%%%%%%%%


Do generic image captions provide enough information to answer the questions?
\tableref{table:commonsense_acc} (Question + Caption) shows the percentage of questions answered
correctly when human subjects are given the question and a human-provided caption
describing the image, but not the image. As expected, the results are better than when humans are shown the questions alone.
However, the accuracies are significantly lower than when subjects are shown the actual image.
This demonstrates that in order to answer the questions correctly, deeper image understanding 
(beyond what image captions typically capture) is necessary. In fact, we find that the distributions of nouns, verbs, and adjectives mentioned in captions is statistically significantly different from those mentioned in our questions + answers (Kolmogorov-Smirnov test, $p<.001$) for both real images and abstract scenes. See the appendix for details. 
%This motivates the VQA task as a way to learn further information about visual scenes.

\section{Conclusion}\label{sec:conclusion}
%\vspace{-.1in}
In this work, we apply the attentional encoder-decoder for the task of abstractive summarization with very promising results, outperforming state-of-the-art results significantly on two different datasets. Each of our proposed novel models addresses a specific problem in abstractive summarization, yielding further improvement in performance. We also propose a new dataset for multi-sentence summarization and establish benchmark numbers on it. As part of our future work, we plan to focus our efforts on this data and build more robust models for summaries consisting of multiple sentences.


%Our results strongly demonstrate that sequence-to-sequence models are extremely promising for summarization. Some of the other lessons we learned from our experiments are: (i) the LVT-trick is very useful for summarization as it improves training speed while not sacrificing performance; (ii) traditional methods such as vocabulary expansion and syntax-based features can boost performance of deep learning based models as well. As part of our ongoing work, we are investigating on ways to effectively generate rare words in the summary, which appears to be a glaring weakness in the existing models.  


%\section*{Acknowledgments}
%Suppressed for review.
%{\small 
%We thank Kyunghyun Cho for his easy-to-understand neural machine translation code and Alexander Rush for providing us %with the test data and for helping us make as fair a comparison as possible with their system.
%}
%\pagebreak
% include your own bib file like this:
%\bibliographystyle{acl}
%\bibliography{acl2016}
\bibliography{summarization}
\bibliographystyle{acl2016}

\end{document}

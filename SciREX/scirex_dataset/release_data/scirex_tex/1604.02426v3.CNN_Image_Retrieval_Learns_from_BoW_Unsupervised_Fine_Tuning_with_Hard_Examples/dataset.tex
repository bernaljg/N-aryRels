\section{Training dataset}
\label{sec:dataset}
%
In this section we briefly summarize the tightly-coupled BoW and SfM reconstruction system~\cite{SRCF15,RSJFCM16} that is employed to automatically select our training data. 
Then, we describe how we exploit the 3D information to select harder matching pairs and hard non-matching pairs with larger variability. 

\subsection{BoW and 3D reconstruction}
%
The retrieval engine used in the work of Schonberger \etal~\cite{SRCF15} builds upon BoW with fast spatial verification~\cite{PCISZ07}. 
It uses Hessian affine local features~\cite{MTSZMSKG05}, \mbox{RootSIFT} descriptors~\cite{AZ12}, and a fine vocabulary of 16M visual words~\cite{MPCM13}.
Then, query images are chosen via min-hash and spatial verification, as in~\cite{CM10a}. 
Image retrieval based on BoW is used to collect images of the objects/landmarks.
These images serve as the initial matching graph for the succeeding SfM reconstruction, which is performed using state-of-the-art SfM~\cite{FGGJR10,AFSS+11}. Different mining techniques, \eg zoom in, zoom out~\cite{MCM13,MRCM14}, sideways crawl~\cite{SRCF15}, help to build larger and complete model. 

In this work, we exploit the outcome of such a system. 
Given a large unannotated image collection, images are clustered and a 3D model is constructed per cluster.
We use the terms \emph{3D model}, \emph{model} and \emph{cluster} interchangeably.
For each image, the estimated camera position is known, as well as the local features registered on the 3D model. 
We drop redundant (overlapping) 3D models, that might have been constructed from different seeds.
Models reconstructing the same landmark but from different and disjoint viewpoints are considered as non-overlapping.

\subsection{Selection of training image pairs}

A 3D model is described as a bipartite visibility graph $\bG = (\cI \cup \cP,\cE)$~\cite{LSH10}, where images $\cI$ and points $\cP$ are the vertices of the graph. 
Edges of this graph are defined by visibility relations between cameras and points, \ie if a point $p\in \cP$ is visible in an image $i\in \cI$, then there exists an edge $(i,p) \in \cE$. 
The set of points observed by an image $i$ is given by
%
\begin{equation}
\label{equ:observed_points}
\cP(i) = \{ p \in \cP: (i,p) \in \cE \}.
\end{equation} 
%

We create a dataset of tuples $\left(q, m(q), \cN(q)\right)$, where $q$ represents a query image, $m(q)$ is a positive image that matches the query, and $\cN(q)$ is a set of negative images that do not match the query.
These tuples are used to form training image pairs, where each tuple corresponds to $|\cN(q)|+1$ pairs. 
For a query image $q$, a pool $\cM(q)$ of candidate positive images is constructed based on the camera positions in the cluster of $q$.
It consists of the $k$ images with closest camera centers to the query.
Due to the wide range of camera orientations, these do not necessarily depict the same object. 
We therefore propose three different ways to sample the positive image.
The positives examples are fixed during the whole training process for all three strategies.
%

\paragraph{Positive images: MAC distance.} 
The image that has the lowest MAC distance to the query is chosen as positive, formally
%
\begin{equation}
m_1(q) = \argmin_{i \in \cM(q)} ||\mac(q)-\mac(i)||.
\label{equ:mac_pos}
\end{equation} 
%
This strategy is similar to the one followed by Arandjelovic \etal~\cite{AGTPS15}. 
They adopt this choice since only GPS coordinates are available and not camera orientations.
Downside of this approach is that the chosen matching examples already have low distance, thus not forcing network to learn much out of the positive samples.


\paragraph{Positive images: maximum inliers.} 
%
In this approach, the 3D information is exploited to choose the positive image, independently of the CNN descriptor. In particular, the image that has the highest number of co-observed 3D points with the query is chosen.
That is, 
%
\begin{equation}
\label{equ:ninl_pos}
m_2(q) = \argmax_{i \in \cM(q)} |\cP(q) \cap \cP(i)|.
\end{equation} 
%
This measure corresponds to the number of spatially verified features between two images, a measure commonly used for ranking in BoW-based retrieval. As this choice is independent of the CNN representation, it delivers more challenging positive examples.
%

%
\begin{figure}[t]
% \vspace{10pt}
\centering

\def\imheight{1.1}
\def\qnumone{1021}
\def\qnumtwo{1441}
\def\qnumthree{2461}
\def\qnumfour{6511}
\def\qnumfive{3481}
\def\qnumsix{3991}
\def\qnumseven{1111}
\def\qnumeight{5311}

\def\vnum{5}
\def\raisenum{0}

\setlength{\fboxsep}{0pt}%
\setlength{\fboxrule}{2pt}%

\setlength\tabcolsep{1.5mm}

\begin{tabular}{lclc}
% \hline
\fcolorbox{green}{black}{\includegraphics[height=\imheight cm]{fig/positives/q\qnumone.jpg}} &
\includegraphics[height=\imheight cm]{fig/positives/q\qnumone_p1.jpg} 
\includegraphics[height=\imheight cm]{fig/positives/q\qnumone_p2.jpg} 
\includegraphics[height=\imheight cm]{fig/positives/q\qnumone_p3.jpg}
% \hline
&
\fcolorbox{green}{black}{\includegraphics[height=\imheight cm]{fig/positives/q\qnumtwo.jpg}} &
\includegraphics[height=\imheight cm]{fig/positives/q\qnumtwo_p1.jpg} 
\includegraphics[height=\imheight cm]{fig/positives/q\qnumtwo_p2.jpg} 
\includegraphics[height=\imheight cm]{fig/positives/q\qnumtwo_p3.jpg}
\\
\fcolorbox{green}{black}{\includegraphics[height=\imheight cm]{fig/positives/q\qnumthree.jpg}} & 
\includegraphics[height=\imheight cm]{fig/positives/q\qnumthree_p1.jpg} 
\includegraphics[height=\imheight cm]{fig/positives/q\qnumthree_p2.jpg} 
\includegraphics[height=\imheight cm]{fig/positives/q\qnumthree_p3.jpg}
&
\fcolorbox{green}{black}{\includegraphics[height=\imheight cm]{fig/positives/q\qnumfour.jpg}} & 
\includegraphics[height=\imheight cm]{fig/positives/q\qnumfour_p1.jpg} 
\includegraphics[height=\imheight cm]{fig/positives/q\qnumfour_p2.jpg} 
\includegraphics[height=\imheight cm]{fig/positives/q\qnumfour_p3.jpg}
\\
\fcolorbox{green}{black}{\includegraphics[height=\imheight cm]{fig/positives/q\qnumfive.jpg}} & 
\includegraphics[height=\imheight cm]{fig/positives/q\qnumfive_p1.jpg} 
\includegraphics[height=\imheight cm]{fig/positives/q\qnumfive_p2.jpg} 
\includegraphics[height=\imheight cm]{fig/positives/q\qnumfive_p3.jpg}
&
\fcolorbox{green}{black}{\includegraphics[height=\imheight cm]{fig/positives/q\qnumsix.jpg}} & 
\includegraphics[height=\imheight cm]{fig/positives/q\qnumsix_p1.jpg} 
\includegraphics[height=\imheight cm]{fig/positives/q\qnumsix_p2.jpg} 
\includegraphics[height=\imheight cm]{fig/positives/q\qnumsix_p3.jpg}
\\
\fcolorbox{green}{black}{\includegraphics[height=\imheight cm]{fig/positives/q\qnumseven.jpg}} & 
\includegraphics[height=\imheight cm]{fig/positives/q\qnumseven_p1.jpg} 
\includegraphics[height=\imheight cm]{fig/positives/q\qnumseven_p2.jpg} 
\includegraphics[height=\imheight cm]{fig/positives/q\qnumseven_p3.jpg}
&
\fcolorbox{green}{black}{\includegraphics[height=\imheight cm]{fig/positives/q\qnumeight.jpg}} & 
\includegraphics[height=\imheight cm]{fig/positives/q\qnumeight_p1.jpg} 
\includegraphics[height=\imheight cm]{fig/positives/q\qnumeight_p2.jpg} 
\includegraphics[height=\imheight cm]{fig/positives/q\qnumeight_p3.jpg}
\\

\end{tabular}
\caption{Examples of training query images (green border) and matching images selected as positive examples by methods (from left to right) $m_1(q)$, $m_2(q)$, and $m_3(q)$.
\label{fig:positives}}
\end{figure}

%

\paragraph{Positive images: relaxed inliers.}
%
Even though both previous methods choose positive images depicting the same object as the query, the variance of viewpoints is limited.
Instead of using a pool of images with similar camera position, the positive example is selected at random from a set of images that co-observe enough points with the query, but do not exhibit too extreme scale change. 
The positive example in this case is  
\begin{equation}
\label{equ:relaxed_pos}
m_3(q) = \texttt{random}\left\{ i \in \cM(q): \frac{|\cP(i) \cap \cP(q)|}{|\cP(q)|} \geq t_i,~\texttt{scale}(i,q) \leq t_s \right\},
\end{equation} 
%
where $\texttt{scale}(i,q)$ is the scale change between the two images.
This method results in selecting harder matching examples which are still guaranteed to depict the same object. Method $m_3$ chooses different image than $m_1$ on 86.5\% of the queries.
In Figure~\ref{fig:positives} we present examples of query images and the corresponding positives selected with the three different methods. The relaxed method increases the variability of viewpoints. 
%

\paragraph{Negative images.} 
%
Negative examples are selected from clusters different than the cluster of the query image, as the clusters are non-overlaping. 
Following a well-known procedure, we choose hard negatives~\cite{STFKM14,GDDM14}, that is, non-matching images with the most similar descriptor. Two different strategies are proposed. In the first, $\cN_1(q)$, k-nearest neighbors from all non-matching images are selected. In the other, $\cN_2(q)$, the same criterion is used, but at most one image per cluster is allowed. While $\cN_1(q)$ often leads to multiple, and very similar, instances of the same object, $\cN_2(q)$ provides higher variability of the negative examples, see Figure~\ref{fig:negatives}. While positives examples are fixed during the whole training process, hard negatives depend on the current CNN parameters and are re-mined multiple times per epoch. 

%
\setlength{\fboxsep}{0pt}%
\setlength{\fboxrule}{2pt}%

\begin{figure}[t]
% \vspace{10pt}
\centering

\def\imheight{1cm}
\def\qnumone{253}
\def\qnumtwo{1611}
\def\qnumthree{1609}

\hspace{-10pt}

\setlength\tabcolsep{2mm}
% \def\arraystretch{1.2}

\begin{tabular}{cccc}

% \hline

\fcolorbox{green}{black}{\includegraphics[height=\imheight]{fig/negatives/q\qnumone.jpg}} &
\includegraphics[height=\imheight]{fig/negatives/q\qnumone_m2_n1.jpg} &
\foreach \neg in {2,3}  { 
	\includegraphics[height=\imheight]{fig/negatives/q\qnumone_m2_n\neg.jpg}
} $\ldots$& 
\foreach \neg in {4,5}  { 
	\includegraphics[height=\imheight]{fig/negatives/q\qnumone_m1_n\neg.jpg}
}$\ldots$\\

% \hline

\fcolorbox{green}{black}{\includegraphics[height=\imheight]{fig/negatives/q\qnumtwo.jpg}} &
\includegraphics[height=\imheight]{fig/negatives/q\qnumtwo_m2_n1.jpg} &
\foreach \neg in {3,4}  { 
	\includegraphics[height=\imheight]{fig/negatives/q\qnumtwo_m2_n\neg.jpg}
}$\ldots$ & 
\foreach \neg in {2,5}  { 
	\includegraphics[height=\imheight]{fig/negatives/q\qnumtwo_m1_n\neg.jpg}
}$\ldots$\\

% \hline

\fcolorbox{green}{black}{\includegraphics[height=\imheight]{fig/negatives/q\qnumthree.jpg}} &
\includegraphics[height=\imheight]{fig/negatives/q\qnumthree_m2_n1.jpg} &
\foreach \neg in {2,3,4}  { 
	\includegraphics[height=\imheight]{fig/negatives/q\qnumthree_m2_n\neg.jpg}
}$\ldots$ & 
\foreach \neg in {2,3,4}  { 
	\includegraphics[height=\imheight]{fig/negatives/q\qnumthree_m1_n\neg.jpg}
}$\ldots$\\

$q$ & $n(q)$ & $\cN_1(q)$ & $\cN_2(q)$ \\

% \hline

\end{tabular}\\[-.7\baselineskip]
\vspace{5pt}
\caption{Examples of training query images $q$ (green border), hardest non-matching images $n(q)$ that are always selected as negative examples, and additional non-matching images selected as negative examples by $\cN_1(q)$ and $\cN_2(q)$ methods respectively.
\label{fig:negatives}
\vspace{10pt}}
\end{figure}

%
% last updated in April 2002 by Antje Endemann
% Based on CVPR 07 and LNCS, with modifications by DAF, AZ and elle, 2008 and AA, 2010, and CC, 2011; TT, 2014; AAS, 2016

%pdflatex -shell-escape paper.tex

\documentclass[runningheads]{llncs}
\usepackage{graphicx}
\usepackage{amsmath,amssymb} % define this before the line numbering.
\usepackage{dsfont}
% \usepackage{ruler} % comment out for camera-ready
\usepackage{color}
\usepackage{lettrine}
\usepackage[width=122mm,left=12mm,paperwidth=146mm,height=193mm,top=12mm,paperheight=217mm]{geometry}
\usepackage{tikz}
\usepackage{multirow}

\usepackage{xspace}
\usepackage{bbm} 
\usepackage{fixltx2e}
\usepackage{contour}
\usepackage{colortbl}

% If you comment hyperref and then uncomment it, you should delete
% egpaper.aux before re-running latex.  (Or just hit 'q' on the first latex
% run, let it finish, and you should be clear).
\usepackage[pagebackref=true,breaklinks=true,colorlinks,bookmarks=false]{hyperref} % for arXiv
% \usepackage{hyperref}

\DeclareMathOperator*{\argmin}{\arg\!\min}
\DeclareMathOperator*{\argmax}{\arg\!\max}
\DeclareMathOperator*{\knn}{\mathit{kNN}}

% \usepackage{pgfplots}
\usepackage{pgfplotstable}
\usepackage{pgf,tikz}


%------------------------------------------------------------------------------
\usepgfplotslibrary{external}
\tikzexternalize
\newcommand{\extfig}[2]{\tikzsetnextfilename{fig/extern/#1}{#2}}
\newcommand{\extdata}[1]{\input{#1}}
%------------------------------------------------------------------------------

\newcommand{\leg}[1]{\addlegendentry{#1}}

\usepackage{tikzextern}
%\usepackage{pgffor}
\newcommand{\extfig}[2]{\tikzsetnextfilename{fig/extern/#1}{#2}}
\newcommand{\extdata}[1]{}

%\newcommand{\extfig}[2]{!}
\newcommand{\extdata}[1]{}
 

\begin{document}
% \renewcommand\thelinenumber{\color[rgb]{0.2,0.5,0.8}\normalfont\sffamily\scriptsize\arabic{linenumber}\color[rgb]{0,0,0}}
% \renewcommand\makeLineNumber {\hss\thelinenumber\ \hspace{6mm} \rlap{\hskip\textwidth\ \hspace{6.5mm}\thelinenumber}}
% \linenumbers
\pagestyle{headings}
\mainmatter
\def\ECCV16SubNumber{825}  % Insert your submission number here


\title{CNN Image Retrieval Learns from BoW:\\ Unsupervised Fine-Tuning with Hard Examples}

% \titlerunning{ECCV-16 submission ID \ECCV16SubNumber}
\titlerunning{CNN Image Retrieval Learns from BoW} % for camera-ready

% \authorrunning{ECCV-16 submission ID \ECCV16SubNumber}
\authorrunning{F. Radenovi{\'c}, G. Tolias, and O. Chum} % for camera ready

% \author{Anonymous ECCV submission}
\newcommand{\namespace}{\hspace{5mm}} \author{Filip Radenovi{\'c} \namespace Giorgos Tolias \namespace Ond{\v r}ej Chum} % for camera ready

% \institute{Paper ID \ECCV16SubNumber}
\institute{CMP, Faculty of Electrical Engineering, Czech Technical University in Prague \\ \email{ \{filip.radenovic,giorgos.tolias,chum\}@cmp.felk.cvut.cz}} % for camera ready

\maketitle

\def\ie{\emph{i.e.}\xspace}
\def\eg{\emph{e.g.}\xspace}
\def\wrt{\emph{w.r.t.}\xspace}
\def\etal{\emph{et al.}\xspace}

\definecolor{darkred}{rgb}{0.8,0,0}

\newcommand{\real}{\mathbb{R}}
\newcommand{\realnn}{{\mathbb{R}^{+}_{0}}}
\newcommand{\nat}{\mathbb{N}}
\newcommand{\natzero}{{\mathbb{N}_{0}}}

\newcommand{\loss}{\mathcal{L}}

\newcommand{\cX}{\mathcal{X}}
\newcommand{\cI}{\mathcal{I}}
\newcommand{\cP}{\mathcal{P}}
\newcommand{\cE}{\mathcal{E}}
\newcommand{\cN}{\mathcal{N}}
\newcommand{\cM}{\mathcal{M}}

\newcommand{\bG}{\mathbb{G}}
\newcommand{\cG}{\boldsymbol{\mathcal{G}}}


\newcommand{\vf}{\mathbf{f}}
\newcommand{\f}{\mathrm{f}}
\newcommand{\mac}{\bar{\vf}}

\def\l2{$\ell_2$}

\xspaceaddexceptions{+}
\def\cpl2{L\textsubscript{w}\xspace}
\def\pcawhiten{PCA\textsubscript{w}\xspace}

\def\cropI{$\texttt{Crop}_\cI$\xspace}
\def\cropA{$\texttt{Crop}_\cX$\xspace}

\renewcommand{\paragraph}[1]{{\medskip \noindent \bf #1}}
\newcommand{\pari}[1]{{\medskip \noindent \it #1}}
\newcommand{\equ}[1]{Equation~(\ref{#1})\xspace}

\newcommand{\alert}[1]{{\color{red}{#1}}}
\newcommand{\todo}[1]{{\color{blue}{#1}}}

\renewcommand{\b}[1]{\textbf{#1}}
\newcommand{\w}[1]{\color{blue}{#1}}
\newcommand{\ww}[1]{\textbf{\color{blue}{#1}}}

\newcommand{\nb}[1]{\textbf{\color{darkred}{#1}}}
% \renewcommand{\sb}[1]{\textbf{#1}}
% \renewcommand{\sb}[1]{\textbf{\color{black}{\contour{darkred}{#1}}}}
\renewcommand{\sb}[1]{{\color{black}{\contour{darkred}{#1}}}}
\newcommand{\ob}[1]{\textbf{#1}}
\newcommand{\bo}{\cellcolor{gray!15}}

\def\sssp{\hspace{1pt}}
\def\ssp{\hspace{3pt}}
\def\msp{\hspace{5pt}}
\def\bsp{\hspace{8pt}}


\begin{abstract}
%
Convolutional Neural Networks (CNNs) achieve state-of-the-art performance in many computer vision tasks. However, this achievement is preceded by extreme manual annotation in order to perform either training from scratch or fine-tuning for the target task. In this work, we propose to fine-tune CNN for image retrieval from a large collection of unordered images in a fully automated manner.
%
We employ state-of-the-art retrieval and Structure-from-Motion (SfM) methods to obtain 3D models, which are used to guide the selection of the training data for CNN fine-tuning. We show that both hard positive and hard negative examples enhance the final performance in particular object retrieval with compact codes.
%
\keywords{CNN fine-tuning, unsupervised learning, image retrieval}
\end{abstract}

\section{Introduction}
\label{sec:intro}

Language modeling is among the important problems that require modeling long-term dependency, with successful applications such as unsupervised pretraining~\citep{dai2015semi,peters2018deep,radford2018improving,devlin2018bert}.
However, it has been a challenge to equip neural networks with the capability to model long-term dependency in sequential data.
Recurrent neural networks (RNNs), in particular Long Short-Term Memory (LSTM) networks~\citep{hochreiter1997long}, have been a standard solution to language modeling and obtained strong results on multiple benchmarks.
Despite the wide adaption, RNNs are difficult to optimize due to gradient vanishing and explosion~\citep{hochreiter2001gradient}, and the introduction of gating in LSTMs and the gradient clipping technique~\citep{graves2013generating} might not be sufficient to fully address this issue.
% ,pascanu2012understanding
Empirically, previous work has found that LSTM language models use 200 context words on average~\citep{khandelwal2018sharp}, indicating room for further improvement.

On the other hand, the direct connections between long-distance word pairs baked in attention mechanisms might ease optimization and enable the learning of long-term dependency~\citep{bahdanau2014neural,vaswani2017attention}.
Recently, \citet{al2018character} designed a set of auxiliary losses to train deep Transformer networks for character-level language modeling, which outperform LSTMs by a large margin.
Despite the success, the LM training in~\citet{al2018character} is performed on separated fixed-length segments of a few hundred characters, without any information flow across segments.
As a consequence of the fixed context length, the model cannot capture any longer-term dependency beyond the predefined context length.
In addition, the fixed-length segments are created by selecting a consecutive chunk of symbols without respecting the sentence or any other semantic boundary.
Hence, the model lacks necessary contextual information needed to well predict the first few symbols, leading to inefficient optimization and inferior performance.
We refer to this problem as \textit{context fragmentation}.

%However, the context length is fixed to hundreds of characters and thus it is not possible to model longer-term dependency. Moreover, it is not clear how the model performs on word-level language modeling data, as the granularity changes.

% Moreover, using auxiliary losses brings additional challenges such as properly tuning the mixture weights and the loss decay schedule.

To address the aforementioned limitations of fixed-length contexts, we propose a new architecture called Transformer-XL (meaning extra long).
We introduce the notion of recurrence into our deep self-attention network. In particular, instead of computing the hidden states from scratch for each new segment, we reuse the hidden states obtained in previous segments.
The reused hidden states serve as memory for the current segment, which builds up a recurrent connection between the segments.
As a result, modeling very long-term dependency becomes possible because information can be propagated through the recurrent connections.
Meanwhile, passing information from the previous segment can also resolve the problem of context fragmentation.
More importantly, we show the necessity of using relative positional encodings rather than absolute ones, in order to enable state reuse without causing temporal confusion.
Hence, as an additional technical contribution, we introduce a simple but more effective relative positional encoding formulation that generalizes to attention lengths longer than the one observed during training.

Transformer-XL obtained strong results on five datasets, varying from word-level to character-level language modeling.
Transformer-XL is also able to generate relatively coherent long text articles with \textit{thousands of} tokens (see Appendix \ref{sec:gen}), trained on only 100M tokens.
% Transformer-XL improves the previous state-of-the-art (SoTA) results from 1.06 to 0.99 in bpc on enwiki8, from 1.13 to 1.08 in bpc on text8, from 20.5 to 18.3 in perplexity on WikiText-103, and from 23.7 to 21.8 in perplexity on One Billion Word.
% Transformer-XL improves the previous state-of-the-art (SoTA) results to 0.99 in bpc on enwiki8, 1.08 in bpc on text8, 18.3 in perplexity on WikiText-103, and 21.8 in perplexity on One Billion Word.
% On small data, Transformer-XL also achieves a perplexity of 54.5 on Penn Treebank without finetuning, which is SoTA when comparable settings are considered.

Our main technical contributions include introducing the notion of recurrence in a purely self-attentive model and deriving a novel positional encoding scheme. These two techniques form a complete set of solutions, as any one of them alone does not address the issue of fixed-length contexts. Transformer-XL is the first self-attention model that achieves substantially better results than RNNs on both character-level and word-level language modeling.

% On WikiText-103, Transformer-XL improves the previous state-of-the-art (SoTA) results from 33 perplexity to 24, with a relative reduction of 27\%. On enwiki8 character-level language modeling, Transformer-XL achieves a SoTA bpc of 1.03, which outperforms \cite{al2018character} by 0.03 with 60+\% fewer parameters. Given a more common model size with 40+M parameters, Transformer-XL achieves a bpc of 1.06, compared to 1.11 by \cite{al2018character}. Transformer-XL also achieves perplexities of 54.5 on Penn Treebank and 29.4 on One Billion Word, which are SoTA when comparable settings are considered.

% Due to the ability of modeling long-range context, our best model uses attention lengths of 1,600 and 3,800 on WikiText-103 and enwiki8 respectively. We also devise a metric called \textit{Relative Effective Context Length} (RECL) that aims to fairly compare the ability of long-range dependency modeling.
% % perform a fair comparison of the gains brought by increasing the context lengths for different models.
% In this setting, Transformer-XL learns a RECL of 900 words on WikiText-103, while the numbers for recurrent networks and Transformer are only 500 and 128.

% We use two methods to quantitatively study the effective lengths of Transformer-XL and the baselines. Similar to \cite{khandelwal2018sharp}, we gradually increase the attention length at test time until no further noticeable improvement ($\sim$0.1\% relative gains) can be observed. Our best model in this settings use attention lengths of 1,600 and 3,800 on WikiText-103 and enwiki8 respectively.
% %In addition, since the effective context length of Transformer-XL can be longer than the attention length due to our recurrent formulation, we devise a metric called \textit{Relative Effective Context Length} (RECL) that aims to perform a fair comparison of the gains brought by increasing the context lengths for different models.
% In addition, we devise a metric called \textit{Relative Effective Context Length} (RECL) that aims to perform a fair comparison of the gains brought by increasing the context lengths for different models.
% In this setting, Transformer-XL learns a RECL of 900 words on WikiText-103, while the numbers for recurrent networks and Transformer are only 500 and 128.

\paragraph{3D Object Detection from RGB-D Data} Researchers have approached the 3D detection problem by taking various ways to represent RGB-D data.

\emph{Front view image based methods:} ~\cite{chen2016monocular, mousavian20163d, xiang2015data} take monocular RGB images and shape priors or occlusion patterns to infer 3D bounding boxes. ~\cite{li2016vehicle, deng2017amodal} represent depth data as 2D maps and apply CNNs to localize objects in 2D image. In comparison we represent depth as a point cloud and use advanced 3D deep networks (PointNets) that can exploit 3D geometry more effectively.

\emph{Bird's eye view based methods:} MV3D~\cite{cvpr17chen} projects LiDAR point cloud to bird's eye view and trains a region proposal network (RPN~\cite{ren2015faster}) for 3D bounding box proposal. However, the method lags behind in detecting small objects, such as pedestrians and cyclists and cannot easily adapt to scenes with multiple objects in vertical direction.
%Our method shares the idea with~\cite{cvpr17chen} in reducing 3D search cost by 2D search first. What differentiates our method from \cite{cvpr17chen} is that, \hao{???} instead of projecting point cloud to images costing loss in 3D geometry, we directly apply PointNet to point clouds that correspond to the 2D regions. % Besides, our method and MV3D can potentially be combined in the bird's eye setting. 3D proposals from our frustum-based PointNet and MV3D can be combined and our 3D network can also be used for bounding box estimation for point cloud in the bird's eye 2D region.

\emph{3D based methods:} ~\cite{wang2015voting, song2014sliding} train 3D object classifiers by SVMs on hand-designed geometry features extracted from point cloud and then localize objects using sliding-window search. \cite{engelcke2017vote3deep} extends ~\cite{wang2015voting} by replacing SVM with 3D CNN on voxelized 3D grids. \cite{ren2016three} designs new geometric features for 3D object detection in a point cloud. \cite{song2016deep, li20163d} convert a point cloud of the entire scene into a volumetric grid and use 3D volumetric CNN for object proposal and classification. Computation cost for those method is usually quite high due to the expensive cost of 3D convolutions and large 3D search space.
%In comparison, we use 2D region proposals from RGB images to reduce the search space from the entire 3D scenes into 3D frustums. Since the points cloud in the frustums have largely varying depth ranges and can be very sparse, it's not applicable to apply CNN on bird's eye view or apply 3D CNN in grids. Our frustum-based PointNet, on the other hand, suits well for this type of data and is able to accurately estimate 3D bounding box with good efficiency.
Recently, \cite{lahoud20172d} proposes a 2D-driven 3D object detection method that is similar to ours in spirit. However, they use hand-crafted features (based on histogram of point coordinates) with simple fully connected networks to regress 3D box location and pose, which is sub-optimal in both speed and performance. In contrast, we propose a more flexible and effective solution with deep 3D feature learning (PointNets).
%In addition we also get 3D instance segmentation as intermediate outputs. Evaluated on SUN-RGBD we show our method is \emph{8.9\%} better than theirs in mAP and \emph{34x} faster at the same time.


% \begin{enumerate}
%     \item ZOOX~\cite{mousavian20163d} image based
%     \item Vote3Deep~\cite{engelcke2017vote3deep} 3d cnn. Recent LIDAR-based methods place 3D windows in 3D voxel grids to score the point cloud
%     \item Voting for Voting~\cite{wang2015voting} Recent LIDAR-based methods place 3D windows in 3D voxel grids to score the point cloud. apply SVM classifers on 3D grids encoded with geometry features
%     \item MV3D~\cite{cvpr17chen}
%     \item VeloFCN~\cite{li2016vehicle} apply convolutional networks to the front view point map in a dense box prediction scheme
%     \item 3DOP~\cite{chen20153d} image based. reconstructs depth from stereo images and uses an energy minimization approach to generate 3D box proposals, which are fed to an R-CNN [10] pipeline for object recognition
%     \item Mono3D~\cite{chen2016monocular} image based. shares the same pipeline with 3DOP, it generates 3D proposals from monocular images.
%     \item 3DFCN~\cite{li20163d} 3d cnn.
%     \item 3DVP~\cite{xiang2015data} introduces 3D voxel patterns and employ a set of ACF detectors to do 2D detection and 3D pose estimation
%     \item Are Cars just 3D Box?~\cite{zeeshan2014cars} fit model to image patch
%     \item ~\cite{zia2013detailed} fit model to image patch
% \end{enumerate}
% \begin{enumerate}
%     \item SlidingShapes~\cite{song2014sliding} apply SVM classifers on 3D grids encoded with geometry features
%     \item DeepSlidingShapes~\cite{song2015sun} 3d cnn.
%     \item 2D-driven~\cite{lahoud20172d}
%     \item ~\cite{deng2017amodal} rgb-d images
%     \item COG feature~\cite{ren2016three}
%     \item Align 3D model in RGB-D~\cite{gupta2015aligning}
% \end{enumerate}

\paragraph{Deep Learning on Point Clouds}
Most existing works convert point clouds to images or volumetric forms before feature learning. \cite{wu20153d, maturana2015voxnet, qi2016volumetric} voxelize point clouds into volumetric grids and generalize image CNNs to 3D CNNs. ~\cite{li2016fpnn, riegler2016octnet, wang2017cnn, engelcke2017vote3deep} design more efficient 3D CNN or neural network architectures that exploit sparsity in point cloud.
However, these CNN based methods still require quantitization of point clouds with certain voxel resolution.
Recently, a few works~\cite{qi2017pointnet,qi2017pointnetplusplus} propose a novel type of network architectures (PointNets) that directly consumes raw point clouds without converting them to other formats. While PointNets have been applied to single object classification and semantic segmentation, our work explores how to extend the architecture for the purpose of 3D object detection.
\section{Network architecture and image representation}
\label{sec:network}
%
In this section we describe the derived image representation that is based on CNN and we present the network architecture used to perform the end-to-end learning in a siamese fashion.
Finally, we describe how, after fine-tuning, we use the same training data to learn projections that appear to be an effective post-processing step.

\subsection{Image representation}
\vspace{-2pt}
We adopt a compact representation that is derived from activations of convolutional layers and is shown to be effective for particular object retrieval~\cite{ARSM+14,TSJ16}.
We assume that a network is fully convolutional~\cite{PKS15} or that all fully connected layers are discarded.
Now, given an input image, the output is a 3D tensor $\cX$ of $W\times H \times K$ dimensions, where $K$ is the number of feature maps in the last layer. 
Let $\cX_k$ be the set of all $W\times H$ activations for feature map $k \in \{1 \ldots K$\}.
The network output consists of $K$ such sets of activations.
The image representation, called Maximum Activations of Convolutions (MAC)~\cite{RSMC14,TSJ16}, is simply constructed by max-pooling over all dimensions per feature map and is given by
%
\vspace{-2pt}
\begin{equation}
\vspace{-2pt}
\vf = [\f_1 \ldots \f_k \ldots \f_K]^\top \text{,~with~} \f_k = \max_{x\in \cX_{k}}~x \cdot \mathds{1}(x>0).
\label{equ:mac}
\end{equation}
%
\noindent
The indicator function $\mathds{1}$ takes care that the feature vector $\vf$ is non-negative, as if the last network layer was a Rectified Linear Unit (ReLU).
The feature vector finally consists of the maximum activation per feature map and its dimensionality  is equal to $K$.
For many popular networks this is equal to 256 or 512, which makes it a compact image representation.
MAC vectors are subsequently \l2-normalized and similarity between two images is evaluated with inner product. 
The contribution of a feature map to the image similarity is measured by the product of the corresponding MAC vector components. 
In Figure~\ref{fig:mac_matches} we show the image patches in correspondence that contribute most to the similarity. 
Such implicit correspondences are improved after fine-tuning. Moreover, the CNN fires less to ImageNet classes, \eg cars and bicycles. 
%
%%%%%%%%%%%%%%%%%%%%%%%%%%%%%%%%%%%%%%%%%%%%%%%%%%%%%%%%%%%%%%%%%%%%%%%%
\begin{figure}[t]
\centering

\def\queryone{11}
\def\dbimageone{3885}
% \def\queryone{16}
% \def\dbimageone{1201}
% \def\queryone{16}
% \def\dbimageone{1911}
% \def\queryone{21}
% \def\dbimageone{3289}
% \def\queryone{21}
% \def\dbimageone{3303}
\def\querytwo{42}
\def\dbimagetwo{211}

\setlength{\tabcolsep}{0pt}
\hspace{-20pt}
\begin{tabular}{ccc}
\raisebox{8pt}{\includegraphics[height=65pt]{fig/correspondences/q\queryone_db\dbimageone_net0_1} }  &
\includegraphics[height=72pt]{fig/correspondences/q\queryone_db\dbimageone_net0_2}  &
\raisebox{30pt}{
	\begin{tabular}{c}
		\multicolumn{1}{c}{VGG off-the-shelf} \\	
		\foreach \patch in {1,2,3,4,5,6,7,8,9,10}  {
		\includegraphics[height=19pt]{fig/correspondences/q\queryone_db\dbimageone_net0_p\patch_1.png}
		\hspace{-8pt} 
		}\\
		\foreach \patch in {1,2,3,4,5,6,7,8,9,10}  {
		\includegraphics[height=19pt]{fig/correspondences/q\queryone_db\dbimageone_net0_p\patch_2.png} 
		\hspace{-8pt} 
		}\\
	\end{tabular}
}
\\
%
\includegraphics[height=60pt]{fig/correspondences/q\queryone_db\dbimageone_net1_1} & %\hspace{3pt} 
\includegraphics[height=58pt]{fig/correspondences/q\queryone_db\dbimageone_net1_2} ~~~&
\raisebox{30pt}{
	\begin{tabular}{c}
		\multicolumn{1}{c}{VGG ours} \\	
		\foreach \patch in {1,2,3,4,5,6,7,8,9,10}  {
		\includegraphics[height=19pt]{fig/correspondences/q\queryone_db\dbimageone_net1_p\patch_1.png}
		\hspace{-8pt} 
		}\\
		\foreach \patch in {1,2,3,4,5,6,7,8,9,10}  {
		\includegraphics[height=19pt]{fig/correspondences/q\queryone_db\dbimageone_net1_p\patch_2.png} 
		\hspace{-8pt} 
		}\\
	\end{tabular}
}
\\
%
%
\includegraphics[height=70pt]{fig/correspondences/q\querytwo_db\dbimagetwo_net0_1} &
\raisebox{8pt}{\includegraphics[height=70pt]{fig/correspondences/q\querytwo_db\dbimagetwo_net0_2}} &
\raisebox{30pt}{
	\begin{tabular}{c}
		\multicolumn{1}{c}{VGG off-the-shelf} \\	
		\foreach \patch in {1,2,3,4,5,6,7,8,9,10}  {
		\includegraphics[height=19pt]{fig/correspondences/q\querytwo_db\dbimagetwo_net0_p\patch_1.png}
		\hspace{-8pt} 
		}\\
		\foreach \patch in {1,2,3,4,5,6,7,8,9,10}  {
		\includegraphics[height=19pt]{fig/correspondences/q\querytwo_db\dbimagetwo_net0_p\patch_2.png} 
		\hspace{-8pt} 
		}\\
	\end{tabular}
}
\\
\includegraphics[height=66pt]{fig/correspondences/q\querytwo_db\dbimagetwo_net1_1} &
~\includegraphics[height=65pt]{fig/correspondences/q\querytwo_db\dbimagetwo_net1_2} &
\raisebox{30pt}{
	\begin{tabular}{c}
		\multicolumn{1}{c}{VGG ours} \\	
		\foreach \patch in {1,2,3,4,5,6,7,8,9,10}  {
		\includegraphics[height=19pt]{fig/correspondences/q\querytwo_db\dbimagetwo_net1_p\patch_1.png}
		\hspace{-8pt} 
		}\\
		\foreach \patch in {1,2,3,4,5,6,7,8,9,10}  {
		\includegraphics[height=19pt]{fig/correspondences/q\querytwo_db\dbimagetwo_net1_p\patch_2.png} 
		\hspace{-8pt} 
		}\\
	\end{tabular}
}
\\
\end{tabular}
%
\vspace{-10pt}
\caption{Visualization of patches corresponding to the MAC vector components that have the highest contribution to the pairwise image similarity. Examples shown use CNN before (top) and after (bottom) fine-tuning of VGG. The same color corresponds to the same vector component (feature map) per image pair. The patch size is equal to the receptive field of the last pooling layer.
\label{fig:mac_matches}
\vspace{-15pt}}
\end{figure}
%%%%%%%%%%%%%%%%%%%%%%%%%%%%%%%%%%%%%%%%%%%%%%%%%%%%%%%%%%%%%%%%%%%%%%%%	
%
\vspace{-6pt}
\subsection{Network and siamese learning}
%
The proposed approach is applicable to any CNN that consists of only convolutional layers. 
In this paper, we focus on re-training (\ie fine-tuning) state-of-the-art CNNs for classification, in particular AlexNet and VGG. 
Fully connected layers are discarded and the pre-trained networks constitute the initialization for our convolutional layers.
Now, the last convolutional layer is followed by a MAC layer that performs MAC vector computation (\ref{equ:mac}).
The input of a MAC layer is a 3D tensor of activation and the output is a non-negative vector. 
Then, an \l2-normalization block takes care that output vectors are normalized. 
In the rest of the paper, MAC corresponds to the \l2-normalized vector $\mac$.

We adopt a siamese architecture and train a two branch network. 
Each branch is a clone of the other, meaning that they share the same parameters. 
Training input consists of image pairs $(i,j)$ and labels $Y(i,j)\in \{0, 1\}$ declaring whether a pair is non-matching (label 0) or matching (label 1). 
We employ the contrastive loss~\cite{CHL05} that acts on the (non-)matching pairs and is defined as
%
\begin{equation}
\small
\loss(i,j) = \frac{1}{2}\left(Y(i,j) ||\mac(i)-\mac(j)||^2 + \left(1-Y(i,j)\right) \left(\max\{0, \tau - ||\mac(i)-\mac(j)||\}\right)^2\right),
\end{equation}
%
where $\mac(i)$ is the \l2-normalized MAC vector of image $i$, and $\tau$ is a parameter defining when non-matching pairs have large enough distance in order not to be taken into account in the loss.
We train the network using Stochastic Gradient Descent (SGD) and a large training set created automatically (see Section~\ref{sec:dataset}). 
%
\subsection{Whitening and dimensionality reduction}
\label{ref:projections}
\vspace{-5pt}
%
In this section, the post-processing of fine-tuned MAC vectors is considered. 
%
Previous methods~\cite{BL15,TSJ16} use PCA of an independent set for whitening and dimensionality reduction, that is the covariance matrix of all descriptors is analyzed. We propose to take advantage of the labeled data provided by the 3D models and use linear discriminant projections originally proposed by Mikolajczyk and Matas~\cite{MM07}. The projection is decomposed into two parts, whitening and rotation. 
The whitening part is the inverse of the square-root of the intraclass (matching pairs) covariance matrix $C_S^{-\frac{1}{2}}$, where 
\vspace{-6pt}
\begin{equation}
% \small
C_S = \sum_{Y(i,j)=1} \left(\mac(i) - \mac(j)\right)\left(\mac(i) - \mac(j)\right)^\top.
\end{equation}
The rotation part is the PCA of the interclass (non-matching pairs) covariance matrix in the whitened space $\mathrm{eig}(C_S^{-\frac{1}{2}} C_D C_S^{-\frac{1}{2}})$, where 
\vspace{-6pt}
\begin{equation}
% \small
C_D = \sum_{Y(i,j)=0} \left(\mac(i) - \mac(j)\right)\left(\mac(i) - \mac(j)\right)^\top.
\end{equation}
The projection $P = C_S^{-\frac{1}{2}} \mathrm{eig}(C_S^{-\frac{1}{2}} C_D C_S^{-\frac{1}{2}})$ is then applied as $P^\top (\mac(i)-\mu)$, where $\mu$ is the mean MAC vector to perform centering. To reduce the descriptor dimensionality to $D$ dimensions, only eigenvectors corresponding to $D$ largest eigenvalues are used.
Projected vectors are subsequently \l2-normalized.
\section{Training dataset}
\label{sec:dataset}
%
In this section we briefly summarize the tightly-coupled BoW and SfM reconstruction system~\cite{SRCF15,RSJFCM16} that is employed to automatically select our training data. 
Then, we describe how we exploit the 3D information to select harder matching pairs and hard non-matching pairs with larger variability. 

\subsection{BoW and 3D reconstruction}
%
The retrieval engine used in the work of Schonberger \etal~\cite{SRCF15} builds upon BoW with fast spatial verification~\cite{PCISZ07}. 
It uses Hessian affine local features~\cite{MTSZMSKG05}, \mbox{RootSIFT} descriptors~\cite{AZ12}, and a fine vocabulary of 16M visual words~\cite{MPCM13}.
Then, query images are chosen via min-hash and spatial verification, as in~\cite{CM10a}. 
Image retrieval based on BoW is used to collect images of the objects/landmarks.
These images serve as the initial matching graph for the succeeding SfM reconstruction, which is performed using state-of-the-art SfM~\cite{FGGJR10,AFSS+11}. Different mining techniques, \eg zoom in, zoom out~\cite{MCM13,MRCM14}, sideways crawl~\cite{SRCF15}, help to build larger and complete model. 

In this work, we exploit the outcome of such a system. 
Given a large unannotated image collection, images are clustered and a 3D model is constructed per cluster.
We use the terms \emph{3D model}, \emph{model} and \emph{cluster} interchangeably.
For each image, the estimated camera position is known, as well as the local features registered on the 3D model. 
We drop redundant (overlapping) 3D models, that might have been constructed from different seeds.
Models reconstructing the same landmark but from different and disjoint viewpoints are considered as non-overlapping.

\subsection{Selection of training image pairs}

A 3D model is described as a bipartite visibility graph $\bG = (\cI \cup \cP,\cE)$~\cite{LSH10}, where images $\cI$ and points $\cP$ are the vertices of the graph. 
Edges of this graph are defined by visibility relations between cameras and points, \ie if a point $p\in \cP$ is visible in an image $i\in \cI$, then there exists an edge $(i,p) \in \cE$. 
The set of points observed by an image $i$ is given by
%
\begin{equation}
\label{equ:observed_points}
\cP(i) = \{ p \in \cP: (i,p) \in \cE \}.
\end{equation} 
%

We create a dataset of tuples $\left(q, m(q), \cN(q)\right)$, where $q$ represents a query image, $m(q)$ is a positive image that matches the query, and $\cN(q)$ is a set of negative images that do not match the query.
These tuples are used to form training image pairs, where each tuple corresponds to $|\cN(q)|+1$ pairs. 
For a query image $q$, a pool $\cM(q)$ of candidate positive images is constructed based on the camera positions in the cluster of $q$.
It consists of the $k$ images with closest camera centers to the query.
Due to the wide range of camera orientations, these do not necessarily depict the same object. 
We therefore propose three different ways to sample the positive image.
The positives examples are fixed during the whole training process for all three strategies.
%

\paragraph{Positive images: MAC distance.} 
The image that has the lowest MAC distance to the query is chosen as positive, formally
%
\begin{equation}
m_1(q) = \argmin_{i \in \cM(q)} ||\mac(q)-\mac(i)||.
\label{equ:mac_pos}
\end{equation} 
%
This strategy is similar to the one followed by Arandjelovic \etal~\cite{AGTPS15}. 
They adopt this choice since only GPS coordinates are available and not camera orientations.
Downside of this approach is that the chosen matching examples already have low distance, thus not forcing network to learn much out of the positive samples.


\paragraph{Positive images: maximum inliers.} 
%
In this approach, the 3D information is exploited to choose the positive image, independently of the CNN descriptor. In particular, the image that has the highest number of co-observed 3D points with the query is chosen.
That is, 
%
\begin{equation}
\label{equ:ninl_pos}
m_2(q) = \argmax_{i \in \cM(q)} |\cP(q) \cap \cP(i)|.
\end{equation} 
%
This measure corresponds to the number of spatially verified features between two images, a measure commonly used for ranking in BoW-based retrieval. As this choice is independent of the CNN representation, it delivers more challenging positive examples.
%

%
\begin{figure}[t]
% \vspace{10pt}
\centering

\def\imheight{1.1}
\def\qnumone{1021}
\def\qnumtwo{1441}
\def\qnumthree{2461}
\def\qnumfour{6511}
\def\qnumfive{3481}
\def\qnumsix{3991}
\def\qnumseven{1111}
\def\qnumeight{5311}

\def\vnum{5}
\def\raisenum{0}

\setlength{\fboxsep}{0pt}%
\setlength{\fboxrule}{2pt}%

\setlength\tabcolsep{1.5mm}

\begin{tabular}{lclc}
% \hline
\fcolorbox{green}{black}{\includegraphics[height=\imheight cm]{fig/positives/q\qnumone.jpg}} &
\includegraphics[height=\imheight cm]{fig/positives/q\qnumone_p1.jpg} 
\includegraphics[height=\imheight cm]{fig/positives/q\qnumone_p2.jpg} 
\includegraphics[height=\imheight cm]{fig/positives/q\qnumone_p3.jpg}
% \hline
&
\fcolorbox{green}{black}{\includegraphics[height=\imheight cm]{fig/positives/q\qnumtwo.jpg}} &
\includegraphics[height=\imheight cm]{fig/positives/q\qnumtwo_p1.jpg} 
\includegraphics[height=\imheight cm]{fig/positives/q\qnumtwo_p2.jpg} 
\includegraphics[height=\imheight cm]{fig/positives/q\qnumtwo_p3.jpg}
\\
\fcolorbox{green}{black}{\includegraphics[height=\imheight cm]{fig/positives/q\qnumthree.jpg}} & 
\includegraphics[height=\imheight cm]{fig/positives/q\qnumthree_p1.jpg} 
\includegraphics[height=\imheight cm]{fig/positives/q\qnumthree_p2.jpg} 
\includegraphics[height=\imheight cm]{fig/positives/q\qnumthree_p3.jpg}
&
\fcolorbox{green}{black}{\includegraphics[height=\imheight cm]{fig/positives/q\qnumfour.jpg}} & 
\includegraphics[height=\imheight cm]{fig/positives/q\qnumfour_p1.jpg} 
\includegraphics[height=\imheight cm]{fig/positives/q\qnumfour_p2.jpg} 
\includegraphics[height=\imheight cm]{fig/positives/q\qnumfour_p3.jpg}
\\
\fcolorbox{green}{black}{\includegraphics[height=\imheight cm]{fig/positives/q\qnumfive.jpg}} & 
\includegraphics[height=\imheight cm]{fig/positives/q\qnumfive_p1.jpg} 
\includegraphics[height=\imheight cm]{fig/positives/q\qnumfive_p2.jpg} 
\includegraphics[height=\imheight cm]{fig/positives/q\qnumfive_p3.jpg}
&
\fcolorbox{green}{black}{\includegraphics[height=\imheight cm]{fig/positives/q\qnumsix.jpg}} & 
\includegraphics[height=\imheight cm]{fig/positives/q\qnumsix_p1.jpg} 
\includegraphics[height=\imheight cm]{fig/positives/q\qnumsix_p2.jpg} 
\includegraphics[height=\imheight cm]{fig/positives/q\qnumsix_p3.jpg}
\\
\fcolorbox{green}{black}{\includegraphics[height=\imheight cm]{fig/positives/q\qnumseven.jpg}} & 
\includegraphics[height=\imheight cm]{fig/positives/q\qnumseven_p1.jpg} 
\includegraphics[height=\imheight cm]{fig/positives/q\qnumseven_p2.jpg} 
\includegraphics[height=\imheight cm]{fig/positives/q\qnumseven_p3.jpg}
&
\fcolorbox{green}{black}{\includegraphics[height=\imheight cm]{fig/positives/q\qnumeight.jpg}} & 
\includegraphics[height=\imheight cm]{fig/positives/q\qnumeight_p1.jpg} 
\includegraphics[height=\imheight cm]{fig/positives/q\qnumeight_p2.jpg} 
\includegraphics[height=\imheight cm]{fig/positives/q\qnumeight_p3.jpg}
\\

\end{tabular}
\caption{Examples of training query images (green border) and matching images selected as positive examples by methods (from left to right) $m_1(q)$, $m_2(q)$, and $m_3(q)$.
\label{fig:positives}}
\end{figure}

%

\paragraph{Positive images: relaxed inliers.}
%
Even though both previous methods choose positive images depicting the same object as the query, the variance of viewpoints is limited.
Instead of using a pool of images with similar camera position, the positive example is selected at random from a set of images that co-observe enough points with the query, but do not exhibit too extreme scale change. 
The positive example in this case is  
\begin{equation}
\label{equ:relaxed_pos}
m_3(q) = \texttt{random}\left\{ i \in \cM(q): \frac{|\cP(i) \cap \cP(q)|}{|\cP(q)|} \geq t_i,~\texttt{scale}(i,q) \leq t_s \right\},
\end{equation} 
%
where $\texttt{scale}(i,q)$ is the scale change between the two images.
This method results in selecting harder matching examples which are still guaranteed to depict the same object. Method $m_3$ chooses different image than $m_1$ on 86.5\% of the queries.
In Figure~\ref{fig:positives} we present examples of query images and the corresponding positives selected with the three different methods. The relaxed method increases the variability of viewpoints. 
%

\paragraph{Negative images.} 
%
Negative examples are selected from clusters different than the cluster of the query image, as the clusters are non-overlaping. 
Following a well-known procedure, we choose hard negatives~\cite{STFKM14,GDDM14}, that is, non-matching images with the most similar descriptor. Two different strategies are proposed. In the first, $\cN_1(q)$, k-nearest neighbors from all non-matching images are selected. In the other, $\cN_2(q)$, the same criterion is used, but at most one image per cluster is allowed. While $\cN_1(q)$ often leads to multiple, and very similar, instances of the same object, $\cN_2(q)$ provides higher variability of the negative examples, see Figure~\ref{fig:negatives}. While positives examples are fixed during the whole training process, hard negatives depend on the current CNN parameters and are re-mined multiple times per epoch. 

%
\setlength{\fboxsep}{0pt}%
\setlength{\fboxrule}{2pt}%

\begin{figure}[t]
% \vspace{10pt}
\centering

\def\imheight{1cm}
\def\qnumone{253}
\def\qnumtwo{1611}
\def\qnumthree{1609}

\hspace{-10pt}

\setlength\tabcolsep{2mm}
% \def\arraystretch{1.2}

\begin{tabular}{cccc}

% \hline

\fcolorbox{green}{black}{\includegraphics[height=\imheight]{fig/negatives/q\qnumone.jpg}} &
\includegraphics[height=\imheight]{fig/negatives/q\qnumone_m2_n1.jpg} &
\foreach \neg in {2,3}  { 
	\includegraphics[height=\imheight]{fig/negatives/q\qnumone_m2_n\neg.jpg}
} $\ldots$& 
\foreach \neg in {4,5}  { 
	\includegraphics[height=\imheight]{fig/negatives/q\qnumone_m1_n\neg.jpg}
}$\ldots$\\

% \hline

\fcolorbox{green}{black}{\includegraphics[height=\imheight]{fig/negatives/q\qnumtwo.jpg}} &
\includegraphics[height=\imheight]{fig/negatives/q\qnumtwo_m2_n1.jpg} &
\foreach \neg in {3,4}  { 
	\includegraphics[height=\imheight]{fig/negatives/q\qnumtwo_m2_n\neg.jpg}
}$\ldots$ & 
\foreach \neg in {2,5}  { 
	\includegraphics[height=\imheight]{fig/negatives/q\qnumtwo_m1_n\neg.jpg}
}$\ldots$\\

% \hline

\fcolorbox{green}{black}{\includegraphics[height=\imheight]{fig/negatives/q\qnumthree.jpg}} &
\includegraphics[height=\imheight]{fig/negatives/q\qnumthree_m2_n1.jpg} &
\foreach \neg in {2,3,4}  { 
	\includegraphics[height=\imheight]{fig/negatives/q\qnumthree_m2_n\neg.jpg}
}$\ldots$ & 
\foreach \neg in {2,3,4}  { 
	\includegraphics[height=\imheight]{fig/negatives/q\qnumthree_m1_n\neg.jpg}
}$\ldots$\\

$q$ & $n(q)$ & $\cN_1(q)$ & $\cN_2(q)$ \\

% \hline

\end{tabular}\\[-.7\baselineskip]
\vspace{5pt}
\caption{Examples of training query images $q$ (green border), hardest non-matching images $n(q)$ that are always selected as negative examples, and additional non-matching images selected as negative examples by $\cN_1(q)$ and $\cN_2(q)$ methods respectively.
\label{fig:negatives}
\vspace{10pt}}
\end{figure}

%
% !TEX root = ../multi_task.tex

We evaluate the presented MTL method on a number of problems. First, we use MultiMNIST \citep{multi_mnist}, an MTL adaptation of MNIST \citep{mnist}. Next, we tackle multi-label classification on the CelebA dataset \citep{celeba} by considering each label as a distinct binary classification task. These problems include both classification and regression, with the number of tasks ranging from 2 to 40. Finally, we experiment with scene understanding, jointly tackling the tasks of semantic segmentation, instance segmentation, and depth estimation on the Cityscapes dataset \citep{cityscapes}. We discuss each experiment separately in the following subsections.

The baselines we consider are (i) \textbf{uniform scaling:} minimizing a uniformly weighted sum of loss functions \mbox{$\frac{1}{T}\sum_t \lL^t$}, \mbox{(ii) \textbf{single task:}} solving tasks independently, \mbox{(iii) \textbf{grid search:}} exhaustively trying various values from $\{ c^t \in [0,1] | \sum_t c^t = 1\}$ and optimizing for $\frac{1}{T}\sum_t c^t \lL^t$, \mbox{(iv) \textbf{\citet{Kendall2018}:}} using the uncertainty weighting proposed by \citet{Kendall2018}, and \mbox{(v) \textbf{GradNorm:}} using the normalization proposed by \citet{Chen2018}.



\subsection{MultiMNIST}
\label{sec:multi_mnist_exp}

Our initial experiments are on MultiMNIST, an MTL version of the MNIST dataset \citep{multi_mnist}. In order to convert digit classification into a multi-task problem, \citet{multi_mnist} overlaid multiple images together. We use a similar construction. For each image, a different one is chosen uniformly in random. Then one of these images is put at the top-left and the other one is at the bottom-right. The resulting tasks are: classifying the digit on the top-left (task-L) and classifying the digit on the bottom-right (task-R). We use 60K examples and directly apply existing single-task MNIST models. The MultiMNIST dataset is illustrated in the supplement.

We use the LeNet architecture \citep{mnist}. We treat all layers except the last as the representation function $g$ and put two fully-connected layers as task-specific functions (see the supplement for details). We visualize the performance profile as a scatter plot of accuracies on task-L and task-R in Figure~\ref{fig:multi_mnist_performance_curve}, and list the results in Table~\ref{tab:multi_mnist}.

In this setup, any static scaling results in lower accuracy than solving each task separately (the single-task baseline). The two tasks appear to compete for model capacity, since increase in the accuracy of one task results in decrease in the accuracy of the other. Uncertainty weighting \citep{Kendall2018} and GradNorm \citep{Chen2018} find solutions that are slightly better than grid search but distinctly worse than the single-task baseline. In contrast, our method finds a solution that efficiently utilizes the model capacity and yields accuracies that are as good as the single-task solutions. This experiment demonstrates the effectiveness of our method as well as the necessity of treating MTL as multi-objective optimization. Even after a large hyper-parameter search, \emph{any} scaling of tasks does not approach the effectiveness of our method.



\subsection{Multi-Label Classification}

\begin{figure}[t]
\includegraphics[width=\textwidth]{radar_full_new}
\vspace{1mm}
\caption{Radar charts of percentage error per attribute on CelebA \citep{celeba}. Lower is better. We divide attributes into two sets for legibility: easy on the left, hard on the right. Zoom in for details.}
\label{fig:multi_label_radar}
\end{figure}


\begin{wraptable}{r}{0.3\textwidth}
%\vspace{-4mm}
\captionof{table}{Mean of error per category of MTL algorithms in multi-label classification on CelebA \citep{celeba}.}
\begin{tabular}{r@{\hspace{2mm}}c@{}}
\toprule
& Average  \\
&  error \\
\midrule
Single task & $8.77$ \\
Uniform scaling & $9.62$ \\
\citealt{Kendall2018} & $9.53$ \\
GradNorm & $8.44$ \\
Ours & $\mathbf{8.25}$  \\
\bottomrule
\end{tabular}
\label{table:multi_label_bar}
%\vspace{-5mm}
\end{wraptable}

Next, we tackle multi-label classification. Given a set of attributes, multi-label classification calls for deciding whether each attribute holds for the input. We use the CelebA dataset \citep{celeba}, which includes 200K face images annotated with 40 attributes. Each attribute gives rise to a binary classification task and we cast this as a 40-way MTL problem. We use ResNet-18 \citep{resnet} without the final layer as a shared representation function, and attach a linear layer for each attribute (see the supplement for further details).


We plot the resulting error for each binary classification task as a radar chart in Figure~\ref{fig:multi_label_radar}. The average over them is listed in Table~\ref{table:multi_label_bar}. We skip grid search since it is not feasible over 40 tasks. Although uniform scaling is the norm in the multi-label classification literature, single-task performance is significantly better. Our method outperforms baselines for significant majority of tasks and achieves comparable performance in rest. This experiment also shows that our method remains effective when the number of tasks is high.


\subsection{Scene Understanding}

To evaluate our method in a more realistic setting, we use scene understanding. Given an RGB image, we solve three tasks: semantic segmentation (assigning pixel-level class labels), instance segmentation (assigning pixel-level instance labels), and monocular depth estimation (estimating continuous disparity per pixel). We follow the experimental procedure of \citet{Kendall2018} and use an encoder-decoder architecture. The encoder is based on ResNet-50 \citep{resnet} and is shared by all three tasks. The decoders are task-specific and are based on the pyramid pooling module \citep{pspnet} (see the supplement for further implementation details).

Since the output space of instance segmentation is unconstrained (the number of instances is not known in advance), we use a proxy problem as in \citet{Kendall2018}. For each pixel, we estimate the location of the center of mass of the instance that encompasses the pixel. These center votes can then be clustered to extract the instances. In our experiments, we directly report the MSE in the proxy task. Figure~\ref{fig:cityscapes_performance_profile} shows the performance profile for each pair of tasks, although we perform all experiments on all three tasks jointly. The pairwise performance profiles shown in Figure~\ref{fig:cityscapes_performance_profile} are simply 2D projections of the three-dimensional profile, presented this way for legibility. The results are also listed in Table~\ref{tab:cityscapes_results}.

MTL outperforms single-task accuracy, indicating that the tasks cooperate and help each other. Our method outperforms all baselines on all tasks.


\subsection{Role of the Approximation}

In order to understand the role of the approximation proposed in Section~\ref{sec:approximation}, we compare the final performance and training time of our algorithm with and without the presented approximation in Table~\ref{tab:approximation_tradeoff} (runtime measured on a single Titan Xp GPU). For a small number of tasks (3 for scene understanding), training time is reduced by 40\%. For the multi-label classification experiment (40 tasks), the presented approximation accelerates learning by a factor of 25.

On the accuracy side, we expect both methods to perform similarly as long as the full-rank assumption is satisfied. As expected, the accuracy of both methods is very similar. Somewhat surprisingly, our approximation results in slightly improved accuracy in all experiments. While counter-intuitive at first, we hypothesize that this is related to the use of SGD in the learning algorithm. Stability analysis in convex optimization suggests that if gradients are computed with an error $\hat{\nabla}_\btheta \mathcal{L}^t = \nabla_\btheta \mathcal{L}^t + \mathbf{e}^t$ ($\btheta$ corresponds to $\btheta^{sh}$ in (\ref{eq:kkt_opt})), as opposed to $\mathbf{Z}$ in the approximate problem in \ref{eq:approx}, the error in the solution is bounded as $\|\hat{\mathbf{\alpha}} - \mathbf{\alpha} \|_2 \leq \mathcal{O}(\max_t \|\mathbf{e}^t\|_2)$. Considering the fact that the gradients are computed over the full parameter set (millions of dimensions) for the original problem and over a smaller space for the approximation (batch size times representation which is in the thousands), the dimension of the error vector is significantly higher in the original problem. We expect the $l_2$ norm of such a random vector to depend on the dimension.

In summary, our quantitative analysis of the approximation suggests that (i) the approximation does not cause an accuracy drop and (ii) by solving an equivalent problem in a lower-dimensional space, our method achieves both better computational efficiency and higher stability.

  {\small
  \begin{table}[t]
%  \vspace{-4mm}
  \caption{Effect of the MGDA-UB approximation. We report the final accuracies as well as training times for our method with and without the approximation.}
  %\vspace{1mm}
  \centering
  \begin{tabular}{@{}r@{\hspace{3mm}}c@{\hspace{3mm}}c@{\hspace{2mm}}c@{\hspace{2mm}}c@{}c@{\hspace{5mm}}c@{\hspace{2mm}}c@{}}
  \toprule
  & \multicolumn{4}{c}{Scene understanding (3 tasks)} &  & \multicolumn{2}{c}{Multi-label (40 tasks)}  \\
  \cmidrule(r){2-5} \cmidrule(lr){7-8}
                  & Training & Segmentation & Instance  & Disparity      & & Training & Average \\
                 & time     &  mIoU [\%]       & error [px] & error [px] & & time (hour)      & error \\
  \midrule
  Ours (w/o approx.) & $38.6$ & $66.13$ & $10.28$ & $2.59$ & & $429.9$ & $8.33$ \\
  Ours & $\mathbf{23.3}$ & $\mathbf{66.63}$ & $\mathbf{10.25}$ & $\mathbf{2.54}$  & & $\mathbf{16.1}$ & $\mathbf{8.25}$ \\
  \bottomrule
  \end{tabular}
  %\vspace{-2mm}
  \label{tab:approximation_tradeoff}
  \end{table}}


\vspace{-2pt}
\section{Conclusions}
% \vspace{-5pt}
%
We addressed fine-tuning of CNN for image retrieval. The training data are selected from an automated 3D reconstruction system applied on a large unordered photo collection. The proposed method does not require any manual annotation and yet outperforms the state of the art on a number of standard benchmarks for wide range (16 to 512) of descriptor dimensionality. The achieved results are reaching the level of the best systems based on local features with spatial matching and query expansion, while being faster and requiring less memory. Training data, fine-tuned networks and evaluation code are publicly available\footnote{\href{http://cmp.felk.cvut.cz/~radenfil/projects/siamac.html}{http://cmp.felk.cvut.cz/\~{}radenf{}i{}l/projects/siamac.html}}.

\clearpage

\paragraph{Acknowledgment}. Work was supported by the MSMT LL1303 ERC-CZ grant.

\bibliographystyle{splncs}
\bibliography{egbib}
\end{document}

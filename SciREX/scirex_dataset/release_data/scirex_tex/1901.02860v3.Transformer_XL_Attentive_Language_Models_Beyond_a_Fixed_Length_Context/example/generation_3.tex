after the French conquest of Italy. This victory marked the beginning of the Second Coalition. Napoleon's rapid advance caused Austria – Russia, Britain and Russia to make peace negotiations. The Russian army under Prince Mikhail Mikhailovich Mikhailovich Chaffee, commander of the Imperial Russian Army in Switzerland, was able to drive off the Austrians in the Battle of Stockach (1805) and to take Pressburg (modern \"{O}3 Austria) from the Austrians. At an early stage the Austro-Hungarian fleet had been damaged in a storm off Cape Matapan; this allowed the Allies to launch a full-scale invasion of Italy in February 1805. The Battle of Austerlitz was the decisive French victory against Napoleon and the largest naval battle in history, involving more modern-day European navies. 
The French military situation worsened as Napoleon faced several threats related to his newly formed Army of Europe, particularly Russia itself, which was now on the verge of collapse. The Russian general and friend of Napoleon, Alexander, had been dissatisfied with the conduct of the Austrians; he was still in conflict with Emperor Napoleon, the French Republic's king, who had declared war on Austria during the summer of 1804. With the war over, Napoleon decided to launch a second invasion of Italy in October 1805. 

= = Prelude = = 

\magenta{In July 1805}, the French 1st Army entered southern Italy. The army, under the command of Marshal Marmont, were reinforced by a few battalions of infantry under Claude General Auguste de Marmont at the town of Philippsburg and another battalion at Belluno. On \magenta{17 September 1805}, the army marched from Belluno towards Krems. By \magenta{29 September}, they had reached Belluno and conducted its advance against a small Austrian force. By \magenta{31 September}, the whole force had been reinforced by a brigade from the Army of Tyrol under the command of Pierre Augereau. 
The Austrians were now under the command of Marshal Jean Victor Marie Moreau, a member of the Directory. Moreau had taken command of the Austrian invasion force in the spring of 1805. His command included the VI Corps commanded by Jean Baptiste Drouet de Ney and the VI Corps commanded by Generals Jean Victor Marie Moreau and Joseph Souham. Ney's corps consisted of the III. Corps and VI. Corps, which consisted of the III Corps and VI. Corps, located in the Austrian Netherlands, was commanded by Friedrich Joseph, Count Baillet de Latour. Moreau's army consisted of six divisions and several associated brigades. 

= = Aftermath = = 


= = = First Coalition forces = = = 

On \magenta{9 October 1805} the French Army of the Danube was attacked by an Austrian army under Archduke Charles at the Battle of Austerlitz. Although Charles and Charles had not had much time to regroup, on \magenta{10 October}, he launched his attack on the Polish forces under Friedrich Joseph, Count of Lauenburg. After three days, Charles' army captured Lauenburg. The French forces pursued the Austrians to the Silesian border, where they encountered strong Austrian resistance. These conflicts forced the Austrians to retreat into Tyrol and Austria agreed to a truce. 
The Austrian army, commanded by Wenzel Anton Karl, Count of Merveldt, was reduced to around 10,000 men. It was initially planned that Archduke Charles would launch a counter-attack against the French army on the same day, as Napoleon had hoped, but this was not carried out. On \magenta{25 October}, Merveldt left Styria for Tyrol. On the same day, Austria launched its new offensive against the French at Ulm. Charles withdrew his army from the region for a third time at the Battle of Elchingen, under the overall command of the Austrian generals, Ferdinand and Friedrich Wilhelm of J\"{u}lich-Cleves-Berg. To prevent Archduke Charles from escaping from the battlefield, the commander of the Habsburg army, Archduke Charles, planned to occupy the fortress Linz; instead, he decided to force Franz von Hipper to surrender the city. However, as Charles moved to the south, Moreau arrived on the scene with additional soldiers – including the entire Imperial Guard – and defeated the Austrians at the Battle of Hohenlinden on \magenta{28 October}. 
The loss of Linz resulted in Austria's complete defeat at Hohenlinden. In the meantime, the French Army of Observation and Preparedness was reorganized into the Army of the Danube under Feldzeugmeister (Colonel-General) Friedrich Freiherr von Hotze. The army was composed of the I, IV, VI, VI, VII, VIII and IX Corps. With reinforcements from Italy and France, it formed new battalions, companies, and squadrons in the Austrian army. On \magenta{17 November 1804}, at the Battle of Jena-Auerstadt the Army of Silesia and the Army of Silesia joined forces, but by the time that the %French approached Vienna, the Prussians had already surrendered. As the Austrians did not want to allow the war to continue, they decided to abandon their territories in the north and move their army to the north and west, cutting off Charles from Vienna. The Battle of Warsaw was fought on 23 October 1805 between the French Army of the Danube and the Austrian Army of Styria in the vicinity of Warsaw and Pressburg (modern Trnava, Slovakia). At that time Habsburg forces 
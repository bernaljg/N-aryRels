%<unk> undergo <unk> reactions with <unk> to afford a number of unique five membered <unk>, as depicted in the figure below. This reactivity is due to the strained three membered ring and weak N-O bond. 
= Battle of D\"{u}renstein = 

The Battle of D\"{u}renstein (also known as the Battle of <unk>, Battle of <unk> and Battle of <unk>; German: <unk> bei <unk> ), on 11 November 1805 was an engagement in the Napoleonic Wars during the War of the Third Coalition. Dürenstein (modern <unk>) is located in the <unk> Valley, on the River Danube, 73 kilometers (45 mi) upstream from Vienna, Austria. The river makes a crescent-shaped curve between <unk> and nearby Krems an der Donau and the battle was fought in the flood plain between the river and the mountains. 
At Dürenstein a combined force of Russian and Austrian troops trapped a French division commanded by Théodore Maxime Gazan. The French division was part of the newly created VIII Corps, the so-called Corps Mortier, under command of \'{E}douard Mortier. In pursuing the Austrian retreat from Bavaria, Mortier had over-extended his three divisions along the north bank of the Danube. Mikhail <unk> Kutuzov, commander of the Coalition force, enticed Mortier to send Gazan's division into a trap and French troops were caught in a valley between two Russian columns. They were rescued by the timely arrival of a second division, under command of Pierre Dupont de l 'Étang. The battle extended well into the night. Both sides claimed victory. The French lost more than a third of their participants, and Gazan's division experienced over 40 percent losses. The Austrians and Russians also had heavy <unk> to 16 <unk> perhaps the most significant was the death in action of Johann Heinrich von Schmitt, one of Austria's most capable chiefs of staff. 
The battle was fought three weeks after the Austrian capitulation at Ulm and three weeks before the Russo-Austrian defeat at the Battle of Austerlitz. After Austerlitz Austria withdrew from the war. The French demanded a high indemnity and Francis II abdicated as Holy Roman Emperor, releasing the German states from their allegiance to the Holy Roman Empire. 

= = Background = = 

In a series of conflicts from 1803-15 known as the Napoleonic Wars, various European powers formed five coalitions against the First French Empire. Like the wars sparked by the French Revolution (1789 ), these further revolutionized the formation, organization and training of European armies and led to an unprecedented militarization, mainly due to mass conscription. Under the leadership of Napoleon, French power rose quickly as the Grande Armée conquered most of Europe, and collapsed rapidly 

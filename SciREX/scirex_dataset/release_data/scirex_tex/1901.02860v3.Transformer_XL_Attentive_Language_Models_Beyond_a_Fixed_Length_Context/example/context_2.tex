= = Distribution = = 

Species range across the Neotropics from Mexico in the north to Bolivia, Paraguay, and southern Brazil in the south. According to <unk> and coauthors, three species are found in Mexico, four in Central America, and 62 in South America. Three species are present in the Caribbean — two in Trinidad and Tobago, along the southern edge of the region, and one in Haiti. 

= = Habitat and ecology = = 

<unk> includes both large trees and small acaulescent palms which occupy a number of different ecological niches. Dense stands of some of the larger species are conspicuous elements on the landscape, while smaller species are found in both in the forest understorey and in savannas. 
Disturbance has been implicated in the formation of vegetation dominated by large <unk> species. In seasonally dry Amazonian forests the density of large adult A. <unk> palms was correlated with canopy openness; the species also dominates savannas formed by repeated forest fires in Trinidad and Tobago. <unk> speciosa forms pure stands in many parts of Brazil where natural forest vegetation has been cleared. Similarly, stands of A. <unk> in Bahia, Brazil (which are cultivated for <unk> fibre) are managed using fire — the seedlings survive cutting and burning, and are able to dominate burned forest patches. 
The fruit are dispersed by animals; fruit which are not dispersed frequently suffer seed predation by <unk> beetles. Certain species of <unk> have been mentioned as examples of " anachronistic " species which are adapted for dispersal by now-extinct Pleistocene megafauna. On <unk> Island, <unk>, in the Brazilian Amazon, <unk> <unk> fruit were consumed by tapirs, collared peccaries, deer and primates. Rodents, including agoutis, fed upon the fruit and, as the fruit availability declined, they fed on the seeds. Other dispersers of <unk> fruit include Crested <unk> which consume the fruit and disperse the seeds of A. <unk> in the Brazilian Pantanal. 

= = Uses = = 

<unk> species have a long history of human utilisation. <unk> <unk> <unk> seeds have been found in archaeological sites in Colombia dating back to 9000 BP. A variety of species remain important sources of edible oil, thatch, edible seeds and fibre. The leaves of <unk> <unk> and A. <unk> are used extensively for thatching. Several species are oil palms, with A. speciosa among the most important economically. Products extracted from A. speciosa were reported to support over 300,000 households in the Brazilian state of Maranhão in 2005, and in 1985 it was estimated to support over 450,000 households throughout the Brazil. <unk> fibres, extracted from the leaf bases of A. <unk>, are commercially important, and generated about US \$
after the disastrous invasion of Russia in 1812. Napoleon's empire ultimately suffered complete military defeat in the 1813 – 14 campaigns, resulting in the restoration of the Bourbon monarchy in France. Although Napoleon made a spectacular return in 1815, known as the Hundred Days, his defeat at the Battle of Waterloo, the pursuit of his army and himself, his abdication and banishment to the Island of Saint Helena concluded the Napoleonic Wars. 

= = Danube campaign = = 

From 1803-06 the Third Coalition fought the First French Empire and its client states (see table at right ). Although several naval battles determined control of the seas, the outcome of the war was decided on the continent, predominantly in two major land operations in the Danube valley: the Ulm campaign in the upper Danube and the Vienna campaign, in the middle Danube valley. 
Political conflicts in Vienna delayed Austria's entry into the Third Coalition until 1805. After hostilities of the War of the Second Coalition ended in 1801, Archduke <unk> emperor's <unk> advantage of the subsequent years of peace to develop a military restructuring plan. He carefully put this plan into effect beginning in 1803 – 04, but implementation was incomplete in 1805 when Karl Mack, Lieutenant Field Marshal and Quartermaster-General of the Army, implemented his own restructuring. Mack bypassed Charles ' methodical approach. Occurring in the field, Mack's plan also undermined the overall command and organizational structure. Regardless, Mack sent an enthusiastic report to Vienna on the military's readiness. Furthermore, after misreading Napoleon's maneuvers in W\"{u}rttemberg, Mack also reported to Vienna on the weakness of French dispositions. His reports convinced the war party advising the emperor, Francis II, to enter the conflict against France, despite Charles ' own advice to the contrary. Responding to the report and rampant anti-French fever in Vienna, Francis dismissed Charles from his post as generalissimo and appointed his <unk> brother-in-law, Archduke Ferdinand, as commander. 
The inexperienced Ferdinand was a poor choice of replacement for the capable Charles, having neither maturity nor aptitude for the assignment. Although Ferdinand retained nominal command, day-to-day decisions were placed in the hands of Mack, equally ill-suited for such an important assignment. When Mack was wounded early in the campaign, he was unable to take full charge of the army. Consequently, command further devolved to Lieutenant Field Marshal Karl Philipp, Prince of Schwarzenberg, an able cavalry officer but inexperienced in the command of such a large army. 

= = = Road to Ulm = = = 

The campaign in the upper Danube valley began in October, with several clashes in Swabia. Near the Bavarian town of Wertingen, 40 kilometers (25 mi) northwest of Augsburg, on 8 October the 1st Regiment of dragoons, part of Murat's Reserve Cavalry Corps, and grenadiers of Lannes ' V Corps surprised an Austrian force half its size. The Austrians were arrayed in a line and unable to form their defensive squares quickly enough to protect themselves from the 4,000 dragoons and 8,000 grenadiers. Nearly 3,000 Austrians were captured and over 400 were killed or wounded. A day later, at another small town, <unk> south of the Danube <unk> French 59th Regiment of the Line stormed a bridge over the Danube and, humiliatingly, chased two large Austrian columns toward Ulm. 
The campaign was not entirely bad news for Vienna. At Haslach, Johann von Klenau arranged his 25,000 infantry and cavalry in a prime defensive position and, on 11 October, the overly confident General of Division Pierre Dupont de l'\'{E}tang attacked Klenau's force with fewer than 8,000 men. The French lost 1,500 men killed and wounded. Aside from taking the Imperial Eagles and <unk> of the 15th and 17th Dragoons, Klenau's force also captured 900 men, 11 guns and 18 ammunition wagons. 
Klenau's victory was a singular success. On 14 October Mack sent two columns out of Ulm in preparation for a breakout to the north: one under Johann Sigismund Riesch headed toward Elchingen to secure the bridge there, and the other under Franz von Werneck went north with most of the heavy artillery. Recognizing the opportunity, Marshal Michel Ney hurried the rest of his VI Corps forward to re-establish contact with Dupont, who was still north of the Danube. In a two-pronged attack Ney sent one division to the south of Elchingen on the right bank of the Danube. This division began the assault at Elchingen. At the same time another division crossed the river to the east and moved west against Riesch's position. After clearing Austrian pickets from a bridge, the French attacked and captured a strategically located abbey at %the top of the hill at bayonet point. The Austrian cavalry unsuccessfully tried to fend off the French, but the Austrian infantry broke and ran. In this engagement alone, the Austrians lost more than half their reserve artillery park, 6,000 (out of 8,000 total participants) dead, wounded or captured and four colors. Reisch's column also failed to destroy the bridges across the Danube. 
%Napoleon's lightning campaign exposed the Austrian indecisive command structure and poor supply apparatus. Mack 
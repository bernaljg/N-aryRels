700,000 in 2013 on the black market. In 2004 the plant was reported to be responsible for a net profit of around US \$ 3 million in Brazil. A 2012 study found A. speciosa to cause agricultural production at least 2 – 3 times greater than those achieved without prior knowledge of land management. 


\underline{= The Tale of Mrs. Tittlemouse =}

The Tale of Mrs. Tittlemouse is a children's book written and illustrated by H. M. Potter. It was first published in February 1910 by Frederick Warne \& Co. and included illustrations by Frederick Warne. Mrs. Tittlemouse was intended to be a children's book, rather than a musical, and the tale describes her attempts to find her own home. Mrs. Tittlemouse is an average kitten and the tale follows a mischievous rabbit named Mr. Tittlemouse who tries to kill her in an attempt to get a ride on a horse. Potter later wrote of the rabbit and her attempts to kill her, " If [ she ] were a fox, I should like to have been able to show her the way. " 
Potter's work was inspired by The Tale of Little Black Sambo and her two nieces, Petunia and Petunia. It was the first book in the Potter family to be published in a hardback format, and was also the first book in the Peter Rabbit series. The tale was released on 13 November 1910 and was accompanied by a greeting card by Potter's brother-in-law, Frederick Warne. 
The tale was well received by critics and children, and sold over 600,000 copies in its first few years. In the first year of its publication, Potter's The Tale of Mrs. Tittlemouse won children's book awards in England, America and Australia. It was selected as a " recommended reading " by children in the US and Canada. 

= = Plot summary = = 

The tale opens with the pet rabbit Mrs. Tittlemouse wandering through a forest in a small litter of four. He is shot and badly injured by a deer after she attempts to escape. Mrs. Tittlemouse decides that she wants to find her own home, because she is eager to go on her own. She goes alone to the farm where she makes a little money by selling a few seeds and building a small cabin in the woods. She is approached by a wealthy hunter named Mr. Tittlemouse, who tries to kill her but Mrs. Tittlemouse kills him by stuffing a rope into his nose and killing him. She is rescued by Mr. Tittlemouse's wife Ruth, but Mrs. Tittlemouse then leaves the woodland with the baby. When she is spotted by 
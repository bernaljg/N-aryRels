\usepackage{xspace}
\usepackage{bbold}
\usepackage{xcolor}

\newcommand{\samuel}[1]{\textcolor{red}{(S: #1)}}
\newcommand{\peter}[1]{\textcolor{green}{(P: #1)}}
\newcommand{\lorenzo}[1]{\textcolor{blue}{(L: #1)}}

\newcommand{\vct}[1]{\ensuremath{\boldsymbol{#1}}}
\newcommand{\mat}[1]{\mathtt{#1}}
\newcommand{\set}[1]{\ensuremath{\mathcal{#1}}}
\newcommand{\con}[1]{#1} %\ensuremath{\mathsf{#1}}}
\newcommand{\T}{\ensuremath{^\top}}
\newcommand{\ind}[1]{\ensuremath{\mathbb 1_{#1}}}
\newcommand{\argmax}{\operatornamewithlimits{\arg\,\max}}
\newcommand{\argmin}{\operatornamewithlimits{\arg\,\min}}


\newtheorem{theorem}{Theorem}
\newtheorem{lemma}{Lemma}
\newtheorem{proposition}{Proposition}
\newtheorem{corollary}{Corollary}
\newtheorem{remark}{Remark}

\newcommand{\aka}{\emph{a.k.a.}\xspace }



\documentclass[10pt,journal,compsoc]{IEEEtran}
%

\usepackage{graphicx}

\DeclareGraphicsExtensions{.pdf,.png,.jpg}

\usepackage{subfigure}
\usepackage[subfigure]{graphfig}
\usepackage{array}
\usepackage{amsmath}
\usepackage{multirow}
\graphicspath{ {./figs/} }


\begin{document}

\title
{
%
    Image Restoration Using  Convolutional Auto-encoders with Symmetric Skip Connections
}


\author
{
	Xiao-Jiao Mao, Chunhua Shen, Yu-Bin Yang
	\IEEEcompsocitemizethanks
	{
		\IEEEcompsocthanksitem
        X.-J. Mao and Y.-B. Yang are with the State Key Laboratory
		for Novel Software Technology, Nanjing University, China.
        E-mail: {\tt xjmgl.nju@gmail.com, yangyubin@nju.edu.cn}.
		\IEEEcompsocthanksitem
        C. Shen is with the School of Computer Science,
		University of Adelaide, Australia.
        E-mail: {\tt chunhua.shen@adelaide.edu.au}.
        \IEEEcompsocthanksitem
        X.-J. Mao's contribution was made when visiting University of Adelaide.
        Correspondence should be addressed to C. Shen.
	}
}



%
\markboth{Manuscript}%
{Mao \MakeLowercase{\textit{et al.}}: Image Restoration Using  Convolutional Auto-encoders}


\IEEEtitleabstractindextext{
\begin{abstract}


    Image restoration, including image denoising, super resolution,  inpainting,
    and so on, is a well-studied problem in computer vision and image processing,
as well as a test bed for low-level image modeling algorithms. In this work, we propose
a very deep fully convolutional  auto-encoder  network  for image
restoration, which is a
 encoding-decoding framework with symmetric con\-vo\-lu\-tion\-al-de\-con\-vo\-lutional layers.
 In other words, the network is composed of multiple layers of convolution and de-convolution
operators, learning end-to-end mappings from corrupted images to the original ones. The
convolutional layers  capture the abstraction of image
contents while eliminating corruptions. Deconvolutional layers have the capability to upsample the
feature maps and  recover
the image details. To deal with the problem that deeper networks tend to be more difficult
to train, we propose to symmetrically link convolutional and deconvolutional layers with
skip-layer connections, with which the training converges much faster and attains better results.
The skip connections from convolutional layers to their mirrored corresponding
deconvolutional layers exhibit two main advantages. First, they allow the signal to be back-propagated
to bottom layers directly, and thus tackles the problem of gradient vanishing, making training deep
networks easier and achieving restoration performance gains consequently. Second, these skip
connections pass image details from convolutional layers to deconvolutional layers, which is
beneficial in recovering the clean image. Significantly, with the large capacity, we show it is
possible to cope with
different levels of corruptions using a single model.
Using the same framework, we train models on tasks of image denoising,  super resolution
removing JPEG compression artifacts, non-blind image deblurring and image inpainting.
Our experiment results on  benchmark
datasets show that our network can achieve best reported performance on all of the four tasks, and
set new state-of-the-art.

\end{abstract}




\begin{IEEEkeywords}
    Image restoration, auto-encoder, convolutional/de-convolutional  networks, skip connection, image
    denoising,  super resolution, image inpainting.
\end{IEEEkeywords}}



\maketitle
\IEEEdisplaynontitleabstractindextext
\IEEEpeerreviewmaketitle



\tableofcontents
\clearpage


\input introduction.tex
\input relatedwork.tex
\input method.tex
\input discussion.tex
\input experiments.tex
\input conclusions.tex

{
    \medskip
    \bibliographystyle{IEEEtran}
    \bibliography{CSRef}
}


\end{document}




\section{Introduction}




Image restoration~\cite{DBLP:conf/iccv/MairalBPSZ09,DBLP:journals/tip/DongZSL13,
DBLP:journals/pami/SchmidtJNRR16,DBLP:conf/cvpr/SchmidtR14,DBLP:conf/iccv/ZoranW11}
is a classical problem in low-level vision, which has been widely studies in the literature.
Yet, it remains an active research topic and provides a test bed for many image modeling techniques.

Generally speaking, image restoration is the operation of taking a corrupted image and
estimating the original image, which is known to be an ill-posed inverse problem. A corrupted
image $Y$ can be represented as
\begin{equation}
    y = H(x) + n
\end{equation}
where $x$ is the clean version of $y$;  $H$ is the degradation function and $n$ is the additive
noise. By accommodating different types of degradation operators and noise distributions, the same mathematical model
applies to most  low-level imaging problems such as image denoising~\cite{DBLP:conf/cvpr/LiuXZG15,
DBLP:conf/iccv/ChenZY15,DBLP:conf/iccv/XuZZZF15,DBLP:conf/cvpr/GuZZF14}, super-resolution
    \cite{DBLP:conf/accv/TimofteSG14,DBLP:conf/cvpr/YangLC13,DBLP:conf/cvpr/ZhuZY14,DBLP:conf/cvpr/ZhuZ0Y15,
DBLP:conf/iccv/RieglerSRB15,DBLP:conf/iccv/GuZXMFZ15,DBLP:conf/iccv/WangLYHH15}, inpainting
    \cite{DBLP:conf/nips/XieXC12,DBLP:journals/ijcv/RothB09,DBLP:journals/tip/MairalES08} and recovering raw
images from compressed images~\cite{DBLP:conf/iccv/DongDLT15,DBLP:journals/tip/FoiKE07,DBLP:conf/eccv/JancsaryNR12}.
In the past decades, extensive studies have been carried out to develop various of image restoration methods.




    Recently, deep neural networks (DNNs) have shown their superior performance in image processing
and computer vision tasks, ranging from high-level recognition, semantic segmentation to low-level
denoising and super-resolution. One of the early deep learning models which has been used for image
denoising is the Stacked Denoising Auto-encoders (SdA)~\cite{DBLP:conf/icml/VincentLBM08}. It is
an extension of the stacked auto-encoder~\cite{DBLP:conf/nips/BengioLPL06} and was originally designed
for unsupervised feature learning.  Denoising auto-encoders can be stacked to form a deep network
by feeding output of the previous layer to the current layer as input. Jain and Seung~\cite{DBLP:conf/nips/JainS08}
proposed to use Convolutional Neural Networks (CNN) to denoise natural images. Their framework is
the same as the recent Fully Convolutional Neural Networks (FCN) for semantic image segmentation \cite{DBLP:conf/cvpr/LongSD15}
and other tasks such as super-resolution \cite{DBLP:journals/pami/DongLHT16}, although their network
is not as deep as today's models. Their network accepts an image as the input and produces an
entire image as the output through four hidden layers of convolutional filters.
The weights are learned by
minimizing the difference between the output and the clean image.



By observing recent superior performance of CNN on image processing tasks, here we propose a very deep fully convolutional
CNN-based framework for image restoration. The input of our framework is a corrupted image, and
the output is its clean version.  We observe that it is beneficial to train a very deep model for
low-level tasks like denoising, super-resolution and JPEG deblocking. Our network is much
deeper compared to that in \cite{DBLP:conf/nips/JainS08} and recent low-level image processing models
such as \cite{DBLP:journals/pami/DongLHT16}.  Instead of using image priors, the proposed framework
learns  fully convolutional and deconvolutional mappings from corrupted images to the clean
ones in an end-to-end fashion. The network is composed of multiple layers of convolution and deconvolution operators.
As deeper networks tend to be more difficult to train, we further propose to symmetrically link convolutional
and deconvolutional layers with multiple skip-layer connections, with which the training converges much faster
and better performance is achieved.

%




Our main contributions can be summarized as follows.

\begin{itemize}
  
\item We propose a very deep network architecture for image restoration. The network consists of a
chain of symmetric convolutional layers and deconvolutional layers. The convolutional layers act as
the feature extractor which encode the primary components of image contents while eliminating the
corruptions. The deconvolutional layers then decode the image abstraction to recover the image content
details. To the best of our knowledge, the proposed framework is the first attempt to used both
convolution and deconvolution for low-level image restoration.

\item To better train the deep network, we propose to add skip connections between corresponding
convolutional and deconvolutional layers. These shortcuts divide the network into several blocks.
These skip connections help to back-propagate the gradients to bottom layers and pass image details
to the top layers. These two characteristics make training of the end-to-end mapping from corrupted image
to the clean one easier and more effective, and thus achieve performance improvement while the network going deeper.

\item We apply the same network for tasks such as image denoising, image super-resolution, JPEG
    deblocking, non-blind image deblurring and image inpainting.
Experiments on a few widely-used  benchmark datasets demonstrate the advantages of our network over
other recent state-of-the-art methods. Moreover, relying on the large capacity and fitting
ability, our network can be trained to obtain good restoration performance on different levels
of corruption even using a single model.
\end{itemize}



The remaining content
is organized as follows. We provide a brief review of related work in Section \ref{sec:related}.
We present the architecture of the proposed network, as well as training, testing details in Section \ref{sec:main}.
In Section \ref{sec:disc}, we discuss some relevant issues.
Experimental results and analysis are provided in Section \ref{sec:exp}.

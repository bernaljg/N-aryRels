\section{Refining target object localization}
\label{refineRefPoint_Section}

To localize keypoints, such as hand or human body joints, a cubic box that contains the hand or human body in 3D space is a prerequisite. This cubic box is usually placed around the reference point, which is obtained using ground-truth joint position~\cite{oberweger2015hands,oberweger2015training,zhou2016model} or the center-of-mass after simple depth thresholding around the hand region~\cite{guo2017ren,guo2017towards,chen2017pose}. However, utilizing the ground-truth joint position is infeasible in real-world applications. Also, in general, using the center-of-mass calculated by simple depth thresholding does not guarantee that the object is correctly contained in the acquired cubic box due to the error in the center-of-mass calculations in cluttered scenes. For example, if other objects are near the target object, then the simple depth thresholding method cannot properly filter the other objects because it applies the same threshold value to all input data. Hence, the computed center-of-mass becomes erroneous, which results in a cubic box that contains only some part of the target object. To overcome these limitations, we train a simple 2D CNN following Oberweger \etal~\cite{Oberweger_2017_ICCV_Workshops} to obtain an accurate reference point as shown in Figure~\ref{fig:ref_refine_net}. This network takes a depth image, whose reference point is calculated by the simple depth thresholding around the hand region, and outputs 3D offset from the calculated reference point to the center of ground-truth joint locations. The refined reference point can be obtained by adding the output offset value of the network to the calculated reference point.
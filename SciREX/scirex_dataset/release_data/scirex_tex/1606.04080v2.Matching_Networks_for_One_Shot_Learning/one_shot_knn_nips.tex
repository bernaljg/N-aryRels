\documentclass{article}
\pdfoutput=1
% if you need to pass options to natbib, use, e.g.:
\PassOptionsToPackage{numbers, compress}{natbib}
% before loading nips_2016
%
% to avoid loading the natbib package, add option nonatbib:
% \usepackage[nonatbib]{nips_2016}

\usepackage[arxiv]{nips_2016}

% to compile a camera-ready version, add the [final] option, e.g.:
% \usepackage[final]{nips_2016}

\usepackage[utf8]{inputenc} % allow utf-8 input
\usepackage[T1]{fontenc}    % use 8-bit T1 fonts
\usepackage[bookmarks=false]{hyperref}       % hyperlinks
\usepackage{url}            % simple URL typesetting
\usepackage{booktabs}       % professional-quality tables
\usepackage{amsfonts}       % blackboard math symbols
\usepackage{nicefrac}       % compact symbols for 1/2, etc.
\usepackage{microtype}      % microtypography
\usepackage{textcomp}
\usepackage{amsmath}
\usepackage{xcolor}
\usepackage{multirow}
\usepackage{graphicx}

\newcommand\todo[1]{\textcolor{blue}{\textbf{TODO:} #1}}
\def\colspaceDS{1.25mm}
\def\colspaceS{2.25mm}
\def\colspaceM{4.0mm}
\def\colspaceL{4.25mm}
\def\colspaceD{5.25mm}
\def\colspaceDD{9.25mm}
\def\t#1{#1}
\def\b#1{\t{\textbf{#1}}}
\newcommand{\cev}[1]{\reflectbox{\ensuremath{\vec{\reflectbox{\ensuremath{#1}}}}}}
\newcommand\abbr[1]{\textsc{#1}}

\newcommand{\Strn}{\ensuremath{S_a}}
\newcommand{\Stst}{\ensuremath{S_b}}

\title{Matching Networks for One Shot Learning}

\author{
  Oriol Vinyals \\
  Google DeepMind \\ 
  \texttt{vinyals@google.com}
  \And
  Charles Blundell \\
  Google DeepMind \\
  \texttt{cblundell@google.com}
  \And
  Timothy Lillicrap \\
  Google DeepMind \\
  \texttt{countzero@google.com}
  \AND
  Koray Kavukcuoglu \\
  Google DeepMind \\
  \texttt{korayk@google.com}
  \And
  Daan Wierstra \\
  Google DeepMind \\
  \texttt{wierstra@google.com}
  %% examples of more authors
  %% \And
  %% Coauthor \\
  %% Affiliation \\
  %% Address \\
  %% \texttt{email} \\
  %% \AND
  %% Coauthor \\
  %% Affiliation \\
  %% Address \\
  %% \texttt{email} \\
  %% \And
  %% Coauthor \\
  %% Affiliation \\
  %% Address \\
  %% \texttt{email} \\
  %% \And
  %% Coauthor \\
  %% Affiliation \\
  %% Address \\
  %% \texttt{email} \\
}

\begin{document}
% \nipsfinalcopy is no longer used

\maketitle

\begin{abstract}
Learning from a few examples remains a key challenge in machine learning. Despite recent advances in important domains such as vision and language, the standard supervised deep learning paradigm does not offer a satisfactory solution for learning new concepts rapidly from little data. In this work, we employ ideas from metric learning based on deep neural features and from recent advances that augment neural networks with external memories. Our framework learns a network that maps a small labelled support set and an unlabelled example to its label, obviating the need for fine-tuning to adapt to new class types. We then define one-shot learning problems on vision (using Omniglot, ImageNet) and language tasks. Our algorithm improves one-shot accuracy on ImageNet from 87.6\% to 93.2\% and from 88.0\% to 93.8\% on Omniglot compared to competing approaches. We also demonstrate the usefulness of the same model on language modeling by introducing a one-shot task on the Penn Treebank.
\end{abstract}

\section{Introduction}
\label{sec:intro}

Language modeling is among the important problems that require modeling long-term dependency, with successful applications such as unsupervised pretraining~\citep{dai2015semi,peters2018deep,radford2018improving,devlin2018bert}.
However, it has been a challenge to equip neural networks with the capability to model long-term dependency in sequential data.
Recurrent neural networks (RNNs), in particular Long Short-Term Memory (LSTM) networks~\citep{hochreiter1997long}, have been a standard solution to language modeling and obtained strong results on multiple benchmarks.
Despite the wide adaption, RNNs are difficult to optimize due to gradient vanishing and explosion~\citep{hochreiter2001gradient}, and the introduction of gating in LSTMs and the gradient clipping technique~\citep{graves2013generating} might not be sufficient to fully address this issue.
% ,pascanu2012understanding
Empirically, previous work has found that LSTM language models use 200 context words on average~\citep{khandelwal2018sharp}, indicating room for further improvement.

On the other hand, the direct connections between long-distance word pairs baked in attention mechanisms might ease optimization and enable the learning of long-term dependency~\citep{bahdanau2014neural,vaswani2017attention}.
Recently, \citet{al2018character} designed a set of auxiliary losses to train deep Transformer networks for character-level language modeling, which outperform LSTMs by a large margin.
Despite the success, the LM training in~\citet{al2018character} is performed on separated fixed-length segments of a few hundred characters, without any information flow across segments.
As a consequence of the fixed context length, the model cannot capture any longer-term dependency beyond the predefined context length.
In addition, the fixed-length segments are created by selecting a consecutive chunk of symbols without respecting the sentence or any other semantic boundary.
Hence, the model lacks necessary contextual information needed to well predict the first few symbols, leading to inefficient optimization and inferior performance.
We refer to this problem as \textit{context fragmentation}.

%However, the context length is fixed to hundreds of characters and thus it is not possible to model longer-term dependency. Moreover, it is not clear how the model performs on word-level language modeling data, as the granularity changes.

% Moreover, using auxiliary losses brings additional challenges such as properly tuning the mixture weights and the loss decay schedule.

To address the aforementioned limitations of fixed-length contexts, we propose a new architecture called Transformer-XL (meaning extra long).
We introduce the notion of recurrence into our deep self-attention network. In particular, instead of computing the hidden states from scratch for each new segment, we reuse the hidden states obtained in previous segments.
The reused hidden states serve as memory for the current segment, which builds up a recurrent connection between the segments.
As a result, modeling very long-term dependency becomes possible because information can be propagated through the recurrent connections.
Meanwhile, passing information from the previous segment can also resolve the problem of context fragmentation.
More importantly, we show the necessity of using relative positional encodings rather than absolute ones, in order to enable state reuse without causing temporal confusion.
Hence, as an additional technical contribution, we introduce a simple but more effective relative positional encoding formulation that generalizes to attention lengths longer than the one observed during training.

Transformer-XL obtained strong results on five datasets, varying from word-level to character-level language modeling.
Transformer-XL is also able to generate relatively coherent long text articles with \textit{thousands of} tokens (see Appendix \ref{sec:gen}), trained on only 100M tokens.
% Transformer-XL improves the previous state-of-the-art (SoTA) results from 1.06 to 0.99 in bpc on enwiki8, from 1.13 to 1.08 in bpc on text8, from 20.5 to 18.3 in perplexity on WikiText-103, and from 23.7 to 21.8 in perplexity on One Billion Word.
% Transformer-XL improves the previous state-of-the-art (SoTA) results to 0.99 in bpc on enwiki8, 1.08 in bpc on text8, 18.3 in perplexity on WikiText-103, and 21.8 in perplexity on One Billion Word.
% On small data, Transformer-XL also achieves a perplexity of 54.5 on Penn Treebank without finetuning, which is SoTA when comparable settings are considered.

Our main technical contributions include introducing the notion of recurrence in a purely self-attentive model and deriving a novel positional encoding scheme. These two techniques form a complete set of solutions, as any one of them alone does not address the issue of fixed-length contexts. Transformer-XL is the first self-attention model that achieves substantially better results than RNNs on both character-level and word-level language modeling.

% On WikiText-103, Transformer-XL improves the previous state-of-the-art (SoTA) results from 33 perplexity to 24, with a relative reduction of 27\%. On enwiki8 character-level language modeling, Transformer-XL achieves a SoTA bpc of 1.03, which outperforms \cite{al2018character} by 0.03 with 60+\% fewer parameters. Given a more common model size with 40+M parameters, Transformer-XL achieves a bpc of 1.06, compared to 1.11 by \cite{al2018character}. Transformer-XL also achieves perplexities of 54.5 on Penn Treebank and 29.4 on One Billion Word, which are SoTA when comparable settings are considered.

% Due to the ability of modeling long-range context, our best model uses attention lengths of 1,600 and 3,800 on WikiText-103 and enwiki8 respectively. We also devise a metric called \textit{Relative Effective Context Length} (RECL) that aims to fairly compare the ability of long-range dependency modeling.
% % perform a fair comparison of the gains brought by increasing the context lengths for different models.
% In this setting, Transformer-XL learns a RECL of 900 words on WikiText-103, while the numbers for recurrent networks and Transformer are only 500 and 128.

% We use two methods to quantitatively study the effective lengths of Transformer-XL and the baselines. Similar to \cite{khandelwal2018sharp}, we gradually increase the attention length at test time until no further noticeable improvement ($\sim$0.1\% relative gains) can be observed. Our best model in this settings use attention lengths of 1,600 and 3,800 on WikiText-103 and enwiki8 respectively.
% %In addition, since the effective context length of Transformer-XL can be longer than the attention length due to our recurrent formulation, we devise a metric called \textit{Relative Effective Context Length} (RECL) that aims to perform a fair comparison of the gains brought by increasing the context lengths for different models.
% In addition, we devise a metric called \textit{Relative Effective Context Length} (RECL) that aims to perform a fair comparison of the gains brought by increasing the context lengths for different models.
% In this setting, Transformer-XL learns a RECL of 900 words on WikiText-103, while the numbers for recurrent networks and Transformer are only 500 and 128.

\section{MT-DNN-1}
\label{sec:mt-dnn-1}

\subsection{Preliminaries}
\label{subsec:prelim}
In this work, our multi-task model combines classification, regression and pair-wise ranking tasks, which are summerised in Table~\ref{tab:task}. We briefly introduce the definition of each task as follows: 
\begin{table}[htb!]
	\begin{center}
		\begin{tabular}{@{\hskip1pt}l@{\hskip1pt}|@{\hskip1pt}c@{\hskip1pt}|@{\hskip1pt}c@{\hskip1pt}|@{\hskip1pt}c}
			\hline \bf Input &Classification&Regression &Ranking\\ \hline \hline
			single sentence &$\checkmark$&& \\
			pairwise text &$\checkmark$&$\checkmark$&$\checkmark$ \\ \hline
		\end{tabular}
	\end{center}
	\lgspace
	\caption{Summary of tasks in our multi-task framework.
	}
	\label{tab:task}
\lgspace
\end{table}
\begin{figure}[!t]
\centering
\adjustbox{trim={.065\width} {.01\height} {.05\width} {.01\height},clip}
{\includegraphics[scale=0.7]{mtl_model}}
\caption{Model architecture.}
\label{fig:mtl_model} 
\end{figure}

\begin{figure}[!t]
\centering
\adjustbox{trim={.05\width} {.01\height} {.05\width} {.01\height},clip}
{\includegraphics[scale=0.7]{mtl_model_v2}}
\caption{Model architecture version 2.}
\label{fig:mtl_model_v2} 
\end{figure}

\textbf{Task definition}

\textbf{Objective}

\textbf{Single classification}
\xiaodl{Need to cluster different tasks..}

\textbf{Sentence-pair classification}: given a pair of sentence, $(S_1, S_2)$, the model predicts a label indicating the relation of this pair of sentences: $P(C|S_1, S_2)$. For example, natural language inference is a typical instance of the sentence-pair classification task: a premise and a hypothesis are denoted by $S_1$ and $S_2$, respectively; the label, $C$, belongs one of three relations (\textit{contradiction}, \textit{neutral} and \textit{entailment}). 

\textbf{Regression}


\textbf{Pair-wise Ranking}
\begin{algorithm}[ht!]
 \SetAlgoLined
Initialize model parameters $\Theta$ randomly  \\
Set M \quad\textit{//the number of updates for the shared layer} \\
%\textit{Counter} = 0\\
 \For{$iteration$ in $0 ... \infty$}{
 	 %1. \textit{Counter} += 1\\
     1. Pick a task $t$ randomly \\
     2. Pick sample(s) from task $t$, i.e., \\
     \hspace{0.4cm}$(Q,C=\{0,1\})$ for classification \\
     \hspace{0.4cm}$(Q, D)$ for ranking\\
     3. Compute loss: $L(\Theta)$, i.e.,\\
     \hspace{0.4cm} the \textit{cross-entropy} for classification \\
     \hspace{0.4cm} the ranking loss for ranking\cite{learning-to-rank2005burges}\\

     4. Compute gradient: $\nabla(\Theta)$ \\
     5. Update model: $\Theta = \Theta - \epsilon \nabla(\Theta)$ \quad\textit{}
     % \eIf{Counter $<$ M}{
  	 %5. Update model: $\Theta = \Theta - \epsilon \nabla(\Theta)$ \quad\textit{//update both $\Theta^s$ and $\Theta^t$} \\
   %}{
   	% 6. Update model: $\Theta^t = \Theta^t - \epsilon \nabla(\Theta^t)$ 
  %}
 }
 \caption{\label{algo:mtdnn} Training a Multi-task model.}
 \algorithmfootnote{Note that $\Theta$ denotes the model parameters. \textcolor{red}{TODO: update alg based on task defination.}}
\end{algorithm}
\section{Related Work}
\label{sec:relwork}

\subsection{Memory Augmented Neural Networks}

A recent surge of models which go beyond ``static'' classification of fixed vectors onto their classes has reshaped current research and industrial applications alike. This is most notable in the massive adoption of LSTMs \cite{hochreiter} in a variety of tasks such as speech \cite{hinton2012deep}, translation \cite{seq2seqilya,montreal} or learning programs \cite{ntm, ptrnets}.
A key component which allowed for more expressive models was the introduction of ``content'' based attention in \cite{montreal}, and ``computer-like'' architectures such as the Neural Turing Machine \cite{ntm} or Memory Networks \cite{memnets}.
Our work takes the metalearning paradigm of \cite{mann}, where an LSTM learnt to learn quickly from data presented sequentially, but we treat the data as a set.
The one-shot learning task we defined on the Penn Treebank \cite{marcus1993building} relates to evaluation techniques and models presented in \cite{hill2015goldilocks}, and we discuss this in Section~\ref{sec:results}.


\subsection{Metric Learning}

As discussed in Section~\ref{sec:model}, there are many links between content based attention, kernel based nearest neighbor and metric learning \cite{lwl}.  The most relevant work is Neighborhood Component Analysis (NCA) \cite{nca}, and the follow up non-linear version \cite{salakhutdinov2007learning}. The loss is very similar to ours, except we use the whole support set $S$ instead of pair-wise comparisons which is more amenable to one-shot learning. Follow-up work in the form of deep convolutional siamese \cite{siamese} networks included much more powerful non-linear mappings. Other losses which include the notion of a set (but use less powerful metrics) were proposed in \cite{weinberger2009distance}.

Lastly, the work in one-shot learning in \cite{omniglot} was inspirational and also provided us with the invaluable Omniglot dataset -- referred to as the ``transpose'' of MNIST. Other works used zero-shot learning on ImageNet, e.g. \cite{norouzi2013zero}. However, there is not much one-shot literature on ImageNet, which we hope to amend via our benchmark and task definitions in the following section.
\iffalse
\bibliography{ref.bib}
\fi

\section{Results}
We perform a variety of experiments across different tasks and neural network architectures in natural language processing as well as image classification. We report our experimental findings on language tasks in section~\ref{sec:NLP}, and image classification
in section~\ref{sec:image_class}. We illustrate that CBS schedules can alleviate sub-optimal initialization in section~\ref{sec:bad_init}. We follow the baseline training method for each task (for details please see Appendix~\ref{sec:training_outline}). 
Alongside testing/validation performance, we also report the number of training iterations (lower values are preferred).

\subsection{Language Results}\label{sec:NLP}

Language modeling is a challenging problem due to the complex and long-range interactions between distant words~\cite{merity2016pointer}.
One hope is that large/deep models might be able to 
capture these complex interactions, but large models easily overfit on these tasks and exhibit large gaps between training set and testing set performance. 
CBS schedules effectively help us avoid overfitting, and in addition snapshot ensembling enables even greater performance.

\begin{table}[!htbp]
\caption{\footnotesize Testing perplexity and number of parameter updates of L1 and L2 models on Penn 
Tree Bank (PTB) and WikiText~2 (WT2) datasets. The best perplexity and lowest number of 
updates are \textbf{bolded}. }
\label{tab:lm-results}
\centering
\begin{tabular}{lcc|cc|cc|cc} \toprule
  &\multicolumn{2}{c}{L1 on PTB}       &\multicolumn{2}{c}{L1 on WT2}    &\multicolumn{2}{c}{L2 on PTB} &\multicolumn{2}{c}{L2 on WT2}\\              
\midrule
Schedule    & {Per.}    & {\# Iters}    & {Per.}    & {\# Iters}    & {Per.}     & {\# Iters}   & {Per.}    & {\# Iters}      \\
\midrule
\Gc	BL\footnotemark	&	83.13	&	52k	&	96.41	&	116k	&	79.34	&	73k	&	99.69	&	164k	\\
\midrule
\Ga	CBS-10	&	80.49	&	49k	&	94.93	&	111k	&	79.37	&	83k	&	95.43	&	187k	\\
\Gc	CBS-5	&	80.78	&	49k	&	\textbf{94.31}	&	111k	&	78.61	&	73k	&	94.32	&	164k	\\
\Ga	CBS-1	&	81.56	&	49k	&	94.52	&	111k	&	77.56	&	69k	&	\textbf{91.78}	&	156k	\\
\midrule
\Gc	CBS-10-A	&	\textbf{80.28}	&	35k	&	95.91	&	79k	&	81.47	&	65k	&	95.28	&	146k	\\
\Ga	CBS-5-A	&	82.03	&	35k	&	95.23	&	79k	&	79.48	&	53k	&	93.63	&	118k	\\
\Gc	CBS-1-A	&	84.41	&	35k	&	95.66	&	79k	&	81.32	&	49k	&	93.19	&	111k	\\
\midrule
\Ga	CBS-10-T	&	80.49	&	49k	&	94.93	&	111k	&	79.42	&	83k	&	94.39	&	187k	\\
\Gc	CBS-5-T	&	80.94	&	53k	&	94.9	&	120k	&	78.95	&	63k	&	94.68	&	142k	\\
\Ga	CBS-1-T	&	81.82	&	46k	&	95.38	&	104k	&	\textbf{77.39}	&	65k	&	93.78	&	147k	\\
\bottomrule 
\end{tabular}
\end{table}

We evaluate a large variety of CBS schedules to positive results as shown in~\tref{tab:lm-results}. 
Results are measured in perplexity, a standard figure of merit for evaluating the quality of language models by measuring its prediction of the empirical distribution of words (lower perplexity value is better). 
As we can see, the best performing CBS schedules result in significant improvements in
perplexity (up to 7.91) over the baseline schedules and also offer reductions in the number of SGD training iterations (up to $33\%$). For example, CBS schedules achieve improvement of 7.91 perplexity improvement on WikiText~2 via CBS-1-T and reduce the SGD iterations from 164k to 111k via the CBS-1-A schedule.
Notice that almost all CBS schedules outperform the baseline schedule.
\begin{figure}[!htbp]
  \centering
\includegraphics[width=.4\textwidth]{fig/train_l2_ptb.pdf}
\includegraphics[width=.4\textwidth]{fig/test_l2_ptb.pdf}
  \caption{\footnotesize Training (left) and testing (right) perplexity as a function of iterations for the L2 model on PTB.}
  \label{fig:l2_ptb}
\end{figure}

\fref{fig:l2_ptb} shows the training and testing perplexity of the L2 model on PTB and WikiTest~2 as trained via the baseline schedule 
along with our best CBS schedule (from \tref{tab:lm-results}). Notice the cyclical spikes in training and testing perplexity. The peaks occur during  decreases in batch size, i.e., increases in noise scale, which could help to escape sub-optimal local minima, and the troughs occur during increases in batch size, i.e., decreases with noise scale.

In order to support our claim that CBS schedules are especially useful for counteracting overfitting, we conducted additional language modeling experiments on models L1', L2' with PTB and WT2 which use significantly lower dropout (0.2 and 0.3) than the original L1, L2 models (0.5 and 0.65). Because these models heavily overfit the training data, we report both the final testing perplexity as well as the best testing perplexity achieve during training. 
As seen in \tref{tab:lm-results-overfit} (in Appendix~\ref{sxn:app:additional results}), with L2' CBS yields improvements of a staggering 60.3 on final testing perplexity and 36.2 on best testing perplexity. CBS yields smaller improvements on L1' of 26.0 and 25.3, which are still much larger than the improvement achieved by CBS on L1 and L2. 


As mentioned above the goal of every cycle is to get an approximate MAP point. A very interesting idea
proposed in~\citep{huang2017snapshot} is to ensemble these MAP points by saving snapshots of the model
at the end of every cycle. We follow that strategy with the only difference that we use a batch size cycle
instead of cyclical learning rate proposed in~\citep{huang2017snapshot} due to higher parallelization opportunities for the former.
We perform experiments on snapshot ensembling with the L2 model with the respective best performing CBS schedules on PTB and WikiText~2 (CBS-1-T and CBS-1), as well as the fixed batch size baseline.
The CBS ensembles on PTB and WikiText~2 result in test set perplexity of 76.14 and 88.47, outperforming baseline ensembles on both datasets (76.52, 89.99 respectively) and CBS single models (77.39, 91.78 respectively).


To further explore the properties of cyclical batch size schedules, we also evaluate these schedules on natural language inference tasks, as shown in~\tref{tab:nli-results}. In our experiments, CBS schedules do not yield large performance improvements on models like E1 which exhibit smaller disparities between training and testing performance.
This is in line with our limitation in that CBS is more effective for models which tend to overfit. On the other hand, we see a large reduction in training 
iterations by up to 62\% which is due to higher effective batch size used in CBS than baseline.


\footnotetext{\citep{zaremba2014recurrent} reports testing perplexity of 82.7 and 78.4 
for L1 and L2 respectively on PTB, which we could not reproduce. The best perplexity and lowest number of updates are \textbf{bolded}.}

\begin{wraptable}{r}{7cm}
\caption{\footnotesize Validation accuracy and number of parameter updates of E1 on MultiNLI and SNLI 
datasets. The best accuracy and lowest number of updates are \textbf{bolded}. }
\label{tab:nli-results}
\centering
\begin{tabular}{lcc|cc} \toprule
  &\multicolumn{2}{c}{MultiNLI}       &\multicolumn{2}{c}{SNLI}    \\              
\midrule
Strategy    & {Acc.}    & {\# Iters}    & {Acc.}    & {\# Iters}    \\
\midrule
\Gc	BL	    &	72.87	&	123k	&	\textbf{86.86}	&	172k	\\
\Ga	CBS-1	&	\textbf{73.17}	&	64k	&	86.73   &   90k   \\
\Gc	CBS-2	&	73.07	&	71k	&	86.56   &   99k	\\
\Ga CBS-1-A &   72.23   & \textbf{48k}     & 86.26     & \textbf{67k}     \\
\Gc CBS-2-A &   72.04   & 57k     & 85.83     & 80k     \\
\bottomrule 
\end{tabular}
\end{wraptable}


\subsection{Image Classification Results}\label{sec:image_class}
We also test our CBS schedules on Cifar-10 and ImageNet. Table.~\ref{tab:cbs_cifar10} reports the testing accuracy and the number of training iterations for different models on Cifar-10. We see that the CBS schedules match baseline performance, but the number of training iterations used in CBS schedules is up to $2\times$ fewer. 

As seen in \fref{fig:wresnet_cifar10}, the training curves of CBS schedules also 
exhibit the aforementioned cyclical spikes both in training loss and testing 
accuracy. Similarly in the previously discussed language experiments, these spikes correspond to cycles in the CBS schedules and can 
be thought of as re-initializations of the neural network weights. We observe that CBS achieves similar performance to the baseline. 


\fref{fig:cbs_imagenet} shows the results of ResNet50 on ImageNet. The baseline trains in  $450k$ 
iterations and reaches $76.134\%$ validation accuracy. With CBS, the final validation accuracy is 
$76.336\%$, trained in $262k$ parameter updates.
CBS outperforms the baseline on both training loss and validation accuracy. 

We offer further support for the hypothesis that CBS schedules are more effective for overfitting neural networks with experiments on model C4, which achieves 94.35\% training 
accuracy 
and 55.55\% testing accuracy on Cifar-10. With CBS-15, we see 90.71\% training 
accuracy 
and 56.44\% testing accuracy, which is a larger improvement than that offered by CBS on convolutional models on Cifar-10.

We also explore combining CBS with the recent adversarial regularization proposed by~\cite{yao2018large}.
Combining CBS-15 on C2 with this strategy improves accuracy to $94.82\%$. This outperforms other schedules shown in  Table~\ref{tab:cbs_cifar10}.
Applying snapshot ensembling on C3 trained with CBS-15-2 leads to improved accuracy of $93.56\%$ as compared to $92.58\%$.
After ensembling ResNet50 on Imagenet with snapshots from the last two cycles, the performance increases to 76.401\% from 75.336\%.

\begin{table}[!htbp]
\caption{\footnotesize Accuracy and number of parameter updates of different models on Cifar-10. The best accuracy and lowest number of iterations are \textbf{bolded}.}
\label{tab:cbs_cifar10}
\centering
\begin{tabular}{lcc|lcc|lcc} \toprule
\multicolumn{3}{c}{AlexNet-like (C1)}  &\multicolumn{3}{c}{WResNet (C2)}  &\multicolumn{3}{c}{ResNet18 (C3)} \\              
\midrule
    {Strategy}                  & {Acc.} & {\# Iters}               &{Strategy}           & {Acc.} & {\# Iters}        &{Strategy}           & {Acc.} & {\# Iters}                     \\
    \midrule
\Gc  Baseline             &86.94            & 35k            & Baseline     &94.53   & 78k                   & Baseline      & \textbf{92.71}  & 63k          \\
\Ga  CBS-10-3          &86.83            & 20k               & CBS-15       &94.46   & 40k                   & CBS-10 & 92.47 & \textbf{32k}         \\
\Gc  CBS-15-2          &86.87            & 26k               & CBS-10-3     &\textbf{94.56}   & 45k          & CBS-5-3  & 92.45 & 37k        \\
\Ga  CBS-5-3           &\textbf{87.03}   & 20k      & CBS-5-3      &94.44   & 45k                   & CBS-15-2 & 92.58 & 48k         \\
\Gc  CBS-5-3-A         &86.75            & \textbf{15k}               & CBS-5-3-A    &94.34   & \textbf{33k}          & CBS-15-2-A & 92.27 & 39k             \\
     \bottomrule 
\end{tabular}
\end{table}



\begin{figure}[!htbp]
  \centering
\includegraphics[width=.4\textwidth]{fig/loss_c1.pdf}
\includegraphics[width=.4\textwidth]{fig/acc_c1.pdf}
  \caption{\footnotesize C2 model (WResNet) on Cifar-10. Training set loss (left), and testing set accuracy (right), evaluated as a function of epochs}

  \label{fig:wresnet_cifar10}
\end{figure}

\begin{figure}[!htbp]
  \centering
\includegraphics[width=.4\textwidth]{fig/loss_iter_I1.pdf}
\includegraphics[width=.4\textwidth]{fig/acc_iter_I1.pdf}
\includegraphics[width=.4\textwidth]{fig/loss_I1.pdf}
\includegraphics[width=.4\textwidth]{fig/acc_I1.pdf}

  \caption{\footnotesize I1 model (ResNet50) on ImageNet. Training set loss (left), and testing set accuracy (right), evaluated as a function of iterations (above) and epochs (below).}
  \label{fig:cbs_imagenet}
\end{figure}

\subsection{Sub-optimal Initialization}\label{sec:bad_init}
Various effective initialization methods~\cite{glorot2010understanding,he2015delving,saxe2013exact,mishkin2015all}
have been proposed previously; however, when presented with new architectures and new tasks,
initialization still needs to be explored empirically and often the final
performance varies greatly with different initializations. In this section, we test if
CBS schedules can alleviate the problem of sub-optimal initialization.

We test a Gaussian initialization with mean $0$ and standard deviation $0.1$ on 
an AlexNet-like model (C1). The baseline (BL) training follows the same setting as described in 
Appendix~\ref{sec:training_outline} and achieves final accuracy $84.27\%$. For CBS, we use cycle width of 10 with 3 steps.
In particular, CBS$_1$ denotes a constant learning rate, and achieves final accuracy $85.41\%$. CBS$_2$ decays the learning rate by a factor of 5 at
epoch 75 and achieves final accuracy $84.95\%$. 
We keep learning rate high during training because a high noise 
level helps $\theta$ escape sub-optimal local minima. 
Notice that all CBS methods achieve better generalization performance than the baseline.
\vspace{-0.09in}
\section{Conclusion}

In this paper we introduced Matching Networks, a new neural architecture that, by way of its corresponding training regime, is capable of state-of-the-art performance on a variety of one-shot classification tasks.
There are a few key insights in this work.
Firstly, one-shot learning is much easier if you train the network to do one-shot learning.
Secondly, non-parametric structures in a neural network make it easier for networks to remember and adapt to new training sets in the same tasks.
Combining these observations together yields Matching Networks.
%
Further, we have defined new one-shot tasks on ImageNet, a reduced version of ImageNet (for rapid experimentation), and a language modeling task.
%
An obvious drawback of our model is the fact that, as the support set $S$ grows in size, the computation for each gradient update becomes more expensive. Although there are sparse and sampling-based methods to alleviate this, much of our future efforts will concentrate around this limitation. Further, as exemplified in the ImageNet dogs subtask, when the label distribution has obvious biases (such as being fine grained), our model suffers. We feel this is an area with exciting challenges which we hope to keep improving in future work.

\section*{Acknowledgements}
We would like to thank Nal Kalchbrenner for brainstorming around the design of the function $g$, and Sander Dieleman and Sergio Guadarrama for their help setting up ImageNet. We would also like thank Simon Osindero for useful discussions around the tasks discussed in this paper, and Theophane Weber and Remi Munos for following some early developments. Karen Simonyan and David Silver helped with the manuscript, as well as many at Google DeepMind. Thanks also to Geoff Hinton and Alex Toshev for discussions about our results.

{\small
\setlength{\bibsep}{0pt plus 1pt}
\bibliography{refs}
\bibliographystyle{plain}}

\section{Positive Definiteness of~$K$}\label{sec:appendixA}
To show that the kernel~$K$ defined in~(\ref{eq:kernel}) is positive definite
(p.d.), we simply use elementary rules from the kernel literature described in
Sections 2.3.2 and 3.4.1 of~\cite{shawe2004}.  A linear combination of p.d. kernels with non-negative weights is also p.d. (see Proposition 3.22
of\cite{shawe2004}), and thus it is sufficient to show that for all $\z,\z'$
in~$\Omega$, the following kernel on $\Omega \to \HH$ is p.d.:
\begin{displaymath}
   (\varphi,\varphi') \mapsto \big\|\varphi(\z)\big\|_\HH  \normH{\varphi'(\z')} e^{-\frac{1}{2\sigma^2} \normH{\tildephi(\z)-\tildephi'(\z')}^2}.
\end{displaymath}
Specifically, it is also sufficient to
show that the following kernel on $\HH$ is p.d.:
\begin{displaymath}
   (\phi,\phi') \mapsto \big\|{\phi}\big\|_\HH  \normH{\phi'} e^{-\frac{1}{2\sigma^2} \normH{\frac{\phi}{\|\phi\|_\HH}-\frac{\phi'}{\|\phi'\|_\HH}}^2}.
\end{displaymath}
with the convention $\phi/\|\phi\|_\HH=0$ if~$\phi=0$.
This is a pointwise product of two kernels and is p.d. when each of the two
kernels is p.d. The first one is obviously p.d.: $(\phi,\phi') \mapsto
\|{\phi}\|_\HH  \normH{\phi'}$. The second one is a composition of the Gaussian
kernel---which is p.d.---, with feature maps $\phi/\|\phi\|_\HH$ of a
normalized linear kernel in~$\HH$.  This composition is p.d. according to
Proposition 3.22, item (v) of~\cite{shawe2004} since the normalization does
not remove the positive-definiteness property.

\section{List of Architectures Reported in the Experiments}\label{appendix:arch}
We present in details the architectures used in the paper in Table~\ref{table:arch}.
\begin{table}[hbtp]
   \centering
   \begin{tabular}{|*{9}{c|}}
      \hline
      Arch. & $N$ & $m_1$  & $p_1$  &  $\gamma_1$ & $m_2$ &  $p_2$ & $S$  &  $\sharp$ param\\
      \hline
      \hline
      \multicolumn{9}{|c|}{MNIST} \\
      \hline
      CKN-GM1 & 2 &  $1 \times 1$  &  12  & 2 &  $3 \times 3$ &  50 &  $4 \times 4$ & $5\,400$\\
      \hline
      CKN-GM2 & 2 &  $1 \times 1$  &  12  & 2 &  $3 \times 3$ &  400 &  $3 \times 3$& $43\,200$ \\
      \hline
      CKN-PM1 & 1 &  $5 \times 5$  &  200  & 2 &  - &  - &  $4 \times 4$  & $5\,000$ \\
      \hline
      CKN-PM2 & 2 &  $5 \times 5$  &  50  & 2 &  $2 \times 2$ &  200 &  $6 \times 6$ & $41\,250$ \\
      \hline
      \hline
      \multicolumn{9}{|c|}{CIFAR-10} \\
      \hline
      CKN-GM & 2 &  $1 \times 1$  &  12  & 2 &  $2 \times 2$ & 800 &  $4 \times 4$ & $38\,400$\\
      \hline
      CKN-PM & 2 &  $2 \times 2$  &  100  & 2 &  $2 \times 2$ &  800 &  $4 \times 4$ & $321\,200$\\
      \hline
      \hline
      \multicolumn{9}{|c|}{STL-10} \\
      \hline
      CKN-GM & 2 &  $1 \times 1$  &  12  & 2 &  $3 \times 3$ & 800 &  $4 \times 4$ & $86\,400$\\
      \hline
      CKN-PM & 2 &  $3 \times 3$  &  50  & 2 &  $3 \times 3$ &  800 &  $3 \times 3$ & $361\,350$\\
      \hline

   \end{tabular}
   \caption{List of architectures reported in the paper. $N$ is the number of layers; $p_1$ and~$p_2$ represent the number of filters are each layer; $m_1$ and~$m_2$ represent the size of the patches~$\NN_1$ and~$\NN_2$ that are of size~$m_1 \times m_1$ and~$m_2 \times m_2$ on their respective feature maps~$\zeta_1$ and~$\zeta_2$; $\gamma_1$ is the subsampling factor between layer 1 and layer 2; $S$ is the size of the output feature map, and the last column indicates the number of parameters that the network has to learn.}
   \label{table:arch}
\end{table}



\end{document}

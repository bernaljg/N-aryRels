\begin{abstract}
Meta-learning has been proposed as a framework to address the challenging few-shot learning setting. The key idea is to leverage a large number of similar few-shot tasks in order to learn how to adapt a base-learner to a new task for which only a few labeled samples are available. As deep neural networks (DNNs) tend to overfit using a few samples only, meta-learning typically uses shallow neural networks (SNNs), thus limiting its effectiveness. 
In this paper we propose a novel few-shot learning method called \textbf{meta-transfer learning (MTL)} which learns to adapt a \textbf{deep NN} for \textbf{few shot learning tasks}. Specifically, \emph{meta} refers to training multiple tasks, and \emph{transfer} is achieved by learning scaling and shifting functions of DNN weights for each task.
In addition, we introduce the \textbf{hard task (HT) meta-batch} scheme as an effective learning curriculum for MTL. We conduct experiments using (5-class, 1-shot) and (5-class, 5-shot) recognition tasks on two challenging few-shot learning benchmarks: miniImageNet and Fewshot-CIFAR100. Extensive comparisons to related works validate that our \textbf{meta-transfer learning} approach trained with the proposed \textbf{HT meta-batch} scheme achieves top performance. An ablation study also shows that both components contribute to fast convergence and high accuracy\footnote{Code: \href{https://github.com/y2l/meta-transfer-learning-tensorflow}{https://github.com/y2l/meta-transfer-learning-tensorflow}}.

% Meta-learning has been proposed as a framework to address the challenging few-shot learning setting. The key idea is to leverage a large number of similar few-shot tasks in order to learn how to adapt a base-learner to a new task for which only a few labeled samples are available. As deep neural networks (DNNs) tend to overfit using a few samples only, meta-learning typically uses shallow neural networks (SNNs), thus limiting its effectiveness. In this paper we propose a novel few-shot learning method called meta-transfer learning (MTL) which learns to adapt a deep NN for few shot learning tasks. Specifically, "meta" refers to training multiple tasks, and "transfer" is achieved by learning scaling and shifting functions of DNN weights for each task. In addition, we introduce the hard task (HT) meta-batch scheme as an effective learning curriculum for MTL. We conduct experiments using (5-class, 1-shot) and (5-class, 5-shot) recognition tasks on two challenging few-shot learning benchmarks: miniImageNet and Fewshot-CIFAR100. Extensive comparisons to related works validate that our meta-transfer learning approach trained with the proposed HT meta-batch scheme achieves top performance. An ablation study also shows that both components contribute to fast convergence and high accuracy.

\end{abstract}

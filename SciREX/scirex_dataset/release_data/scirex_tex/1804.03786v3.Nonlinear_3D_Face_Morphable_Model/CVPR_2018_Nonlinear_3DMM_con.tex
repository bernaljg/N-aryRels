%!TEX root = CVPR_2017_Face_Frontalization.tex
\Section{Conclusions}
\label{sec:con}

Since its debut in 1999, 3DMM has became a cornerstone of facial analysis research % in the vision community,
with applications to many problems.
Despite its impact, it has drawbacks in requiring training data of 3D scans, learning from controlled 2D images, and limited representation power due to linear bases. % for both shape and texture.
These drawbacks could be formidable when fitting 3DMM to unconstrained faces, or learning 3DMM for generic objects such as shoes.
This paper demonstrates that there exists an alternative approach to 3DMM learning, where a nonlinear 3DMM can be learned from a large set of unconstrained face images without collecting 3D face scans.
Further, the model fitting algorithm can be learnt jointly with 3DMM, in an end-to-end fashion.

Our experiments cover a diverse aspects of our learnt model, some of which might need the subjective judgment of the readers.
We hope that both the judgment and quantitative results could be viewed under the context that, unlike linear 3DMM, no genuine 3D scans are used in our learning.
Finally, we believe that unsupervisedly learning 3D models from large-scale in-the-wild 2D images is one promising research direction. 
This work is one step along this direction.
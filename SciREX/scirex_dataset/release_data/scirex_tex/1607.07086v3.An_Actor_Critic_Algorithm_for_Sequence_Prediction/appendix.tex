\section{Hyperparameters}
\label{sec:hyperparameters}

For machine translation experiments the variance penalty coefficient $\lambda$
was set to $10^{-4}$, and the delay coefficients $\gamma_{\theta}$ and
$\gamma_{\phi}$ were both set to $10^{-4}$. For REINFORCE with the critic we did not use 
a delayed actor, i.e. $\gamma_{\theta}$ was set to 1. For the spelling correction
task we used the same $\gamma_{\theta}$ and $\gamma_{\phi}$ but a different 
$\lambda=10^{-3}$. When we used a combined training criterion, the weight of the log-likelihood gradient $\lambda_{LL}$ was always 0.1. All initial weights were sampled from a centered uniform distribution with
width $0.1$.

In some of our experiments we provided the actor states as additional inputs to
the critic.  Specifically, we did so in our spelling correction experiments and
in our WMT 14 machine translation study. All the other results were obtained
without this technique.

For decoding with beam search we substracted the length of a candidate
times $\rho$ from the log-likelihood cost. The exact value of $\rho$ 
was selected on the validation set and was equal to $0.8$
for models trained by log-likelihood and REINFORCE and to $1.0$
for models trained by actor-critic and REINFORCE-critic.

For some of the hyperparameters we performed an ablation study. The results are reported in Table \ref{tab:ablation}.

\begin{table}
    \caption{Results of an ablation study. We tried varying the actor update speed $\gamma_{\theta}$, 
             the critic update speed $\gamma_{\phi}$, the value penalty coefficient $\lambda$, 
             whether or not reward shaping is used, whether or not temporal difference (TD) learning is used
             for the critic. Reported are the best training and validation BLEU score obtained in the course
             of the first 10 training epochs. Some of the validation scores would still improve with longer training.
             Greedy search was used for decoding.}
    \centering
    \begin{tabular}{c  c  c  c  c  c c}
        $\gamma_{\theta}$ & $\gamma_{\phi}$ & $\lambda$ & reward shaping & TD & train BLEU & valid BLEU \\
        \hline
        \multicolumn{7}{c}{baseline} \\
        % ted16a & 
        % 0.001 & 0.001 & $10^{-3}$ & yes & yes & 33.66 & 23.01 \\
        % \hline
        % ted16k & 
        0.001 & 0.001 & $10^{-3}$ & yes & yes & 33.73 & 23.16 \\
        \hline
        \multicolumn{7}{c}{with different $\gamma_{\phi}$} \\
        % ted16f & 
        0.001 & 0.01 & $10^{-3}$ & yes & yes & 33.52 & 23.03 \\
        % ted16g & 
        0.001 & 0.1 & $10^{-3}$ & yes & yes & 32.63 & 22.80 \\
        % ted16l & 
        0.001 & 1 & $10^{-3}$ & yes & yes & 9.59 & 8.14 \\
        \hline
        \multicolumn{7}{c}{with different $\gamma_{\theta}$} \\
        % ted19 & 
        1 & 0.001 & $10^{-3}$ & yes & yes & 32.9 & 22.88 \\
        \hline
        \multicolumn{7}{c}{without reward shaping} \\
        % ted16o2 &
        0.001 & 0.001 & $10^{-3}$ & no & yes & 32.74 & 22.61 \\
        \hline
        \multicolumn{7}{c}{without temporal difference learning} \\
        % ted16i2 & 
        0.001 & 0.001 & $10^{-3}$ & yes & no & 23.2 & 16.36 \\
        \hline
        \multicolumn{7}{c}{with different $\lambda$} \\
        % ted16n & 
        0.001 & 0.001 & $3 * 10^{-3}$ & yes & yes & 32.4 & 22.48 \\
        % ted16m2 & 
        0.001 & 0.001 & $10^{-4}$ & yes & yes & 34.10 & 23.15 \\
        % ted16m4 & 
        0.001 & 0.001 & $10^{-6}$ & yes & yes & 35.00 & 23.10 \\
        % ted16m7 & 
        0.001 & 0.001 & $10^{-8}$ & yes & yes & 33.6 & 22.72 \\
        % ted16m5 & 
        0.001 & 0.001 & $0$ & yes & yes & 27.41 & 20.55 \\
    \end{tabular}
    \label{tab:ablation}
\end{table}

\section{Data}
\label{sec:data}
For the IWSLT 2014 data the sizes of validation and tests set were 6,969 and 6,750, 
respectively. We limited the number of words in the English and German
vocabularies to the 22,822 and 32,009 most frequent words, respectively, and
replaced all other words with a special token. The maximum sentence length
in our dataset was 50. For WMT14 we used vocabularies of 30,000 words for both English and French, and the maximum sentence length was also 50.


\section{Generated Q-values}

In Table \ref{tab:qvalues} we provide an example of value predictions
that the critic outputs for candidate next words. One can see that the critic has
indeed learnt to assign larger values for the appropriate next words.
While the critic does not always produce sensible estimates and can often predict
a high return for irrelevant rare words, this is greatly reduced using the variance
penalty term from Equation \eqref{eq:variance_penalty}.

\begin{figure}[h]%{r}{0.7\textwidth}
\centering
\label{tab:qvalues}
\caption{The best 3 words according to the critic at intermediate
        steps of generating a translation. The numbers in parentheses
        are the value predictions $\hat{Q}$. The German original is 
        ``\"{u}ber eine davon will ich hier erz\"{a}hlen .''
        The reference translation is 
        ``and there's one I want to talk about''.}

\begin{tabular}{c|c}
    \textrm{Word} & \textrm{Words with largest $\hat{Q}$} \\
    \hline
    % BOS  &
    %#,(3.582) and(3.529) \&apos;m(3.529) \\
    one &
    and(6.623) there(6.200) but(5.967) \\
    of &
    that(6.197) one(5.668) \&apos;s(5.467) \\
    them &
    that(5.408) one(5.118) i(5.002) \\
    i &
    that(4.796) i(4.629) ,(4.139) \\
    want &
    want(5.008) i(4.160) \&apos;t(3.361) \\
    to &
    to(4.729) want(3.497) going(3.396) \\
    tell &
    talk(3.717) you(2.407) to(2.133) \\
    you &
    about(1.209) that(0.989) talk(0.924) \\
    about &
    about(0.706) .(0.660) right(0.653) \\
    here &
    .(0.498) ?(0.291) --(0.285) \\
    . &
    .(0.195) there(0.175) know(0.087) \\
     $\emptyset$  &
    .(0.168)  $\emptyset$ (-0.093) ?(-0.173) \\
\end{tabular}
\end{figure}

\newpage

\section{Proof of Equation \eqref{eq:ac}}

\begin{align*}
    \frac{dV}{d\theta} = 
    \frac{d}{d\theta} \Exp_{\hat{Y} \sim p(\hat{Y})} R(\hat{Y}) =
    \sum\limits_{\hat{Y}}
    \frac{d}{d\theta}
    \left[
    p(\hat{y}_1) p(\hat{y}_2|\hat{y}_1) 
    \ldots
    p(\hat{y}_T|\hat{y}_1 \ldots \hat{y}_{T-1})
    \right]
    R(\hat{Y}) = \notag\\
    \sum\limits_{t=1}^T
    \sum\limits_{\hat{Y}}
    p(\hat{Y}_{1 \ldots t-1})
    \frac{dp(\hat{y}_t|\hat{Y}_{1 \ldots t - 1})}{d\theta}
    p(\hat{Y}_{t+1 .. T}|\hat{Y}_{1 \ldots t})
    R(\hat{Y}) = \notag\\
    \begin{aligned}
        & \sum\limits_{t=1}^T
          \sum\limits_{\hat{Y}_{1 \ldots t}}
          p(\hat{Y}_{1 \ldots t-1})
          \frac{dp(\hat{y}_t|\hat{Y}_{1 \ldots t - 1})}{d\theta}
        &  \sum\limits_{\hat{Y}_{t + 1 \ldots T}}
           p(\hat{Y}_{t+1 .. T}|\hat{Y}_{1 \ldots t})
           \sum\limits_{\tau=1}^T 
           r_{\tau}(\hat{y}_{\tau}; \hat{Y}_{1 \ldots \tau - 1})
    \end{aligned} 
    =\notag\\
    \begin{aligned}
        & \sum\limits_{t=1}^T
          \sum\limits_{\hat{Y}_{1 \ldots t}}
          p(\hat{Y}_{1 \ldots t-1})
          \frac{dp(\hat{y}_t|\hat{Y}_{1 \ldots t - 1})}{d\theta} \\
        &  \left[
          r_t(\hat{y}_t; \hat{Y}_{1 \ldots t - 1}) +
          \sum\limits_{\hat{Y}_{t + 1 \ldots T}}
           p(\hat{Y}_{t+1 .. T}|\hat{Y}_{1 \ldots t})
           \sum\limits_{\tau=t+1}^T 
           r_{\tau}(\hat{y}_{\tau}; \hat{Y}_{1 \ldots \tau - 1})
           \right]
    \end{aligned} 
    = \notag\\
    \sum\limits_{t=1}^T
    \Exp_{\hat{Y}_{1 \ldots t - 1} \sim p(\hat{Y}_{1 \ldots t - 1})}
    \sum\limits_{a \in A}
    \frac{dp(a|\hat{Y}_{1 \ldots t - 1})}{d\theta}
    Q(a;\hat{Y}_{1 \ldots t - 1}) 
    = \notag\\
    \Exp_{\hat{Y} \sim p(\hat{Y})} 
    \sum\limits_{t=1}^T
    \sum\limits_{a \in \mathcal{A}}
    \frac{dp(a|\hat{Y}_{1 \ldots t - 1})}{d\theta}
    Q(a;\hat{Y}_{1 \ldots t - 1})
\end{align*}    


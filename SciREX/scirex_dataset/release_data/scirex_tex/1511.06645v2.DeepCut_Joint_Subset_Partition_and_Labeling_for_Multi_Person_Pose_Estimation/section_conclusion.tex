\section{Conclusion}
Articulated pose estimation of multiple people in uncontrolled real
world images is challenging but of real world interest. In this work,
we proposed a new formulation as a joint subset partitioning and
labeling problem (SPLP). Different to previous two-stage strategies
that separate the detection and pose estimation steps, the SPLP model
jointly infers the number of people, their poses, spatial proximity,
and part level occlusions. Empirical results on four diverse and
challenging datasets show significant improvements over all previous
methods not only for the multi person, but also for the single person
pose estimation problem. On multi person WAF dataset we improve by
$30$\% PCP over the traditional two-stage approach. This shows that a
joint formulation is crucial to disambiguate multiple and potentially
overlapping persons. Models and code available at \myurl.

%% \pg{which
%%   is
%% \pg{exclude:
%%   ,which shows the effectiveness of our model.}  \pg{include the
%%   message of the paper, for example: This paper finds that single
%%   person approaches can be used for the multi-person case only to some
%%   extend. A joint formulation of the problem is crucial to disambiguate
%%   multiple and potential overlapping persons.  We have demonstrated
%%   that the SPLP model, by integrating meaningful constraints severely
%%   improves on this important task.}


%% For example on the upper body pose estimation our
%% approach improves by $9.3$ percent points of mAP over the strong
%% CNN-based part detector, which is more than 30\% relative
%% improvement. In the future we plan to improve the performance of our
%% approach by employing more powerful unary and pairwise terms.
%% \pg{That is probably the weakest sentence to end this paper
%%   with. Nobody will ask for future extensions, especially not the most
%%   boring ones.}

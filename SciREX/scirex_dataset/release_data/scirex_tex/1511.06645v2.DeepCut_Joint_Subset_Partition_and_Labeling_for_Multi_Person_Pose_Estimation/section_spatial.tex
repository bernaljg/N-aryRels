\section{Pairwise Probabilities}
\label{sec:unary-and-pairwise}
Here we describe the estimation of the pairwise terms.
% , while unary probabilities are described in
% Sec.~\ref{sec:unary-prob}.
We define pairwise features $f_{dd'}$ for
the variable $z_{dd'cc'}$ (Sec.~\ref{section:problem}).
% where $dd' \in \tbinom{D}{2}$ and $cc'\in C^2$.
Each part detection $d$ includes the probabilities $f_{p_{dc}}$
(Sec.~\ref{sec:unary-prob}), its location $(x_d,y_d)$, scale $h_d$ and
bounding box $B_d$ coordinates.  Given two detections $d$ and $d'$,
and the corresponding features $( f_{p_{dc}}, x_d, y_d, h_d, B_d)$ and
$(f_{p_{d'c}}, x_{d'}, y_{d'}, h_{d'}, B_{d'})$, we define two sets of
auxiliary variables for $z_{dd'cc'}$, one set for $c = c'$ (same body
part class clustering) and one for $c\neq c'$ (across two body part
classes labeling).  These features capture the proximity, kinematic
relation and appearance similarity between body parts.
% \pgcomment{These are designed as to capture...}

\myparagraph{The same body part class ($c=c'$).}
% \paragraph{$c = c'$} We leverage the spatial relations between the
%detections $d$ and $d'$ for the pairwise term within the same body
%part class.
Two detections denoting the same body part of the same person should
be in close proximity to each other. We introduce the following
auxiliary variables that capture the spatial relations: $\Delta x =
|x_d-x_{d'}|/\bar{h}$, $\Delta y = |y_d-y_{d'}|/\bar{h}$, $\Delta h =
|h_d-h_{d'}|/\bar{h}$, $IOUnion$, $IOMin$, $IOMax$. The latter three
are intersections over union/minimum/maximum of the two detection
boxes, respectively, and $\bar{h} = (h_d + h_{d'})/2$.
%% \begin{alignat}{4}
%% \small
%% \smallskip
%%   \Delta x &= \frac{|x_d-x_{d'}|}{\bar{h}}   &\quad IOUnion &= \frac{|B_d \cap B_{d'}|}{|B_d \cup B_{d'}|}     \nonumber \\
%%   \Delta y &= \frac{|y_d-y_{d'}|}{\bar{h}}  &\quad IOMin &= \frac{|B_d \cap B_{d'}|}{\min(|B_d|, |B_{d'}|)}     \\
%%   \Delta h &= \frac{|h_d-h_{d'}|}{\bar{h}}   &\quad IOMax &= \frac{|B_d \cap B_{d'}|}{\max(|B_d|, |B_{d'}|)} \nonumber
%% \end{alignat}
%% where $\bar{h} = \frac{(h_d + h_{d'})}{2}$, $IOUnion$, $IOMin$ and
%% $IOMax$ are intersections over union/minimum/maximum of the two
%% detection bounding box areas, respectively.

{\it Non-linear Mapping.}
We augment the feature representation by appending quadratic and exponential terms.
The final pairwise feature  $f_{dd'}$ for the variable $z_{dd'cc}$ is
$(\Delta x, \Delta y, \Delta h,IOUnion, IOMin, IOMax,{(\Delta x)}^2, \\ \ldots, {(IOMax)}^2, \exp{(-{\Delta x})}, \ldots, \exp{(-{IOMax})})$.

\myparagraph{Two different body part classes ($c\neq c'$).}
% \paragraph{$c\neq c'$}
We encode the kinematic body constraints into the pairwise feature by
introducing auxiliary variables $ S_{dd'}$ and $R_{dd'}$, where
$S_{dd'}$ and $R_{dd'}$ are the Euclidean distance and the angle
between two detections, respectively. To capture the joint
distribution of $S_{dd'}$ and $R_{dd'}$, instead of using $S_{dd'}$
and $R_{dd'}$ directly, we employ the posterior probability
$p(z_{dd'cc'} = 1|S_{dd'},R_{dd'})$ as pairwise feature for
$z_{dd'cc'}$ to encode the geometric relations between the body part
class $c$ and $c'$.  More specifically, assuming the prior probability
$p(z_{dd'cc'} = 1) = p(z_{dd'cc'} = 0) = 0.5$, the posterior
probability of detection $d$ and $d'$ have the body part label $c$ and
$c'$, namely $z_{dd'cc'} = 1$, is
\vspace{-0.1cm}
\begin{align}
\small
\noindent
&p(z_{dd'cc'} = 1|S_{dd'},R_{dd'}) \nonumber \\ &=
\frac{p(S_{dd'},R_{dd'}|z_{dd'cc'} = 1)}{p(S_{dd'},R_{dd'}|z_{dd'cc'}
  = 1) + p(S_{dd'},R_{dd'}|z_{dd'cc'} = 0)}, \nonumber
\enspace
\end{align}
\vspace{-0.1cm}

\noindent where $p(S_{dd'},R_{dd'}|z_{dd'cc'} = 1)$ is obtained by
conducting a normalized 2D histogram of $S_{dd'}$ and $R_{dd'}$ from
positive training examples, analogous to the negative likelihood
$p(S_{dd'},R_{dd'}|z_{dd'cc'} = 0)$. In Sec.~\ref{sec:multicut:single}
we also experiment with encoding the appearance into the pairwise
feature by concatenating the feature $f_{p_{dc}}$ from $d$ and
$f_{p_{d'c}}$ from $d'$, as $f_{p_{dc}}$ is the output of the CNN-based part
detectors. The final pairwise feature is $(p(z_{dd'cc'} =
1|S_{dd'},R_{dd'}),f_{p_{dc}},f_{p_{d'c}})$.

%Regarding to the unary feature, we implicitly employ the CNN feature
%since we directly use the R-CNN softmax output as the probability
%estimation for $x_{dc}$, which is trained based on the CNN feature.
%We discuss it in the next section.
\subsection{Probability Estimation} % {\todo{should it be here?}}}
\label{sec:probability-estimation}
% the unary cost $\alpha_{dc}$ and the pairwise cost $\beta_{dd'cc'}$
% are defined in terms of the probability ratio.
The coefficients $\alpha$ and $\beta $ of the objective function
(Eq.~\ref{eq:ilp}) are defined by the probability ratio
in the log space (Eq.~\ref{eq:cost-definition-1} and
Eq.~\ref{eq:cost-definition-2}).  Here we describe the estimation of
the corresponding probability density: {\it (1)} For every pair of
detection and part classes, namely for any $(d, c) \in D \times C$, we
estimate a probability $p_{dc} \in (0,1)$ of the detection $d$ being a
body part of class $c$.  {\it (2)} For every combination of two
distinct detections and two body part classes, namely for any $dd' \in
\tbinom{D}{2}$ and any $cc' \in C^2$, we estimate a probability
$p_{dd'cc'} \in (0,1)$ of $d$ and $d'$ belonging to the same person,
meanwhile $d$ and $d'$ are body parts of classes $c$ and $c'$,
respectively.

\myparagraph{Learning.} 
%% The output of the R-CNN softmax layer
%% contains the jointly estimated probability of the detection $d$ being
%% each body part class $c$, which we directly use as the probability
%% estimation for $p_{dc}$.  Learning the parameters of the softmax
%% classifier from the training data is done in the R-CNN framework in
%% the standard way.
%%As for $p_{dd'cc'}$, 
Given the features $f_{dd'}$ and a Gaussian prior $p(\theta_{cc'}) =
\mathcal{N}(0,\sigma^2)$ on the parameters, logistic model is
%logistic model
\begin{align}
\label{eq:z-probability}
p(z_{dd'cc'} = 1 | f_{dd'}, \theta_{cc'}) & = \frac{1}{1 + \exp(-\langle \theta_{cc'}, f_{dd'} \rangle)}.
\end{align}
$(|C|\times (|C| + 1))/2 $ parameters are estimated using ML.
%Maximum Likelihood. % on the training set.

%}{Estimating the maximally probable model parameter $\theta_{cc'}$
%  from the training data is formulated as s logistic regression
%  problem. We perform the logistic regression for learning the model
%  parameter $\theta_{cc'}$ for every pair of body part classes, in
%  total $(|C|\times (|C| + 1))/2 $.}
\myparagraph{Inference}
Given two detections $d$ and $d'$,
the coefficients $\alpha_{dc}$ for $x_{dc}$ and $\alpha_{d'c}$ for $x_{d'c}$ are obtained by Eq.~\ref{eq:cost-definition-1},
the coefficient $\beta_{dd'cc'}$ for $z_{dd'cc'}$ has the form
\begin{align}
\small
  \beta_{dd'cc'} = \log \frac{1 - p_{dd'cc'}}{p_{dd'cc'}} = -\langle f_{dd'},  \theta_{cc'}\rangle.
  \label{cost-probability-feature}
\end{align}
%with pairwise features $f_{dd'}$, and 
%learned model parameters $\theta_{cc'}$ from logistic regression.
Model parameters $\theta_{cc'}$ are learned using logistic regression.
% \subsection{Implementation Details}
% \label{sec:features:details}
% Using all $2000$ detections in our model is prohibitive due to
% exponential complexity of the fully connected part graph. We thus
% select a representative subset of detections

% \begin{itemize}

% \item selecting representative set of unaries
% \item min probability for unaries
% \end{itemize}

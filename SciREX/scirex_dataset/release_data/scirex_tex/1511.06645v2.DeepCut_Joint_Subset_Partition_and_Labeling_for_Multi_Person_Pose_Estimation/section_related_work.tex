\myparagraph{Related work.} 
Most work on pose estimation targets the 
single person case. Methods progressed from
simple part detectors and elaborate body models
\cite{Ren:2005:RHB,Ramanan:2006:LPI,Jiang:2008:GPE} to tree-structured
pictorial structures (PS) models with strong
part detectors
\cite{pishchulin13iccv,yang12pami,chen14nips,sapp13cvpr}. 
Impressive results are obtained predicting locations
of parts with convolutional neural networks (CNN)~\cite{Toshev:2014:DHP,Tompson:2015:EOL}. 
%For example
%\cite{Tompson:2015:EOL} proposes a model that does not rely on
%explicit body modeling, and instead encodes appearance of part
%configurations via a convolutional multi-scale image representation.
While body models are not a necessary component for effective
part localization, constraints among parts allow to assemble
independent detections into body configurations as 
demonstrated in~\cite{chen14nips} by combining CNN-based body part
detectors with a body model \cite{yang12pami}.
% while primarily
%focusing on the single-person case.

A popular approach to multi-person pose estimation is to detect people first and then estimate body
pose independently 
\cite{Sun:2011:APM,pishchulin12cvpr,yang12pami,Gkioxari:2014:UKP}. \cite{yang12pami} proposes a
flexible mixture-of-parts model for detection and pose estimation. 
\cite{yang12pami} obtains multiple pose hypotheses 
%by computing locations of body limbs 
corresponding to
different root part positions %(\eg , torso) 
and then performing non-maximum suppression.
\cite{Gkioxari:2014:UKP} detects people using a flexible configuration of
poselets and  
%. Given a person detection 
the body pose is predicted as a weighted average 
%over predictions
%of each activated 
of activated poselets. \cite{pishchulin12cvpr} detects people and then predicts poses of each
person using a PS model. \cite{Belagiannis:2014:3DP} estimates poses of multiple
people in 3D by constructing a shared space of 3D body part hypotheses, but uses 2D person detections
to establish the number of people in the scene. 
%While aiming to predict poses of multiple people these
These approaches are limited to cases with people sufficiently far from each other that do not have
overlapping body parts.

Our work is closely related to \cite{eichner10eccv,Ladicky:2013:HPE}
who also propose a joint objective to estimate poses of multiple
people. \cite{eichner10eccv} proposes a multi-person PS model that
explicitly models depth ordering and person-person occlusions. Our
formulation is not limited by a number of occlusion states among
people. \cite{Ladicky:2013:HPE} proposes a joint model for pose
estimation and body segmentation coupling pose estimates of
individuals by image segmentation.
\cite{eichner10eccv,Ladicky:2013:HPE} uses a person detector
to generate initial hypotheses for the joint
model. \cite{Ladicky:2013:HPE} resorts to a greedy approach of adding
one person hypothesis at a time until the joint objective can be
reduced, whereas our formulation can be solved with a certified optimality gap. In
addition \cite{Ladicky:2013:HPE} relies on expensive labeling of body
part segmentation, which the proposed approach does not require.

Similarly to \cite{Chen:2015:POC} we aim to distinguish between
visible and occluded body parts. \cite{Chen:2015:POC} primarily
focuse on the single-person case and handles multi-person scenes
akin to \cite{yang12pami}. 
% by performing part-based non-maximum
%suppression on the set of pose estimates. 
We consider the
 more difficult problem of full-body pose estimation,
whereas \cite{eichner10eccv,Chen:2015:POC} focus on upper-body poses
and consider a simplified case of people seen from the front.

Our work is related to early work on pose estimation that also relies
on integer linear programming to assemble candidate body part
hypotheses into valid configurations \cite{Jiang:2008:GPE}. Their
single person method employs a tree graph augmented with weaker
non-tree repulsive edges and expects the same number of parts. In
contrast, our novel formulation relies on fully connected model
%is proposed that extends
%beyond \cite{Jiang:2008:GPE} 
to deal with unknown number of people per image and body parts per
person.
%%  and multiple people, deal with a
%% variable number of parts and multiple candidate hypotheses per body
%% part.

The Minimum Cost Multicut Problem
\cite{chopra-1993,deza-1997},
known in machine learning as correlation clustering
\cite{bansal-2004},
has been used in computer vision for image segmentation 
\cite{alush-2012,andres-2011,kim-2014,yarkony-2012}
but has not been used before in the context of pose estimation.
It is known to be NP-hard
\cite{demaine-2006}.


%\cite{Gkioxari:2014:UKP} use agglomerative clustering to generate a set of final person detections
%from an initial candidate set. %This is similar to our approach (TODO).

% \begin{itemize}
% \item approaches that consider one person only and don't model multiple people jointly, assume that location/scale known 

% \item recent Yuille CVPR'15 - they do occlusion only, we focus on inferring poses of multiple people
%   - our model handles occlusions as well.

% \item We are the Family - Eichner ECCV, we deal with more complex poses, build on a stronger detector

% \item Leonid's CVPR'12, we jointly reason about multiple people, they first detect and then estimate
%   people poses.

% \item joint detection and pose estimation, Ramanan, Savarese, Gkioxari

% \item Ladicky CVPR'13 - similar idea, but based on segmentation, greedy selection of number of candidate person hypothesis. require segmentation labels at training time !!!

% \item Belagianis CVPR'13, multiple people in 3D, but greedy two step process (detect first)
% \end{itemize}

% Relation to agglomerative clustering used in Gkioxari et al.

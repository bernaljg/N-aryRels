\section{The Label Bias Problem}
\label{sec:label_bias}

Intuitively, we would like the model to be able to revise an earlier
decision made during search, when later evidence becomes available
that rules out the earlier decision as incorrect. At first glance, it
might appear that a locally normalized model used in conjunction with
beam search or exact search is able to revise earlier
decisions. However the label bias problem 
(see \newcite{bottou},
\newcite{collins99}
pages 222-226,
\newcite{crf},
\newcite{bottou-lecun-2005}, 
\newcite{smithJohnson07}) means that locally normalized models often have a very
weak ability to revise earlier decisions.
%% Our experiments show consistent improvements in accuracy for globally
%% normalized models over locally normalized models with beam search; we
%% suspect that the label bias problem is a major source of this
%% difference.

This section gives a formal perspective on the label bias problem,
through a proof that globally normalized models are strictly more
expressive than locally normalized models. The theorem was originally
proved\footnote{More precisely \newcite{smithJohnson07} prove the
  theorem for models with potential functions of the form
  $\score(d_{i-1}, d_i, x_i)$; the generalization to potential
  functions of the form $\score(d_{1:i-1}, d_i, x_{1:i})$ is
  straightforward.}  by \newcite{smithJohnson07}.
The example
underlying the proof gives a clear illustration of the label bias
problem.\footnote{\newcite{smithJohnson07} cite Michael Collins
as the source of the example underlying the proof.  
Note that the theorem refers to {\em conditional} models of the form
$p(d_{1:n} | x_{1:n})$ with global or local normalization.
Equivalence (or non-equivalence) results for {\em joint} models of the
form $p(d_{1:n}, x_{1:n})$ are quite different: for example
results from \newcite{chi99} and \newcite{abneyEtal99} imply that weighted
context-free grammars (a globally normalized joint model) and
probabilistic context-free grammars (a locally normalized joint model)
are equally expressive.}


\paragraph{Global Models can be Strictly More Expressive than
Local Models}
Consider a tagging problem where the task is to map an input sequence
$x_{1:n}$ to a decision sequence $d_{1:n}$.
First, consider a locally normalized model
where we restrict the scoring function to access only the first
$i$ input symbols $x_{1:i}$ when scoring decision $d_i$.
We will return to this restriction soon.
The scoring function $\score$ can be an otherwise arbitrary function of the
tuple $\langle d_{1:i-1}, d_i, x_{1:i} \rangle$:
\begin{align*}
p_L(d_{1:n} | x_{1:n})
&=\prod_{i=1}^n p_L(d_i | d_{1:i-1}, x_{1:i}) \\
&= \frac{\exp \sum_{i=1}^n \score(d_{1:i-1},d_i, x_{1:i}) }
{\prod_{i=1}^n Z_L(d_{1:i-1}, x_{1:i})}.
\end{align*}

Second, consider a globally normalized model
\begin{align*}
&p_G(d_{1:n} | x_{1:n}) 
= \frac{\exp \sum_{i=1}^n \score(d_{1:i-1},d_i, x_{1:i}) }
{Z_G(x_{1:n})}.
\end{align*}
This model again makes use of a scoring function $\score(d_{1:i-1},
d_i, x_{1:i})$ restricted to the first $i$ input symbols when
scoring decision $d_i$.

Define ${\cal P}_L$ to be the set of all possible distributions
$p_L(d_{1:n} | x_{1:n})$ under the local model obtained
as the scores $\score$ vary. Similarly, define ${\cal P}_G$ to be the
set of all possible distributions $p_G(d_{1:n} | x_{1:n})$
under the global model. Here a ``distribution'' is a function
from a pair $(x_{1:n}, d_{1:n})$ to a probability
$p(d_{1:n} | x_{1:n})$.
Our main result is the following:

\begin{theorem} See also \newcite{smithJohnson07}.
%\hspace*{1cm}%\newline
\hspace*{0.5cm}${\cal P}_L$ is a strict subset of ${\cal P}_G$, that is ${\cal P}_L
  \subsetneq {\cal P}_G$.
\end{theorem}

To prove this we will first prove that ${\cal P}_L \subseteq {\cal
  P}_G$. This step is straightforward. We then show that
${\cal P}_G \nsubseteq {\cal P}_L$; that is, there are distributions
in ${\cal P}_G$ that are not in ${\cal P}_L$.
The proof that ${\cal P}_G \nsubseteq {\cal P}_L$ gives a clear
illustration of the label bias problem.


{\em Proof that ${\cal P}_L \subseteq {\cal P}_G$:} 
We need to show that for any locally normalized
distribution $p_L$, we can construct a globally normalized model $p_G$
such that $p_G = p_L$.
Consider a locally normalized model with scores
$\score(d_{1:i-1}, d_i, x_{1:i})$.
Define a global model $p_G$ with scores
\[
\score'(d_{1:i-1}, d_i, x_{1:i}) = 
\log p_L(d_i | d_{1:i-1}, x_{1:i}).
\]
Then it is easily verified that 
\[p_G(d_{1:n} | x_{1:n}) = 
p_L(d_{1:n} | x_{1:n}) \]
for all $x_{1:n}, d_{1:n}$.
$\qed$

In proving ${\cal P}_G \nsubseteq {\cal P}_L$ we will use a
simple problem where every example seen in training or test data is
one of the following two tagged sentences:
\begin{align}
x_1 x_2 x_3 = \hbox{a b c}, \;\;d_1 d_2 d_3 = \hbox{A B C}
\nonumber \\
x_1 x_2 x_3 = \hbox{a b e}, \;\;d_1 d_2 d_3 = \hbox{A D E}
\label{eq:lbexample}
\end{align}

Note that the input $x_2 = \hbox{b}$ is ambiguous: it can take tags
$\hbox{B}$ or $\hbox{D}$. This ambiguity is resolved when the next
input symbol, {\tt c} or {\tt e}, is observed.

Now consider a globally normalized model, where the scores
$\score(d_{1:i-1}, d_i, x_{1:i})$ are defined as follows. 
Define ${\cal T}$ as the set $\{ (A, B), (B, C), (A, D), (D, E)\}$
of bigram tag transitions seen in the data. Similarly, define ${\cal
  E}$ as the set $\{ (a, A), (b, B), (c, C), (b, D), (e, E)\}$ of
(word, tag) pairs seen in the data. We define 
\begin{align}
&\score(d_{1:i-1}, d_i, x_{1:i})
\label{eq:alpha} \\
&=\alpha \times \llbracket(d_{i-1}, d_i) \in {\cal T} \rrbracket
+ \alpha \times \llbracket (x_i, d_i) \in {\cal E} \rrbracket
\nonumber
\end{align}
where $\alpha$ is the single scalar parameter of the model,
and $\llbracket \pi \rrbracket = 1$ if $\pi$ is true, $0$ otherwise.


{\em Proof that ${\cal P}_G \nsubseteq {\cal P}_L$:} We
will construct a globally normalized model $p_G$ such that there is
no locally normalized model such that $p_L = p_G$.

Under the definition in Eq.~\eqref{eq:alpha},
it is straightforward to show that
\begin{align*}\small
\lim_{\alpha \rightarrow \infty} p_G(\hbox{A B C} | \hbox{a b c}) =
\lim_{\alpha \rightarrow \infty} p_G(\hbox{A D E} | \hbox{a b e}) = 1 .
%% \lim_{\rule{0pt}{1.5ex}\mathclap{\alpha \rightarrow \infty}} p_G(\hbox{A B C} | \hbox{a b c}) 
%% =
%% \lim_{\rule{0pt}{1.5ex}\mathclap{\alpha \rightarrow \infty}} p_G(\hbox{A D E} | \hbox{a b e}) = 1 .
\end{align*}

In contrast, under {\em any} definition for $\score(d_{1:i-1}, d_i,
x_{1:i})$, we must have
\begin{equation}
p_L(\hbox{A B C} | \hbox{a b c}) + 
p_L(\hbox{A D E} | \hbox{a b e}) \leq 1
\label{eq:localbad}
\end{equation}
This follows because $p_L(\hbox{A B C} | \hbox{a b c})
= p_L(\hbox{A} | \hbox{a}) \times
p_L(\hbox{B} | \hbox{A}, \hbox{a b}) \times p_L(\hbox{C} | \hbox{A B},
\hbox{a b c})$
and $p_L(\hbox{A D E} | \hbox{a b e})
= p_L(\hbox{A} | \hbox{a}) \times
p_L(\hbox{D} | \hbox{A}, \hbox{a b}) \times
p_L(\hbox{E} | \hbox{A D}, \hbox{a b e})$.
%% \begin{align}
%% &p_L(\hbox{A B C} | \hbox{a b c}) \nonumber \\
%% &= p_L(\hbox{A} | \hbox{a}) \times 
%% p_L(\hbox{B} | \hbox{A}, \hbox{a b}) \times 
%% p_L(\hbox{C} | \hbox{A B}, \hbox{a b c}) \nonumber
%% \end{align} 
%% and
%% \begin{align}
%% &p_L(\hbox{A D E} | \hbox{a b e}) \nonumber \\
%% &= p_L(\hbox{A} | \hbox{a}) \times 
%% p_L(\hbox{D} | \hbox{A}, \hbox{a b}) \times 
%% p_L(\hbox{E} | \hbox{A D}, \hbox{a b e}) \nonumber
%% \end{align} 
The inequality $p_L(\hbox{B} | \hbox{A}, \hbox{a b}) + p_L(\hbox{D} |
\hbox{A}, \hbox{a b}) \leq 1$ then immediately implies
Eq.~\eqref{eq:localbad}.

It follows that for sufficiently large values of $\alpha$, we have 
$p_G(\hbox{A B C} | \hbox{a b c}) + p_G(\hbox{A D E} | \hbox{a b e}) >
1$, and given Eq.~\eqref{eq:localbad}
it is impossible to define a locally normalized
model with $p_L(\hbox{A B C} | \hbox{a b c}) = p_G(\hbox{A B C} |
\hbox{a b c})$ and $p_L(\hbox{A D E} | \hbox{a b e}) = 
p_G(\hbox{A D E} | \hbox{a b e})$.
$\qed$

%%%%%%%%%%%%%%%
% Inserted here so that it appears in the right place in the paper.
\begin{table*}[t]
  \centering%
  \scalebox{1.0}{%
    \small%
    \setlength{\tabcolsep}{4pt}%
    \centering%
    \begin{tabular}{lcccccccccccccccc}
      \toprule
      &\hspace*{0.1cm}& En &\hspace*{0.1cm}& \multicolumn{3}{c}{En-Union} &\hspace*{0.1cm}& \multicolumn{7}{c}{CoNLL '09}&\hspace*{0.1cm}& Avg\\
      Method       && WSJ      &&   News  &   Web   &   QTB   &&    Ca    &   Ch    &   Cz    &   En    &   Ge    &   Ja    &   Sp    &&-  \\
      \midrule
      Linear CRF   &&    97.17 &&   97.60 &   94.58 &   96.04  &&   98.81 &   94.45 &   98.90 &   97.50 &   97.14 &   97.90 &   98.79 &&   97.17\\
      \newcite{ling-EtAl:2015:EMNLP}&& \bf97.78 &&   97.44 &   94.03 &   96.18  &&   98.77 &   94.38 &   99.00 &   97.60 &\bf97.84 &   97.06 &   98.71 &&   97.16\\
      \midrule
      Our Local (B=1) && 97.44 &&   97.66 &   94.46 &   96.59  &&   98.91 &   94.56 &   98.96 &   97.36 &   97.35 &   98.02 &   98.88 &&   97.29\\
      Our Local (B=8) && 97.45 &&   97.69 &   94.46 &   96.64  &&   98.88 &   94.56 &   98.96 &   97.40 &   97.35 &   98.02 &   98.89 &&   97.30\\
      Our Global (B=8)&& 97.44 &&\bf97.77 &\bf94.80 &\bf96.86  &&\bf99.03 &\bf94.72 &\bf99.02 &\bf97.65 &   97.52 &\bf98.37 &\bf98.97 &&\bf97.47\\
            \midrule
      Parsey McParseface\hspace*{-.3cm} && - && 97.52 & 94.24 & 96.45 & - & - & - & - & - & - & - & - && - \\
      \bottomrule
    \end{tabular}
  }%\vspace*{-0.1cm}
  \caption{\label{tab:pos}
    Final POS tagging test set results on English WSJ and Treebank Union as well as CoNLL'09. We also show the performance of our pre-trained open source model, ``Parsey McParseface.''
}
\end{table*}

%%% Local Variables:
%%% mode: latex
%%% TeX-master: "../paper"
%%% End:

%%%%%%%%%%%%%%

Under the restriction that scores $\score(d_{1:i-1}, d_i, x_{1:i})$
depend only on the first $i$ input symbols,
the globally normalized model is still able to model the data in
Eq.~\eqref{eq:lbexample}, while the locally normalized model
fails (see Eq.~\ref{eq:localbad}). The ambiguity at input symbol
{\tt b} is naturally resolved when the next symbol ({\tt c} or {\tt
  e}) is observed, but the locally normalized model is not able to
revise its prediction.

It is easy to fix the locally normalized model for the example
in Eq.~\eqref{eq:lbexample} by
allowing scores $\score(d_{1:i-1}, d_i, x_{1:i+1})$ that 
take into account the input symbol $x_{i+1}$. 
More generally we can have a model of the form $\score(d_{1:i-1}, d_i, x_{1:i+k})$
where the integer $k$ specifies the amount of lookahead in the model.
Such lookahead is common in practice, but insufficient in general.
For every amount of lookahead $k$, 
we can construct examples that cannot be modeled
with a locally normalized model
by duplicating the middle input {\tt b} in (\ref{eq:lbexample}) $k+1$ times.
Only a local model with scores
$\score(d_{1:i-1}, d_i, x_{1:n})$ that considers the entire
input can capture any distribution $p(d_{1:n} | x_{1:n})$:
in this case the decomposition
$ p_L(d_{1:n} | x_{1:n}) = \prod_{i=1}^n p_L(d_i | d_{1:i-1}, x_{1:n}) $
makes no independence assumptions.

However, increasing the amount of context used as input comes
at a cost, requiring more powerful learning algorithms, and
potentially more training data. For a detailed analysis of the trade-offs
between structural features in CRFs and more powerful local classifiers
without structural constraints,
see \newcite{liang08structure}; in these experiments local classifiers
are unable to reach the performance of CRFs on problems such as parsing
and named entity recognition where structural constraints are important.
Note that there is nothing to preclude an approach that makes use of
both global normalization and more powerful scoring functions
$\score(d_{1:i-1}, d_i, x_{1:n})$, obtaining the best of both worlds.
The experiments that follow make use of both.

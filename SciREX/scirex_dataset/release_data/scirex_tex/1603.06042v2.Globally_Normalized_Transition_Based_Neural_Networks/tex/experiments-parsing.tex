\begin{table*}
  \centering
%  \hspace*{-0.7em}
  \scalebox{0.9}{
%    \renewcommand{\arraystretch}{1.0}%
    \setlength\tabcolsep{5pt}%
    \begin{tabular}{l@{\hskip 0.6cm}ccc@{\hskip 0.7cm}ccc@{\hskip 0.7cm}cc@{\hskip 0.7cm}cc}
      \toprule
      &&\multicolumn{2}{c@{\hskip 0.7cm}}{WSJ} && \multicolumn{2}{c@{\hskip 0.7cm}}{Union-News} & \multicolumn{2}{c@{\hskip 0.9cm}}{Union-Web} & \multicolumn{2}{c@{\hskip 0.0cm}}{Union-QTB\hspace*{0.5cm}}\\
      Method && UAS & LAS && UAS & LAS & UAS & LAS & UAS & LAS\\
      \midrule
      %\quad {\em Supervised} \\
      %\newcite{bohnet:2010:COLING} & 93.29 & 91.38 & 88.22 &  85.22 & 94.01 &  91.49 \\
      \newcite{martins-etAl:2013:ACL}$^\star$ && 92.89 & 90.55 && 93.10 &  91.13 & 88.23 &  85.04 & 94.21 &  91.54 \\
      \newcite{zhang-mcdonald:2014:ACL}$^\star$ && 93.22 & 91.02 && 93.32 & 91.48 & 88.65 &  85.59 & 93.37 &  90.69 \\
      \newcite{weiss-etAl:2015:ACL} && 93.99 & 92.05 && 93.91 &  92.25 & 89.29 &  86.44 & 94.17 &  92.06 \\
      \newcite{alberti-EtAl:2015:EMNLP} && 94.23 & 92.36 && 94.10 & 92.55 & 89.55 & 86.85 & 94.74 & 93.04 \\
      \midrule
      Our Local (B=1) && 92.95 & 91.02 && 93.11 & 91.46 & 88.42 & 85.58 & 92.49 & 90.38 \\
      Our Local (B=32) && 93.59 & 91.70 && 93.65 & 92.03 & 88.96 & 86.17 & 93.22 & 91.17 \\
      Our Global (B=32) && {\bf 94.61} & {\bf 92.79} && {\bf 94.44} & {\bf 92.93} & {\bf 90.17} & {\bf 87.54} & {\bf 95.40} & {\bf 93.64}  \\
      \midrule
      Parsey McParseface (B=8) && - & - && 94.15 & 92.51 & 89.08 & 86.29 & 94.77 & 93.17 \\
      %      \midrule
%      \quad {\em Semi-supervised} \\
      %\newcite{weiss-etAl:2015:ACL} tri-training && 94.16 & 92.41 && 94.16 &  92.62 & 89.72 &  87.00 & {\bf 95.58} &  93.05 \
%      \newcite{weiss-etAl:2015:ACL} && 94.26 & 92.41 \\
%      Our structured && {\bf 94.54} & {\bf 92.70} \\
      \bottomrule
    \end{tabular}
  }
  \caption{\label{tab:english_parsing}
    Final English dependency parsing test set results. We note that
    training our system using only the WSJ corpus (i.e. no pre-trained embeddings or other external resources) 
    yields 94.08\% UAS and 92.15\% LAS for our global model with beam 32.
  }
\end{table*}

%%% Local Variables:
%%% mode: latex
%%% TeX-master: "../paper"
%%% End:

\begin{table*}[t]
  %\hspace*{-0.29cm}
  \centering
  \scalebox{0.9}{%
    \small
    \setlength{\tabcolsep}{2pt}%
    \begin{tabular}{lcc@{\hskip 0.4cm}cc@{\hskip 0.4cm}cc@{\hskip 0.4cm}cc@{\hskip 0.4cm}cc@{\hskip 0.4cm}cc@{\hskip 0.4cm}cc}
      \toprule
      & \multicolumn{2}{c@{\hskip 0.4cm}}{Catalan} & \multicolumn{2}{c@{\hskip 0.4cm}}{Chinese} & \multicolumn{2}{c@{\hskip 0.4cm}}{Czech} & \multicolumn{2}{c@{\hskip 0.4cm}}{English} & \multicolumn{2}{c@{\hskip 0.4cm}}{German} & \multicolumn{2}{c@{\hskip 0.4cm}}{Japanese} & \multicolumn{2}{c@{\hskip 0.4cm}}{Spanish} \\
      Method & UAS & LAS & UAS & LAS & UAS & LAS & UAS & LAS & UAS & LAS & UAS & LAS & UAS & LAS \\
      \midrule
      Best Shared Task Result & - & 87.86 & - & 79.17 & -  & 80.38 & -  & 89.88 & -  & 87.48 & -  & 92.57 & -  & 87.64 \\
      \midrule
      \newcite{BallesterosDS15} & 90.22 & 86.42 & 80.64 & 76.52 & 79.87 & 73.62 & 90.56 & 88.01 & 88.83 & 86.10 & 93.47 & 92.55 & 90.38 & 86.59 \\
      \newcite{zhang-mcdonald:2014:ACL} & 91.41 & 87.91 & 82.87 & 78.57 & 86.62 & 80.59 & 92.69 & 90.01 & 89.88 & 87.38 & 92.82 & 91.87 & 90.82 & 87.34 \\
      \newcite{lei-EtAl:2014:ACL} &  91.33 & 87.22 & 81.67 & 76.71 & 88.76 & 81.77 & 92.75 & 90.00 & 90.81 & 87.81 & {\bf 94.04} & 91.84 & 91.16 & 87.38 \\
  \newcite{bohnet-nivre:2012:EMNLP-CoNLL} & 92.44 & 89.60 & 82.52 & 78.51 & 88.82 & 83.73 &  92.87 & 90.60 & {\bf 91.37} & {\bf 89.38} & 93.67 & 92.63 & 92.24 & 89.60 \\
      \newcite{alberti-EtAl:2015:EMNLP} & 92.31 & 89.17 & 83.57 & 79.90 & 88.45 & 83.57 & 92.70 & 90.56 & 90.58 & 88.20 & 93.99 & {\bf 93.10} & 92.26 & 89.33 \\
      \midrule
      Our Local (B=1) & 91.24 & 88.21 & 81.29 & 77.29 & 85.78 & 80.63 & 91.44 & 89.29 & 89.12 & 86.95 & 93.71 & 92.85 & 91.01 & 88.14 \\
      Our Local (B=16) & 91.91 & 88.93 & 82.22 & 78.26 & 86.25 & 81.28 & 92.16 & 90.05 & 89.53 & 87.4 & 93.61 & 92.74 & 91.64 & 88.88 \\
      Our Global (B=16) & {\bf 92.67} & {\bf 89.83} & {\bf 84.72} & {\bf 80.85} & {\bf 88.94} & {\bf 84.56} & {\bf 93.22} & {\bf 91.23} & 90.91 & 89.15 & 93.65 & 92.84 & {\bf 92.62} & {\bf 89.95} \\
      \bottomrule
    \end{tabular}
  }
  \caption{\label{tab:conll09_final}
    Final CoNLL '09 dependency parsing test set results.}
\end{table*}


\subsection{Dependency Parsing}

In dependency parsing the goal is to produce a directed tree representing
the syntactic structure of the input sentence.

\paragraph{Data \& Evaluation.}

We use the same corpora as in our POS tagging experiments, except that we use
the standard parsing splits of the WSJ. To avoid over-fitting to the development
set (Sec.~22), we use Sec.~24 for tuning the hyperparameters of our models.
We convert the English constituency trees to Stanford style dependencies
\cite{stanford_dependencies} using version 3.3.0 of the converter.
For English, we use predicted POS tags (the same POS tags are used for
all models) and exclude punctuation from the evaluation, as is standard.
For the CoNLL '09 datasets
we follow standard practice and include all punctuation in the evaluation.
We follow \newcite{alberti-EtAl:2015:EMNLP} and 
use our own predicted POS tags so that we can include a k-best tag
feature (see below) but use the supplied predicted morphological features.
We report unlabeled and labeled attachment scores (UAS/LAS).

\paragraph{Model Configuration.}

Our model configuration is basically the same as the one originally proposed
by \newcite{chen-manning:2014:EMNLP} and then refined by
\newcite{weiss-etAl:2015:ACL}. 
In particular, we use the arc-standard
transition system and extract the same set of features
as prior work: words, part of speech tags, and
dependency arcs and labels in the surrounding context of the state, 
as well as k-best tags as proposed by \newcite{alberti-EtAl:2015:EMNLP}.
We use two hidden layers of 1,024 dimensions each.

\paragraph{Results.}

Tables \ref{tab:english_parsing} and \ref{tab:conll09_final} show
our final parsing results and a comparison to the best systems from the literature.
We obtain the best ever published results on almost all datasets, including the WSJ.
Our main results use the same pre-trained word embeddings
as \newcite{weiss-etAl:2015:ACL} and \newcite{alberti-EtAl:2015:EMNLP}, but no tri-training.
When we artificially restrict ourselves to not use pre-trained word embeddings, 
we observe only a modest drop of $\sim$0.5\% UAS;
for example, training only on the WSJ yields 94.08\% UAS and 92.15\% LAS 
for our global model with a beam of size 32.

Even though we do not use tri-training, our model compares favorably to the 94.26\% LAS and 92.41\% UAS
reported by \newcite{weiss-etAl:2015:ACL} with tri-training.
As we show in Sec.~\ref{sec:discussion}, these gains can be attributed to the full backpropagation
training that differentiates our approach from that of \newcite{weiss-etAl:2015:ACL}
and \newcite{alberti-EtAl:2015:EMNLP}.
Our results also significantly outperform the LSTM-based approaches of
\newcite{dyer-etAl:2015:ACL} and \newcite{BallesterosDS15}.

%%% Local Variables:
%%% mode: latex
%%% TeX-master: "../paper"
%%% End:

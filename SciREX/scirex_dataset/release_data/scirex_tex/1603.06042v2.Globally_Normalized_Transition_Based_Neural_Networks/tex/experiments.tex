\section{Experiments}
\label{sec:experiments}

To demonstrate the flexibility and modeling power of our approach, we provide
experimental results on a diverse set of structured prediction tasks.
We apply our approach to POS tagging, syntactic dependency parsing, and sentence
compression.

While directly optimizing the global model defined by Eq.~\eqref{eq:global-cost} works well,
we found that training the model in two steps
achieves the same precision much faster:
we first pretrain the network using the local objective given in Eq.~\eqref{eq:local-beam-cost},
and then perform additional training steps using the global objective given in Eq.~\eqref{eq:global-beam}.
We pretrain all layers except the softmax layer in this way.
We purposefully abstain from complicated hand engineering
of input features, 
which might improve performance further \cite{durrett-klein:2015:ACL}.

We use the training recipe from \newcite{weiss-etAl:2015:ACL} for each training
stage of our model. Specifically, we use averaged stochastic gradient descent
with momentum, and we tune the learning rate, learning rate schedule,
momentum, and early stopping time using a separate held-out corpus for each
task. We tune again with a different set of hyperparameters for training with
the global objective. 

\subsection{Part of Speech Tagging}
\label{subsection:tagging}

Part of speech (POS) tagging is a classic NLP task,
where modeling the structure of the output
is important for achieving state-of-the-art performance.

\paragraph{Data \& Evaluation.}

We conducted experiments on a number of different datasets:
(1) the English Wall Street Journal (WSJ) part
of the Penn Treebank \cite{marcus:1993:CL}
with standard POS tagging splits;
(2) the English ``Treebank Union'' multi-domain corpus containing
data from the OntoNotes corpus version 5 \cite{hovy-EtAl:2006:NAACL},
the English Web Treebank \cite{petrov-mcdonald:2012:SANCL}, and the
updated and corrected Question Treebank \cite{judge-etAl:2006:ACL}
with identical setup to \newcite{weiss-etAl:2015:ACL}; and
(3) the CoNLL '09 multi-lingual shared 
task \cite{hajic-EtAl:2009:CoNLL}.%\footnote{http://ufal.mff.cuni.cz/conll2009-st/results/results.php}

\paragraph{Model Configuration.}

Inspired by the integrated POS tagging and parsing transition 
system of \newcite{bohnet-nivre:2012:EMNLP-CoNLL}, 
we employ a simple transition system that uses only a {\sc Shift} action and 
predicts the POS tag of the current word on the buffer
as it gets shifted to the stack. 
We extract the following features on a window $\pm 3$ tokens centered
at the current focus token: word, cluster, character n-gram up to length 3.
We also extract the tag predicted for the previous 4 tokens.
The network in these experiments has a single hidden layer with
256 units on WSJ and Treebank Union and 64 on CoNLL'09.

\paragraph{Results.}

In Table \ref{tab:pos} we compare our model
to a linear CRF and to the compositional
character-to-word LSTM model of \newcite{ling-EtAl:2015:EMNLP}.
The CRF is a first-order linear model with exact inference and
the same emission features as our model. It additionally also
has transition features of the
word, cluster and character n-gram up to length 3 on both endpoints of the
transition.
The results for \newcite{ling-EtAl:2015:EMNLP}
were solicited from the authors.

Our local model already compares favorably against these methods on average.
Using beam search with a locally normalized model does not help, but with global normalization
it leads to a 7\% reduction in relative error, empirically demonstrating the effect of label bias.
The set of character ngrams feature is very important, increasing average
accuracy on the CoNLL'09 datasets by about 0.5\% absolute. 
This shows that character-level modeling can also be done with a simple feed-forward
network without recurrence.

\begin{table*}
  \centering
%  \hspace*{-0.7em}
  \scalebox{0.9}{
%    \renewcommand{\arraystretch}{1.0}%
    \setlength\tabcolsep{5pt}%
    \begin{tabular}{l@{\hskip 0.6cm}ccc@{\hskip 0.7cm}ccc@{\hskip 0.7cm}cc@{\hskip 0.7cm}cc}
      \toprule
      &&\multicolumn{2}{c@{\hskip 0.7cm}}{WSJ} && \multicolumn{2}{c@{\hskip 0.7cm}}{Union-News} & \multicolumn{2}{c@{\hskip 0.9cm}}{Union-Web} & \multicolumn{2}{c@{\hskip 0.0cm}}{Union-QTB\hspace*{0.5cm}}\\
      Method && UAS & LAS && UAS & LAS & UAS & LAS & UAS & LAS\\
      \midrule
      %\quad {\em Supervised} \\
      %\newcite{bohnet:2010:COLING} & 93.29 & 91.38 & 88.22 &  85.22 & 94.01 &  91.49 \\
      \newcite{martins-etAl:2013:ACL}$^\star$ && 92.89 & 90.55 && 93.10 &  91.13 & 88.23 &  85.04 & 94.21 &  91.54 \\
      \newcite{zhang-mcdonald:2014:ACL}$^\star$ && 93.22 & 91.02 && 93.32 & 91.48 & 88.65 &  85.59 & 93.37 &  90.69 \\
      \newcite{weiss-etAl:2015:ACL} && 93.99 & 92.05 && 93.91 &  92.25 & 89.29 &  86.44 & 94.17 &  92.06 \\
      \newcite{alberti-EtAl:2015:EMNLP} && 94.23 & 92.36 && 94.10 & 92.55 & 89.55 & 86.85 & 94.74 & 93.04 \\
      \midrule
      Our Local (B=1) && 92.95 & 91.02 && 93.11 & 91.46 & 88.42 & 85.58 & 92.49 & 90.38 \\
      Our Local (B=32) && 93.59 & 91.70 && 93.65 & 92.03 & 88.96 & 86.17 & 93.22 & 91.17 \\
      Our Global (B=32) && {\bf 94.61} & {\bf 92.79} && {\bf 94.44} & {\bf 92.93} & {\bf 90.17} & {\bf 87.54} & {\bf 95.40} & {\bf 93.64}  \\
      \midrule
      Parsey McParseface (B=8) && - & - && 94.15 & 92.51 & 89.08 & 86.29 & 94.77 & 93.17 \\
      %      \midrule
%      \quad {\em Semi-supervised} \\
      %\newcite{weiss-etAl:2015:ACL} tri-training && 94.16 & 92.41 && 94.16 &  92.62 & 89.72 &  87.00 & {\bf 95.58} &  93.05 \
%      \newcite{weiss-etAl:2015:ACL} && 94.26 & 92.41 \\
%      Our structured && {\bf 94.54} & {\bf 92.70} \\
      \bottomrule
    \end{tabular}
  }
  \caption{\label{tab:english_parsing}
    Final English dependency parsing test set results. We note that
    training our system using only the WSJ corpus (i.e. no pre-trained embeddings or other external resources) 
    yields 94.08\% UAS and 92.15\% LAS for our global model with beam 32.
  }
\end{table*}

%%% Local Variables:
%%% mode: latex
%%% TeX-master: "../paper"
%%% End:

\begin{table*}[t]
  %\hspace*{-0.29cm}
  \centering
  \scalebox{0.9}{%
    \small
    \setlength{\tabcolsep}{2pt}%
    \begin{tabular}{lcc@{\hskip 0.4cm}cc@{\hskip 0.4cm}cc@{\hskip 0.4cm}cc@{\hskip 0.4cm}cc@{\hskip 0.4cm}cc@{\hskip 0.4cm}cc}
      \toprule
      & \multicolumn{2}{c@{\hskip 0.4cm}}{Catalan} & \multicolumn{2}{c@{\hskip 0.4cm}}{Chinese} & \multicolumn{2}{c@{\hskip 0.4cm}}{Czech} & \multicolumn{2}{c@{\hskip 0.4cm}}{English} & \multicolumn{2}{c@{\hskip 0.4cm}}{German} & \multicolumn{2}{c@{\hskip 0.4cm}}{Japanese} & \multicolumn{2}{c@{\hskip 0.4cm}}{Spanish} \\
      Method & UAS & LAS & UAS & LAS & UAS & LAS & UAS & LAS & UAS & LAS & UAS & LAS & UAS & LAS \\
      \midrule
      Best Shared Task Result & - & 87.86 & - & 79.17 & -  & 80.38 & -  & 89.88 & -  & 87.48 & -  & 92.57 & -  & 87.64 \\
      \midrule
      \newcite{BallesterosDS15} & 90.22 & 86.42 & 80.64 & 76.52 & 79.87 & 73.62 & 90.56 & 88.01 & 88.83 & 86.10 & 93.47 & 92.55 & 90.38 & 86.59 \\
      \newcite{zhang-mcdonald:2014:ACL} & 91.41 & 87.91 & 82.87 & 78.57 & 86.62 & 80.59 & 92.69 & 90.01 & 89.88 & 87.38 & 92.82 & 91.87 & 90.82 & 87.34 \\
      \newcite{lei-EtAl:2014:ACL} &  91.33 & 87.22 & 81.67 & 76.71 & 88.76 & 81.77 & 92.75 & 90.00 & 90.81 & 87.81 & {\bf 94.04} & 91.84 & 91.16 & 87.38 \\
  \newcite{bohnet-nivre:2012:EMNLP-CoNLL} & 92.44 & 89.60 & 82.52 & 78.51 & 88.82 & 83.73 &  92.87 & 90.60 & {\bf 91.37} & {\bf 89.38} & 93.67 & 92.63 & 92.24 & 89.60 \\
      \newcite{alberti-EtAl:2015:EMNLP} & 92.31 & 89.17 & 83.57 & 79.90 & 88.45 & 83.57 & 92.70 & 90.56 & 90.58 & 88.20 & 93.99 & {\bf 93.10} & 92.26 & 89.33 \\
      \midrule
      Our Local (B=1) & 91.24 & 88.21 & 81.29 & 77.29 & 85.78 & 80.63 & 91.44 & 89.29 & 89.12 & 86.95 & 93.71 & 92.85 & 91.01 & 88.14 \\
      Our Local (B=16) & 91.91 & 88.93 & 82.22 & 78.26 & 86.25 & 81.28 & 92.16 & 90.05 & 89.53 & 87.4 & 93.61 & 92.74 & 91.64 & 88.88 \\
      Our Global (B=16) & {\bf 92.67} & {\bf 89.83} & {\bf 84.72} & {\bf 80.85} & {\bf 88.94} & {\bf 84.56} & {\bf 93.22} & {\bf 91.23} & 90.91 & 89.15 & 93.65 & 92.84 & {\bf 92.62} & {\bf 89.95} \\
      \bottomrule
    \end{tabular}
  }
  \caption{\label{tab:conll09_final}
    Final CoNLL '09 dependency parsing test set results.}
\end{table*}


\subsection{Dependency Parsing}

In dependency parsing the goal is to produce a directed tree representing
the syntactic structure of the input sentence.

\paragraph{Data \& Evaluation.}

We use the same corpora as in our POS tagging experiments, except that we use
the standard parsing splits of the WSJ. To avoid over-fitting to the development
set (Sec.~22), we use Sec.~24 for tuning the hyperparameters of our models.
We convert the English constituency trees to Stanford style dependencies
\cite{stanford_dependencies} using version 3.3.0 of the converter.
For English, we use predicted POS tags (the same POS tags are used for
all models) and exclude punctuation from the evaluation, as is standard.
For the CoNLL '09 datasets
we follow standard practice and include all punctuation in the evaluation.
We follow \newcite{alberti-EtAl:2015:EMNLP} and 
use our own predicted POS tags so that we can include a k-best tag
feature (see below) but use the supplied predicted morphological features.
We report unlabeled and labeled attachment scores (UAS/LAS).

\paragraph{Model Configuration.}

Our model configuration is basically the same as the one originally proposed
by \newcite{chen-manning:2014:EMNLP} and then refined by
\newcite{weiss-etAl:2015:ACL}. 
In particular, we use the arc-standard
transition system and extract the same set of features
as prior work: words, part of speech tags, and
dependency arcs and labels in the surrounding context of the state, 
as well as k-best tags as proposed by \newcite{alberti-EtAl:2015:EMNLP}.
We use two hidden layers of 1,024 dimensions each.

\paragraph{Results.}

Tables \ref{tab:english_parsing} and \ref{tab:conll09_final} show
our final parsing results and a comparison to the best systems from the literature.
We obtain the best ever published results on almost all datasets, including the WSJ.
Our main results use the same pre-trained word embeddings
as \newcite{weiss-etAl:2015:ACL} and \newcite{alberti-EtAl:2015:EMNLP}, but no tri-training.
When we artificially restrict ourselves to not use pre-trained word embeddings, 
we observe only a modest drop of $\sim$0.5\% UAS;
for example, training only on the WSJ yields 94.08\% UAS and 92.15\% LAS 
for our global model with a beam of size 32.

Even though we do not use tri-training, our model compares favorably to the 94.26\% LAS and 92.41\% UAS
reported by \newcite{weiss-etAl:2015:ACL} with tri-training.
As we show in Sec.~\ref{sec:discussion}, these gains can be attributed to the full backpropagation
training that differentiates our approach from that of \newcite{weiss-etAl:2015:ACL}
and \newcite{alberti-EtAl:2015:EMNLP}.
Our results also significantly outperform the LSTM-based approaches of
\newcite{dyer-etAl:2015:ACL} and \newcite{BallesterosDS15}.

%%% Local Variables:
%%% mode: latex
%%% TeX-master: "../paper"
%%% End:

%!TEX root = ../paper.tex

\subsection{Sentence Compression}

Our final structured prediction task is extractive sentence compression.

\paragraph{Data \& Evaluation.}

We follow \newcite{filippova-emnlp15}, where a large news collection is used to
heuristically generate compression instances.
Our final corpus contains about 2.3M compression instances: we use 2M examples
for training, 130k for development and 160k for the final test.
We report per-token F1 score and per-sentence accuracy (A), i.e.~percentage of
instances that fully match the golden compressions.
Following~\newcite{filippova-emnlp15} we also run a human evaluation on 200
sentences where we ask the raters to score compressions for \textit{readability}
(\texttt{read}) and \textit{informativeness} (\texttt{info}) on a scale from 0
to 5.

\paragraph{Model Configuration.}

The transition system for sentence compression is similar to POS
tagging: we scan sentences from left-to-right and label each token as
\textit{keep} or \textit{drop}.
We extract features from words, POS tags, and dependency labels from a window of
tokens centered on the input, as well as features from the history of
predictions.
We use a single hidden layer of size 400.

\begin{table}
  \centering%
  \scalebox{0.85}{%
    \begin{tabular}{lcccc}
      \toprule
      & \multicolumn{2}{c}{Generated corpus} & \multicolumn{2}{c}{Human eval} \\
      Method & A & F1 & read & info \\
      \midrule
      \newcite{filippova-emnlp15} & {\bf 35.36} & {\bf 82.83} & 4.66 & 4.03 \\
      Automatic & - & - & 4.31 & 3.77  \\
      \midrule
      Our Local (B=1) & 30.51 & 78.72 & 4.58 & 4.03 \\
      Our Local (B=8) & 31.19 & 75.69 & - & - \\
      Our Global (B=8) & 35.16 & 81.41 & \textbf{4.67} & \textbf{4.07} \\
      \bottomrule
    \end{tabular}
  }
  \caption{\label{tab:compression-eval}
    Sentence compression results on News data.
    % on Test-5k
    {\em Automatic} refers to application of the same automatic extraction rules used to generate
    the News training corpus.
  }
\end{table}

\paragraph{Results.}

Table~\ref{tab:compression-eval} shows our sentence compression results.
Our globally normalized model again significantly outperforms the local model.
Beam search with a locally normalized model suffers from severe label bias issues
that we discuss on a concrete example in Section~\ref{sec:discussion}.
We also compare to the sentence compression system
from \newcite{filippova-emnlp15}, a 3-layer stacked LSTM which uses dependency
label information.
The LSTM and our global model perform on par on both the automatic evaluation
as well as the human ratings, but our model is roughly 100$\times$ faster.
All compressions kept approximately 42\% of the tokens on average and
all the models are significantly better than the automatic extractions ($p < 0.05$).

%%% Local Variables:
%%% mode: latex
%%% TeX-master: "../paper"
%%% End:


%%% Local Variables:
%%% mode: latex
%%% TeX-master: "../paper"
%%% End:

\documentclass[11pt]{article}
\usepackage{acl2016}
\usepackage{times}
\usepackage{url}
\usepackage{latexsym}
\usepackage{xcolor}
\usepackage{balance}
\usepackage{graphicx}
\usepackage{epstopdf}
\usepackage{amsmath}
\usepackage{color}
\usepackage{amsfonts}
\usepackage{amssymb}
\usepackage{booktabs}
%\usepackage{txfonts}
\usepackage{enumitem}
\usepackage{mathtools}
\usepackage{multirow}
\usepackage[font=small]{caption}
\usepackage{stmaryrd}

\newcommand{\denselist}{\setlength{\itemsep}{1pt}
  \setlength{\parskip}{0pt} \setlength{\parsep}{0pt}}
\newcommand{\bitem}{\begin{itemize}[noitemsep,topsep=2pt]\denselist}
\newcommand{\eitem}{\end{itemize}}
\newcommand{\qed}{\square}

\newtheorem{theorem}{Theorem}[section]

\setlength\titlebox{5.5cm}

\DeclareMathOperator*{\argmax}{argmax}

\aclfinalcopy

\title{Globally Normalized Transition-Based Neural Networks}

\author{
  Daniel Andor, Chris Alberti, David Weiss, Aliaksei Severyn, \\
    {\bf Alessandro Presta, Kuzman Ganchev, Slav Petrov and Michael Collins\thanks{$\;\;$On leave from Columbia University.}}\\
  Google Inc\\
  New York, NY\\
  {\footnotesize \tt \{andor,chrisalberti,djweiss,severyn,apresta,kuzman,slav,mjcollins\}@google.com}
}

\date{}

\begin{document}
\maketitle

% Abstract.
\begin{abstract}
We introduce a globally normalized transition-based neural network
model that achieves state-of-the-art part-of-speech tagging,
dependency parsing and sentence compression results.  Our model is a
simple feed-forward neural network that operates on a task-specific
transition system, yet achieves comparable or better accuracies than
recurrent models.
We discuss the importance of global as opposed to local normalization:
a key insight is that the label bias problem implies that
globally
normalized models can be strictly more expressive 
than locally normalized models.
\end{abstract}

% Utility commands.
\graphicspath{{./figs/}}
\newcommand{\bX}{\mathbf{X}}
\newcommand{\bE}{\mathbf{E}}
\newcommand{\bb}{\mathbf{b}}
\newcommand{\bH}{\mathbf{H}}
\newcommand{\bW}{\mathbf{W}}
\newcommand{\bh}{\mathbf{h}}
\newcommand{\mwords}{\mathrm{word}}
\newcommand{\mtags}{\mathrm{tag}}
\newcommand{\mlabels}{\mathrm{label}}
\newcommand{\todo}[1]{{\bf \color{red}{TODO: #1}}}
\newcommand{\eat}[1]{\ignorespaces}
\newcommand{\commentout}[1]{}
\newcommand\T{\rule{0pt}{4ex}}  % Top strut for tables

% Sections of the paper.
\section{Introduction}
\label{sec:intro}

Language modeling is among the important problems that require modeling long-term dependency, with successful applications such as unsupervised pretraining~\citep{dai2015semi,peters2018deep,radford2018improving,devlin2018bert}.
However, it has been a challenge to equip neural networks with the capability to model long-term dependency in sequential data.
Recurrent neural networks (RNNs), in particular Long Short-Term Memory (LSTM) networks~\citep{hochreiter1997long}, have been a standard solution to language modeling and obtained strong results on multiple benchmarks.
Despite the wide adaption, RNNs are difficult to optimize due to gradient vanishing and explosion~\citep{hochreiter2001gradient}, and the introduction of gating in LSTMs and the gradient clipping technique~\citep{graves2013generating} might not be sufficient to fully address this issue.
% ,pascanu2012understanding
Empirically, previous work has found that LSTM language models use 200 context words on average~\citep{khandelwal2018sharp}, indicating room for further improvement.

On the other hand, the direct connections between long-distance word pairs baked in attention mechanisms might ease optimization and enable the learning of long-term dependency~\citep{bahdanau2014neural,vaswani2017attention}.
Recently, \citet{al2018character} designed a set of auxiliary losses to train deep Transformer networks for character-level language modeling, which outperform LSTMs by a large margin.
Despite the success, the LM training in~\citet{al2018character} is performed on separated fixed-length segments of a few hundred characters, without any information flow across segments.
As a consequence of the fixed context length, the model cannot capture any longer-term dependency beyond the predefined context length.
In addition, the fixed-length segments are created by selecting a consecutive chunk of symbols without respecting the sentence or any other semantic boundary.
Hence, the model lacks necessary contextual information needed to well predict the first few symbols, leading to inefficient optimization and inferior performance.
We refer to this problem as \textit{context fragmentation}.

%However, the context length is fixed to hundreds of characters and thus it is not possible to model longer-term dependency. Moreover, it is not clear how the model performs on word-level language modeling data, as the granularity changes.

% Moreover, using auxiliary losses brings additional challenges such as properly tuning the mixture weights and the loss decay schedule.

To address the aforementioned limitations of fixed-length contexts, we propose a new architecture called Transformer-XL (meaning extra long).
We introduce the notion of recurrence into our deep self-attention network. In particular, instead of computing the hidden states from scratch for each new segment, we reuse the hidden states obtained in previous segments.
The reused hidden states serve as memory for the current segment, which builds up a recurrent connection between the segments.
As a result, modeling very long-term dependency becomes possible because information can be propagated through the recurrent connections.
Meanwhile, passing information from the previous segment can also resolve the problem of context fragmentation.
More importantly, we show the necessity of using relative positional encodings rather than absolute ones, in order to enable state reuse without causing temporal confusion.
Hence, as an additional technical contribution, we introduce a simple but more effective relative positional encoding formulation that generalizes to attention lengths longer than the one observed during training.

Transformer-XL obtained strong results on five datasets, varying from word-level to character-level language modeling.
Transformer-XL is also able to generate relatively coherent long text articles with \textit{thousands of} tokens (see Appendix \ref{sec:gen}), trained on only 100M tokens.
% Transformer-XL improves the previous state-of-the-art (SoTA) results from 1.06 to 0.99 in bpc on enwiki8, from 1.13 to 1.08 in bpc on text8, from 20.5 to 18.3 in perplexity on WikiText-103, and from 23.7 to 21.8 in perplexity on One Billion Word.
% Transformer-XL improves the previous state-of-the-art (SoTA) results to 0.99 in bpc on enwiki8, 1.08 in bpc on text8, 18.3 in perplexity on WikiText-103, and 21.8 in perplexity on One Billion Word.
% On small data, Transformer-XL also achieves a perplexity of 54.5 on Penn Treebank without finetuning, which is SoTA when comparable settings are considered.

Our main technical contributions include introducing the notion of recurrence in a purely self-attentive model and deriving a novel positional encoding scheme. These two techniques form a complete set of solutions, as any one of them alone does not address the issue of fixed-length contexts. Transformer-XL is the first self-attention model that achieves substantially better results than RNNs on both character-level and word-level language modeling.

% On WikiText-103, Transformer-XL improves the previous state-of-the-art (SoTA) results from 33 perplexity to 24, with a relative reduction of 27\%. On enwiki8 character-level language modeling, Transformer-XL achieves a SoTA bpc of 1.03, which outperforms \cite{al2018character} by 0.03 with 60+\% fewer parameters. Given a more common model size with 40+M parameters, Transformer-XL achieves a bpc of 1.06, compared to 1.11 by \cite{al2018character}. Transformer-XL also achieves perplexities of 54.5 on Penn Treebank and 29.4 on One Billion Word, which are SoTA when comparable settings are considered.

% Due to the ability of modeling long-range context, our best model uses attention lengths of 1,600 and 3,800 on WikiText-103 and enwiki8 respectively. We also devise a metric called \textit{Relative Effective Context Length} (RECL) that aims to fairly compare the ability of long-range dependency modeling.
% % perform a fair comparison of the gains brought by increasing the context lengths for different models.
% In this setting, Transformer-XL learns a RECL of 900 words on WikiText-103, while the numbers for recurrent networks and Transformer are only 500 and 128.

% We use two methods to quantitatively study the effective lengths of Transformer-XL and the baselines. Similar to \cite{khandelwal2018sharp}, we gradually increase the attention length at test time until no further noticeable improvement ($\sim$0.1\% relative gains) can be observed. Our best model in this settings use attention lengths of 1,600 and 3,800 on WikiText-103 and enwiki8 respectively.
% %In addition, since the effective context length of Transformer-XL can be longer than the attention length due to our recurrent formulation, we devise a metric called \textit{Relative Effective Context Length} (RECL) that aims to perform a fair comparison of the gains brought by increasing the context lengths for different models.
% In addition, we devise a metric called \textit{Relative Effective Context Length} (RECL) that aims to perform a fair comparison of the gains brought by increasing the context lengths for different models.
% In this setting, Transformer-XL learns a RECL of 900 words on WikiText-103, while the numbers for recurrent networks and Transformer are only 500 and 128.

\section{MT-DNN-1}
\label{sec:mt-dnn-1}

\subsection{Preliminaries}
\label{subsec:prelim}
In this work, our multi-task model combines classification, regression and pair-wise ranking tasks, which are summerised in Table~\ref{tab:task}. We briefly introduce the definition of each task as follows: 
\begin{table}[htb!]
	\begin{center}
		\begin{tabular}{@{\hskip1pt}l@{\hskip1pt}|@{\hskip1pt}c@{\hskip1pt}|@{\hskip1pt}c@{\hskip1pt}|@{\hskip1pt}c}
			\hline \bf Input &Classification&Regression &Ranking\\ \hline \hline
			single sentence &$\checkmark$&& \\
			pairwise text &$\checkmark$&$\checkmark$&$\checkmark$ \\ \hline
		\end{tabular}
	\end{center}
	\lgspace
	\caption{Summary of tasks in our multi-task framework.
	}
	\label{tab:task}
\lgspace
\end{table}
\begin{figure}[!t]
\centering
\adjustbox{trim={.065\width} {.01\height} {.05\width} {.01\height},clip}
{\includegraphics[scale=0.7]{mtl_model}}
\caption{Model architecture.}
\label{fig:mtl_model} 
\end{figure}

\begin{figure}[!t]
\centering
\adjustbox{trim={.05\width} {.01\height} {.05\width} {.01\height},clip}
{\includegraphics[scale=0.7]{mtl_model_v2}}
\caption{Model architecture version 2.}
\label{fig:mtl_model_v2} 
\end{figure}

\textbf{Task definition}

\textbf{Objective}

\textbf{Single classification}
\xiaodl{Need to cluster different tasks..}

\textbf{Sentence-pair classification}: given a pair of sentence, $(S_1, S_2)$, the model predicts a label indicating the relation of this pair of sentences: $P(C|S_1, S_2)$. For example, natural language inference is a typical instance of the sentence-pair classification task: a premise and a hypothesis are denoted by $S_1$ and $S_2$, respectively; the label, $C$, belongs one of three relations (\textit{contradiction}, \textit{neutral} and \textit{entailment}). 

\textbf{Regression}


\textbf{Pair-wise Ranking}
\begin{algorithm}[ht!]
 \SetAlgoLined
Initialize model parameters $\Theta$ randomly  \\
Set M \quad\textit{//the number of updates for the shared layer} \\
%\textit{Counter} = 0\\
 \For{$iteration$ in $0 ... \infty$}{
 	 %1. \textit{Counter} += 1\\
     1. Pick a task $t$ randomly \\
     2. Pick sample(s) from task $t$, i.e., \\
     \hspace{0.4cm}$(Q,C=\{0,1\})$ for classification \\
     \hspace{0.4cm}$(Q, D)$ for ranking\\
     3. Compute loss: $L(\Theta)$, i.e.,\\
     \hspace{0.4cm} the \textit{cross-entropy} for classification \\
     \hspace{0.4cm} the ranking loss for ranking\cite{learning-to-rank2005burges}\\

     4. Compute gradient: $\nabla(\Theta)$ \\
     5. Update model: $\Theta = \Theta - \epsilon \nabla(\Theta)$ \quad\textit{}
     % \eIf{Counter $<$ M}{
  	 %5. Update model: $\Theta = \Theta - \epsilon \nabla(\Theta)$ \quad\textit{//update both $\Theta^s$ and $\Theta^t$} \\
   %}{
   	% 6. Update model: $\Theta^t = \Theta^t - \epsilon \nabla(\Theta^t)$ 
  %}
 }
 \caption{\label{algo:mtdnn} Training a Multi-task model.}
 \algorithmfootnote{Note that $\Theta$ denotes the model parameters. \textcolor{red}{TODO: update alg based on task defination.}}
\end{algorithm}
\section{The Label Bias Problem}
\label{sec:label_bias}

Intuitively, we would like the model to be able to revise an earlier
decision made during search, when later evidence becomes available
that rules out the earlier decision as incorrect. At first glance, it
might appear that a locally normalized model used in conjunction with
beam search or exact search is able to revise earlier
decisions. However the label bias problem 
(see \newcite{bottou},
\newcite{collins99}
pages 222-226,
\newcite{crf},
\newcite{bottou-lecun-2005}, 
\newcite{smithJohnson07}) means that locally normalized models often have a very
weak ability to revise earlier decisions.
%% Our experiments show consistent improvements in accuracy for globally
%% normalized models over locally normalized models with beam search; we
%% suspect that the label bias problem is a major source of this
%% difference.

This section gives a formal perspective on the label bias problem,
through a proof that globally normalized models are strictly more
expressive than locally normalized models. The theorem was originally
proved\footnote{More precisely \newcite{smithJohnson07} prove the
  theorem for models with potential functions of the form
  $\score(d_{i-1}, d_i, x_i)$; the generalization to potential
  functions of the form $\score(d_{1:i-1}, d_i, x_{1:i})$ is
  straightforward.}  by \newcite{smithJohnson07}.
The example
underlying the proof gives a clear illustration of the label bias
problem.\footnote{\newcite{smithJohnson07} cite Michael Collins
as the source of the example underlying the proof.  
Note that the theorem refers to {\em conditional} models of the form
$p(d_{1:n} | x_{1:n})$ with global or local normalization.
Equivalence (or non-equivalence) results for {\em joint} models of the
form $p(d_{1:n}, x_{1:n})$ are quite different: for example
results from \newcite{chi99} and \newcite{abneyEtal99} imply that weighted
context-free grammars (a globally normalized joint model) and
probabilistic context-free grammars (a locally normalized joint model)
are equally expressive.}


\paragraph{Global Models can be Strictly More Expressive than
Local Models}
Consider a tagging problem where the task is to map an input sequence
$x_{1:n}$ to a decision sequence $d_{1:n}$.
First, consider a locally normalized model
where we restrict the scoring function to access only the first
$i$ input symbols $x_{1:i}$ when scoring decision $d_i$.
We will return to this restriction soon.
The scoring function $\score$ can be an otherwise arbitrary function of the
tuple $\langle d_{1:i-1}, d_i, x_{1:i} \rangle$:
\begin{align*}
p_L(d_{1:n} | x_{1:n})
&=\prod_{i=1}^n p_L(d_i | d_{1:i-1}, x_{1:i}) \\
&= \frac{\exp \sum_{i=1}^n \score(d_{1:i-1},d_i, x_{1:i}) }
{\prod_{i=1}^n Z_L(d_{1:i-1}, x_{1:i})}.
\end{align*}

Second, consider a globally normalized model
\begin{align*}
&p_G(d_{1:n} | x_{1:n}) 
= \frac{\exp \sum_{i=1}^n \score(d_{1:i-1},d_i, x_{1:i}) }
{Z_G(x_{1:n})}.
\end{align*}
This model again makes use of a scoring function $\score(d_{1:i-1},
d_i, x_{1:i})$ restricted to the first $i$ input symbols when
scoring decision $d_i$.

Define ${\cal P}_L$ to be the set of all possible distributions
$p_L(d_{1:n} | x_{1:n})$ under the local model obtained
as the scores $\score$ vary. Similarly, define ${\cal P}_G$ to be the
set of all possible distributions $p_G(d_{1:n} | x_{1:n})$
under the global model. Here a ``distribution'' is a function
from a pair $(x_{1:n}, d_{1:n})$ to a probability
$p(d_{1:n} | x_{1:n})$.
Our main result is the following:

\begin{theorem} See also \newcite{smithJohnson07}.
%\hspace*{1cm}%\newline
\hspace*{0.5cm}${\cal P}_L$ is a strict subset of ${\cal P}_G$, that is ${\cal P}_L
  \subsetneq {\cal P}_G$.
\end{theorem}

To prove this we will first prove that ${\cal P}_L \subseteq {\cal
  P}_G$. This step is straightforward. We then show that
${\cal P}_G \nsubseteq {\cal P}_L$; that is, there are distributions
in ${\cal P}_G$ that are not in ${\cal P}_L$.
The proof that ${\cal P}_G \nsubseteq {\cal P}_L$ gives a clear
illustration of the label bias problem.


{\em Proof that ${\cal P}_L \subseteq {\cal P}_G$:} 
We need to show that for any locally normalized
distribution $p_L$, we can construct a globally normalized model $p_G$
such that $p_G = p_L$.
Consider a locally normalized model with scores
$\score(d_{1:i-1}, d_i, x_{1:i})$.
Define a global model $p_G$ with scores
\[
\score'(d_{1:i-1}, d_i, x_{1:i}) = 
\log p_L(d_i | d_{1:i-1}, x_{1:i}).
\]
Then it is easily verified that 
\[p_G(d_{1:n} | x_{1:n}) = 
p_L(d_{1:n} | x_{1:n}) \]
for all $x_{1:n}, d_{1:n}$.
$\qed$

In proving ${\cal P}_G \nsubseteq {\cal P}_L$ we will use a
simple problem where every example seen in training or test data is
one of the following two tagged sentences:
\begin{align}
x_1 x_2 x_3 = \hbox{a b c}, \;\;d_1 d_2 d_3 = \hbox{A B C}
\nonumber \\
x_1 x_2 x_3 = \hbox{a b e}, \;\;d_1 d_2 d_3 = \hbox{A D E}
\label{eq:lbexample}
\end{align}

Note that the input $x_2 = \hbox{b}$ is ambiguous: it can take tags
$\hbox{B}$ or $\hbox{D}$. This ambiguity is resolved when the next
input symbol, {\tt c} or {\tt e}, is observed.

Now consider a globally normalized model, where the scores
$\score(d_{1:i-1}, d_i, x_{1:i})$ are defined as follows. 
Define ${\cal T}$ as the set $\{ (A, B), (B, C), (A, D), (D, E)\}$
of bigram tag transitions seen in the data. Similarly, define ${\cal
  E}$ as the set $\{ (a, A), (b, B), (c, C), (b, D), (e, E)\}$ of
(word, tag) pairs seen in the data. We define 
\begin{align}
&\score(d_{1:i-1}, d_i, x_{1:i})
\label{eq:alpha} \\
&=\alpha \times \llbracket(d_{i-1}, d_i) \in {\cal T} \rrbracket
+ \alpha \times \llbracket (x_i, d_i) \in {\cal E} \rrbracket
\nonumber
\end{align}
where $\alpha$ is the single scalar parameter of the model,
and $\llbracket \pi \rrbracket = 1$ if $\pi$ is true, $0$ otherwise.


{\em Proof that ${\cal P}_G \nsubseteq {\cal P}_L$:} We
will construct a globally normalized model $p_G$ such that there is
no locally normalized model such that $p_L = p_G$.

Under the definition in Eq.~\eqref{eq:alpha},
it is straightforward to show that
\begin{align*}\small
\lim_{\alpha \rightarrow \infty} p_G(\hbox{A B C} | \hbox{a b c}) =
\lim_{\alpha \rightarrow \infty} p_G(\hbox{A D E} | \hbox{a b e}) = 1 .
%% \lim_{\rule{0pt}{1.5ex}\mathclap{\alpha \rightarrow \infty}} p_G(\hbox{A B C} | \hbox{a b c}) 
%% =
%% \lim_{\rule{0pt}{1.5ex}\mathclap{\alpha \rightarrow \infty}} p_G(\hbox{A D E} | \hbox{a b e}) = 1 .
\end{align*}

In contrast, under {\em any} definition for $\score(d_{1:i-1}, d_i,
x_{1:i})$, we must have
\begin{equation}
p_L(\hbox{A B C} | \hbox{a b c}) + 
p_L(\hbox{A D E} | \hbox{a b e}) \leq 1
\label{eq:localbad}
\end{equation}
This follows because $p_L(\hbox{A B C} | \hbox{a b c})
= p_L(\hbox{A} | \hbox{a}) \times
p_L(\hbox{B} | \hbox{A}, \hbox{a b}) \times p_L(\hbox{C} | \hbox{A B},
\hbox{a b c})$
and $p_L(\hbox{A D E} | \hbox{a b e})
= p_L(\hbox{A} | \hbox{a}) \times
p_L(\hbox{D} | \hbox{A}, \hbox{a b}) \times
p_L(\hbox{E} | \hbox{A D}, \hbox{a b e})$.
%% \begin{align}
%% &p_L(\hbox{A B C} | \hbox{a b c}) \nonumber \\
%% &= p_L(\hbox{A} | \hbox{a}) \times 
%% p_L(\hbox{B} | \hbox{A}, \hbox{a b}) \times 
%% p_L(\hbox{C} | \hbox{A B}, \hbox{a b c}) \nonumber
%% \end{align} 
%% and
%% \begin{align}
%% &p_L(\hbox{A D E} | \hbox{a b e}) \nonumber \\
%% &= p_L(\hbox{A} | \hbox{a}) \times 
%% p_L(\hbox{D} | \hbox{A}, \hbox{a b}) \times 
%% p_L(\hbox{E} | \hbox{A D}, \hbox{a b e}) \nonumber
%% \end{align} 
The inequality $p_L(\hbox{B} | \hbox{A}, \hbox{a b}) + p_L(\hbox{D} |
\hbox{A}, \hbox{a b}) \leq 1$ then immediately implies
Eq.~\eqref{eq:localbad}.

It follows that for sufficiently large values of $\alpha$, we have 
$p_G(\hbox{A B C} | \hbox{a b c}) + p_G(\hbox{A D E} | \hbox{a b e}) >
1$, and given Eq.~\eqref{eq:localbad}
it is impossible to define a locally normalized
model with $p_L(\hbox{A B C} | \hbox{a b c}) = p_G(\hbox{A B C} |
\hbox{a b c})$ and $p_L(\hbox{A D E} | \hbox{a b e}) = 
p_G(\hbox{A D E} | \hbox{a b e})$.
$\qed$

%%%%%%%%%%%%%%%
% Inserted here so that it appears in the right place in the paper.
\begin{table*}[t]
  \centering%
  \scalebox{1.0}{%
    \small%
    \setlength{\tabcolsep}{4pt}%
    \centering%
    \begin{tabular}{lcccccccccccccccc}
      \toprule
      &\hspace*{0.1cm}& En &\hspace*{0.1cm}& \multicolumn{3}{c}{En-Union} &\hspace*{0.1cm}& \multicolumn{7}{c}{CoNLL '09}&\hspace*{0.1cm}& Avg\\
      Method       && WSJ      &&   News  &   Web   &   QTB   &&    Ca    &   Ch    &   Cz    &   En    &   Ge    &   Ja    &   Sp    &&-  \\
      \midrule
      Linear CRF   &&    97.17 &&   97.60 &   94.58 &   96.04  &&   98.81 &   94.45 &   98.90 &   97.50 &   97.14 &   97.90 &   98.79 &&   97.17\\
      \newcite{ling-EtAl:2015:EMNLP}&& \bf97.78 &&   97.44 &   94.03 &   96.18  &&   98.77 &   94.38 &   99.00 &   97.60 &\bf97.84 &   97.06 &   98.71 &&   97.16\\
      \midrule
      Our Local (B=1) && 97.44 &&   97.66 &   94.46 &   96.59  &&   98.91 &   94.56 &   98.96 &   97.36 &   97.35 &   98.02 &   98.88 &&   97.29\\
      Our Local (B=8) && 97.45 &&   97.69 &   94.46 &   96.64  &&   98.88 &   94.56 &   98.96 &   97.40 &   97.35 &   98.02 &   98.89 &&   97.30\\
      Our Global (B=8)&& 97.44 &&\bf97.77 &\bf94.80 &\bf96.86  &&\bf99.03 &\bf94.72 &\bf99.02 &\bf97.65 &   97.52 &\bf98.37 &\bf98.97 &&\bf97.47\\
            \midrule
      Parsey McParseface\hspace*{-.3cm} && - && 97.52 & 94.24 & 96.45 & - & - & - & - & - & - & - & - && - \\
      \bottomrule
    \end{tabular}
  }%\vspace*{-0.1cm}
  \caption{\label{tab:pos}
    Final POS tagging test set results on English WSJ and Treebank Union as well as CoNLL'09. We also show the performance of our pre-trained open source model, ``Parsey McParseface.''
}
\end{table*}

%%% Local Variables:
%%% mode: latex
%%% TeX-master: "../paper"
%%% End:

%%%%%%%%%%%%%%

Under the restriction that scores $\score(d_{1:i-1}, d_i, x_{1:i})$
depend only on the first $i$ input symbols,
the globally normalized model is still able to model the data in
Eq.~\eqref{eq:lbexample}, while the locally normalized model
fails (see Eq.~\ref{eq:localbad}). The ambiguity at input symbol
{\tt b} is naturally resolved when the next symbol ({\tt c} or {\tt
  e}) is observed, but the locally normalized model is not able to
revise its prediction.

It is easy to fix the locally normalized model for the example
in Eq.~\eqref{eq:lbexample} by
allowing scores $\score(d_{1:i-1}, d_i, x_{1:i+1})$ that 
take into account the input symbol $x_{i+1}$. 
More generally we can have a model of the form $\score(d_{1:i-1}, d_i, x_{1:i+k})$
where the integer $k$ specifies the amount of lookahead in the model.
Such lookahead is common in practice, but insufficient in general.
For every amount of lookahead $k$, 
we can construct examples that cannot be modeled
with a locally normalized model
by duplicating the middle input {\tt b} in (\ref{eq:lbexample}) $k+1$ times.
Only a local model with scores
$\score(d_{1:i-1}, d_i, x_{1:n})$ that considers the entire
input can capture any distribution $p(d_{1:n} | x_{1:n})$:
in this case the decomposition
$ p_L(d_{1:n} | x_{1:n}) = \prod_{i=1}^n p_L(d_i | d_{1:i-1}, x_{1:n}) $
makes no independence assumptions.

However, increasing the amount of context used as input comes
at a cost, requiring more powerful learning algorithms, and
potentially more training data. For a detailed analysis of the trade-offs
between structural features in CRFs and more powerful local classifiers
without structural constraints,
see \newcite{liang08structure}; in these experiments local classifiers
are unable to reach the performance of CRFs on problems such as parsing
and named entity recognition where structural constraints are important.
Note that there is nothing to preclude an approach that makes use of
both global normalization and more powerful scoring functions
$\score(d_{1:i-1}, d_i, x_{1:n})$, obtaining the best of both worlds.
The experiments that follow make use of both.

% !TEX root = ../multi_task.tex

We evaluate the presented MTL method on a number of problems. First, we use MultiMNIST \citep{multi_mnist}, an MTL adaptation of MNIST \citep{mnist}. Next, we tackle multi-label classification on the CelebA dataset \citep{celeba} by considering each label as a distinct binary classification task. These problems include both classification and regression, with the number of tasks ranging from 2 to 40. Finally, we experiment with scene understanding, jointly tackling the tasks of semantic segmentation, instance segmentation, and depth estimation on the Cityscapes dataset \citep{cityscapes}. We discuss each experiment separately in the following subsections.

The baselines we consider are (i) \textbf{uniform scaling:} minimizing a uniformly weighted sum of loss functions \mbox{$\frac{1}{T}\sum_t \lL^t$}, \mbox{(ii) \textbf{single task:}} solving tasks independently, \mbox{(iii) \textbf{grid search:}} exhaustively trying various values from $\{ c^t \in [0,1] | \sum_t c^t = 1\}$ and optimizing for $\frac{1}{T}\sum_t c^t \lL^t$, \mbox{(iv) \textbf{\citet{Kendall2018}:}} using the uncertainty weighting proposed by \citet{Kendall2018}, and \mbox{(v) \textbf{GradNorm:}} using the normalization proposed by \citet{Chen2018}.



\subsection{MultiMNIST}
\label{sec:multi_mnist_exp}

Our initial experiments are on MultiMNIST, an MTL version of the MNIST dataset \citep{multi_mnist}. In order to convert digit classification into a multi-task problem, \citet{multi_mnist} overlaid multiple images together. We use a similar construction. For each image, a different one is chosen uniformly in random. Then one of these images is put at the top-left and the other one is at the bottom-right. The resulting tasks are: classifying the digit on the top-left (task-L) and classifying the digit on the bottom-right (task-R). We use 60K examples and directly apply existing single-task MNIST models. The MultiMNIST dataset is illustrated in the supplement.

We use the LeNet architecture \citep{mnist}. We treat all layers except the last as the representation function $g$ and put two fully-connected layers as task-specific functions (see the supplement for details). We visualize the performance profile as a scatter plot of accuracies on task-L and task-R in Figure~\ref{fig:multi_mnist_performance_curve}, and list the results in Table~\ref{tab:multi_mnist}.

In this setup, any static scaling results in lower accuracy than solving each task separately (the single-task baseline). The two tasks appear to compete for model capacity, since increase in the accuracy of one task results in decrease in the accuracy of the other. Uncertainty weighting \citep{Kendall2018} and GradNorm \citep{Chen2018} find solutions that are slightly better than grid search but distinctly worse than the single-task baseline. In contrast, our method finds a solution that efficiently utilizes the model capacity and yields accuracies that are as good as the single-task solutions. This experiment demonstrates the effectiveness of our method as well as the necessity of treating MTL as multi-objective optimization. Even after a large hyper-parameter search, \emph{any} scaling of tasks does not approach the effectiveness of our method.



\subsection{Multi-Label Classification}

\begin{figure}[t]
\includegraphics[width=\textwidth]{radar_full_new}
\vspace{1mm}
\caption{Radar charts of percentage error per attribute on CelebA \citep{celeba}. Lower is better. We divide attributes into two sets for legibility: easy on the left, hard on the right. Zoom in for details.}
\label{fig:multi_label_radar}
\end{figure}


\begin{wraptable}{r}{0.3\textwidth}
%\vspace{-4mm}
\captionof{table}{Mean of error per category of MTL algorithms in multi-label classification on CelebA \citep{celeba}.}
\begin{tabular}{r@{\hspace{2mm}}c@{}}
\toprule
& Average  \\
&  error \\
\midrule
Single task & $8.77$ \\
Uniform scaling & $9.62$ \\
\citealt{Kendall2018} & $9.53$ \\
GradNorm & $8.44$ \\
Ours & $\mathbf{8.25}$  \\
\bottomrule
\end{tabular}
\label{table:multi_label_bar}
%\vspace{-5mm}
\end{wraptable}

Next, we tackle multi-label classification. Given a set of attributes, multi-label classification calls for deciding whether each attribute holds for the input. We use the CelebA dataset \citep{celeba}, which includes 200K face images annotated with 40 attributes. Each attribute gives rise to a binary classification task and we cast this as a 40-way MTL problem. We use ResNet-18 \citep{resnet} without the final layer as a shared representation function, and attach a linear layer for each attribute (see the supplement for further details).


We plot the resulting error for each binary classification task as a radar chart in Figure~\ref{fig:multi_label_radar}. The average over them is listed in Table~\ref{table:multi_label_bar}. We skip grid search since it is not feasible over 40 tasks. Although uniform scaling is the norm in the multi-label classification literature, single-task performance is significantly better. Our method outperforms baselines for significant majority of tasks and achieves comparable performance in rest. This experiment also shows that our method remains effective when the number of tasks is high.


\subsection{Scene Understanding}

To evaluate our method in a more realistic setting, we use scene understanding. Given an RGB image, we solve three tasks: semantic segmentation (assigning pixel-level class labels), instance segmentation (assigning pixel-level instance labels), and monocular depth estimation (estimating continuous disparity per pixel). We follow the experimental procedure of \citet{Kendall2018} and use an encoder-decoder architecture. The encoder is based on ResNet-50 \citep{resnet} and is shared by all three tasks. The decoders are task-specific and are based on the pyramid pooling module \citep{pspnet} (see the supplement for further implementation details).

Since the output space of instance segmentation is unconstrained (the number of instances is not known in advance), we use a proxy problem as in \citet{Kendall2018}. For each pixel, we estimate the location of the center of mass of the instance that encompasses the pixel. These center votes can then be clustered to extract the instances. In our experiments, we directly report the MSE in the proxy task. Figure~\ref{fig:cityscapes_performance_profile} shows the performance profile for each pair of tasks, although we perform all experiments on all three tasks jointly. The pairwise performance profiles shown in Figure~\ref{fig:cityscapes_performance_profile} are simply 2D projections of the three-dimensional profile, presented this way for legibility. The results are also listed in Table~\ref{tab:cityscapes_results}.

MTL outperforms single-task accuracy, indicating that the tasks cooperate and help each other. Our method outperforms all baselines on all tasks.


\subsection{Role of the Approximation}

In order to understand the role of the approximation proposed in Section~\ref{sec:approximation}, we compare the final performance and training time of our algorithm with and without the presented approximation in Table~\ref{tab:approximation_tradeoff} (runtime measured on a single Titan Xp GPU). For a small number of tasks (3 for scene understanding), training time is reduced by 40\%. For the multi-label classification experiment (40 tasks), the presented approximation accelerates learning by a factor of 25.

On the accuracy side, we expect both methods to perform similarly as long as the full-rank assumption is satisfied. As expected, the accuracy of both methods is very similar. Somewhat surprisingly, our approximation results in slightly improved accuracy in all experiments. While counter-intuitive at first, we hypothesize that this is related to the use of SGD in the learning algorithm. Stability analysis in convex optimization suggests that if gradients are computed with an error $\hat{\nabla}_\btheta \mathcal{L}^t = \nabla_\btheta \mathcal{L}^t + \mathbf{e}^t$ ($\btheta$ corresponds to $\btheta^{sh}$ in (\ref{eq:kkt_opt})), as opposed to $\mathbf{Z}$ in the approximate problem in \ref{eq:approx}, the error in the solution is bounded as $\|\hat{\mathbf{\alpha}} - \mathbf{\alpha} \|_2 \leq \mathcal{O}(\max_t \|\mathbf{e}^t\|_2)$. Considering the fact that the gradients are computed over the full parameter set (millions of dimensions) for the original problem and over a smaller space for the approximation (batch size times representation which is in the thousands), the dimension of the error vector is significantly higher in the original problem. We expect the $l_2$ norm of such a random vector to depend on the dimension.

In summary, our quantitative analysis of the approximation suggests that (i) the approximation does not cause an accuracy drop and (ii) by solving an equivalent problem in a lower-dimensional space, our method achieves both better computational efficiency and higher stability.

  {\small
  \begin{table}[t]
%  \vspace{-4mm}
  \caption{Effect of the MGDA-UB approximation. We report the final accuracies as well as training times for our method with and without the approximation.}
  %\vspace{1mm}
  \centering
  \begin{tabular}{@{}r@{\hspace{3mm}}c@{\hspace{3mm}}c@{\hspace{2mm}}c@{\hspace{2mm}}c@{}c@{\hspace{5mm}}c@{\hspace{2mm}}c@{}}
  \toprule
  & \multicolumn{4}{c}{Scene understanding (3 tasks)} &  & \multicolumn{2}{c}{Multi-label (40 tasks)}  \\
  \cmidrule(r){2-5} \cmidrule(lr){7-8}
                  & Training & Segmentation & Instance  & Disparity      & & Training & Average \\
                 & time     &  mIoU [\%]       & error [px] & error [px] & & time (hour)      & error \\
  \midrule
  Ours (w/o approx.) & $38.6$ & $66.13$ & $10.28$ & $2.59$ & & $429.9$ & $8.33$ \\
  Ours & $\mathbf{23.3}$ & $\mathbf{66.63}$ & $\mathbf{10.25}$ & $\mathbf{2.54}$  & & $\mathbf{16.1}$ & $\mathbf{8.25}$ \\
  \bottomrule
  \end{tabular}
  %\vspace{-2mm}
  \label{tab:approximation_tradeoff}
  \end{table}}

\section{Discussion}
\label{sec:discussion}

We derived a 
proof for the label bias problem
and the advantages of global models.
We then emprirically verified this theoretical superiority
by demonstrating state-of-the-art performance on three
different tasks.
%Our experiments showed globally normalized models consistently improving on locally normalized ones.
In this section we situate and compare our model to
previous work and provide two examples of the label bias problem
in practice.

\subsection{Related Neural CRF Work}

Neural network models have been been combined with 
conditional random fields and globally normalized models before.
\newcite{bottou-97} and \newcite{lecun-98h} describe global training of
neural network models for structured prediction problems.
\newcite{conditional_neural_fields} add a non-linear neural
network layer to a linear-chain CRF and
\newcite{neural_crf} apply a similar approach
to more general Markov network structures.
\newcite{recurrentCRF} and \newcite{Zheng_2015_ICCV}
introduce recurrence into the model and
\newcite{huang2015bidirectional} finally combine
CRFs and LSTMs.
These neural CRF models are limited to
sequence labeling tasks where exact inference is possible,
while our model works well when exact inference is intractable.
%In fact, we obtain our strongest results on dependency parsing
%where beam search is necessary because of the large
%tree structured output space.

\subsection{Related Transition-Based Parsing Work}

For early work on neural-networks for transition-based parsing,
see Henderson \shortcite{henderson:2003:NAACL,henderson:2004:ACL}.
Our work is closest to the work of
\newcite{weiss-etAl:2015:ACL}, \newcite{zhou-etAl:2015:ACL}
and \newcite{watanabe-sumita:2015:ACL};
in these approaches global normalization is added to the
local model of \newcite{chen-manning:2014:EMNLP}.
Empirically, \newcite{weiss-etAl:2015:ACL}
achieves the best performance, even though
their model keeps the parameters of the locally
normalized neural network fixed and only
trains a perceptron that uses the activations as features.
Their model is therefore limited in its ability to
revise the predictions of the locally normalized model.
In Table~\ref{tab:depth} we show that full backpropagation
training all the way to the word embeddings
is very important and significantly contributes
to the performance of our model.
We also compared training under the CRF objective with a 
Perceptron-like hinge loss between the gold and best elements of the beam.
When we limited the backpropagation depth to training only the top layer $\theta^{(d)}$,
we found negligible differences in accuracy:
93.20\% and 93.28\% for the CRF objective and hinge loss respectively.
However, when training with full backpropagation the CRF accuracy 
is 0.2\% higher and training converged more than 4$\times$
faster.

\begin{table}[t]
  \centering
  \scalebox{0.95}{%
    \renewcommand{\arraystretch}{1.0}%
    \setlength\tabcolsep{6pt}%
    \begin{tabular}[h]{lcc}
      \toprule
       Method & UAS & LAS \\
      \midrule
      Local (B=1)         & 92.85 & 90.59 \\
      Local (B=16)        & 93.32 & 91.09 \\
      \midrule
      Global (B=16) $\{\theta^{(d)}\}$    & 93.45 & 91.21 \\
      Global (B=16) $\{W_2, \theta^{(d)}\}$& 94.01 & 91.77 \\
      Global (B=16) $\{W_1, W_2, \theta^{(d)}\}$& 94.09 & 91.81 \\
      Global (B=16) (full)                 & 94.38 & 92.17 \\
      \bottomrule
    \end{tabular}
  }
  \caption{WSJ dev set scores for successively deeper levels of backpropagation.
    The {\em full} parameter set corresponds to backpropagation all the way to the embeddings.
    $W_i$: hidden layer $i$ weights.
  }
  \label{tab:depth}
\end{table}


\begin{table*}[t]
  \centering%
  \small
  \scalebox{0.95}{%
    \setlength{\tabcolsep}{2pt}%
    %\begin{tabular}{p{7.5cm}cl}
    \begin{tabular}{llcl}
      \toprule
      Method & Predicted compression & $p_L$ & $p_G$ \\
      \midrule
      Local (B=1) & \textcolor{gray}{In Pakistan, former leader} Pervez Musharraf has appeared in court \textcolor{gray}{for the first time, on treason charges}. & $0.13$ & $0.05$\\
      Local (B=8) &\textcolor{gray}{In Pakistan, former leader Pervez Musharraf has appeared in court for the first time, on treason charges}. & {\bf $0.16$} &% $e^{-9}$ 
$<$$10^{-4}$\\
      Global (B=8) & \textcolor{gray}{In Pakistan, former leader} Pervez Musharraf has appeared \textcolor{gray}{in court for the first time,} on treason charges. & $0.06$  & {\bf $0.07$} \\
      \bottomrule
    \end{tabular}
  }
  \caption{\label{tab:sent-compression-label-bias-example}
    Example sentence compressions where the label bias of the locally normalized 
    model leads to a breakdown during beam search.
    The probability of each compression under the local ($p_L$) and global ($p_G$) models shows that only the global model can properly represent zero probability for the empty compression.
  }
\end{table*}


\newcite{zhou-etAl:2015:ACL} perform full 
backpropagation training like us, 
but even with a much larger beam, their performance is significantly
lower than ours. We also apply our model to two additional tasks,
while they experiment only with dependency parsing.
Finally, \newcite{watanabe-sumita:2015:ACL} introduce recurrent
components and additional techniques like max-violation updates
for a corresponding constituency parsing model.
In contrast, our model does not require any recurrence
or specialized training.

\subsection{Label Bias in Practice}

We observed several instances of severe label bias in the sentence
compression task.  Although using beam search with the local model
outperforms greedy inference on average, beam search leads the local
model to occasionally produce empty compressions
(Table~\ref{tab:sent-compression-label-bias-example}).  It is important to
note that these are {\em not} search errors: the empty compression has
higher probability under $p_L$ than the prediction from greedy
inference. However, the more expressive globally normalized model
does not suffer from this limitation, and correctly gives the empty
compression almost zero probability.

We also present some
empirical evidence that the label bias problem is severe in
parsing. We trained models where the scoring functions in parsing
at position $i$ in the sentence are limited to considering only tokens
$x_{1:i}$; hence unlike the full parsing model, there is no
ability to look ahead in the sentence when making a
decision.\footnote{This setting may be important in some applications,
  where for example parse structures for sentence prefixes are
  required, or where the input is received one word at a time and
  online processing is beneficial.} The result for a greedy model
under this constraint is 76.96\% UAS; for a locally normalized model
with beam search is 81.35\%; and for a globally normalized model
is 93.60\%. Thus the globally normalized model gets very close to the
performance of a model with full lookahead, while the locally
normalized model with a beam gives dramatically lower performance.  In
our final experiments with full lookahead, the globally normalized
model achieves 94.01\% accuracy, compared to 93.07\% accuracy for a
local model with beam search. Thus adding lookahead allows the local
model to close the gap in performance to the global model; however
there is still a significant difference in accuracy, which may in
large part be due to the label bias problem.

A number of authors have considered modified training procedures for
greedy models, or for locally normalized models.
\newcite{daume09searn} introduce Searn, an algorithm that allows a
classifier making greedy decisions to become more robust to errors
made in previous decisions. \newcite{goldberg2013training} describe
improvements to a greedy parsing approach that makes use of methods
from imitation learning \cite{bagnell2011imitation} to augment the
training set. Note that these methods are focused on greedy
models: they are unlikely to solve the label bias problem when used in
conjunction with beam search, given that the problem is one of
expressivity of the underlying model. More recent work
\cite{henderson2015,vaswani2016} has augmented locally normalized
models with {\em correctness probabilities} or {\em error states},
effectively adding a step after every decision where the probability
of correctness of the resulting structure is evaluated. This gives
considerable gains over a locally normalized model, although
performance is lower than our full globally normalized approach.

\section{Conclusions}

Our work is motivated by two major deficiencies in training the current generative models for text generation: exposure bias and a loss which does not operate at the sequence level.
While Reinforcement learning can potentially address these issues, it struggles in settings when 
there are very large action spaces, such as in text generation. Towards that end, 
we propose the MIXER algorithm, which deals with these issues and enables successful training of reinforcement learning models for text generation. 
We achieve this by replacing the initial random policy with the optimal policy of a cross-entropy trained model and by gradually exposing the model more and more to its own predictions in an incremental learning framework.




%. First, the exposure bias affecting the commonly used cross-entropy loss. 
%While the model sees only ground truth inputs at training time, at test time model predictions are fed back as input to generate a full sequence. 
%Second, current text generation systems are often trained to predict the next word in the sequence without taking into account the quality of the % overall sequence. 
% These discrepancies make the generation process brittle.
%Reinforcement learning is a framework that can address these issues. 
%First, at training time the model is used to generate an entire sequence of actions. 
%Second, the reward does not need to factor over individual words nor does it need to be differentiable. 
%Therefore, we can easily and directly operate at the sequence level, generate at training time and optimize our model towards any desired metric, such as BLEU and ROUGE. 
%One challenge with reinforcement learning is that it struggles with very large action spaces such as for text generation.

% Mixed Incremental Cross-Entropy Reinforce (MIXER) 
%The algorithm we propose, MIXER, 
%deals with this issue and enables successful training of reinforcement learning models for text generation. 
%We achieve this by replacing the initial random policy with the optimal policy of a cross-entropy trained model and by gradually exposing the model more and more to its own predictions in an incremental learning framework.

Our results show that MIXER outperforms three strong baselines for greedy generation and it is very competitive with beam search. 
The approach we propose is agnostic to the underlying model or the form of the reward function. 
% We are free to use any other metric as reward such as ROUGE or METEOR instead of BLEU. 
% Similarly, we may use a different parametric model such as a feed- forward network or an LSTM \citep{lstm}.
In future work we would like to design better estimation techniques for the average reward $\bar{r}_t$, because poor estimates can lead to slow convergence of both REINFORCE and MIXER. 
Finally, our training algorithm relies on a single sample while it would be interesting to investigate the effect of more comprehensive search methods at training time.


% Our work addresses two major deficiencies in training the current generative models for text generation. First, it addresses the {\it exposure bias} affecting the commonly used cross-entropy loss. 
% %While the model sees only ground truth inputs at training 
% %time, at test time model predictions are fed back as input to generate a full sequence. 
% Second, it directly tries to optimize for the final evaluation metric, namely, BLEU. 
% %current text generation systems are often trained to predict the next word in the sequence without taking into account the quality of the overall sequence. These discrepancies make the generation process brittle. 
% Both these objectives are accomplished by the proposed Mixed Incremental Cross-Entropy Reinforce (MIXER) algorithm. 
% %Reinforcement learning is a framework that can address these issues. First, at training time the model is used to generate an entire sequence of actions. Second, the reward does not need to factor over individual words nor does it need to be differentiable. Therefore, we can easily and directly operate at the sequence level, generate at training time and optimize our model towards BLEU, our test time evaluation metric. One challenge with reinforcement learning is that it struggles with very large action spaces such as for text generation.
% MIXER is an extension of the REINFORCE algorithm applied to text generation, which 
% %Mixed Incremental Cross-Entropy Reinforce (MIXER) deals with this issue and enables
% %successful training of reinforcement learning models for text generation.
% replaces the initial random policy with the optimal policy of
% a cross-entropy trained model and it gradually exposes the model more and more to its own predictions in an incremental learning framework.

% Our results show that MIXER outperforms three strong baselines for greedy generation and it is very competitive with beam search. 
% The approach we propose is agnostic to the underlying model or the form of the reward function. 
% We are free to use any other metric as reward such as ROUGE or METEOR instead of BLEU. 
% Similarly, we may use a different parametric model such as a feed-forward network or an LSTM~\citep{lstm}.

% For future we would like to design better estimation techniques for the average reward $\bar{r}_t$, because poor estimates can lead to slow convergence of both REINFORCE and MIXER.
% Finally, our training algorithm relies on a single sample while it would be interesting to investigate the effect of more comprehensive search methods at training time.

% In this study, we investigated sequence level training algorithms for RNNs with the goal to improve text generation.
% Today, the dominant training protocol is cross-entropy loss, which optimizes the prediction of the next word in the sequence. However, at test time the model is asked to predict several words in the future by re-circulating its own prediction back to the input. 
% The problem of predicting several steps in the future while obtaining delayed feedback, and to perform prediction via a discrete sequence of actions inspired us to apply reinforcement learning techniques. Unfortunately, reinforcement learning techniques do not usually handle well large action spaces, like those we encounter in typical language modeling applications.

% MIXER addresses these limitations through pre-training and incremental learning. 

% MIXER addresses these limitations by leveraging both the fact that we have access to the optimal policy and by using incremental learning.
% Since we have examples of ground truth generation, we can "pre-train" the model for next step prediction via cross-entropy. This drastically reduces the actual search space. By using incremental learning, the model is then able to gradually produce stable sequences and to make effective use of its own predictions.

% Our empirical validation shows that the model we propose achieves the best BLEU score compared to three strong baselines. Moreover, generations can be further improved by using beam search. Note that the approach we proposed is agnostic of the particular underlying model and metric. We can easily replace BLEU with ROUGE, METEOR, \etc by simply swapping the function that computes rewards within the training loop. Similarly, the training algorithm applies to any type of model and RNN, LSTM~\citep{lstm} included.



% There are several avenues of future investigation. First, REINFORCE upon which we build, requires careful estimation of the average reward. Poor estimation of this value can yield very slow convergence. More generally, searching at training time is still an unsolved problem. In particular, it would be very powerful to include beam search also at training time. 


\ifaclfinal
% No acknowledgements in the anonymous submission.
\section*{Acknowledgements}

We would like to thank Ling Wang for training his C2W part-of-speech tagger on our setup,
and Emily Pitler, Ryan McDonald, Greg Coppola and
Fernando Pereira for tremendously helpful discussions.
Finally, we are grateful to all members of the Google Parsing Team.
\else\fi

% Bibliography.
\balance
\bibliographystyle{acl2016}
\bibliography{paper}

\end{document}

%
% File emnlp2016.tex
%

\documentclass[11pt,letterpaper]{article}
\usepackage{emnlp2016}
\usepackage{times}
\usepackage{latexsym}
\usepackage{basecommon}
\usepackage{graphicx}
\usepackage{url}
\usepackage[font={small}]{caption}

\newcommand{\RNN}{\mathrm{\mathbf{RNN}}}
\newcommand{\BPTT}{\mathrm{\mathbf{BPTT}}}
\newcommand{\BRNN}{\mathrm{\mathbf{BRNN}}}
\newcommand{\longpfx}[1]{\ensuremath{w_1 \cdots w_{#1}}}
\newcommand{\longgoldpfx}[1]{\ensuremath{y_1 \cdots y_{#1}}}
\newcommand{\pfx}[1]{\ensuremath{w_{1:{#1}}}}
\newcommand{\goldpfx}[1]{\ensuremath{y_{1:{#1}}}}
\newcommand{\beampred}[2]{\ensuremath{\hat{y}_{1:{#1}}^{({#2})}}}
\DeclareMathOperator{\suk}{succ}
\DeclareMathOperator{\topK}{topK}
\DeclareMathOperator{\score}{score}
\newcommand{\nicein}{\ensuremath{\,{\in}\,}}
\newcommand{\niceq}{\ensuremath{\,{=}\,}}

% Uncomment this linlle for the final submission:
\emnlpfinalcopy

%  Enter the EMNLP Paper ID here:
\def\emnlppaperid{***}

% To expand the titlebox for more authors, uncomment
% below and set accordingly.
% \addtolength\titlebox{.5in}    

\newcommand\BibTeX{B{\sc ib}\TeX}


\title{Sequence-to-Sequence Learning \\ as Beam-Search Optimization}

% Author information can be set in various styles:
% For several authors from the same institution:
% \author{Author 1 \and ... \and Author n \\
%         Address line \\ ... \\ Address line}
% if the names do not fit well on one line use
%         Author 1 \\ {\bf Author 2} \\ ... \\ {\bf Author n} \\
% For authors from different institutions:
% \author{Author 1 \\ Address line \\  ... \\ Address line
%         \And  ... \And
%         Author n \\ Address line \\ ... \\ Address line}
% To start a seperate ``row'' of authors use \AND, as in
% \author{Author 1 \\ Address line \\  ... \\ Address line
%         \AND
%         Author 2 \\ Address line \\ ... \\ Address line \And
%         Author 3 \\ Address line \\ ... \\ Address line}
% If the title and author information does not fit in the area allocated,
% place \setlength\titlebox{<new height>} right after
% at the top, where <new height> can be something larger than 2.25in
\author{Sam Wiseman \and Alexander M. Rush\\
 School of Engineering and Applied Sciences \\ Harvard University \\ Cambridge, MA, USA \\ {\tt \{swiseman,srush\}@seas.harvard.edu }}

\date{}

\begin{document}

\maketitle

\begin{abstract}
  Sequence-to-Sequence (seq2seq) modeling has rapidly become an
  important general-purpose NLP tool that has proven effective for
  many text-generation and sequence-labeling tasks. Seq2seq builds on deep neural language modeling and inherits its
  remarkable accuracy in estimating local, next-word
  distributions. In this work, we introduce a model and beam-search training
  scheme, based on the work of \newcite{daume05learning}, that extends
  seq2seq to learn global sequence scores. This
  structured approach avoids classical biases associated with local
  training and unifies the training loss with the test-time usage,
  while preserving the proven model architecture of seq2seq and
  its efficient training approach. We show that our system outperforms a
  highly-optimized attention-based seq2seq system and other baselines
  on three different sequence to sequence tasks: word ordering,
  parsing, and machine translation.
\end{abstract}


\section{Introduction}
\section{Introduction}
\label{sec:intro}

Language modeling is among the important problems that require modeling long-term dependency, with successful applications such as unsupervised pretraining~\citep{dai2015semi,peters2018deep,radford2018improving,devlin2018bert}.
However, it has been a challenge to equip neural networks with the capability to model long-term dependency in sequential data.
Recurrent neural networks (RNNs), in particular Long Short-Term Memory (LSTM) networks~\citep{hochreiter1997long}, have been a standard solution to language modeling and obtained strong results on multiple benchmarks.
Despite the wide adaption, RNNs are difficult to optimize due to gradient vanishing and explosion~\citep{hochreiter2001gradient}, and the introduction of gating in LSTMs and the gradient clipping technique~\citep{graves2013generating} might not be sufficient to fully address this issue.
% ,pascanu2012understanding
Empirically, previous work has found that LSTM language models use 200 context words on average~\citep{khandelwal2018sharp}, indicating room for further improvement.

On the other hand, the direct connections between long-distance word pairs baked in attention mechanisms might ease optimization and enable the learning of long-term dependency~\citep{bahdanau2014neural,vaswani2017attention}.
Recently, \citet{al2018character} designed a set of auxiliary losses to train deep Transformer networks for character-level language modeling, which outperform LSTMs by a large margin.
Despite the success, the LM training in~\citet{al2018character} is performed on separated fixed-length segments of a few hundred characters, without any information flow across segments.
As a consequence of the fixed context length, the model cannot capture any longer-term dependency beyond the predefined context length.
In addition, the fixed-length segments are created by selecting a consecutive chunk of symbols without respecting the sentence or any other semantic boundary.
Hence, the model lacks necessary contextual information needed to well predict the first few symbols, leading to inefficient optimization and inferior performance.
We refer to this problem as \textit{context fragmentation}.

%However, the context length is fixed to hundreds of characters and thus it is not possible to model longer-term dependency. Moreover, it is not clear how the model performs on word-level language modeling data, as the granularity changes.

% Moreover, using auxiliary losses brings additional challenges such as properly tuning the mixture weights and the loss decay schedule.

To address the aforementioned limitations of fixed-length contexts, we propose a new architecture called Transformer-XL (meaning extra long).
We introduce the notion of recurrence into our deep self-attention network. In particular, instead of computing the hidden states from scratch for each new segment, we reuse the hidden states obtained in previous segments.
The reused hidden states serve as memory for the current segment, which builds up a recurrent connection between the segments.
As a result, modeling very long-term dependency becomes possible because information can be propagated through the recurrent connections.
Meanwhile, passing information from the previous segment can also resolve the problem of context fragmentation.
More importantly, we show the necessity of using relative positional encodings rather than absolute ones, in order to enable state reuse without causing temporal confusion.
Hence, as an additional technical contribution, we introduce a simple but more effective relative positional encoding formulation that generalizes to attention lengths longer than the one observed during training.

Transformer-XL obtained strong results on five datasets, varying from word-level to character-level language modeling.
Transformer-XL is also able to generate relatively coherent long text articles with \textit{thousands of} tokens (see Appendix \ref{sec:gen}), trained on only 100M tokens.
% Transformer-XL improves the previous state-of-the-art (SoTA) results from 1.06 to 0.99 in bpc on enwiki8, from 1.13 to 1.08 in bpc on text8, from 20.5 to 18.3 in perplexity on WikiText-103, and from 23.7 to 21.8 in perplexity on One Billion Word.
% Transformer-XL improves the previous state-of-the-art (SoTA) results to 0.99 in bpc on enwiki8, 1.08 in bpc on text8, 18.3 in perplexity on WikiText-103, and 21.8 in perplexity on One Billion Word.
% On small data, Transformer-XL also achieves a perplexity of 54.5 on Penn Treebank without finetuning, which is SoTA when comparable settings are considered.

Our main technical contributions include introducing the notion of recurrence in a purely self-attentive model and deriving a novel positional encoding scheme. These two techniques form a complete set of solutions, as any one of them alone does not address the issue of fixed-length contexts. Transformer-XL is the first self-attention model that achieves substantially better results than RNNs on both character-level and word-level language modeling.

% On WikiText-103, Transformer-XL improves the previous state-of-the-art (SoTA) results from 33 perplexity to 24, with a relative reduction of 27\%. On enwiki8 character-level language modeling, Transformer-XL achieves a SoTA bpc of 1.03, which outperforms \cite{al2018character} by 0.03 with 60+\% fewer parameters. Given a more common model size with 40+M parameters, Transformer-XL achieves a bpc of 1.06, compared to 1.11 by \cite{al2018character}. Transformer-XL also achieves perplexities of 54.5 on Penn Treebank and 29.4 on One Billion Word, which are SoTA when comparable settings are considered.

% Due to the ability of modeling long-range context, our best model uses attention lengths of 1,600 and 3,800 on WikiText-103 and enwiki8 respectively. We also devise a metric called \textit{Relative Effective Context Length} (RECL) that aims to fairly compare the ability of long-range dependency modeling.
% % perform a fair comparison of the gains brought by increasing the context lengths for different models.
% In this setting, Transformer-XL learns a RECL of 900 words on WikiText-103, while the numbers for recurrent networks and Transformer are only 500 and 128.

% We use two methods to quantitatively study the effective lengths of Transformer-XL and the baselines. Similar to \cite{khandelwal2018sharp}, we gradually increase the attention length at test time until no further noticeable improvement ($\sim$0.1\% relative gains) can be observed. Our best model in this settings use attention lengths of 1,600 and 3,800 on WikiText-103 and enwiki8 respectively.
% %In addition, since the effective context length of Transformer-XL can be longer than the attention length due to our recurrent formulation, we devise a metric called \textit{Relative Effective Context Length} (RECL) that aims to perform a fair comparison of the gains brought by increasing the context lengths for different models.
% In addition, we devise a metric called \textit{Relative Effective Context Length} (RECL) that aims to perform a fair comparison of the gains brought by increasing the context lengths for different models.
% In this setting, Transformer-XL learns a RECL of 900 words on WikiText-103, while the numbers for recurrent networks and Transformer are only 500 and 128.


\section{Related Work}
\label{sec:relatedwork}
\section{Related Work}

% Various approaches for document image classification have been proposed over the years. Generally, document image classification approaches are divided into two major groups, structure/layout based, and content based. This section provides an overview of some important works which have been reported in reference to structure or content based document classification.% \subsection{Content based Document Classification}

Over the years, different methods have been proposed for document image classification. The overall classification methods can be divided into three distinct categories.
The first category exploits structure and layout similarities, while the second focuses on developing local and global features that could be used for document classification. The third category is based on deep \ac{cnn}s that extract the features automatically for document classification. This section provides a summary of the important related work regarding the above mentioned three categories.

\begin{figure*}
    \begin{subfigure}{0.22\linewidth}
        \centering
        \fbox{\includegraphics[height=0.65\textheight]{architectures/alexnet.pdf}}
        \subcaption{AlexNet}
        \label{fig:alexnet}
    \end{subfigure}
    \begin{subfigure}{0.21\linewidth}
        \centering
        \fbox{\includegraphics[height=0.65\textheight]{architectures/vgg.pdf}}
        \subcaption{VGG-16}
        \label{fig:vgg}
    \end{subfigure}
    \begin{subfigure}{0.36\linewidth}
        \centering
        \fbox{\includegraphics[height=0.65\textheight]{architectures/googlenet.pdf}}
        \subcaption{GoogLeNet}
        \label{fig:googlenet}
    \end{subfigure}
    \begin{subfigure}{0.19\linewidth}
        \centering
        \fbox{\includegraphics[height=0.65\textheight]{architectures/resnet.pdf}}
        \subcaption{ResNet-50}
        \label{fig:resnet}
    \end{subfigure}
    \caption{Deep CNN architectures used in this work}
    \label{fig:deepcnn}
\end{figure*}

Dengel and Dubiel~\cite{doclass_Dengel95} used layout structure printed documents. They used top-down induction in decision trees to convert printed documents into a complementary logical structure.
Bagdanov and Worring~\cite{doclass_Bagdanov2001} classify machine-printed documents by using the Attributed Relational Graphs (ARGs).
Byun and Lee~\cite{doclass_Byun2000} used parts of the documents for the recognition. They reasoned that processing complete documents is time-consuming. The document classification was performed on parts of the documents using DP algorithm. Their approach was fast but the applicability is limited to forms. Shin and Doermann~\cite{doclass_shin} proposed an approach that used layout structural similarity for full or partial image matching for retrieval. 
Kevyn and Nickolov~\cite{Collins-thompson02aclustering-based} used both the layout and the text features for matching the documents for retrieval. 

Jayant et al.~\cite{doclass_Kumar12} propose a method that relies on the patch code words derived from the document images. The code book is learned independently of the class labels of the documents. In the first step, the images are recursively partitioned both in horizontal and vertical direction for modeling spatial relationships. Subsequently, a histogram for each partition is calculated that is used for the classification.
Following the same idea of developing the code book, another work presented by Jayant et al.~\cite{doclass_Kumar14} build a codebook of SURF descriptors extracted from training images. Then, histograms of codewords are created similar to~\cite{doclass_Kumar12}. A Random Forest classifier is used for classification. The applicability of the approach is shown in the presence of limited data.
Chen et al.~\cite{doclass_Chen12} propose a method based on low-level image features to classify documents. The approach is limited to structured documents. An important point is that one could obtain the registration of two images by matching the feature points.
Joutel et al.~\cite{doclass_Joutel2007} proposed a method that used curvelet transformation for indexing and querying the documents at different image scales. Their method is designed particularly for large databases of handwritten manuscripts. Kochi and Saitoh~\cite{doclass_Kochi99} used textual descriptions of document images for information extraction from documents. The method is limited to semi-structured documents and assumes a pre-defined knowledge is available for the document classes.
Reddy and Govindaraju~\cite{doclass_umamaheswara08} used binary images for the classification of the documents. They use pixel information and calculate pixel densities.  They used k-means clustering supported by adaptive boosting. The method is evaluated on the benchmark NIST scanned special tax form databases $2$ and $6$.

%The approach only deals with structured documents which are mostly-text and with pre-printed contents.  doclass_Kumar12,doclass_Chen12, doclass_Kumar14

% Jayant et al.~\cite{doclass_Kumar12} propose a method based on statistics of patch code words. Starting with a set of wanted and a random set of unwanted images, raw-image patches extracted from the unlabeled images to learn a code book. Spatial relationships between patches are modelled by recursively partitioning the image horizontally and vertically and a histogram of patch-codewords is computed for each partition. 

% In another work, Jayant et al.~\cite{doclass_Kumar14} build a codebook of SURF descriptors extracted from some training images. Then histograms of codewords are created similar to~\cite{doclass_Kumar12}. Later a Random Forest classifier is used for classification. The system performs reasonably even for limited training data. Chen et al.~\cite{doclass_Chen12} propose a method based on SIFT descriptors to classify documents. The approach only deals with structured documents which are mostly-text and with pre-printed contents.


%low level pixel density information from the binary images. The proposed system is based on the k-means algorithm and supported by adaptive boosting. The results are reported on the benchmark NIST scanned special tax form databases $2$ and $6$.


% Joutel et al.~\cite{doclass_Joutel2007} use Curvelet transform as a multiscale method for indexing linear singularities and curved handwritten shapes in documents images. 
% Their method detects oriented and curved fragments at different scales and searches for similar handwritten samples in large manuscripts databases. 
% Kochi and Saitoh~\cite{doclass_Kochi99} compare  the textual elements of document images. The presented system is robust against shifts or noise in the target documents and can handle semi-formatted documents..
% Reddy and Govindaraju~\cite{doclass_umamaheswara08} use low level pixel density information from the binary images. The proposed system is based on the k-means algorithm and supported by adaptive boosting. The results are reported on the benchmark NIST scanned special tax form databases $2$ and $6$.
% \subsection{Structure based Document Classification}



%Based on Layout Structural Similarity doclass_Byun2000, doclass_shin, Collins-thompson02aclustering-based


The pioneering work that performed document classification using \ac{cnn}s used a rather shallow network for classification~\cite{lekang_14_a}. Nevertheless, the proposed approach outperformed structural similarity based methods and shows the potential of automatic feature learning for document classification using \ac{cnn}s. The reason may be that deep networks require a lot of data for training and at that time the standard challenging dataset consisted of only $3,482$ images.
Afzal et. al.~\cite{afzal2015deepdocclassifier} and Harley et. al.~\cite{harley2015evaluation} provided a breakthrough when they showed that it is possible to use transfer learning and the features that are learned from general (daily life) images can be used for the classification of document images~\cite{afzal2015deepdocclassifier}. They achieved a significant improvement over \ac{cnn} based methods that were the \sota at that time.
%Another significant contribution is presented by Harley et. al.~\cite{harley2015evaluation}. 
%The concept of their work was the same as Afzal et. al., however, they used the features from deep \ac{cnn} for document retrieval and showed that validity of their approach.
Another notable contribution by Harley et. al.~\cite{harley2015evaluation} was that they introduced a dataset consisting of $400,000$ documents divided into $16$ classes.
This allowed for the evaluation of deep neural networks using a significant amount of data. 

The \sota in deep \ac{cnn}s has advanced significantly in recent years and there has been no comprehensive study regarding the impact of deep architectures for document classification. Moreover, there is no study that explores transfer learning from document images and also there is no report of the impact of the amount of training images. The presented work takes into account these issues and performs a comprehensive set of experiments to fill the gaps that exist. Eventually, this study leads to an approach that can reduce the error by more than half and therefore provides another leap forward in the domain of document image classification.


% Deep convolutional networks have shown successful results on large benchmark datasets consisting of more than one million images, such as ImageNet~\cite{cnn_alexnet_nips2014}.
% Alex et al.~\cite{cnn_alexnet_nips2014} trained a large, deep convolutional neural network to classify the $1.2$ million high-resolution images in the ImageNet LSVRC-2010 contest into $1000$ different classes. An important finding was that the learned deep representation is transferable across different tasks.
% Sermanet et al.~\cite{conf/cvpr/SermanetKCL13} suggested to use unsupervised pre-training, followed by supervised fine-tuning for pedestrian detection. Similarly, supervised pre-training was proved helpful in different computer vision and multimedia settings w.r.t. a concept-bank paradigm~\cite{TorresaniSzummerFitzgibbon10}. Recently, Girshick et al.~\cite{journals/corr/GirshickDDM13} showed that, for dealing with scarce data, supervised pre-training on larger data and then fine-tuning on smaller problem-specific dataset improves classification results. Based on this background, to the best of our knowledge, this paper is the first attempt ever when pre-trained CNN are optimized for document image classification.



\section{Background and Notation}
\label{sec:background}
In the simplest seq2seq scenario, we are given a collection of source-target
sequence pairs and tasked with learning to generate
target sequences from source sequences. For instance, we might view machine translation in this way, where in particular we attempt to generate English sentences from (corresponding) French sentences. Seq2seq models are part of the broader class of ``encoder-decoder'' models~\cite{cho14on}, which first use an encoding model to transform a source object into an encoded representation $\boldx$. Many different sequential
(and non-sequential) encoders have proven to be effective for
different source domains. In this work we are agnostic to the
form of the encoding model, and simply assume an abstract source
representation $\boldx$. %In experiments we utilize an attention-based LSTM encoder \cite{} which has shown to be effective for many tasks \cite{}.

Once the input sequence is encoded, seq2seq models generate a target
sequence using a \textit{decoder}. The decoder is tasked with
generating a target sequence of words from a target vocabulary $\mcV$. In particular, words are generated sequentially by conditioning on the input representation $\boldx$ and on the previously generated words or \textit{history}. We use the notation $\pfx{T}$ to refer to an arbitrary word sequence of length $T$, and the notation $\goldpfx{T}$ to refer to the \textit{gold} (i.e., correct) target word sequence for an input $\boldx$. 

Most seq2seq systems utilize a recurrent neural network (RNN) for the decoder model. Formally, a recurrent neural network is a parameterized non-linear
function $\RNN$ that recursively maps a sequence of vectors to a
sequence of hidden states. Let $\boldm_1, \ldots, \boldm_T$ be a
sequence of $T$ vectors, and let $\boldh_0$ be some initial state
vector. Applying an RNN to any such sequence yields hidden states
$\boldh_t$ at each time-step $t$, as follows:
\begin{align*}
\boldh_t \gets \RNN(\boldm_t, \boldh_{t-1}; \btheta),
\end{align*}
where $\btheta$ is the set of model parameters, which are shared over time. In this work, the vectors $\boldm_t$ will always correspond to the embeddings of a target word sequence $\pfx{T}$, and so we will also write $\boldh_t \gets \RNN(w_t, \boldh_{t-1}; \btheta)$, with $w_t$ standing in for its embedding.
 
%To back-propagate errors through a recurrent neural network, we accumulate the 
%gradients of each state with respect to subsequent states by running a backward procedure we will refer to as $\BRNN$ at each time-step (starting at the penultimate step): 
%\begin{align*}
%\nabla_{\boldh_t} \mcL \gets \BRNN(y_{t+1}, \boldh_{t},\nabla_{\boldh_{t+1}} \mcL),
%\end{align*}
%$\BRNN$ takes into account $\boldh_t$'s contribution to any loss incurred from its next-step prediction, as well as to any loss incurred through $\boldh_{t+1}$. In what follows, we will often abbreviate $\nabla_{\boldh_t} \mcL$ as $\nabla_{\boldh_t}$.  
%%\begin{align*}
%%\nabla_{\boldh_t} \mcL \gets \nabla_{\boldh_t} \mcL + \BRNN(\nabla_{\boldh_{t+1}} \mcL, \boldm_t, \boldh_{t}).
%%\end{align*}
%%Note that $\boldm_t$ is the embedding corresponding to output word $w_t$. 
%Running this $\BRNN$ procedure from $t \niceq T$ to $t \niceq 1$ is known as back-propagation through time (BPTT).

%\textbf{something about BPTT}

%  which takes the form of a recurrent
% neural network (RNN). 

% where a
% decoder RNN generates a target sequence of T
% words w1 · · · wT (such as a translation or summary),
% from an

% As RNN decoding is the main focus of this work,
% we now describe this process in greater detail.  

RNN decoders are typically trained to act as conditional language
models. That is, one attempts to model the probability of the $t$'th target
word conditioned on $\boldx$ and the target history by stipulating that $p(w_{t} | \pfx{t-1}, \boldx) \niceq g(w_{t},
\boldh_{t-1}, \boldx)$, for some parameterized function $g$ typically computed with an affine layer followed by a softmax. In computing these probabilities, the state $\boldh_{t-1}$ represents the target history, and $\boldh_0$ is typically set to be some function of $\boldx$. The complete model (including encoder) is trained,
analogously to a neural language model, to minimize the cross-entropy
loss at each time-step while conditioning on the gold history in the
training data. That is, the model is trained to minimize $-\ln \prod_{t=1}^{T} p(y_{t} |\goldpfx{t-1}, \boldx)$.

Once the decoder is trained, discrete sequence generation can be
performed by approximately maximizing the probability of the target
sequence under the conditional distribution,
$\hat{y}_{1:T} \niceq \mathrm{argbeam}_{w_{1:T}} \prod_{t=1}^{T} p(w_t |\pfx{t-1}, \boldx)$, where we use the notation $\mathrm{argbeam}$ to emphasize that the decoding process requires heuristic search, since the RNN model is non-Markovian. In practice, a simple beam search
procedure that explores $K$ prospective histories at each time-step
has proven to be an effective decoding approach. However, as noted above,
decoding in this manner after conditional language-model style training \textit{potentially} suffers from the issues of exposure bias and label bias, which motivates the work of this paper.

% However we note that this procedure potentially
% suffers from the issues 


% and we will often omit the
% $\boldx$ argument to $f$ when there is only a single $\boldx$ in
% question.


  

% , which often takes the form of a recurrent
% neural network. 



% For the sake of this work the sequential form of the input sequence is
% actually 


% Seq2seq is highly related to the corresponding 
% \textit{encoder-decoder} approached  



%  $w_{1:s}$ 
% $w_{1:t}$


% It has become popular in recent years to use RNNs within an
% ``encoder-decoder'' framework, where a decoder RNN generates a target
% sequence of $T$ words $\longpfx{T}$ (such as a translation or
% summary), from an 


%  The methods we describe below are designed
% specifically for encoder-decoder scenarios where the decoder is an
% RNN; we make no assumption about the encoder.



% \noindent \textbf{RNNs:} A recurrent neural network is a parameterized
% non-linear function $\RNN$ that recursively maps a sequence of vectors
% to a sequence of hidden states. Let $\boldm_1, \ldots, \boldm_t$ be a
% sequence of $t$ vectors, and let $\boldh_0$ be some initial state
% vector. Applying an RNN to any such sequence yields hidden states
% $\boldh_t$ at each time-step, as follows:
% %{\small
% \begin{align*}
% \boldh_t \gets \RNN(\boldm_t, \boldh_{t-1}; \btheta),
% \end{align*}
% %}
% \noindent where $\btheta$ is the set of model parameters, which are shared over time. 


%Accordingly, we consider the generation of target word sequences $\longpfx{T}$ of length $T$, where we have used $\cdot$ as the concatenation operator, and where each word token $w_j$ comes from our target vocabulary $\mcV$. We denote by $\boldx$ the input representation on which the target generation conditions. We refer to the \textit{gold} (i.e., correct) output word sequence for an input $\boldx$ as $\longgoldpfx{T}$. We will often abbreviate sequences  $\longpfx{T}$ as $\pfx{T}$. % (and similarly for $\longgoldpfx{T}$ and $\goldpfx{T}$).\\ %, and we refer to set of all possible $\boldx$'s as $\mcX$.  \\

% When using an RNN decoder, it is typical to model the probability of
% the $t\,{+}\,1$'st target word's type being $w$ given the preceding
% words and the input as a function of $\boldh_t$. That is, one
% stipulates that $p(w_{t+1} \niceq w|\pfx{t}, \boldx) \propto g(w,
% \boldh_t, \boldx)$, for some function $g$ that examines the hidden
% state at time $t$ and $\boldx$. It is then natural to train such a
% model with a cross-entropy loss at each time-step. In this paper we
% will instead be interested in modeling non-probabilistic scores of
% arbitrary \textit{sequences} formed from the target vocabulary
% $\mcV$. We will accordingly define the score of an entire
% \textit{prefix} $\pfx{t}$ followed by a single word $w$ as
% \begin{align} \label{eq:score}
% \score(\pfx{t} \cdot w) \triangleq f(w, \boldh_t, \boldx),
% \end{align} 
% where, analogously, $f$ is some function examining the current hidden-state of the relevant RNN at time $t$ as well as the input representation $\boldx$. Note that we use $\cdot$ as the concatenation operator.  



\section{Beam Search Optimization}
\section{MT-DNN-1}
\label{sec:mt-dnn-1}

\subsection{Preliminaries}
\label{subsec:prelim}
In this work, our multi-task model combines classification, regression and pair-wise ranking tasks, which are summerised in Table~\ref{tab:task}. We briefly introduce the definition of each task as follows: 
\begin{table}[htb!]
	\begin{center}
		\begin{tabular}{@{\hskip1pt}l@{\hskip1pt}|@{\hskip1pt}c@{\hskip1pt}|@{\hskip1pt}c@{\hskip1pt}|@{\hskip1pt}c}
			\hline \bf Input &Classification&Regression &Ranking\\ \hline \hline
			single sentence &$\checkmark$&& \\
			pairwise text &$\checkmark$&$\checkmark$&$\checkmark$ \\ \hline
		\end{tabular}
	\end{center}
	\lgspace
	\caption{Summary of tasks in our multi-task framework.
	}
	\label{tab:task}
\lgspace
\end{table}
\begin{figure}[!t]
\centering
\adjustbox{trim={.065\width} {.01\height} {.05\width} {.01\height},clip}
{\includegraphics[scale=0.7]{mtl_model}}
\caption{Model architecture.}
\label{fig:mtl_model} 
\end{figure}

\begin{figure}[!t]
\centering
\adjustbox{trim={.05\width} {.01\height} {.05\width} {.01\height},clip}
{\includegraphics[scale=0.7]{mtl_model_v2}}
\caption{Model architecture version 2.}
\label{fig:mtl_model_v2} 
\end{figure}

\textbf{Task definition}

\textbf{Objective}

\textbf{Single classification}
\xiaodl{Need to cluster different tasks..}

\textbf{Sentence-pair classification}: given a pair of sentence, $(S_1, S_2)$, the model predicts a label indicating the relation of this pair of sentences: $P(C|S_1, S_2)$. For example, natural language inference is a typical instance of the sentence-pair classification task: a premise and a hypothesis are denoted by $S_1$ and $S_2$, respectively; the label, $C$, belongs one of three relations (\textit{contradiction}, \textit{neutral} and \textit{entailment}). 

\textbf{Regression}


\textbf{Pair-wise Ranking}
\begin{algorithm}[ht!]
 \SetAlgoLined
Initialize model parameters $\Theta$ randomly  \\
Set M \quad\textit{//the number of updates for the shared layer} \\
%\textit{Counter} = 0\\
 \For{$iteration$ in $0 ... \infty$}{
 	 %1. \textit{Counter} += 1\\
     1. Pick a task $t$ randomly \\
     2. Pick sample(s) from task $t$, i.e., \\
     \hspace{0.4cm}$(Q,C=\{0,1\})$ for classification \\
     \hspace{0.4cm}$(Q, D)$ for ranking\\
     3. Compute loss: $L(\Theta)$, i.e.,\\
     \hspace{0.4cm} the \textit{cross-entropy} for classification \\
     \hspace{0.4cm} the ranking loss for ranking\cite{learning-to-rank2005burges}\\

     4. Compute gradient: $\nabla(\Theta)$ \\
     5. Update model: $\Theta = \Theta - \epsilon \nabla(\Theta)$ \quad\textit{}
     % \eIf{Counter $<$ M}{
  	 %5. Update model: $\Theta = \Theta - \epsilon \nabla(\Theta)$ \quad\textit{//update both $\Theta^s$ and $\Theta^t$} \\
   %}{
   	% 6. Update model: $\Theta^t = \Theta^t - \epsilon \nabla(\Theta^t)$ 
  %}
 }
 \caption{\label{algo:mtdnn} Training a Multi-task model.}
 \algorithmfootnote{Note that $\Theta$ denotes the model parameters. \textcolor{red}{TODO: update alg based on task defination.}}
\end{algorithm}

%\section{Practical Considerations}
%\input{sections/practical}

%\section{Experiments}
%\label{sec:experiments}
% !TEX root = ../multi_task.tex

We evaluate the presented MTL method on a number of problems. First, we use MultiMNIST \citep{multi_mnist}, an MTL adaptation of MNIST \citep{mnist}. Next, we tackle multi-label classification on the CelebA dataset \citep{celeba} by considering each label as a distinct binary classification task. These problems include both classification and regression, with the number of tasks ranging from 2 to 40. Finally, we experiment with scene understanding, jointly tackling the tasks of semantic segmentation, instance segmentation, and depth estimation on the Cityscapes dataset \citep{cityscapes}. We discuss each experiment separately in the following subsections.

The baselines we consider are (i) \textbf{uniform scaling:} minimizing a uniformly weighted sum of loss functions \mbox{$\frac{1}{T}\sum_t \lL^t$}, \mbox{(ii) \textbf{single task:}} solving tasks independently, \mbox{(iii) \textbf{grid search:}} exhaustively trying various values from $\{ c^t \in [0,1] | \sum_t c^t = 1\}$ and optimizing for $\frac{1}{T}\sum_t c^t \lL^t$, \mbox{(iv) \textbf{\citet{Kendall2018}:}} using the uncertainty weighting proposed by \citet{Kendall2018}, and \mbox{(v) \textbf{GradNorm:}} using the normalization proposed by \citet{Chen2018}.



\subsection{MultiMNIST}
\label{sec:multi_mnist_exp}

Our initial experiments are on MultiMNIST, an MTL version of the MNIST dataset \citep{multi_mnist}. In order to convert digit classification into a multi-task problem, \citet{multi_mnist} overlaid multiple images together. We use a similar construction. For each image, a different one is chosen uniformly in random. Then one of these images is put at the top-left and the other one is at the bottom-right. The resulting tasks are: classifying the digit on the top-left (task-L) and classifying the digit on the bottom-right (task-R). We use 60K examples and directly apply existing single-task MNIST models. The MultiMNIST dataset is illustrated in the supplement.

We use the LeNet architecture \citep{mnist}. We treat all layers except the last as the representation function $g$ and put two fully-connected layers as task-specific functions (see the supplement for details). We visualize the performance profile as a scatter plot of accuracies on task-L and task-R in Figure~\ref{fig:multi_mnist_performance_curve}, and list the results in Table~\ref{tab:multi_mnist}.

In this setup, any static scaling results in lower accuracy than solving each task separately (the single-task baseline). The two tasks appear to compete for model capacity, since increase in the accuracy of one task results in decrease in the accuracy of the other. Uncertainty weighting \citep{Kendall2018} and GradNorm \citep{Chen2018} find solutions that are slightly better than grid search but distinctly worse than the single-task baseline. In contrast, our method finds a solution that efficiently utilizes the model capacity and yields accuracies that are as good as the single-task solutions. This experiment demonstrates the effectiveness of our method as well as the necessity of treating MTL as multi-objective optimization. Even after a large hyper-parameter search, \emph{any} scaling of tasks does not approach the effectiveness of our method.



\subsection{Multi-Label Classification}

\begin{figure}[t]
\includegraphics[width=\textwidth]{radar_full_new}
\vspace{1mm}
\caption{Radar charts of percentage error per attribute on CelebA \citep{celeba}. Lower is better. We divide attributes into two sets for legibility: easy on the left, hard on the right. Zoom in for details.}
\label{fig:multi_label_radar}
\end{figure}


\begin{wraptable}{r}{0.3\textwidth}
%\vspace{-4mm}
\captionof{table}{Mean of error per category of MTL algorithms in multi-label classification on CelebA \citep{celeba}.}
\begin{tabular}{r@{\hspace{2mm}}c@{}}
\toprule
& Average  \\
&  error \\
\midrule
Single task & $8.77$ \\
Uniform scaling & $9.62$ \\
\citealt{Kendall2018} & $9.53$ \\
GradNorm & $8.44$ \\
Ours & $\mathbf{8.25}$  \\
\bottomrule
\end{tabular}
\label{table:multi_label_bar}
%\vspace{-5mm}
\end{wraptable}

Next, we tackle multi-label classification. Given a set of attributes, multi-label classification calls for deciding whether each attribute holds for the input. We use the CelebA dataset \citep{celeba}, which includes 200K face images annotated with 40 attributes. Each attribute gives rise to a binary classification task and we cast this as a 40-way MTL problem. We use ResNet-18 \citep{resnet} without the final layer as a shared representation function, and attach a linear layer for each attribute (see the supplement for further details).


We plot the resulting error for each binary classification task as a radar chart in Figure~\ref{fig:multi_label_radar}. The average over them is listed in Table~\ref{table:multi_label_bar}. We skip grid search since it is not feasible over 40 tasks. Although uniform scaling is the norm in the multi-label classification literature, single-task performance is significantly better. Our method outperforms baselines for significant majority of tasks and achieves comparable performance in rest. This experiment also shows that our method remains effective when the number of tasks is high.


\subsection{Scene Understanding}

To evaluate our method in a more realistic setting, we use scene understanding. Given an RGB image, we solve three tasks: semantic segmentation (assigning pixel-level class labels), instance segmentation (assigning pixel-level instance labels), and monocular depth estimation (estimating continuous disparity per pixel). We follow the experimental procedure of \citet{Kendall2018} and use an encoder-decoder architecture. The encoder is based on ResNet-50 \citep{resnet} and is shared by all three tasks. The decoders are task-specific and are based on the pyramid pooling module \citep{pspnet} (see the supplement for further implementation details).

Since the output space of instance segmentation is unconstrained (the number of instances is not known in advance), we use a proxy problem as in \citet{Kendall2018}. For each pixel, we estimate the location of the center of mass of the instance that encompasses the pixel. These center votes can then be clustered to extract the instances. In our experiments, we directly report the MSE in the proxy task. Figure~\ref{fig:cityscapes_performance_profile} shows the performance profile for each pair of tasks, although we perform all experiments on all three tasks jointly. The pairwise performance profiles shown in Figure~\ref{fig:cityscapes_performance_profile} are simply 2D projections of the three-dimensional profile, presented this way for legibility. The results are also listed in Table~\ref{tab:cityscapes_results}.

MTL outperforms single-task accuracy, indicating that the tasks cooperate and help each other. Our method outperforms all baselines on all tasks.


\subsection{Role of the Approximation}

In order to understand the role of the approximation proposed in Section~\ref{sec:approximation}, we compare the final performance and training time of our algorithm with and without the presented approximation in Table~\ref{tab:approximation_tradeoff} (runtime measured on a single Titan Xp GPU). For a small number of tasks (3 for scene understanding), training time is reduced by 40\%. For the multi-label classification experiment (40 tasks), the presented approximation accelerates learning by a factor of 25.

On the accuracy side, we expect both methods to perform similarly as long as the full-rank assumption is satisfied. As expected, the accuracy of both methods is very similar. Somewhat surprisingly, our approximation results in slightly improved accuracy in all experiments. While counter-intuitive at first, we hypothesize that this is related to the use of SGD in the learning algorithm. Stability analysis in convex optimization suggests that if gradients are computed with an error $\hat{\nabla}_\btheta \mathcal{L}^t = \nabla_\btheta \mathcal{L}^t + \mathbf{e}^t$ ($\btheta$ corresponds to $\btheta^{sh}$ in (\ref{eq:kkt_opt})), as opposed to $\mathbf{Z}$ in the approximate problem in \ref{eq:approx}, the error in the solution is bounded as $\|\hat{\mathbf{\alpha}} - \mathbf{\alpha} \|_2 \leq \mathcal{O}(\max_t \|\mathbf{e}^t\|_2)$. Considering the fact that the gradients are computed over the full parameter set (millions of dimensions) for the original problem and over a smaller space for the approximation (batch size times representation which is in the thousands), the dimension of the error vector is significantly higher in the original problem. We expect the $l_2$ norm of such a random vector to depend on the dimension.

In summary, our quantitative analysis of the approximation suggests that (i) the approximation does not cause an accuracy drop and (ii) by solving an equivalent problem in a lower-dimensional space, our method achieves both better computational efficiency and higher stability.

  {\small
  \begin{table}[t]
%  \vspace{-4mm}
  \caption{Effect of the MGDA-UB approximation. We report the final accuracies as well as training times for our method with and without the approximation.}
  %\vspace{1mm}
  \centering
  \begin{tabular}{@{}r@{\hspace{3mm}}c@{\hspace{3mm}}c@{\hspace{2mm}}c@{\hspace{2mm}}c@{}c@{\hspace{5mm}}c@{\hspace{2mm}}c@{}}
  \toprule
  & \multicolumn{4}{c}{Scene understanding (3 tasks)} &  & \multicolumn{2}{c}{Multi-label (40 tasks)}  \\
  \cmidrule(r){2-5} \cmidrule(lr){7-8}
                  & Training & Segmentation & Instance  & Disparity      & & Training & Average \\
                 & time     &  mIoU [\%]       & error [px] & error [px] & & time (hour)      & error \\
  \midrule
  Ours (w/o approx.) & $38.6$ & $66.13$ & $10.28$ & $2.59$ & & $429.9$ & $8.33$ \\
  Ours & $\mathbf{23.3}$ & $\mathbf{66.63}$ & $\mathbf{10.25}$ & $\mathbf{2.54}$  & & $\mathbf{16.1}$ & $\mathbf{8.25}$ \\
  \bottomrule
  \end{tabular}
  %\vspace{-2mm}
  \label{tab:approximation_tradeoff}
  \end{table}}


\section*{Acknowledgments} We thank Yoon Kim for helpful discussions and for providing the initial seq2seq code on which our implementations are based. We thank Allen Schmaltz for help with the word ordering experiments. We also gratefully acknowledge the support of a Google Research Award.

\nocite{bahdanau16an}

\bibliography{beamtrain}
\bibliographystyle{emnlp2016}

\end{document}

\documentclass[conference]{IEEEtran}
\usepackage{blindtext, graphicx}
%\usepackage{caption}
\usepackage{subcaption}
\usepackage{amsmath}
\usepackage{color}
\usepackage{colortbl}
%\usepackage[caption=false,font=footnotesize]{subfig}
% \usepackage{subfig}
\usepackage{graphicx}
\usepackage{acronym}
\usepackage[T1]{fontenc}
\usepackage{etoolbox}
% *** GRAPHICS RELATED PACKAGES ***
%
\ifCLASSINFOpdf
  % \usepackage[pdftex]{graphicx}
  % declare the path(s) where your graphic files are
  % \graphicspath{{../pdf/}{../jpeg/}}
  % and their extensions so you won't have to specify these with
  % every instance of \includegraphics
  % \DeclareGraphicsExtensions{.pdf,.jpeg,.png}
\else
  % or other class option (dvipsone, dvipdf, if not using dvips). graphicx
  % will default to the driver specified in the system graphics.cfg if no
  % driver is specified.
  % \usepackage[dvips]{graphicx}
  % declare the path(s) where your graphic files are
  % \graphicspath{{../eps/}}
  % and their extensions so you won't have to specify these with
  % every instance of \includegraphics
  % \DeclareGraphicsExtensions{.eps}
\fi

% correct bad hyphenation here
%\hyphenation{op-tical net-works semi-conduc-tor}

\usepackage[noadjust]{cite}

\acrodef{lstm}		[\textsc{LSTM\xspace}]				{Long Short-Term Memory Neural Networks}
\acrodef{rnn}		[\textsc{RNN\xspace}]				{Recurrent Neural Networks}
\acrodef{dip}		[\textsc{DIP\xspace}]				{Document Image Processing Pipeline}
\acrodef{dipp}		[\textsc{DIPP\xspace}]				{Document Image Processing Pipeline}

\acrodef{cnn}		[\textsc{CNN\xspace}]				{Convolutional Neural Networks}

\acrodef{relu}		[\textsc{ReLUs\xspace}]				{Rectified Linear Units}


\IEEEoverridecommandlockouts



\acrodef{das}[\textsc{DAS}\xspace]{Document Analysis System}
\acrodef{ann}[\textsc{ANN}\xspace]{Artificial Neural Network}
\acrodef{cnn}[\textsc{CNN}\xspace]{Convolutional Neural Network}
\acrodef{fcn}[\textsc{FCN}\xspace]{Fully Connected Network}
\acrodef{gpu}[\textsc{GPU}\xspace]{Graphics Processing Unit}
\acrodef{svm}[\textsc{SVM\xspace}]{Support Vector Machine}
\acrodef{pca}[\textsc{PCA\xspace}]{Principal Component Analysis}
\acrodef{cc}[\textsc{CC\xspace}]{Connected Component}
\acrodef{bvlc}[\textsc{BVLC\xspace}]{Berkeley Vision and Learning Center}
\acrodef{sf}[\textit{smart FIX}]{\textit{smart \textbf{F}or \textbf{I}nformation e\textbf{X}traction}}
\acrodef{dfki}[\textsc{DFKI\xspace}]{German Research Center for Artificial Intelligence}
\acrodef{gt}[\textsc{GT}\xspace]{Ground Truth}
\acrodef{fp}[\textsc{$FP$}\xspace]{False Positives}
\acrodef{fn}[\textsc{$FN$\xspace}]{False Negatives}	
\acrodef{tp}[\textsc{$TP$\xspace}]{True Positives}
\acrodef{tn}[\textsc{$TN$\xspace}]{True Negatives}
\acrodef{ocr}[\textsc{OCR\xspace}]{Optical Character Recognition}
\acrodef{pkv}[\textsc{PKV\xspace}]{Private Health Insurance}







% abbreviations
\newcommand*{\eg}		{e.g.\ }
\newcommand*{\ie}		{i.e.\ }
\newcommand*{\etal}		{et~al.\ }
\newcommand*{\cf}		{cf.\ }
\newcommand*{\sota}		{state-of-the-art\ }

\begin{document}

% \title{The Impact of Very Deep Neural Network and Transfer Learning From Similar Dataset for Document Image Classification}



\title{Cutting the Error by Half: Investigation of Very Deep CNN and Advanced Training Strategies for Document Image Classification}

\makeatletter

\def\footnoterule{\relax%
  \kern-5pt
  \hbox to \columnwidth{\hfill\vrule width 0.5\columnwidth height 0.4pt\hfill}
  \kern4.6pt}

\makeatother

%Anonymous for review!


%\patchcmd{\maketitle}{\@fnsymbol}{\@alph}{}{}  % Footnote numbers from symbols to small letters


% \author{\IEEEauthorblockN{Anonymous Authors}\\
%         \IEEEauthorblockN{Paper ID 424}}
  
% \newcommand{\specificthanks}[1]{\@fnsymbol{#1}}
% \renewcommand{\thefootnote}{\fnsymbol{footnote}}

 
\author{
    \IEEEauthorblockN{ Muhammad Zeshan Afzal\IEEEauthorrefmark{1}\IEEEauthorrefmark{2}\IEEEauthorrefmark{4}\thanks{* These authors contributed equally to this work},
    Andreas K\"olsch\IEEEauthorrefmark{1}\IEEEauthorrefmark{2},
    Sheraz Ahmed\IEEEauthorrefmark{3},
    Marcus Liwicki\IEEEauthorrefmark{2}\IEEEauthorrefmark{5}}

    \IEEEauthorblockA{
        afzal@iupr.com,
        a\_koelsch12@cs.uni-kl.de,
        sheraz.ahmed@dfki.de,
        marcus.liwicki@unifr.ch
    }

    \IEEEauthorblockA{\IEEEauthorrefmark{2}MindGarage, University of Kaiserslautern, Germany}

    \IEEEauthorblockA{\IEEEauthorrefmark{3}DFKI, Kaiserslautern, Germany}

    \IEEEauthorblockA{\IEEEauthorrefmark{4}Insiders Technologies GmbH, Kaiserslautern, Germany}

    \IEEEauthorblockA{\IEEEauthorrefmark{5}University of Fribourg, Switzerland}

}


% make the title area


\maketitle

\IEEEpeerreviewmaketitle
% \begin{figure}[h!]
%     \centering
%     \includegraphics[width=\linewidth,height=5cm]{fig/placeholder}
% \end{figure}

% While object recognition on 2D images is getting more and more mature, 3D understanding is eagerly in demand yet largely underexplored. In this paper, we study the 3D object detection problem from RGB-D data captured by depth sensors in both indoor and outdoor environments. Different from previous deep learning methods that work on 2D RGB-D images or 3D voxels, which often obscure natural 3D patterns and invariances of 3D data, we directly operate on raw point clouds by popping up RGB-D scans. Although recent works such as PointNet performs well for segmentation in small-scale point clouds, one key challenge is how to efficiently detect objects in large-scale scenes. Leveraging the wisdom of dimension reduction and mature 2D object detectors, we develop a Frustum PointNet framework that addresses the challenge. Evaluated on KITTI and SUN RGB-D 3D detection benchmarks, our method outperforms state of the arts by remarkable margins with high efficiency (running at 5 fps).


% While object recognition on 2D images is getting more and more mature, 3D understanding is eagerly in demand yet underexplored.
In this work, we study 3D object detection from RGB-D data in both indoor and outdoor scenes. While previous methods focus on images or 3D voxels, often obscuring natural 3D patterns and invariances of 3D data, we directly operate on raw point clouds by popping up RGB-D scans. However, a key challenge of this approach is how to efficiently localize objects in point clouds of large-scale scenes (region proposal). Instead of solely relying on 3D proposals, our method leverages both mature 2D object detectors and advanced 3D deep learning for object localization, achieving efficiency as well as high recall for even small objects. Benefited from learning directly in raw point clouds, our method is also able to precisely estimate 3D bounding boxes even under strong occlusion or with very sparse points. Evaluated on KITTI and SUN RGB-D 3D detection benchmarks, our method outperforms the state of the art by remarkable margins while having real-time capability.


\section{Introduction}

Humans use different forms of communications such as speech, hand gestures and emotions. Being able to understand one's emotions and the encoded feelings is an important factor for an appropriate and correct understanding.


With the ongoing research in the field of robotics, especially in the field of humanoid robots, it becomes interesting to integrate these capabilities into machines allowing for a more diverse and natural way of communication. One example is the Software called EmotiChat~\cite{Anderson06areal-time}. This is a chat application with emotion recognition. The user is monitored and whenever an emotion is detected (smile, etc.), an emoticon is inserted into the chat window. Besides Human Computer Interaction other fields like surveillance or driver safety could also profit from it. Being able to detect the mood of the driver could help to detect the level of attention, so that automatic systems can adapt.\\
\let\thefootnote\relax\footnote{*F. Trier and P. Burkert contributed equally to this work.}


Many methods rely on extraction of the facial region. This can be realized through manual inference~\cite{4032815} or an automatic detection approach~\cite{Anderson06areal-time}.
Methods often involve the Facial Action Coding System (FACS) which describes the facial expression using Action Units (AU). An Action Unit is a facial action like "raising the Inner Brow". Multiple activations of AUs describe the facial expression~\cite{kumar2009face}. Being able to correctly detect AUs is a helpful step, since it allows making a statement about the activation level of the corresponding emotion. \\
Handcrafted facial landmarks can be used such as done by Kotsia et al.~\cite{4032815}. Detecting such landmarks can be hard, as the distance between them differs depending on the person~\cite{6998925}. Not only AUs can be used to detect emotions, but also texture. When a face shows an emotion the structure changes and different filters can be applied to detect this~\cite{6998925}.\\


\begin{figure}
   \centering
        \includegraphics[width=\columnwidth]{Fig1}
   \caption{Example images from the MMI (top) and CKP (bottom). The emotions from left to right are: \textit{Anger}, \textit{Sadness}, \textit{Disgust}, \textit{Happiness}, \textit{Fear}, \textit{Surprise}. The emotion \textit{Contempt} of the CKP set is not displayed.}\label{fig:example_images}
\end{figure}




The presented approach uses Artificial Neural Networks (ANN). ANNs differ, as they are trained on the data with less need for manual interference. 
Convolutional Neural Networks are a special kind of ANN and have been shown to work well as feature extractor when using images as input~\cite{donahue2013decaf} and are real-time capable. This allows for the usage of the raw input images without any pre- or postprocessing.\\
GoogleNet~\cite{DBLP:journals/corr/SzegedyLJSRAEVR14} is a deep neural network architecture that relies on CNNs. It has been introduced during the Image Net Large Scale Visual Recognition Challenge(ILSVRC) 2014. This challenge analyses the quality of different image classification approaches submitted by different groups. The images are separated into 1000 different classes organized by the WordNet hierarchy. In the challenge "object detection with additional training data" GoogleNet has achieved about 44\% precision~\cite{LSVRC-results}. These results have demonstrated the potential which lies in this kind of architecture. Therefore it has been used as inspiration for the proposed architecture.\\
The proposed network has been evaluated on the Extended Cohn-Kanade Dataset (Section~\ref{sec:ckp}) and on the MMI Dataset (Section~\ref{sec:mmi}). Typical pictures of persons showing emotions can be seen in Fig.~\ref{fig:example_images}.
The emotion \textit{Contempt} of the CKP set is not shown as no subject with consent for publication and an annotated emotion is part of the dataset. Results of experiments on these datasets demonstrate the success of using a deep layered neural network structure. With a 10-fold cross-validation a recognition accuracy of 99.6\% has been achieved. \\

The paper is arranged as follows: After this introduction, Related Work (Section~\ref{sec:related}) is presented which focuses on Emotion/Expression recognition and the various approaches scientists have taken. Next is Section~\ref{sec:background}, Background, which focuses on the main components of the architecture proposed in this article. Section~\ref{sec:datasets} contains a summary of the used Datasets. In Section~\ref{sec:architecture} the architecture is presented. This is followed by the experiments and its results (Section~\ref{sec:experiments}) . Finally, Section~\ref{sec:conclusion} summarizes the article and concludes the article.

\section{Related Work}

% Various approaches for document image classification have been proposed over the years. Generally, document image classification approaches are divided into two major groups, structure/layout based, and content based. This section provides an overview of some important works which have been reported in reference to structure or content based document classification.% \subsection{Content based Document Classification}

Over the years, different methods have been proposed for document image classification. The overall classification methods can be divided into three distinct categories.
The first category exploits structure and layout similarities, while the second focuses on developing local and global features that could be used for document classification. The third category is based on deep \ac{cnn}s that extract the features automatically for document classification. This section provides a summary of the important related work regarding the above mentioned three categories.

\begin{figure*}
    \begin{subfigure}{0.22\linewidth}
        \centering
        \fbox{\includegraphics[height=0.65\textheight]{architectures/alexnet.pdf}}
        \subcaption{AlexNet}
        \label{fig:alexnet}
    \end{subfigure}
    \begin{subfigure}{0.21\linewidth}
        \centering
        \fbox{\includegraphics[height=0.65\textheight]{architectures/vgg.pdf}}
        \subcaption{VGG-16}
        \label{fig:vgg}
    \end{subfigure}
    \begin{subfigure}{0.36\linewidth}
        \centering
        \fbox{\includegraphics[height=0.65\textheight]{architectures/googlenet.pdf}}
        \subcaption{GoogLeNet}
        \label{fig:googlenet}
    \end{subfigure}
    \begin{subfigure}{0.19\linewidth}
        \centering
        \fbox{\includegraphics[height=0.65\textheight]{architectures/resnet.pdf}}
        \subcaption{ResNet-50}
        \label{fig:resnet}
    \end{subfigure}
    \caption{Deep CNN architectures used in this work}
    \label{fig:deepcnn}
\end{figure*}

Dengel and Dubiel~\cite{doclass_Dengel95} used layout structure printed documents. They used top-down induction in decision trees to convert printed documents into a complementary logical structure.
Bagdanov and Worring~\cite{doclass_Bagdanov2001} classify machine-printed documents by using the Attributed Relational Graphs (ARGs).
Byun and Lee~\cite{doclass_Byun2000} used parts of the documents for the recognition. They reasoned that processing complete documents is time-consuming. The document classification was performed on parts of the documents using DP algorithm. Their approach was fast but the applicability is limited to forms. Shin and Doermann~\cite{doclass_shin} proposed an approach that used layout structural similarity for full or partial image matching for retrieval. 
Kevyn and Nickolov~\cite{Collins-thompson02aclustering-based} used both the layout and the text features for matching the documents for retrieval. 

Jayant et al.~\cite{doclass_Kumar12} propose a method that relies on the patch code words derived from the document images. The code book is learned independently of the class labels of the documents. In the first step, the images are recursively partitioned both in horizontal and vertical direction for modeling spatial relationships. Subsequently, a histogram for each partition is calculated that is used for the classification.
Following the same idea of developing the code book, another work presented by Jayant et al.~\cite{doclass_Kumar14} build a codebook of SURF descriptors extracted from training images. Then, histograms of codewords are created similar to~\cite{doclass_Kumar12}. A Random Forest classifier is used for classification. The applicability of the approach is shown in the presence of limited data.
Chen et al.~\cite{doclass_Chen12} propose a method based on low-level image features to classify documents. The approach is limited to structured documents. An important point is that one could obtain the registration of two images by matching the feature points.
Joutel et al.~\cite{doclass_Joutel2007} proposed a method that used curvelet transformation for indexing and querying the documents at different image scales. Their method is designed particularly for large databases of handwritten manuscripts. Kochi and Saitoh~\cite{doclass_Kochi99} used textual descriptions of document images for information extraction from documents. The method is limited to semi-structured documents and assumes a pre-defined knowledge is available for the document classes.
Reddy and Govindaraju~\cite{doclass_umamaheswara08} used binary images for the classification of the documents. They use pixel information and calculate pixel densities.  They used k-means clustering supported by adaptive boosting. The method is evaluated on the benchmark NIST scanned special tax form databases $2$ and $6$.

%The approach only deals with structured documents which are mostly-text and with pre-printed contents.  doclass_Kumar12,doclass_Chen12, doclass_Kumar14

% Jayant et al.~\cite{doclass_Kumar12} propose a method based on statistics of patch code words. Starting with a set of wanted and a random set of unwanted images, raw-image patches extracted from the unlabeled images to learn a code book. Spatial relationships between patches are modelled by recursively partitioning the image horizontally and vertically and a histogram of patch-codewords is computed for each partition. 

% In another work, Jayant et al.~\cite{doclass_Kumar14} build a codebook of SURF descriptors extracted from some training images. Then histograms of codewords are created similar to~\cite{doclass_Kumar12}. Later a Random Forest classifier is used for classification. The system performs reasonably even for limited training data. Chen et al.~\cite{doclass_Chen12} propose a method based on SIFT descriptors to classify documents. The approach only deals with structured documents which are mostly-text and with pre-printed contents.


%low level pixel density information from the binary images. The proposed system is based on the k-means algorithm and supported by adaptive boosting. The results are reported on the benchmark NIST scanned special tax form databases $2$ and $6$.


% Joutel et al.~\cite{doclass_Joutel2007} use Curvelet transform as a multiscale method for indexing linear singularities and curved handwritten shapes in documents images. 
% Their method detects oriented and curved fragments at different scales and searches for similar handwritten samples in large manuscripts databases. 
% Kochi and Saitoh~\cite{doclass_Kochi99} compare  the textual elements of document images. The presented system is robust against shifts or noise in the target documents and can handle semi-formatted documents..
% Reddy and Govindaraju~\cite{doclass_umamaheswara08} use low level pixel density information from the binary images. The proposed system is based on the k-means algorithm and supported by adaptive boosting. The results are reported on the benchmark NIST scanned special tax form databases $2$ and $6$.
% \subsection{Structure based Document Classification}



%Based on Layout Structural Similarity doclass_Byun2000, doclass_shin, Collins-thompson02aclustering-based


The pioneering work that performed document classification using \ac{cnn}s used a rather shallow network for classification~\cite{lekang_14_a}. Nevertheless, the proposed approach outperformed structural similarity based methods and shows the potential of automatic feature learning for document classification using \ac{cnn}s. The reason may be that deep networks require a lot of data for training and at that time the standard challenging dataset consisted of only $3,482$ images.
Afzal et. al.~\cite{afzal2015deepdocclassifier} and Harley et. al.~\cite{harley2015evaluation} provided a breakthrough when they showed that it is possible to use transfer learning and the features that are learned from general (daily life) images can be used for the classification of document images~\cite{afzal2015deepdocclassifier}. They achieved a significant improvement over \ac{cnn} based methods that were the \sota at that time.
%Another significant contribution is presented by Harley et. al.~\cite{harley2015evaluation}. 
%The concept of their work was the same as Afzal et. al., however, they used the features from deep \ac{cnn} for document retrieval and showed that validity of their approach.
Another notable contribution by Harley et. al.~\cite{harley2015evaluation} was that they introduced a dataset consisting of $400,000$ documents divided into $16$ classes.
This allowed for the evaluation of deep neural networks using a significant amount of data. 

The \sota in deep \ac{cnn}s has advanced significantly in recent years and there has been no comprehensive study regarding the impact of deep architectures for document classification. Moreover, there is no study that explores transfer learning from document images and also there is no report of the impact of the amount of training images. The presented work takes into account these issues and performs a comprehensive set of experiments to fill the gaps that exist. Eventually, this study leads to an approach that can reduce the error by more than half and therefore provides another leap forward in the domain of document image classification.


% Deep convolutional networks have shown successful results on large benchmark datasets consisting of more than one million images, such as ImageNet~\cite{cnn_alexnet_nips2014}.
% Alex et al.~\cite{cnn_alexnet_nips2014} trained a large, deep convolutional neural network to classify the $1.2$ million high-resolution images in the ImageNet LSVRC-2010 contest into $1000$ different classes. An important finding was that the learned deep representation is transferable across different tasks.
% Sermanet et al.~\cite{conf/cvpr/SermanetKCL13} suggested to use unsupervised pre-training, followed by supervised fine-tuning for pedestrian detection. Similarly, supervised pre-training was proved helpful in different computer vision and multimedia settings w.r.t. a concept-bank paradigm~\cite{TorresaniSzummerFitzgibbon10}. Recently, Girshick et al.~\cite{journals/corr/GirshickDDM13} showed that, for dealing with scarce data, supervised pre-training on larger data and then fine-tuning on smaller problem-specific dataset improves classification results. Based on this background, to the best of our knowledge, this paper is the first attempt ever when pre-trained CNN are optimized for document image classification.


%\input{elm.tex}
\section{Deep Convolutional Neural Networks}
\label{sec:networks}

This section briefly presents the deep \ac{cnn} architectures used in this work. Furthermore, the image preprocessing and training details are described.



% The method presented in this paper does not utilize document features that require a high resolution, such as optical character recognition. Instead, it solely relies on the structure and layout of the input documents to classify them. Therefore, in a preprocessing step, the high-resolution images are downscaled to a lower resolution of $227\times227$ which is the input size of the CNN.

% The common approach to successfully train CNNs for object recognition is to augment the training data by resizing the images to a larger size and to then randomly crop areas from these images (\cf \cite{cnn_alexnet_nips2014}). This data augmentation technique has proven to be effective for networks trained on the ImageNet data set where the most discriminating elements of the images are typically located close to the center of the image and therefore contained in all crops. However, by this technique the network is effectively presented with less than 80\% of the original image. We intentionally do not augment our training data in this way, because unlike in object recognition, the most discriminating parts of document images often reside in the outer regions of the document, \eg the head of a letter.

% As a second preprocessing step, we subtract the mean values of the training images from both the training and the validation images.

% Lastly, we convert the grayscale document images to RGB images, \ie we copy the values of the single-channel images to generate three-channel images. 

% ToDo, Discuss: Why is this important?

% The original weights which are used to initialize the proposed network were trained using all three channels RGB and therefore extracts different types of features. We introduce the same information in all the channels in order to enrich the features for better classification.


% \subsection{Network Architectures}
% \begin{figure*}
%     \begin{subfigure}{0.23\linewidth}
%         \centering
%         \includegraphics[height=0.65\textheight]{architectures/alexnet.pdf}
%         \subcaption{AlexNet}
%     \end{subfigure}
%     \unskip\vrule
%     \begin{subfigure}{0.21\linewidth}
%         \centering
%         \includegraphics[height=0.65\textheight]{architectures/vgg.pdf}
%         \subcaption{VGG-16}
%     \end{subfigure}
%     \unskip\vrule
%     \begin{subfigure}{0.36\linewidth}
%         \centering
%         \includegraphics[height=0.65\textheight]{architectures/googlenet.pdf}
%         \subcaption{GoogLeNet}
%     \end{subfigure}
%     \unskip\vrule
%     \begin{subfigure}{0.19\linewidth}
%         \centering
%         \includegraphics[height=0.65\textheight]{architectures/resnet.pdf}
%         \subcaption{ResNet-50}
%     \end{subfigure}
    
%     \caption{Deep CNN architectures used in this work}
%     \label{fig:deepcnn}
% \end{figure*}



\subsection{Network Architectures}

The deep \ac{cnn} architectures used in this paper are well known in the domain of object recognition but are not used frequently for document image classification. The networks are of very different nature (\cf~Fig.~\ref{fig:deepcnn}).

\subsubsection{\textbf{AlexNet}}
AlexNet \cite{cnn_alexnet_nips2014} is the eight-layer \ac{cnn} that won the ImageNet Large Scale Visual Recognition Challenge (ILSVRC) in 2012 \cite{russakovsky2015imagenet} by a large margin.
It employs five convolutional layers with optional pooling and local response normalization. These are then followed by three fully-connected layers and a softmax classifier (\cf~Fig.~\ref{fig:alexnet}).

\subsubsection{\textbf{VGG-16}}
VGG-16, as the name suggests is a 16-layer \ac{cnn}~\cite{simonyan2014very}. Unlike AlexNet, it uses only convolutional filters of size $3\times3$. Just like AlexNet, it has a straightforward architecture, but with $13$ convolutional layers and $3$ fully connected layers (\cf~Fig.~\ref{fig:vgg}) it is quite a bit deeper and has a repetitive pattern of layers. This architecture has won the localization category of the ILSVRC 2014.

\subsubsection{\textbf{GoogLeNet}}
GoogLeNet, just like VGG-16, won a category of the ILSVRC 2014, namely the classification category~\cite{szegedy2015going}. The architecture of this network, however, is a bit more sophisticated (\cf Fig.~\ref{fig:googlenet}). Unlike AlexNet and VGG-16, it is not just a stack of Convolution layers and Pooling layers, but rather a stack of building blocks, which themselves consist of Convolution and Pooling layers. It is therefore a Network-in-Network approach~\cite{lin2013network}. Due to its high depth, the network employs three softmax classifiers during training, to enable efficient backpropagation of the error. At test time, the two auxiliary classifiers are discarded.

\subsubsection{\textbf{Resnet-50}}
ResNets are a family of very deep \ac{cnn} architectures which make use of residual connections~\cite{he2016deep} to overcome the challenge of efficient error backpropagation. ResNet-50 is a variant of the network with $50$ layers, which, as in GoogLeNet, are grouped in building blocks (\cf~Fig.~\ref{fig:resnet}). An even deeper variant with $152$ layers won the ILSVRC classification task in 2015. Interestingly, despite its increased depth, the network has fewer parameters to fit than VGG-16.




\subsection{Preprocessing}

As the networks used in this paper require images of a fixed size as input, we first downscale all images to the expected input size of the networks. For AlexNet, the images are resized to $227\times227$ pixels, for the other networks the images are resized to $224\times224$ pixels.
Typically, when training \ac{cnn}s, the training data is augmented by resizing the images to a larger size, \eg $256\times256$ pixels and then cropping random patches of these images in the size of the network input. This approach has shown to be effective for real-world image classification~\cite{cnn_alexnet_nips2014}. In real-world images, the objects are typically close to the center of the image and therefore always contained in the random crops. However, the most discriminative parts of document images are not always close to the center of the image but reside in the outer regions, \eg the head of a letter. Therefore, we do not enlarge our training dataset in this way but train solely with images containing the entire document.

After resizing the images, we compute the mean pixel values of the training images and subtract them from all images to center the training data.

As a last preprocessing step, we convert the grayscale images to RGB images by simply copying the pixel values of the single-channel images to three channels.




\subsection{Training Details}
We train all networks using stochastic gradient descent with a momentum of $0.9$ and a learning rate that is updated every iteration to
\begin{equation}
    lr = initial\_lr * \left( 1 - \frac{iter}{max\_iter}\right) ^ {0.5}
\end{equation}
The initial learning rate is set to a value between $0.01$ and $0.0001$ depending on the network architecture, the training dataset and the weight initialization.

The number of training epochs depends on the task and ranges between $40$ and $80$ epochs.

% - training on large data set: 40 epochs, poly learn rate
% - fine tune from large to small data set: 40 epochs, poly learn rate
% - fine tune from imagenet to small data set: 80 epochs, poly learn rate
% - train from scratch to small data set: 80 epochs, poly learn rate


% The deep \ac{cnn} architecture used in the proposed approach consists of convolutional layer, maxpooling layer, \ac{relu} and the fully connected (FC) layers.
% The convolutional layers filter the input with the kernel to produce the feature maps which are either refined by the next convolutional layers or used for the classification by fully connected layers after applying the \ac{relu}.
% Pooling layers gather the overall response of the neighbouring groups of neurons in the same kernel. A window centered at the pooling location of size $n\times n$ is applied. 
% The dropout~\cite{dropout_Hinton12} layers are used to avoid any type of overfitting which may occur during the training. The dropout layer is preferred over having combinations of different models for the efficiency reasons.
% The fully connected layers are the classifiers which are used to classify the features extracted by the combination of the convolutional and the pooling layers.












% \subsection{Network Architecture}
% The architecture of the proposed neural network model (depicted in Figure~\ref{fig:deepcnn}) is inspired from~\cite{cnn_alexnet_nips2014}. 
% The model consists of eight main layers (five convolutional and the three fully connected layers). 
% The model uses the overlapping pooling \ie the windows used to summarize the outputs are overlapping.
% This can be achieved by setting the distance between the pooling units less than the size of the window.
% As depicted in Figure~\ref{fig:deepcnn} first convolutional layers processes the input of shape $227\times227\times3$.
% The first convolutional layer has $96$ kernels of size $11\times11\times3$.
% The first layer uses the stride(step size) of 4 pixels.
% The output of the first layer is then pooled to the first maxpooling layer where the average pooling and response normalization has been performed.
% The second layer has $256$ kernels of size $5\times5\times48$.
% The pooling and response normalization is perfromed for each layer except the third and the fourth layers because they are connected to each other.
% The number and the shapes of the kernels are (384, $3\times3\times256$),  (384, $3\times3\times192$) and (256, $3\times3\times192$) for third, fourth and fifth layer respectively.
% Each convolutional and fully connected layer is followed by a \ac{relu} layer.
% To avoid overfitting the dropout layer is applied to the sixth and the seventh layer which are fully connected layers.
% The eighth layer corresponds to the actual number of classes of the Tobacco-3428 Legislation dataset and hence contains $10$ output neurons corresponding to each class of the dataset.
% The output of the last layer is provided to a 10-way softmax which models a distribution over the 10 class labels. The data input and output sizes are depicted in Figure~\ref{fig:deepcnn}.


% \subsection{Learning Details}
% The objective is minimized using the stochastic gradient descent. 
% A batch size of $10$ is used.
% The learning rate, momentum and weight decay are set to $0.0001$, $0.9$ and $0.0005$ respectively.
% The weights of the network are initialized using the pretrained model\footnote{https://github.com/BVLC/caffe/tree/master/models/bvlc\_alexnet} except for the last fully connected layer.
% The training of the network has been done using Caffe\footnote{http://caffe.berkeleyvision.org/} deep learning framework~\cite{jia2014caffe}.






% !TEX root = ../multi_task.tex

We evaluate the presented MTL method on a number of problems. First, we use MultiMNIST \citep{multi_mnist}, an MTL adaptation of MNIST \citep{mnist}. Next, we tackle multi-label classification on the CelebA dataset \citep{celeba} by considering each label as a distinct binary classification task. These problems include both classification and regression, with the number of tasks ranging from 2 to 40. Finally, we experiment with scene understanding, jointly tackling the tasks of semantic segmentation, instance segmentation, and depth estimation on the Cityscapes dataset \citep{cityscapes}. We discuss each experiment separately in the following subsections.

The baselines we consider are (i) \textbf{uniform scaling:} minimizing a uniformly weighted sum of loss functions \mbox{$\frac{1}{T}\sum_t \lL^t$}, \mbox{(ii) \textbf{single task:}} solving tasks independently, \mbox{(iii) \textbf{grid search:}} exhaustively trying various values from $\{ c^t \in [0,1] | \sum_t c^t = 1\}$ and optimizing for $\frac{1}{T}\sum_t c^t \lL^t$, \mbox{(iv) \textbf{\citet{Kendall2018}:}} using the uncertainty weighting proposed by \citet{Kendall2018}, and \mbox{(v) \textbf{GradNorm:}} using the normalization proposed by \citet{Chen2018}.



\subsection{MultiMNIST}
\label{sec:multi_mnist_exp}

Our initial experiments are on MultiMNIST, an MTL version of the MNIST dataset \citep{multi_mnist}. In order to convert digit classification into a multi-task problem, \citet{multi_mnist} overlaid multiple images together. We use a similar construction. For each image, a different one is chosen uniformly in random. Then one of these images is put at the top-left and the other one is at the bottom-right. The resulting tasks are: classifying the digit on the top-left (task-L) and classifying the digit on the bottom-right (task-R). We use 60K examples and directly apply existing single-task MNIST models. The MultiMNIST dataset is illustrated in the supplement.

We use the LeNet architecture \citep{mnist}. We treat all layers except the last as the representation function $g$ and put two fully-connected layers as task-specific functions (see the supplement for details). We visualize the performance profile as a scatter plot of accuracies on task-L and task-R in Figure~\ref{fig:multi_mnist_performance_curve}, and list the results in Table~\ref{tab:multi_mnist}.

In this setup, any static scaling results in lower accuracy than solving each task separately (the single-task baseline). The two tasks appear to compete for model capacity, since increase in the accuracy of one task results in decrease in the accuracy of the other. Uncertainty weighting \citep{Kendall2018} and GradNorm \citep{Chen2018} find solutions that are slightly better than grid search but distinctly worse than the single-task baseline. In contrast, our method finds a solution that efficiently utilizes the model capacity and yields accuracies that are as good as the single-task solutions. This experiment demonstrates the effectiveness of our method as well as the necessity of treating MTL as multi-objective optimization. Even after a large hyper-parameter search, \emph{any} scaling of tasks does not approach the effectiveness of our method.



\subsection{Multi-Label Classification}

\begin{figure}[t]
\includegraphics[width=\textwidth]{radar_full_new}
\vspace{1mm}
\caption{Radar charts of percentage error per attribute on CelebA \citep{celeba}. Lower is better. We divide attributes into two sets for legibility: easy on the left, hard on the right. Zoom in for details.}
\label{fig:multi_label_radar}
\end{figure}


\begin{wraptable}{r}{0.3\textwidth}
%\vspace{-4mm}
\captionof{table}{Mean of error per category of MTL algorithms in multi-label classification on CelebA \citep{celeba}.}
\begin{tabular}{r@{\hspace{2mm}}c@{}}
\toprule
& Average  \\
&  error \\
\midrule
Single task & $8.77$ \\
Uniform scaling & $9.62$ \\
\citealt{Kendall2018} & $9.53$ \\
GradNorm & $8.44$ \\
Ours & $\mathbf{8.25}$  \\
\bottomrule
\end{tabular}
\label{table:multi_label_bar}
%\vspace{-5mm}
\end{wraptable}

Next, we tackle multi-label classification. Given a set of attributes, multi-label classification calls for deciding whether each attribute holds for the input. We use the CelebA dataset \citep{celeba}, which includes 200K face images annotated with 40 attributes. Each attribute gives rise to a binary classification task and we cast this as a 40-way MTL problem. We use ResNet-18 \citep{resnet} without the final layer as a shared representation function, and attach a linear layer for each attribute (see the supplement for further details).


We plot the resulting error for each binary classification task as a radar chart in Figure~\ref{fig:multi_label_radar}. The average over them is listed in Table~\ref{table:multi_label_bar}. We skip grid search since it is not feasible over 40 tasks. Although uniform scaling is the norm in the multi-label classification literature, single-task performance is significantly better. Our method outperforms baselines for significant majority of tasks and achieves comparable performance in rest. This experiment also shows that our method remains effective when the number of tasks is high.


\subsection{Scene Understanding}

To evaluate our method in a more realistic setting, we use scene understanding. Given an RGB image, we solve three tasks: semantic segmentation (assigning pixel-level class labels), instance segmentation (assigning pixel-level instance labels), and monocular depth estimation (estimating continuous disparity per pixel). We follow the experimental procedure of \citet{Kendall2018} and use an encoder-decoder architecture. The encoder is based on ResNet-50 \citep{resnet} and is shared by all three tasks. The decoders are task-specific and are based on the pyramid pooling module \citep{pspnet} (see the supplement for further implementation details).

Since the output space of instance segmentation is unconstrained (the number of instances is not known in advance), we use a proxy problem as in \citet{Kendall2018}. For each pixel, we estimate the location of the center of mass of the instance that encompasses the pixel. These center votes can then be clustered to extract the instances. In our experiments, we directly report the MSE in the proxy task. Figure~\ref{fig:cityscapes_performance_profile} shows the performance profile for each pair of tasks, although we perform all experiments on all three tasks jointly. The pairwise performance profiles shown in Figure~\ref{fig:cityscapes_performance_profile} are simply 2D projections of the three-dimensional profile, presented this way for legibility. The results are also listed in Table~\ref{tab:cityscapes_results}.

MTL outperforms single-task accuracy, indicating that the tasks cooperate and help each other. Our method outperforms all baselines on all tasks.


\subsection{Role of the Approximation}

In order to understand the role of the approximation proposed in Section~\ref{sec:approximation}, we compare the final performance and training time of our algorithm with and without the presented approximation in Table~\ref{tab:approximation_tradeoff} (runtime measured on a single Titan Xp GPU). For a small number of tasks (3 for scene understanding), training time is reduced by 40\%. For the multi-label classification experiment (40 tasks), the presented approximation accelerates learning by a factor of 25.

On the accuracy side, we expect both methods to perform similarly as long as the full-rank assumption is satisfied. As expected, the accuracy of both methods is very similar. Somewhat surprisingly, our approximation results in slightly improved accuracy in all experiments. While counter-intuitive at first, we hypothesize that this is related to the use of SGD in the learning algorithm. Stability analysis in convex optimization suggests that if gradients are computed with an error $\hat{\nabla}_\btheta \mathcal{L}^t = \nabla_\btheta \mathcal{L}^t + \mathbf{e}^t$ ($\btheta$ corresponds to $\btheta^{sh}$ in (\ref{eq:kkt_opt})), as opposed to $\mathbf{Z}$ in the approximate problem in \ref{eq:approx}, the error in the solution is bounded as $\|\hat{\mathbf{\alpha}} - \mathbf{\alpha} \|_2 \leq \mathcal{O}(\max_t \|\mathbf{e}^t\|_2)$. Considering the fact that the gradients are computed over the full parameter set (millions of dimensions) for the original problem and over a smaller space for the approximation (batch size times representation which is in the thousands), the dimension of the error vector is significantly higher in the original problem. We expect the $l_2$ norm of such a random vector to depend on the dimension.

In summary, our quantitative analysis of the approximation suggests that (i) the approximation does not cause an accuracy drop and (ii) by solving an equivalent problem in a lower-dimensional space, our method achieves both better computational efficiency and higher stability.

  {\small
  \begin{table}[t]
%  \vspace{-4mm}
  \caption{Effect of the MGDA-UB approximation. We report the final accuracies as well as training times for our method with and without the approximation.}
  %\vspace{1mm}
  \centering
  \begin{tabular}{@{}r@{\hspace{3mm}}c@{\hspace{3mm}}c@{\hspace{2mm}}c@{\hspace{2mm}}c@{}c@{\hspace{5mm}}c@{\hspace{2mm}}c@{}}
  \toprule
  & \multicolumn{4}{c}{Scene understanding (3 tasks)} &  & \multicolumn{2}{c}{Multi-label (40 tasks)}  \\
  \cmidrule(r){2-5} \cmidrule(lr){7-8}
                  & Training & Segmentation & Instance  & Disparity      & & Training & Average \\
                 & time     &  mIoU [\%]       & error [px] & error [px] & & time (hour)      & error \\
  \midrule
  Ours (w/o approx.) & $38.6$ & $66.13$ & $10.28$ & $2.59$ & & $429.9$ & $8.33$ \\
  Ours & $\mathbf{23.3}$ & $\mathbf{66.63}$ & $\mathbf{10.25}$ & $\mathbf{2.54}$  & & $\mathbf{16.1}$ & $\mathbf{8.25}$ \\
  \bottomrule
  \end{tabular}
  %\vspace{-2mm}
  \label{tab:approximation_tradeoff}
  \end{table}}

 
\section{Conclusion}\label{sec:conclusion}
%\vspace{-.1in}
In this work, we apply the attentional encoder-decoder for the task of abstractive summarization with very promising results, outperforming state-of-the-art results significantly on two different datasets. Each of our proposed novel models addresses a specific problem in abstractive summarization, yielding further improvement in performance. We also propose a new dataset for multi-sentence summarization and establish benchmark numbers on it. As part of our future work, we plan to focus our efforts on this data and build more robust models for summaries consisting of multiple sentences.


%Our results strongly demonstrate that sequence-to-sequence models are extremely promising for summarization. Some of the other lessons we learned from our experiments are: (i) the LVT-trick is very useful for summarization as it improves training speed while not sacrificing performance; (ii) traditional methods such as vocabulary expansion and syntax-based features can boost performance of deep learning based models as well. As part of our ongoing work, we are investigating on ways to effectively generate rare words in the summary, which appears to be a glaring weakness in the existing models.  


\ifCLASSOPTIONcaptionsoff
  \newpage
\fi


% can use a bibliography generated by BibTeX as a .bbl file
% BibTeX documentation can be easily obtained at:
% http://www.ctan.org/tex-archive/biblio/bibtex/contrib/doc/
% The IEEEtran BibTeX style support page is at:
% http://www.michaelshell.org/tex/ieeetran/bibtex/
\bibliographystyle{IEEEtran}
% argument is your BibTeX string definitions and bibliography database(s)
\bibliography{sample.bib}


\end{document}



\documentclass[10pt,twocolumn,letterpaper]{article}
\usepackage[utf8]{inputenc}
\usepackage{iccv}
\usepackage{times}
\usepackage[inline]{enumitem}
\usepackage[export]{adjustbox}
\usepackage{amsmath}
\usepackage{amssymb}
\usepackage{bm}  %
\usepackage{multirow}
\usepackage{booktabs}
\usepackage{calc}
\usepackage{chngcntr}  %
\usepackage{silence}
\WarningFilter{fixltx2e}{fixltx2e is not required}
\usepackage{fixltx2e}
\usepackage{dblfloatfix}  %
\usepackage{framed}
\usepackage{graphicx}
\usepackage{microtype}
\usepackage{pifont}%
\usepackage{placeins}  %
\usepackage{xfrac}  %
\usepackage{subcaption}
\usepackage{makecell}
\usepackage{tikz}
\usetikzlibrary{arrows.meta,positioning,fit,backgrounds,calc}
\usepackage[pagebackref=true,breaklinks=true,colorlinks,bookmarks=false]{hyperref} %
%
\urlstyle{sf}
\usepackage{cleveref} %


\usepackage{multibib}
\newcites{suppl}{Supplemental references}%

\setlength{\tabcolsep}{0.45em}  %

%
\makeatletter
\renewcommand\@makefntext[1]{%
  \noindent\makebox[1em][r]{\@makefnmark}#1}
\makeatother

\let\vec\bm

\newcommand{\inaug}{IN-aug\xspace}

\newcommand{\bx}{\vec{x}}
\newcommand{\bz}{\vec{z}}
\newcommand{\be}{\vec{e}}
\newcommand{\bw}{\vec{w}}
\newcommand{\bb}{\vec{b}}
\newcommand{\bW}{\vec{W}}
\newcommand{\ljoint}{\ell^\textrm{joint}}
\newcommand{\lclass}{\ell^\textrm{class}}
\newcommand{\lret}{\ell^\textrm{retr}}
\newcommand{\lmargin}{\ell^\text{margin}}

\newcommand{\cmark}{\ding{51}}%
\newcommand{\xmark}{\ding{55}}%  %

%
\makeatletter
\renewcommand{\paragraph}{%
  \@startsection{paragraph}{4}%
  %
  {\z@}{0.4em}{-1em}%
  {\normalfont\normalsize\bfseries}%
}

%
\setlist[itemize]{%
labelsep=5pt,%
labelindent=0.4\parindent,%
itemindent=0pt,%
leftmargin=*,%
itemsep=-4pt, 
}

\makeatother

\definecolor{darkgreen}{RGB}{0, 140, 0}
\definecolor{antiquefuchsia}{rgb}{0.57, 0.36, 0.51}
\definecolor{auburn}{rgb}{0.43, 0.21, 0.1}

\newcommand{\rv}[1]{{\color{antiquefuchsia}[\textbf{Rv}:#1]}}
\newcommand{\maxim}[1]{{\color{red}[\textbf{MB}:#1]}}
\newcommand{\matthijs}[1]{{\color{blue}[\textbf{MD}:#1]}}
\newcommand{\av}[1]{{\color{auburn}[\textbf{AV}:#1]}}
\newcommand{\iasonas}[1]{{\color{darkgreen}[\textbf{IK}:#1]}}

%
%
%
%
%
%

\newif\ifarxiv
\global\arxivtrue

\iccvfinalcopy
\ificcvfinal\pagestyle{empty}\fi
%
\def\iccvPaperID{5147}
\def\httilde{\mbox{\tt\raisebox{-.5ex}{\symbol{126}}}}

\ifarxiv\pagestyle{plain}\fi

%
\title{MultiGrain: a unified image embedding for classes and instances}
\author{\begin{tabular}{ccccc}
  Maxim Berman\thanks{Did this work during an internship at Facebook AI Research.} & Herv\'e J\'egou & Andrea Vedaldi & Iasonas Kokkinos & Matthijs Douze \\
  ESAT--PSI, KU Leuven & \multicolumn{4}{c}{Facebook AI Research} \\
 %
\end{tabular}}

\ificcvfinal
\hypersetup{  %
pdftitle={MultiGrain: a unified image embedding for classes and instances},
pdfsubject={},
pdfauthor={Maxim Berman, Herve Jegou, Andrea Vedaldi, Iasonas Kokkinos, Matthijs Douze} %
pdfkeywords={}
}
\fi

\makeatletter
\let\inserttitle\@title
%
\makeatother

\begin{document}
\maketitle
\begin{abstract}
MultiGrain is a network architecture producing compact vector representations that are suited both for image classification and particular object retrieval. 
It builds on a standard classification trunk.
The top of the network produces an embedding containing coarse and fine-grained information, so that images can be recognized based on the object class, particular object, or if they are distorted copies.
Our joint training is simple: we minimize a  cross-entropy loss for classification and a ranking loss 
that determines if two images are identical up to data augmentation, with no need for additional labels. 
A key component of MultiGrain is a pooling layer that takes advantage of high-resolution images with a network trained at a lower resolution. 

When fed to a linear classifier, the learned embeddings provide state-of-the-art classification accuracy. 
For instance, we obtain 79.4\% top-1 accuracy with a ResNet-50 learned on Imagenet, which is a +1.8\% absolute improvement over the AutoAugment method. 
When compared with the cosine similarity, the same embeddings perform on par with the state-of-the-art for image retrieval at moderate resolutions. 

 %

%
%
%
%
%
%
%
%
\end{abstract}

\section{Introduction}
\label{sec:intro}

Language modeling is among the important problems that require modeling long-term dependency, with successful applications such as unsupervised pretraining~\citep{dai2015semi,peters2018deep,radford2018improving,devlin2018bert}.
However, it has been a challenge to equip neural networks with the capability to model long-term dependency in sequential data.
Recurrent neural networks (RNNs), in particular Long Short-Term Memory (LSTM) networks~\citep{hochreiter1997long}, have been a standard solution to language modeling and obtained strong results on multiple benchmarks.
Despite the wide adaption, RNNs are difficult to optimize due to gradient vanishing and explosion~\citep{hochreiter2001gradient}, and the introduction of gating in LSTMs and the gradient clipping technique~\citep{graves2013generating} might not be sufficient to fully address this issue.
% ,pascanu2012understanding
Empirically, previous work has found that LSTM language models use 200 context words on average~\citep{khandelwal2018sharp}, indicating room for further improvement.

On the other hand, the direct connections between long-distance word pairs baked in attention mechanisms might ease optimization and enable the learning of long-term dependency~\citep{bahdanau2014neural,vaswani2017attention}.
Recently, \citet{al2018character} designed a set of auxiliary losses to train deep Transformer networks for character-level language modeling, which outperform LSTMs by a large margin.
Despite the success, the LM training in~\citet{al2018character} is performed on separated fixed-length segments of a few hundred characters, without any information flow across segments.
As a consequence of the fixed context length, the model cannot capture any longer-term dependency beyond the predefined context length.
In addition, the fixed-length segments are created by selecting a consecutive chunk of symbols without respecting the sentence or any other semantic boundary.
Hence, the model lacks necessary contextual information needed to well predict the first few symbols, leading to inefficient optimization and inferior performance.
We refer to this problem as \textit{context fragmentation}.

%However, the context length is fixed to hundreds of characters and thus it is not possible to model longer-term dependency. Moreover, it is not clear how the model performs on word-level language modeling data, as the granularity changes.

% Moreover, using auxiliary losses brings additional challenges such as properly tuning the mixture weights and the loss decay schedule.

To address the aforementioned limitations of fixed-length contexts, we propose a new architecture called Transformer-XL (meaning extra long).
We introduce the notion of recurrence into our deep self-attention network. In particular, instead of computing the hidden states from scratch for each new segment, we reuse the hidden states obtained in previous segments.
The reused hidden states serve as memory for the current segment, which builds up a recurrent connection between the segments.
As a result, modeling very long-term dependency becomes possible because information can be propagated through the recurrent connections.
Meanwhile, passing information from the previous segment can also resolve the problem of context fragmentation.
More importantly, we show the necessity of using relative positional encodings rather than absolute ones, in order to enable state reuse without causing temporal confusion.
Hence, as an additional technical contribution, we introduce a simple but more effective relative positional encoding formulation that generalizes to attention lengths longer than the one observed during training.

Transformer-XL obtained strong results on five datasets, varying from word-level to character-level language modeling.
Transformer-XL is also able to generate relatively coherent long text articles with \textit{thousands of} tokens (see Appendix \ref{sec:gen}), trained on only 100M tokens.
% Transformer-XL improves the previous state-of-the-art (SoTA) results from 1.06 to 0.99 in bpc on enwiki8, from 1.13 to 1.08 in bpc on text8, from 20.5 to 18.3 in perplexity on WikiText-103, and from 23.7 to 21.8 in perplexity on One Billion Word.
% Transformer-XL improves the previous state-of-the-art (SoTA) results to 0.99 in bpc on enwiki8, 1.08 in bpc on text8, 18.3 in perplexity on WikiText-103, and 21.8 in perplexity on One Billion Word.
% On small data, Transformer-XL also achieves a perplexity of 54.5 on Penn Treebank without finetuning, which is SoTA when comparable settings are considered.

Our main technical contributions include introducing the notion of recurrence in a purely self-attentive model and deriving a novel positional encoding scheme. These two techniques form a complete set of solutions, as any one of them alone does not address the issue of fixed-length contexts. Transformer-XL is the first self-attention model that achieves substantially better results than RNNs on both character-level and word-level language modeling.

% On WikiText-103, Transformer-XL improves the previous state-of-the-art (SoTA) results from 33 perplexity to 24, with a relative reduction of 27\%. On enwiki8 character-level language modeling, Transformer-XL achieves a SoTA bpc of 1.03, which outperforms \cite{al2018character} by 0.03 with 60+\% fewer parameters. Given a more common model size with 40+M parameters, Transformer-XL achieves a bpc of 1.06, compared to 1.11 by \cite{al2018character}. Transformer-XL also achieves perplexities of 54.5 on Penn Treebank and 29.4 on One Billion Word, which are SoTA when comparable settings are considered.

% Due to the ability of modeling long-range context, our best model uses attention lengths of 1,600 and 3,800 on WikiText-103 and enwiki8 respectively. We also devise a metric called \textit{Relative Effective Context Length} (RECL) that aims to fairly compare the ability of long-range dependency modeling.
% % perform a fair comparison of the gains brought by increasing the context lengths for different models.
% In this setting, Transformer-XL learns a RECL of 900 words on WikiText-103, while the numbers for recurrent networks and Transformer are only 500 and 128.

% We use two methods to quantitatively study the effective lengths of Transformer-XL and the baselines. Similar to \cite{khandelwal2018sharp}, we gradually increase the attention length at test time until no further noticeable improvement ($\sim$0.1\% relative gains) can be observed. Our best model in this settings use attention lengths of 1,600 and 3,800 on WikiText-103 and enwiki8 respectively.
% %In addition, since the effective context length of Transformer-XL can be longer than the attention length due to our recurrent formulation, we devise a metric called \textit{Relative Effective Context Length} (RECL) that aims to perform a fair comparison of the gains brought by increasing the context lengths for different models.
% In addition, we devise a metric called \textit{Relative Effective Context Length} (RECL) that aims to perform a fair comparison of the gains brought by increasing the context lengths for different models.
% In this setting, Transformer-XL learns a RECL of 900 words on WikiText-103, while the numbers for recurrent networks and Transformer are only 500 and 128.

\paragraph{3D Object Detection from RGB-D Data} Researchers have approached the 3D detection problem by taking various ways to represent RGB-D data.

\emph{Front view image based methods:} ~\cite{chen2016monocular, mousavian20163d, xiang2015data} take monocular RGB images and shape priors or occlusion patterns to infer 3D bounding boxes. ~\cite{li2016vehicle, deng2017amodal} represent depth data as 2D maps and apply CNNs to localize objects in 2D image. In comparison we represent depth as a point cloud and use advanced 3D deep networks (PointNets) that can exploit 3D geometry more effectively.

\emph{Bird's eye view based methods:} MV3D~\cite{cvpr17chen} projects LiDAR point cloud to bird's eye view and trains a region proposal network (RPN~\cite{ren2015faster}) for 3D bounding box proposal. However, the method lags behind in detecting small objects, such as pedestrians and cyclists and cannot easily adapt to scenes with multiple objects in vertical direction.
%Our method shares the idea with~\cite{cvpr17chen} in reducing 3D search cost by 2D search first. What differentiates our method from \cite{cvpr17chen} is that, \hao{???} instead of projecting point cloud to images costing loss in 3D geometry, we directly apply PointNet to point clouds that correspond to the 2D regions. % Besides, our method and MV3D can potentially be combined in the bird's eye setting. 3D proposals from our frustum-based PointNet and MV3D can be combined and our 3D network can also be used for bounding box estimation for point cloud in the bird's eye 2D region.

\emph{3D based methods:} ~\cite{wang2015voting, song2014sliding} train 3D object classifiers by SVMs on hand-designed geometry features extracted from point cloud and then localize objects using sliding-window search. \cite{engelcke2017vote3deep} extends ~\cite{wang2015voting} by replacing SVM with 3D CNN on voxelized 3D grids. \cite{ren2016three} designs new geometric features for 3D object detection in a point cloud. \cite{song2016deep, li20163d} convert a point cloud of the entire scene into a volumetric grid and use 3D volumetric CNN for object proposal and classification. Computation cost for those method is usually quite high due to the expensive cost of 3D convolutions and large 3D search space.
%In comparison, we use 2D region proposals from RGB images to reduce the search space from the entire 3D scenes into 3D frustums. Since the points cloud in the frustums have largely varying depth ranges and can be very sparse, it's not applicable to apply CNN on bird's eye view or apply 3D CNN in grids. Our frustum-based PointNet, on the other hand, suits well for this type of data and is able to accurately estimate 3D bounding box with good efficiency.
Recently, \cite{lahoud20172d} proposes a 2D-driven 3D object detection method that is similar to ours in spirit. However, they use hand-crafted features (based on histogram of point coordinates) with simple fully connected networks to regress 3D box location and pose, which is sub-optimal in both speed and performance. In contrast, we propose a more flexible and effective solution with deep 3D feature learning (PointNets).
%In addition we also get 3D instance segmentation as intermediate outputs. Evaluated on SUN-RGBD we show our method is \emph{8.9\%} better than theirs in mAP and \emph{34x} faster at the same time.


% \begin{enumerate}
%     \item ZOOX~\cite{mousavian20163d} image based
%     \item Vote3Deep~\cite{engelcke2017vote3deep} 3d cnn. Recent LIDAR-based methods place 3D windows in 3D voxel grids to score the point cloud
%     \item Voting for Voting~\cite{wang2015voting} Recent LIDAR-based methods place 3D windows in 3D voxel grids to score the point cloud. apply SVM classifers on 3D grids encoded with geometry features
%     \item MV3D~\cite{cvpr17chen}
%     \item VeloFCN~\cite{li2016vehicle} apply convolutional networks to the front view point map in a dense box prediction scheme
%     \item 3DOP~\cite{chen20153d} image based. reconstructs depth from stereo images and uses an energy minimization approach to generate 3D box proposals, which are fed to an R-CNN [10] pipeline for object recognition
%     \item Mono3D~\cite{chen2016monocular} image based. shares the same pipeline with 3DOP, it generates 3D proposals from monocular images.
%     \item 3DFCN~\cite{li20163d} 3d cnn.
%     \item 3DVP~\cite{xiang2015data} introduces 3D voxel patterns and employ a set of ACF detectors to do 2D detection and 3D pose estimation
%     \item Are Cars just 3D Box?~\cite{zeeshan2014cars} fit model to image patch
%     \item ~\cite{zia2013detailed} fit model to image patch
% \end{enumerate}
% \begin{enumerate}
%     \item SlidingShapes~\cite{song2014sliding} apply SVM classifers on 3D grids encoded with geometry features
%     \item DeepSlidingShapes~\cite{song2015sun} 3d cnn.
%     \item 2D-driven~\cite{lahoud20172d}
%     \item ~\cite{deng2017amodal} rgb-d images
%     \item COG feature~\cite{ren2016three}
%     \item Align 3D model in RGB-D~\cite{gupta2015aligning}
% \end{enumerate}

\paragraph{Deep Learning on Point Clouds}
Most existing works convert point clouds to images or volumetric forms before feature learning. \cite{wu20153d, maturana2015voxnet, qi2016volumetric} voxelize point clouds into volumetric grids and generalize image CNNs to 3D CNNs. ~\cite{li2016fpnn, riegler2016octnet, wang2017cnn, engelcke2017vote3deep} design more efficient 3D CNN or neural network architectures that exploit sparsity in point cloud.
However, these CNN based methods still require quantitization of point clouds with certain voxel resolution.
Recently, a few works~\cite{qi2017pointnet,qi2017pointnetplusplus} propose a novel type of network architectures (PointNets) that directly consumes raw point clouds without converting them to other formats. While PointNets have been applied to single object classification and semantic segmentation, our work explores how to extend the architecture for the purpose of 3D object detection.
%
\section{Architecture design}\label{sec:arch}
%
Our goal is to develop a convolutional neural network that is suitable for both image classification and instance retrieval.
%
%
%
In the current best practices, the architectures and training procedures used for class and instance recognition differ in a significant manner.
%

This section describes such technical differences, summarized in \cref{tab:diff_classif_instance}, together with our solutions to bridge them. 
This leads us to a unified architecture, shown in \cref{fig:3arch}, that we jointly train for both tasks in an end-to-end manner.


\begin{table}
\centering
\caption{\label{tab:diff_classif_instance}
Differences between classification and image retrieval:
Retrieval architectures incorporate a final pooling layer that is regionalized (RMAC) or magnifies activations (GeM). The triplet loss requires a batching strategy with pairs of matching images. %
%
}
\vspace{-7pt}
%
%
%
%
%
%
%
%
%
{\small
\begin{tabular}{@{}lc@{\hspace{5pt}}c@{}}
\toprule
                  & classification       & retrieval \\ 
\midrule 
spatial pooling   & avg. pooling         & RMAC~\cite{TOLIAS20143466} or GeM~\cite{radenovic2018fine} \\
loss              & cross-entropy        & triplet~\cite{Gordo2016DeepIR}  \\
batch sampling    & diverse              & similar images in batch  \\
whitening         & no                   & yes  \\
resolution        & low ($224^2$--$300^2$)           & high ($800$--$1$k$\times$scaled) \\
%
%
%
%
\bottomrule
\end{tabular}}

\vspace{-7pt}
\end{table}




\subsection{Spatial pooling operators\label{sec:p-pooling}}
%
%
%
%
%
%
%
%
%
%
%
This section considers the final, global spatial pooling layer.
%
Local pooling operators, usually max pooling, are found throughout the layers of most convolutional networks to achieve local invariance to small translations. 
By contrast, global spatial pooling converts a 3D tensor of activations produced by a convolutional trunk to a vector. 
%

\paragraph{Classification.}  
In early models such as LeNet-5~\cite{lecun1989backpropagation} or AlexNet~\cite{krizhevsky2012imagenet}, the final spatial pooling is just a linearization of the activation map.
It is therefore sensitive to the absolute location. 
Recent architectures such as ResNet and DenseNet employ average pooling, which is permutation invariant and hence offers a more global translation invariance. 
%
%
%

%
\begin{figure}
\centering
\includegraphics[width=0.9\linewidth]{figs/multigrain_rv}
\caption{\label{fig:3arch}
 Overview of our Multigrain architecture.} %
 \vspace{-10pt}
\end{figure}

\paragraph{Image retrieval}\hspace{-1em} requires more localized geometric information: particular objects or landmarks are visually more similar, but the task suffers more from clutter, and a given query image has no specific training data devoted to it. This is why the pooling operator tries to favor more locality. Next we discuss the generalized mean pooling operator. 

%
Let $\bx\in\mathbb{R}^{C\times W \times H}$ be the feature tensor computed by a convolutional neural network for a given image, where $C$ is the number of feature channels and $H$ and $W$ are the height and width of the map, respectively. 
We denote by $u\in\Omega=\{1,\dots,H\}\times\{1,\dots,W\}$ a ``pixel'' in the map, by $c$ the channel, and by $x_{cu}$ the corresponding tensor element: $\bx=[x_{cu}]_{c=1..C,u\in\Omega}$.
%
The generalized mean pooling (GeM) layer computes the generalized mean of each channel in a tensor.
Formally, the GeM embedding is given by
\begin{equation}\label{eq:gem}
 \be
 =
 \left[
 \Big(\frac{1}{|\Omega|}
 \sum_{u\in \Omega}x^p_{cu}
\Big)^\frac{1}{p}
 \right]_{c=1..C}
\end{equation}
where $p > 0$ is a parameter. Setting this exponent as $p>1$ increases the contrast of the pooled feature map and focuses on the salient features of the image~\cite{Bo2009EfficientMK,Boureau2010ATA,dollar2009integral}. 
%
GeM is a generalization of the average pooling commonly used in classification networks ($p=1$) and of spatial max-pooling layer ($p=\infty$). 
It is employed in the original R-MAC as an approximation of max pooling~\cite{dollar2009integral}, yet only recently~\cite{radenovic2018fine} it was shown to be competitive on its own with R-MAC for image retrieval.

%
To the best of our knowledge, this paper is the first to apply and evaluate GeM pooling in an image classification setting.
More importantly, we show later in this paper that \emph{adjusting the exponent is an effective way to change the input image resolution between train and test time} for all tasks, which explains why image retrieval has benefited from it considering that this task employs higher-resolution images. 

%
%
%
%
%
%
%
%
%
%
%
%
%
%
%
%
%

%
%

%

\subsection{Training objective\label{sec:losses}}

In order to combine the classification and retrieval tasks, we use a joint objective function composed of a classification loss and an instance retrieval loss. 
The two-branch architecture is illustrated in ~\cref{fig:3arch} and detailed next.

\paragraph{Classification loss.}

For classification, we adopt the standard cross-entropy loss.
Formally, let $\be_i\in\mathbb{R}^d$ be the embedding computed by the deep network for image $i$, $\bw_c \in \mathbb{R}^{d}$ the parameters of a linear classifier for class $c \in \{1,\dots,C\}$, and $y_i$ be the ground-truth class for that image.
Then
\begin{equation}\label{eq:crossent}
\lclass(\be_i,\bW,y_i)
=
- \langle \bw_{y_i}, \be_i \rangle
+ \log \sum_{c=1}^C
\exp \langle\bw_c, \be_i \rangle,
\end{equation}
where $\bW=[\bw_c]_{c=1..C}$.
We omit it for simplicity, but by adding a constant channel to the feature vector, the  bias of the classification layer is incorporated in its weight matrix. 


\paragraph{Retrieval loss.}

For image retrieval, the embeddings of two matching images (a positive pair) should have distances smaller than embeddings of non-matching images (a negative pair). 
%
This can be enforced in two ways. 
The contrastive loss~\cite{hadsell2006dimensionality} requires distances between positive pairs to be smaller than a threshold, and distances between negative pairs to be greater.
The triplet loss instead requires an image to be closer to a positive sibling than to a negative sibling~\cite{schroff2015facenet}, which is relative property of image triplets.
%
These losses requires adjusting multiple parameters, including how pairs and triplets are sampled. These parameters are sometimes hard to tune, especially for the triplet loss.

Wu~et~al.~\cite{wu2017sampling} proposed an effective method that addresses these difficulties.
Given a batch of images, they re-normalize their embeddings to the unit sphere, sample negative pairs as a function of the embedding similarity, and use those pairs in a margin loss, a variant of contrastive loss that shares some of the benefits of the triplet loss.

In more detail, given images $i,j\in\mathcal{B}$ in a batch with embeddings $\be_i,\be_j\in\mathbb{R}^d$, the margin loss is expressed as
\begin{equation}\label{eq:margin}
  \lret(\be_i, \be_j, \beta, y_{ij})
  = \max\{0, \alpha + y_{ij}(D(\be_i, \be_j) - \beta)\}
\end{equation}
where
$
%
    D(\be_i, \be_j) = \left\| \sfrac{\be_i}{\|\be_i\|} - \sfrac{\be_j}{\|\be_j\|}\right\|
$
%
is the Euclidean distance between the normalized embeddings, the label $y_{ij}$ is equal to 1 if the two images match and $-1$ otherwise, $\alpha > 0$ the margin (a constant hyper-parameter), and $\beta > 0$ is a parameter (learned during training together with the model parameters), controlling the volume of the embedding space occupied embedding vectors.
Due to the normalization, $D(\be_i, \be_j)$ is equivalent to a cosine similarity, which, up to whitening (\cref{sec:whiten}), is also used in retrieval.

%
%
%
%
%
%
%

Loss~\eqref{eq:margin} is computed on a subset of positive and negative pairs $(i,j)\in \mathcal{B}^2$ selected with the sampling~\cite{wu2017sampling} 
%
%
\begin{equation}
\begin{aligned}
\mathcal{P}_+(\mathcal{B})
&= \{ (i,j) \in \mathcal{B}^2: y_{ij} = 1\},
\\
\mathcal{P}_-(\mathcal{B})
&=
  \{ (i,j^*) : (i,j) \in \mathcal{P}_+(\mathcal{B}), 
  j^* \sim p(\cdot|i)
  \},
\\
\mathcal{P}(\mathcal{B})&=\mathcal{P}_+(\mathcal{B})\cup\mathcal{P}_-(\mathcal{B}),
\end{aligned}
\end{equation}
where the conditional probability of choosing a negative $j$ for image $i$ is 
$
 p(j|i) \propto \min\{\tau, q^{-1}(D(\be_i,\be_j))\} 
 \cdot \mathbf{1}_{\{y_{ij}=-1\}},
$
where $\tau > 0$ is a parameter and $q(z) \propto z^{d-2}(1 - z^2/4)^{\frac{d-3}{2}}$ is a PDF that depends on the embedding dimension $d$.

The use of distance weighted-sampling with margin loss is very suited to our joint training setting: this framework tolerates relatively small batch sizes ($|\mathcal{B}|\sim 80$ to $120$ instances) while requiring only a small amount of positives images (3 to 5) of each instance in the batch, without the need for elaborate parameter tuning or offline sampling.

\paragraph{Joint loss and architecture.}

The joint loss is a combination
of classification and retrieval loss weighted by a factor $\lambda \in [0, 1]$.
For a batch $\mathcal{B}$ of images, the joint loss writes as
\begin{equation}\label{eq:joint}
\frac{\lambda}{|\mathcal{B}|}\cdot\sum_{i\in\mathcal{B}}
\lclass(\be_i,\bw,y_i)
+
\frac{1-\lambda}{|\mathcal{P}(\mathcal{B})|}
\cdot\hspace{-1em}
\sum_{\raisebox{-3pt}{
$\scriptstyle(i,j)\in\mathcal{P}(\mathcal{B})$
}}
\hspace{-1em}
\lret(\be_i, \be_j, \beta, y_{ij}),
\end{equation}
\ie, losses are normalized by the number of items in the corresponding summations.


%
%
%

\subsection{Batching with repeated augmentation (RA)\label{sec:data-augmented-batches}}

%
%
Here, we propose to use only a training dataset for image classification, and train instance recognition via data augmentation.
%
The rationale is that data augmentation produces another image that contains the same object instance.
This approach does not require more annotation beyond the standard classification set.

%
%

%
%
%

We introduce a new sampling scheme for training with SGD and data augmentation, which we refer to as~\emph{repeated augmentations}.
In RA we form an image batch $\mathcal{B}$ by sampling $\left\lceil|\mathcal{B}|/m\right\rceil$ different images from the dataset, and transform them up to $m$ times by a set of data augmentations to fill the batch. 
Thus, the instance level ground-truth $y_{ij}=+1$ iff images $i$ and $j$ are two augmented versions of the same training image.
%
The key difference with the standard sampling scheme in SGD is that samples are not independent, as augmented versions of the same image are highly correlated. 
%
%
While this strategy reduces the performance if the batch size is small, for larger batch sizes RA outperforms the standard i.i.d.~scheme -- while using the same batch size and learning rate for both schemes.
This is different from the observation of~\cite{2019arXiv190109335H}, who also consider repeated samples in a batch, but simultaneously increase the size of the latter.

We conjecture that the benefit of correlated RA samples is to facilitate learning features that are invariant to the only difference between the repeated images --- the augmentations.
By comparison, with standard SGD sampling, two versions of the same image are seen only in different epochs. 
A study of an idealized problem illustrates this phenomenon in the supplementary material~\ref{sec:suppl-data-augmented-toy}.

\subsection{PCA whitening}\label{sec:whiten}

In order to transfer features learned via data augmentation to standard retrieval datasets, we apply a step of PCA whitening, in accordance
with previous works in image retrieval~\cite{Gordo2017EndtoEndLO,jegou2012negative}.
The Euclidean distance between transformed features is equivalent to the Mahalanobis distance between the input descriptors.
This is done after training the network, using an external dataset of unlabelled images. 


The effect of PCA whitening can be undone in the parameters of the classification layer, so that the whitened embeddings can be used for both classification and instance retrieval. In detail, let $\be$ be an image embedding vector and $\bw_c$ the weight vector for class $c$, such that $\langle\bw_c,\be\rangle$ are the outputs of the classifier as in~\cref{eq:crossent}.
The whitening operation $\Phi$ can be written as~\cite{Gordo2017EndtoEndLO}
$
    \Phi(\be) = S \left(\frac{\be}{\|\be\|} - \bm{\mu}\right)
$
given the whitening matrix $S$ and centering vector $\bm{\mu}$; hence
\begin{equation*}\label{eq:adapted_classifier}
    \langle\bw_c,\be\rangle = 
    \langle\bw_c,\Phi^{-1}(\Phi(\be))\rangle =
    \|\be\| \left( \langle\bw_c', \Phi(\be)\rangle + b_c' \right) 
\end{equation*}
where 
$
\bw_c' =  
S^{-1 \top} \bw_c
$
and
$
b_c' = \langle\bw_c, \mu \rangle
$
are the modified weight and bias for class $c$.
%
We observed that inducing decorrelation via a loss~\cite{Cogswell2016ReducingOI} is insufficient to ensure that features generalize well, which concurs with prior works~\cite{Gordo2017EndtoEndLO,radenovic2018fine}.


%

%
%
%

\subsection{Input sizes\label{sec:input-size}}

The standard practice in image classification is to resize and center-crop input images to a relatively low resolution, e.g.~$224\times 224$ pixels~\cite{krizhevsky2012imagenet}. 
%
The benefits are a smaller memory footprint, faster inference, and the possibility of batching the inputs if they are cropped to a common size. On the other hand, image retrieval is typically dependent on finer details in the images, as an instance can be seen under a variety of scales, and cover only a small amount of pixels.
The currently best-performing feature extractors for image retrieval therefore commonly use input sizes of $800$~\cite{Gordo2017EndtoEndLO} or $1024$~\cite{radenovic2018fine} pixels for the largest side, without cropping the image to a square.
This is impractical for end-to-end training of a joint classification and retrieval network.

Instead, we train our architecture at the standard $224\times 224$ resolution, and use larger input resolutions at test time only. 
This is possible due to a key advantage of our architecture:
a network trained with a pooling exponent $p$ and resolution $s$ can be evaluated at a larger resolution $s^*>s$ using a larger pooling exponent $p^*>p$, see our validation in \cref{sec:classif-results}. 

\label{sec:expanding-resolution}
\paragraph{Proxy task for cross-validation of $p^*$.} In order to select the exponent $p^*$, suitable for all tasks, we create a synthetic retrieval task {\bf \inaug} in between classification and retrieval. We sample $2,\!000$ images from the training set of ImageNet, $2$ per class, and create 5 augmented copies of each of them, using the ``full'' data augmentation described before. 

We evaluate the retrieval accuracy on \inaug in a fashion similar to UKBench~\cite{nister2006scalable}, with an accuracy ranging from 0 to 5 depending measuring how many of the first 5 augmentations are ranked in top 5 positions. 
We pick the best-performing $p^* \in \{1, 2,\ldots, 10\}$ on \inaug, 
which provides the following choices as a function of $\lambda$ and $s^*$:
%
\begin{equation*}
{\small
\begin{tabular}{c|lrrr}
\toprule
$\lambda$ & $s^*=$ & $224$ & $500$ & $800$ \\
\midrule
$1$   & $p^*=$ & $3$ & $4$ & $4$\\
$0.5$ & $p^*=$ & $3$ & $4$ & $5$\\
\bottomrule
\end{tabular}}
\end{equation*}

The optimal $p^*$ obtained on \inaug provides a trade-off between retrieval and classification. Experimentally, we observed that other choices are suitable for setting this parameter: %
fine-tuning the parameter $p^*$ alone using training inputs at a given resolution by back-propagation of the cross-entropy loss provides similar results and values of $p^*$.  


%

% !TEX root = ../multi_task.tex

We evaluate the presented MTL method on a number of problems. First, we use MultiMNIST \citep{multi_mnist}, an MTL adaptation of MNIST \citep{mnist}. Next, we tackle multi-label classification on the CelebA dataset \citep{celeba} by considering each label as a distinct binary classification task. These problems include both classification and regression, with the number of tasks ranging from 2 to 40. Finally, we experiment with scene understanding, jointly tackling the tasks of semantic segmentation, instance segmentation, and depth estimation on the Cityscapes dataset \citep{cityscapes}. We discuss each experiment separately in the following subsections.

The baselines we consider are (i) \textbf{uniform scaling:} minimizing a uniformly weighted sum of loss functions \mbox{$\frac{1}{T}\sum_t \lL^t$}, \mbox{(ii) \textbf{single task:}} solving tasks independently, \mbox{(iii) \textbf{grid search:}} exhaustively trying various values from $\{ c^t \in [0,1] | \sum_t c^t = 1\}$ and optimizing for $\frac{1}{T}\sum_t c^t \lL^t$, \mbox{(iv) \textbf{\citet{Kendall2018}:}} using the uncertainty weighting proposed by \citet{Kendall2018}, and \mbox{(v) \textbf{GradNorm:}} using the normalization proposed by \citet{Chen2018}.



\subsection{MultiMNIST}
\label{sec:multi_mnist_exp}

Our initial experiments are on MultiMNIST, an MTL version of the MNIST dataset \citep{multi_mnist}. In order to convert digit classification into a multi-task problem, \citet{multi_mnist} overlaid multiple images together. We use a similar construction. For each image, a different one is chosen uniformly in random. Then one of these images is put at the top-left and the other one is at the bottom-right. The resulting tasks are: classifying the digit on the top-left (task-L) and classifying the digit on the bottom-right (task-R). We use 60K examples and directly apply existing single-task MNIST models. The MultiMNIST dataset is illustrated in the supplement.

We use the LeNet architecture \citep{mnist}. We treat all layers except the last as the representation function $g$ and put two fully-connected layers as task-specific functions (see the supplement for details). We visualize the performance profile as a scatter plot of accuracies on task-L and task-R in Figure~\ref{fig:multi_mnist_performance_curve}, and list the results in Table~\ref{tab:multi_mnist}.

In this setup, any static scaling results in lower accuracy than solving each task separately (the single-task baseline). The two tasks appear to compete for model capacity, since increase in the accuracy of one task results in decrease in the accuracy of the other. Uncertainty weighting \citep{Kendall2018} and GradNorm \citep{Chen2018} find solutions that are slightly better than grid search but distinctly worse than the single-task baseline. In contrast, our method finds a solution that efficiently utilizes the model capacity and yields accuracies that are as good as the single-task solutions. This experiment demonstrates the effectiveness of our method as well as the necessity of treating MTL as multi-objective optimization. Even after a large hyper-parameter search, \emph{any} scaling of tasks does not approach the effectiveness of our method.



\subsection{Multi-Label Classification}

\begin{figure}[t]
\includegraphics[width=\textwidth]{radar_full_new}
\vspace{1mm}
\caption{Radar charts of percentage error per attribute on CelebA \citep{celeba}. Lower is better. We divide attributes into two sets for legibility: easy on the left, hard on the right. Zoom in for details.}
\label{fig:multi_label_radar}
\end{figure}


\begin{wraptable}{r}{0.3\textwidth}
%\vspace{-4mm}
\captionof{table}{Mean of error per category of MTL algorithms in multi-label classification on CelebA \citep{celeba}.}
\begin{tabular}{r@{\hspace{2mm}}c@{}}
\toprule
& Average  \\
&  error \\
\midrule
Single task & $8.77$ \\
Uniform scaling & $9.62$ \\
\citealt{Kendall2018} & $9.53$ \\
GradNorm & $8.44$ \\
Ours & $\mathbf{8.25}$  \\
\bottomrule
\end{tabular}
\label{table:multi_label_bar}
%\vspace{-5mm}
\end{wraptable}

Next, we tackle multi-label classification. Given a set of attributes, multi-label classification calls for deciding whether each attribute holds for the input. We use the CelebA dataset \citep{celeba}, which includes 200K face images annotated with 40 attributes. Each attribute gives rise to a binary classification task and we cast this as a 40-way MTL problem. We use ResNet-18 \citep{resnet} without the final layer as a shared representation function, and attach a linear layer for each attribute (see the supplement for further details).


We plot the resulting error for each binary classification task as a radar chart in Figure~\ref{fig:multi_label_radar}. The average over them is listed in Table~\ref{table:multi_label_bar}. We skip grid search since it is not feasible over 40 tasks. Although uniform scaling is the norm in the multi-label classification literature, single-task performance is significantly better. Our method outperforms baselines for significant majority of tasks and achieves comparable performance in rest. This experiment also shows that our method remains effective when the number of tasks is high.


\subsection{Scene Understanding}

To evaluate our method in a more realistic setting, we use scene understanding. Given an RGB image, we solve three tasks: semantic segmentation (assigning pixel-level class labels), instance segmentation (assigning pixel-level instance labels), and monocular depth estimation (estimating continuous disparity per pixel). We follow the experimental procedure of \citet{Kendall2018} and use an encoder-decoder architecture. The encoder is based on ResNet-50 \citep{resnet} and is shared by all three tasks. The decoders are task-specific and are based on the pyramid pooling module \citep{pspnet} (see the supplement for further implementation details).

Since the output space of instance segmentation is unconstrained (the number of instances is not known in advance), we use a proxy problem as in \citet{Kendall2018}. For each pixel, we estimate the location of the center of mass of the instance that encompasses the pixel. These center votes can then be clustered to extract the instances. In our experiments, we directly report the MSE in the proxy task. Figure~\ref{fig:cityscapes_performance_profile} shows the performance profile for each pair of tasks, although we perform all experiments on all three tasks jointly. The pairwise performance profiles shown in Figure~\ref{fig:cityscapes_performance_profile} are simply 2D projections of the three-dimensional profile, presented this way for legibility. The results are also listed in Table~\ref{tab:cityscapes_results}.

MTL outperforms single-task accuracy, indicating that the tasks cooperate and help each other. Our method outperforms all baselines on all tasks.


\subsection{Role of the Approximation}

In order to understand the role of the approximation proposed in Section~\ref{sec:approximation}, we compare the final performance and training time of our algorithm with and without the presented approximation in Table~\ref{tab:approximation_tradeoff} (runtime measured on a single Titan Xp GPU). For a small number of tasks (3 for scene understanding), training time is reduced by 40\%. For the multi-label classification experiment (40 tasks), the presented approximation accelerates learning by a factor of 25.

On the accuracy side, we expect both methods to perform similarly as long as the full-rank assumption is satisfied. As expected, the accuracy of both methods is very similar. Somewhat surprisingly, our approximation results in slightly improved accuracy in all experiments. While counter-intuitive at first, we hypothesize that this is related to the use of SGD in the learning algorithm. Stability analysis in convex optimization suggests that if gradients are computed with an error $\hat{\nabla}_\btheta \mathcal{L}^t = \nabla_\btheta \mathcal{L}^t + \mathbf{e}^t$ ($\btheta$ corresponds to $\btheta^{sh}$ in (\ref{eq:kkt_opt})), as opposed to $\mathbf{Z}$ in the approximate problem in \ref{eq:approx}, the error in the solution is bounded as $\|\hat{\mathbf{\alpha}} - \mathbf{\alpha} \|_2 \leq \mathcal{O}(\max_t \|\mathbf{e}^t\|_2)$. Considering the fact that the gradients are computed over the full parameter set (millions of dimensions) for the original problem and over a smaller space for the approximation (batch size times representation which is in the thousands), the dimension of the error vector is significantly higher in the original problem. We expect the $l_2$ norm of such a random vector to depend on the dimension.

In summary, our quantitative analysis of the approximation suggests that (i) the approximation does not cause an accuracy drop and (ii) by solving an equivalent problem in a lower-dimensional space, our method achieves both better computational efficiency and higher stability.

  {\small
  \begin{table}[t]
%  \vspace{-4mm}
  \caption{Effect of the MGDA-UB approximation. We report the final accuracies as well as training times for our method with and without the approximation.}
  %\vspace{1mm}
  \centering
  \begin{tabular}{@{}r@{\hspace{3mm}}c@{\hspace{3mm}}c@{\hspace{2mm}}c@{\hspace{2mm}}c@{}c@{\hspace{5mm}}c@{\hspace{2mm}}c@{}}
  \toprule
  & \multicolumn{4}{c}{Scene understanding (3 tasks)} &  & \multicolumn{2}{c}{Multi-label (40 tasks)}  \\
  \cmidrule(r){2-5} \cmidrule(lr){7-8}
                  & Training & Segmentation & Instance  & Disparity      & & Training & Average \\
                 & time     &  mIoU [\%]       & error [px] & error [px] & & time (hour)      & error \\
  \midrule
  Ours (w/o approx.) & $38.6$ & $66.13$ & $10.28$ & $2.59$ & & $429.9$ & $8.33$ \\
  Ours & $\mathbf{23.3}$ & $\mathbf{66.63}$ & $\mathbf{10.25}$ & $\mathbf{2.54}$  & & $\mathbf{16.1}$ & $\mathbf{8.25}$ \\
  \bottomrule
  \end{tabular}
  %\vspace{-2mm}
  \label{tab:approximation_tradeoff}
  \end{table}}

\section{Conclusions}

Our work is motivated by two major deficiencies in training the current generative models for text generation: exposure bias and a loss which does not operate at the sequence level.
While Reinforcement learning can potentially address these issues, it struggles in settings when 
there are very large action spaces, such as in text generation. Towards that end, 
we propose the MIXER algorithm, which deals with these issues and enables successful training of reinforcement learning models for text generation. 
We achieve this by replacing the initial random policy with the optimal policy of a cross-entropy trained model and by gradually exposing the model more and more to its own predictions in an incremental learning framework.




%. First, the exposure bias affecting the commonly used cross-entropy loss. 
%While the model sees only ground truth inputs at training time, at test time model predictions are fed back as input to generate a full sequence. 
%Second, current text generation systems are often trained to predict the next word in the sequence without taking into account the quality of the % overall sequence. 
% These discrepancies make the generation process brittle.
%Reinforcement learning is a framework that can address these issues. 
%First, at training time the model is used to generate an entire sequence of actions. 
%Second, the reward does not need to factor over individual words nor does it need to be differentiable. 
%Therefore, we can easily and directly operate at the sequence level, generate at training time and optimize our model towards any desired metric, such as BLEU and ROUGE. 
%One challenge with reinforcement learning is that it struggles with very large action spaces such as for text generation.

% Mixed Incremental Cross-Entropy Reinforce (MIXER) 
%The algorithm we propose, MIXER, 
%deals with this issue and enables successful training of reinforcement learning models for text generation. 
%We achieve this by replacing the initial random policy with the optimal policy of a cross-entropy trained model and by gradually exposing the model more and more to its own predictions in an incremental learning framework.

Our results show that MIXER outperforms three strong baselines for greedy generation and it is very competitive with beam search. 
The approach we propose is agnostic to the underlying model or the form of the reward function. 
% We are free to use any other metric as reward such as ROUGE or METEOR instead of BLEU. 
% Similarly, we may use a different parametric model such as a feed- forward network or an LSTM \citep{lstm}.
In future work we would like to design better estimation techniques for the average reward $\bar{r}_t$, because poor estimates can lead to slow convergence of both REINFORCE and MIXER. 
Finally, our training algorithm relies on a single sample while it would be interesting to investigate the effect of more comprehensive search methods at training time.


% Our work addresses two major deficiencies in training the current generative models for text generation. First, it addresses the {\it exposure bias} affecting the commonly used cross-entropy loss. 
% %While the model sees only ground truth inputs at training 
% %time, at test time model predictions are fed back as input to generate a full sequence. 
% Second, it directly tries to optimize for the final evaluation metric, namely, BLEU. 
% %current text generation systems are often trained to predict the next word in the sequence without taking into account the quality of the overall sequence. These discrepancies make the generation process brittle. 
% Both these objectives are accomplished by the proposed Mixed Incremental Cross-Entropy Reinforce (MIXER) algorithm. 
% %Reinforcement learning is a framework that can address these issues. First, at training time the model is used to generate an entire sequence of actions. Second, the reward does not need to factor over individual words nor does it need to be differentiable. Therefore, we can easily and directly operate at the sequence level, generate at training time and optimize our model towards BLEU, our test time evaluation metric. One challenge with reinforcement learning is that it struggles with very large action spaces such as for text generation.
% MIXER is an extension of the REINFORCE algorithm applied to text generation, which 
% %Mixed Incremental Cross-Entropy Reinforce (MIXER) deals with this issue and enables
% %successful training of reinforcement learning models for text generation.
% replaces the initial random policy with the optimal policy of
% a cross-entropy trained model and it gradually exposes the model more and more to its own predictions in an incremental learning framework.

% Our results show that MIXER outperforms three strong baselines for greedy generation and it is very competitive with beam search. 
% The approach we propose is agnostic to the underlying model or the form of the reward function. 
% We are free to use any other metric as reward such as ROUGE or METEOR instead of BLEU. 
% Similarly, we may use a different parametric model such as a feed-forward network or an LSTM~\citep{lstm}.

% For future we would like to design better estimation techniques for the average reward $\bar{r}_t$, because poor estimates can lead to slow convergence of both REINFORCE and MIXER.
% Finally, our training algorithm relies on a single sample while it would be interesting to investigate the effect of more comprehensive search methods at training time.

% In this study, we investigated sequence level training algorithms for RNNs with the goal to improve text generation.
% Today, the dominant training protocol is cross-entropy loss, which optimizes the prediction of the next word in the sequence. However, at test time the model is asked to predict several words in the future by re-circulating its own prediction back to the input. 
% The problem of predicting several steps in the future while obtaining delayed feedback, and to perform prediction via a discrete sequence of actions inspired us to apply reinforcement learning techniques. Unfortunately, reinforcement learning techniques do not usually handle well large action spaces, like those we encounter in typical language modeling applications.

% MIXER addresses these limitations through pre-training and incremental learning. 

% MIXER addresses these limitations by leveraging both the fact that we have access to the optimal policy and by using incremental learning.
% Since we have examples of ground truth generation, we can "pre-train" the model for next step prediction via cross-entropy. This drastically reduces the actual search space. By using incremental learning, the model is then able to gradually produce stable sequences and to make effective use of its own predictions.

% Our empirical validation shows that the model we propose achieves the best BLEU score compared to three strong baselines. Moreover, generations can be further improved by using beam search. Note that the approach we proposed is agnostic of the particular underlying model and metric. We can easily replace BLEU with ROUGE, METEOR, \etc by simply swapping the function that computes rewards within the training loop. Similarly, the training algorithm applies to any type of model and RNN, LSTM~\citep{lstm} included.



% There are several avenues of future investigation. First, REINFORCE upon which we build, requires careful estimation of the average reward. Poor estimation of this value can yield very slow convergence. More generally, searching at training time is still an unsolved problem. In particular, it would be very powerful to include beam search also at training time. 


\ificcvfinal
\paragraph*{Acknowledgments.} 
We thank Kaiming He for useful feedback and references. 
Maxim Berman is supported by Research Foundation - Flanders (FWO) through project number G0A2716N. PSI--ESAT acknowledges a GPU server donation from FAIR Partnership Program.

\fi
%


{\small\bibliographystyle{ieee}\bibliography{biblio}}

%
%
\newcommand{\vv}[1]{{\texttt{#1}}}
\newcommand{\conv}{\vv{conv}}
\newcommand{\fc}{\vv{fc}}
\newcolumntype{x}{>\small c}

\newcommand{\trainval}{trainval\raisebox{0.2ex}{$\ast$}}
\newcolumntype{L}[1]{>{\raggedright\let\newline\\\arraybackslash\hspace{0pt}}m{#1}}
\newcolumntype{C}[1]{>{\centering\let\newline\\\arraybackslash\hspace{0pt}}m{#1}}
\newcolumntype{R}[1]{>{\raggedleft\let\newline\\\arraybackslash\hspace{0pt}}m{#1}}
\newcommand{\hl}[1]{\underline{\textbf{#1}}}
\newcolumntype{o}{>\small L}
% \renewcommand{\baselinestretch}{0.99}

\clearpage
\onecolumn

% \commentAS{Discuss making the text regular size.}\\
% \commentAS{Standardize table sizes and format.}

\section*{Appendix A}

\subsubsection*{De-duplication Experiments}
A dataset with 300M images is almost guaranteed to contain images that overlap with the validation set of target tasks. In fact, we find that even for ImageNet, there are 890 out of 50K validation images have near-duplicate images in the training.

We use visual embeddings to measure similarities and identify duplicate or near-duplicate images. The embeddings are based on deep learning features. We find there are 5536 out of 50K images in ImageNet validation set, 1648 out of 8K images in COCO \minival, 201 out of 4952 images in Pascal VOC 2007 test set, and 84 out of 1449 images in Pascal VOC 2012 validation set that have near duplicates in JFT-300M. We rerun several experiments by removing near-duplicate images from validation sets and then comparing performance between baselines and learned models. We observe no significant differences in trends. Table~\ref{tab:imagenet_dedup},~\ref{tab:coco_dedup} and ~\ref{tab:pascal_dedup} show that the duplicate images have minimal impact on performance for all experiments.

\begin{table*}[h]
\centering
\renewcommand{\arraystretch}{1.1}
\renewcommand{\tabcolsep}{1.2mm}
% \resizebox{0.7\linewidth}{!}{
\begin{tabular}{@{} l | c  c | c  c @{}}
& \multicolumn{2}{c|}{Original} & \multicolumn{2}{c}{De-duplication} \\
& Top-1 Acc. & Top-5 Acc. & Top-1 Acc. & Top-5 Acc. \\
\hline
MSRA checkpoint & 76.4 & 92.9 & 76.4 & 92.9 \\
Random initialization & 77.5 & 93.9 & 77.5 & 93.8 \\
Fine-tune from JFT-300M & 79.2 & 94.7 & 79.3 & 94.7 \\
\end{tabular}
% }
\vspace{0.05in}
\caption{Top-1 and top-5 classification accuracy on ImageNet validation set, before and after de-duplication. Single model and single crop are used.}
\label{tab:imagenet_dedup}
\end{table*}
\vspace{-0.1in}

\begin{table*}[h]
\centering
\renewcommand{\arraystretch}{1.1}
\renewcommand{\tabcolsep}{1.2mm}
% \resizebox{0.7\linewidth}{!}{
\begin{tabular}{@{} l | c  c | c  c @{}}
& \multicolumn{2}{c|}{Original} & \multicolumn{2}{c}{De-duplication} \\
& mAP@0.5 & mAP@[0.5,0.95] & mAP@0.5 & mAP@[0.5,0.95] \\
\hline
ImageNet & 54.0 & 34.5 & 54.0 & 34.6 \\
300M & 57.1 & 36.8 & 56.8 & 36.7 \\
ImageNet+300M & 58.2 & 37.8 & 58.2 & 37.7 \\
\end{tabular}
% }
\vspace{0.07in}
\caption{mAP@0.5 and mAP@[0.5,0.95] for object detection performance on COCO \minival, before and after de-duplication.}
\label{tab:coco_dedup}
\end{table*}
\vspace{-0.1in}

\begin{table*}[h]
\centering
\renewcommand{\arraystretch}{1.1}
\renewcommand{\tabcolsep}{1.2mm}
% \resizebox{0.7\linewidth}{!}{
\begin{tabular}{@{} l | c  c | c  c @{}}
& \multicolumn{2}{c|}{VOC07 Detection} & \multicolumn{2}{c}{VOC12 Segmentation} \\
& Original & De-duplication & Original & De-duplication \\
\hline
ImageNet & 76.3 & 76.5 & 73.6 & 73.3 \\
300M & 81.4 & 81.5 & 75.3 & 75.1 \\
ImageNet+300M & 81.3 & 81.2 & 76.5 & 76.5 \\
\end{tabular}
% }
\vspace{0.05in}
\caption{Object detection and semantic segmentation performance on Pascal VOC, before and after deduplication. (Left) Object detection mAP@0.5 on Pascal VOC 2007 test set. (Right) Semantic segmentation mIOU on Pascal VOC 2012 validation set.}
\label{tab:pascal_dedup}
\end{table*}

We do not conduct de-duplication experiments of COCO testdev dataset for object detection and pose estimation as their groundtruth annotations are not publicly available.


\clearpage

\section*{Appendix B}

\subsubsection*{Detailed and Per-category Results: Object Detection}
In this section, we present detailed and per-category object detection results for Table 2 (Section 5.2) from the main submission, evaluated on the COCO test-dev split. In Table~\ref{tab:coco_main}, we report detailed AP and AR results using different initializations. In Table~\ref{tab:per_category}, we provide per-category AP and AP@.5 results.  


\begin{table*}[h]
\centering
\renewcommand{\arraystretch}{1.1}
\renewcommand{\tabcolsep}{1.2mm}
% \resizebox{\linewidth}{!}{
\begin{tabular}{@{} l | c  c  c  c  c  c | c  c  c  c  c  c @{}}

& AP & AP@.5 & AP@.75 & AP(S) & AP(M) & AP(L) & AR & AR@.5 & AR@.75 & AR(S) & AR(M) & AR(L)\\
\hline
ImageNet & 34.3 & 53.6 & 36.9 & 15.1 & 37.4 & 48.5 & 30.2 & 47.3 & 49.7 & 26.0 & 54.6 & 68.6\\
300M & 36.7 & 56.9 & 39.5 & 17.1 & 40.0 & 50.7 & 31.5 & 49.3 & 51.9 & 28.6 & 56.9 & 70.4 \\
ImageNet+300M & 37.4 & 58.0 & 40.1 & 17.5 & 41.1 & 51.2 & 31.8 & 49.8 & 52.4 & 29.0 & 57.7 & 70.5 \\
\end{tabular}
% }
\vspace{0.1in}
\caption{Object detection performance on COCO test-dev split using different model initializations.}
\label{tab:coco_main}
\end{table*}
\vspace{-0.1in}

\subsubsection*{Per-category Results: Semantic Segmentation}
In Table~\ref{tab:segmentation_supp}, we report quantitative results on the VOC 2012 segmentation validation set for all classes (refer to Figure 5 (left), Section 5.3 in the main submission). Results are reported for different initializations. We observe more than 7 point improvement for categories like boat and horse.


\begin{table*}[h]
	\centering
	\renewcommand{\arraystretch}{1.1}
	\renewcommand{\tabcolsep}{1.2mm}
	\resizebox{\linewidth}{!}{
		\begin{tabular}{@{}L{2.5cm} c c*{29}{x} @{}}
			\toprule
			Initialization & mIOU & bg & {aero} & {bike} & {bird} & {boat} & {bottle} & {bus} & {car}& {cat} & {chair} & {cow} & {table} & {dog} & {horse} & {mbike} & {persn} & {plant} & {sheep} & {sofa} & {train} & {tv} \\
			\midrule
			ImageNet &
			73.6 & 93.2 & 88.9 & 40.1 & 87.3 & 65.0 & 78.8 & 89.9 & 84.3 & 88.8 & 37.2 & 81.6 & 49.3 & 84.1 & 78.9 & 79.3 & 83.3 & 57.7 & 82.0 & 41.7 & 80.3 & 73.1 \\
			300M &
			75.3 & 93.7 & 89.8 & 40.1 & 89.8 & 70.6 & 78.5 & 89.9 & 86.1 & 92.0 & 36.9 & 80.9 & 52.8 & 87.6 & 82.4 & 80.8 & 84.3 & 61.7 & 84.4 & 44.8 & 80.9 & 72.6\\
			ImageNet+300M &
			76.5 & 94.8 & 90.4 & 41.6 & 89.1 & 73.1 & 80.4 & 92.3 & 86.7 & 92.0 & 39.6 & 82.7 & 52.7 & 86.2 & 86.1 & 83.6 & 85.7 & 61.5 & 83.9 & 45.3 & 84.6 & 73.6 \\
			\bottomrule
		\end{tabular}
	}
	\vspace{0.01in}
	\caption{Per-class semantic segmentation performance on PASCAL VOC 2012 validation set.}
	\label{tab:segmentation_supp}
\end{table*}
\vspace{-0.1in}

\subsubsection*{Detailed Results: Human Pose Estimation}
In Table~\ref{tab:coco_pose_supp}, we present all AP and AR results for the performance reported in Table 7 (Section 5.4) in the main submission.
%Please note that we reported incorrect baseline AR and AR@.5 performance for ImageNet baseline in the main submission (the performance is 66.7 AR and 86.6 AR@.5 instead of 68.6 and 88.1 respectively as reported in Table 7, Section 5.4).

% \vspace{-0.1in}
\begin{table*}[h]
\centering
\begin{tabular}{@{} l | c  c  c  c  c | c  c  c  c  c @{}}
& AP & AP@.5 & AP@.75 & AP(M) & AP(L) & AR & AR@.5 & AR@.75 & AR(M) & AR(L)\\
\hline
CMU Pose~[3] & 61.8 & 84.9 & 67.5 & 57.1 & 68.2 & 66.5 & 87.2 & 71.8 & 60.6 & 74.6 \\
ImageNet~[26] & 62.4 & 84.0 & 68.5 & 59.1 & 68.1 & 66.7 & 86.6 & 72.0 & 60.8 & 74.9\\
300M & 64.8 & 85.8 & 71.5 & 62.2 & 70.3 & 69.4 & 88.4 & 75.2 & 63.9 & 77.0\\
ImageNet+300M & 64.4 & 85.7 & 70.7 & 61.8 & 69.8 & 69.1 & 88.2 & 74.8 & 63.7 & 76.6\\
\end{tabular}
\vspace{0.05in}
\caption{Human pose estimation performance on the COCO test-dev split.}
\label{tab:coco_pose_supp}
\end{table*}


\begin{table*}[h!]
\centering
\caption{Per-class object detection performance on COCO test-dev split using different model initializations.}
\label{tab:allclasses}
\vspace{0.1in}
\renewcommand{\arraystretch}{1.2}
\renewcommand{\tabcolsep}{1.2mm}
\resizebox{0.485\linewidth}{!}{
\footnotesize
\begin{tabular}{@{}
L{1.7cm} 
!{\color{gray}\vrule} cc
!{\color{gray}\vrule} cc
!{\color{gray}\vrule} cc
@{}}
\Xhline{1pt}
\multicolumn{1}{r!{\color{gray}\vrule}}{Initialization $\rightarrow$} & \multicolumn{2}{c!{\color{gray}\vrule}}{ImageNet} & \multicolumn{2}{c!{\color{gray}\vrule}}{300M} &
\multicolumn{2}{c}{{\footnotesize ImageNet+300M}}\\
& \scriptsize AP@.5 & \scriptsize AP &  \scriptsize AP@.5 & \scriptsize AP & \scriptsize AP@.5 & \scriptsize AP \\ 
\Xhline{0.5pt}
person & 71.5 & 47.7 & 73.1 & 49.8 & 72.7 & 49.9 \\
bicycle & 48.9 & 26.4 & 54.9 & 30.0 & 52.7 & 29.9 \\
car & 55.7 & 34.7 & 58.3 & 36.9 & 59.3 & 37.1 \\
motorcycle & 56.5 & 36.7 & 61.6 & 40.5 & 59.9 & 39.6 \\
airplane & 67.9 & 52.0 & 70.1 & 55.0 & 70.4 & 54.7 \\
bus & 77.7 & 62.5 & 79.5 & 64.6 & 79.0 & 64.2 \\
train & 66.8 & 59.2 & 69.7 & 62.8 & 69.7 & 62.1 \\
truck & 46.3 & 29.9 & 49.7 & 33.0 & 52.2 & 34.5 \\
boat & 30.6 & 19.4 & 32.5 & 22.1 & 32.1 & 22.3 \\
traffic light & 48.9 & 22.7 & 49.8 & 24.3 & 49.1 & 24.6 \\
fire hydrant & 75.3 & 59.1 & 74.4 & 59.3 & 74.9 & 59.5 \\
stop sign & 83.2 & 63.6 & 84.4 & 63.8 & 85.6 & 66.4 \\
parking meter & 62.2 & 37.5 & 64.9 & 38.5 & 64.5 & 37.6 \\
bench & 38.1 & 19.6 & 39.3 & 20.1 & 40.6 & 21.4 \\
bird & 60.2 & 29.4 & 61.9 & 33.0 & 63.3 & 34.2 \\
cat & 64.2 & 58.1 & 68.0 & 61.9 & 67.9 & 62.4 \\
dog & 62.6 & 52.9 & 66.1 & 56.2 & 66.9 & 57.3 \\
horse & 67.2 & 53.5 & 70.8 & 57.0 & 71.3 & 57.0 \\
sheep & 64.4 & 43.6 & 64.8 & 45.4 & 66.7 & 46.3 \\
cow & 70.7 & 45.4 & 71.9 & 47.4 & 73.3 & 48.9 \\
elephant & 75.1 & 64.1 & 77.3 & 66.4 & 76.1 & 65.5 \\
bear & 70.5 & 66.9 & 74.5 & 69.8 & 72.7 & 70.0 \\
zebra & 71.0 & 59.3 & 71.5 & 60.4 & 71.3 & 61.0 \\
giraffe & 75.3 & 67.4 & 75.9 & 69.0 & 75.9 & 69.3 \\
backpack & 19.6 & 12.8 & 19.5 & 14.7 & 18.5 & 15.1 \\
umbrella & 46.2 & 28.9 & 50.7 & 32.3 & 50.4 & 32.8 \\
handbag & 14.7 & 9.7 & 13.7 & 10.9 & 16.1 & 12.0 \\
tie & 50.8 & 26.3 & 53.2 & 27.9 & 51.5 & 28.4 \\
suitcase & 40.4 & 26.7 & 44.4 & 30.3 & 46.9 & 32.5 \\
frisbee & 53.4 & 43.8 & 55.3 & 48.6 & 58.6 & 48.3 \\
skis & 1.5 & 18.1 & 3.0 & 20.0 & 2.3 & 20.7 \\
snowboard & 45.7 & 29.3 & 47.0 & 33.3 & 43.9 & 32.1 \\
sports ball & 41.8 & 35.6 & 48.7 & 37.6 & 42.3 & 38.6 \\
kite & 39.4 & 37.5 & 33.9 & 38.9 & 35.9 & 40.0 \\
baseball bat & 8.3 & 23.4 & 6.7 & 25.1 & 9.9 & 27.5 \\
baseball glove & 35.6 & 27.4 & 33.7 & 31.2 & 41.9 & 31.8 \\
skateboard & 42.2 & 40.0 & 48.6 & 44.7 & 49.2 & 44.4 \\
surfboard & 48.5 & 31.1 & 51.7 & 32.8 & 52.4 & 33.9 \\
tennis racket & 53.1 & 42.6 & 55.1 & 44.1 & 55.4 & 45.1 \\
bottle & 61.2 & 28.6 & 61.8 & 30.5 & 61.6 & 30.8 \\
\Xhline{1pt}
\end{tabular}
}
\hspace{0.2cm}
\resizebox{0.485\linewidth}{!}{
\footnotesize
\begin{tabular}{@{}
L{1.7cm} 
!{\color{gray}\vrule} cc
!{\color{gray}\vrule} cc
!{\color{gray}\vrule} cc
@{}}
\Xhline{1pt}
\multicolumn{1}{r!{\color{gray}\vrule}}{Initialization $\rightarrow$} & \multicolumn{2}{c!{\color{gray}\vrule}}{ImageNet} & \multicolumn{2}{c!{\color{gray}\vrule}}{300M} &
\multicolumn{2}{c}{{\footnotesize ImageNet+300M}}\\
& \scriptsize AP@.5 & \scriptsize AP &  \scriptsize AP@.5 & \scriptsize AP & \scriptsize AP@.5 & \scriptsize AP \\ 
\Xhline{0.5pt}
wine glass & 53.8 & 30.2 & 56.3 & 33.3 & 58.7 & 34.7 \\
cup & 64.7 & 32.5 & 67.5 & 35.6 & 68.4 & 35.9 \\
fork & 45.7 & 23.2 & 45.1 & 26.5 & 50.1 & 27.8 \\
knife & 29.9 & 12.8 & 37.1 & 15.7 & 37.2 & 16.4 \\
spoon & 13.0 & 10.0 & 11.4 & 11.7 & 11.6 & 13.3 \\
bowl & 49.4 & 32.1 & 53.6 & 35.4 & 52.2 & 35.4 \\
banana & 38.1 & 18.7 & 39.8 & 20.4 & 40.0 & 21.1 \\
apple & 49.4 & 19.1 & 50.1 & 20.8 & 51.5 & 21.7 \\
sandwich & 44.0 & 29.6 & 45.2 & 31.3 & 47.8 & 34.1 \\
orange & 48.7 & 25.0 & 50.7 & 26.2 & 49.0 & 26.1 \\
broccoli & 30.6 & 22.9 & 32.5 & 24.8 & 31.9 & 24.6 \\
carrot & 25.9 & 14.0 & 28.6 & 16.1 & 21.5 & 16.4 \\
hot dog & 43.7 & 21.8 & 46.5 & 24.8 & 48.2 & 25.8 \\
pizza & 67.9 & 51.1 & 69.0 & 52.3 & 68.7 & 52.8 \\
donut & 60.2 & 40.1 & 64.8 & 43.9 & 66.8 & 46.4 \\
cake & 42.7 & 25.5 & 46.4 & 28.1 & 46.5 & 29.1 \\
chair & 33.0 & 21.1 & 36.7 & 24.0 & 35.9 & 24.4 \\
couch & 41.3 & 36.2 & 44.5 & 38.9 & 44.9 & 39.4 \\
potted plant & 25.6 & 20.1 & 27.3 & 21.9 & 30.0 & 23.4 \\
bed & 44.5 & 40.6 & 45.6 & 41.7 & 47.2 & 43.4 \\
dining table & 33.9 & 25.3 & 36.3 & 27.5 & 36.8 & 27.6 \\
toilet & 61.1 & 54.8 & 61.8 & 56.1 & 63.3 & 57.4 \\
tv & 61.8 & 50.0 & 63.0 & 51.9 & 63.7 & 52.7 \\
laptop & 65.8 & 54.5 & 68.3 & 56.6 & 68.9 & 57.5 \\
mouse & 72.1 & 44.4 & 72.0 & 47.6 & 75.6 & 47.3 \\
remote & 56.4 & 22.1 & 55.8 & 24.4 & 59.1 & 26.0 \\
keyboard & 57.1 & 45.4 & 57.5 & 45.9 & 61.4 & 48.3 \\
cell phone & 54.0 & 23.4 & 58.5 & 26.1 & 57.5 & 26.7 \\
microwave & 53.9 & 50.3 & 53.7 & 50.5 & 58.7 & 53.1 \\
oven & 40.9 & 31.7 & 41.9 & 33.5 & 43.2 & 34.6 \\
toaster & 32.6 & 14.7 & 39.9 & 20.5 & 32.9 & 20.1 \\
sink & 43.2 & 31.0 & 44.8 & 34.4 & 44.0 & 33.9 \\
refrigerator & 48.6 & 42.3 & 51.7 & 44.6 & 52.4 & 46.1 \\
book & 15.2 & 7.4 & 18.7 & 8.8 & 21.3 & 9.8 \\
clock & 56.7 & 43.7 & 56.7 & 45.3 & 55.8 & 45.1 \\
vase & 57.1 & 32.3 & 61.5 & 35.9 & 61.4 & 36.5 \\
scissors & 31.1 & 20.8 & 38.9 & 25.2 & 34.8 & 25.9 \\
teddy bear & 50.4 & 35.4 & 54.7 & 40.2 & 54.7 & 40.4 \\
hair drier & 2.3 & 1.0 & 4.8 & 1.8 & 4.0 & 1.9 \\
toothbrush & 48.5 & 34.3 & 50.7 & 36.7 & 51.2 & 37.4 \\
\Xhline{1pt}
\end{tabular}
}
\label{tab:per_category}
\end{table*}

%
%
\end{document}

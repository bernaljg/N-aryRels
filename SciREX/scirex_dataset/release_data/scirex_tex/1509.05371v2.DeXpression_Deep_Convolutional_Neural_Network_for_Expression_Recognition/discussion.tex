\section{Discussion}
\label{sec:discussion}

The accuracy on the CKP set shows that the chosen approach is robust, misclassification usually occurs on pictures which are the first few instances of an emotion sequence. Often a neutral facial expression is depicted in those frames. Thus those misclassifications are not necessarily an error in the approach, but in the data selection. Other than that no major problem could be detected. The emotion \textit{Surprise} is often confused with \textit{Disgust} with a rate of 0.045\% which is the highest. Of those images, where an emotion is present, only few are wrongly classified.\\ 


As there is no consent for the misclassified images, they cannot be depicted here. However some unique names are provided. \\
Image S119\_001\_00000010 is classified as \textit{Fear} while the annotated emotion corresponds to \textit{Surprise}. The image depicts a person with a wide open mouth and open eyes. Pictures representing \textit{Surprise} are often very similar, since the persons also have wide open mouths and eyes. In image S032\_004\_00000014 the targeted label \textit{Fear} is confused with \textit{Anger}. While the mouth region in pictures with \textit{Anger} differ, the eye regions are alike, since in both situations the eyes and eyebrows are contracted.\\
Similar effects are experienced when dealing with the MMI Dataset. Since the first two frames are discarded most pictures with neutral positions are excluded. In few images a neutral position can still be found which gives rise to errors. For the same reason as the CKP set images will not be displayed. Due to the approach to extract images of the videos, a unique identifier for the misclassified image cannot be provided.\\
The top confusions are observed for \textit{Fear} and \textit{Surprise} with a rate of 0.0159\% where \textit{Fear} is wrongly misclassified as \textit{Surprise}. Session 1937 shows a woman displaying \textit{Fear} but it is classified as \textit{Surprise}. Both share common features like similar eye and mouth movement. In both emotions, participants move the head slightly backwards. This can be identified by wrinkled skin. The second most confusion rate, \textit{Surprise} being mistaken as \textit{Sadness}, is mostly based on neutral position images. Although the first two images are not used, some selected frames still do not contain an emotion. In Session 1985 \textit{Surprise} is being mistaken as \textit{Sadness}. The image depicts a man with his mouth being slightly curved, making him look sad.\\

DeXpression extracts features and uses them to classify images, but in very few cases the emotions are confused. This happens, as discussed, usually in pictures depicting no emotion. DeXpression performs very well on both tested sets, if an emotion is present.


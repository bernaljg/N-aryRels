 %
% File naacl2019.tex
%
%% Based on the style files for ACL 2018 and NAACL 2018, which were
%% Based on the style files for ACL-2015, with some improvements
%%  taken from the NAACL-2016 style
%% Based on the style files for ACL-2014, which were, in turn,
%% based on ACL-2013, ACL-2012, ACL-2011, ACL-2010, ACL-IJCNLP-2009,
%% EACL-2009, IJCNLP-2008...
%% Based on the style files for EACL 2006 by 
%%e.agirre@ehu.es or Sergi.Balari@uab.es
%% and that of ACL 08 by Joakim Nivre and Noah Smith

\documentclass[11pt,a4paper]{article}
\usepackage[hyperref]{acl2019}
\usepackage{acl2019}
\usepackage{times}
\usepackage{latexsym}
\usepackage{url}
\usepackage{multirow}
\usepackage{adjustbox}
\usepackage[ruled]{algorithm2e} %alog
\usepackage{amssymb}% http://ctan.org/pkg/amssymb
\usepackage{pifont}% http://ctan.org/pkg/pifont

\usepackage{url}

\usepackage{mwe}
\usepackage{amsmath}
\usepackage{amssymb}
\usepackage{wasysym}
\usepackage{bbm}
\usepackage{todonotes} % insert [disable] to disable all notes
%% figure
\usepackage{graphicx}
\usepackage{caption}
\usepackage[caption=false]{subfig}
\newcommand{\cmark}{\ding{51}}%
\newcommand{\xmark}{\ding{55}}%
\newcommand{\xiaodl}[2][]{\todo[color=yellow,size=\scriptsize,fancyline,caption={},#1]{Xiaodong:#2}} 


\aclfinalcopy % Uncomment this line for the final submission
\def\aclpaperid{1963} %  Enter the acl Paper ID here

%\setlength\titlebox{5cm}
% You can expand the titlebox if you need extra space
% to show all the authors. Please do not make the titlebox
% smaller than 5cm (the original size); we will check this
% in the camera-ready version and ask you to change it back.


\newcommand\BibTeX{B{\sc ib}\TeX}
\DeclareGraphicsExtensions{.pdf,.png,.jpg}
\newcommand\MNAME{MT-DNN}

% For our own commenting purposes
\newcommand{\JG}[1]{{\color{blue}{[{\bf JG}: #1]}}}

\newcommand{\WZ}[1]{{\color{red}{[{\bf WZ}: #1]}}}

% \title{Large-scale Multi-task Learning for Natural Language Understanding}
\title{Multi-Task Deep Neural Networks for Natural Language Understanding}

\author{Xiaodong Liu\thanks{~~Equal Contribution.}~$^1$, Pengcheng He$^\bold{\ast}$$^2$, Weizhu Chen$^2$, Jianfeng Gao$^1$ \\
  $^1$ Microsoft Research~~~~~~~~
  $^2$ Microsoft Dynamics 365 AI \\
  {\tt \{xiaodl,penhe,wzchen,jfgao\}@microsoft.com}
}
\date{}

\begin{document}
\maketitle

\begin{abstract}
%In this paper, we propose a multi-task learning framework to handle with multiple Natural Language Processing (NLP) tasks based on the pre-trained BERT models. The unique property of our framework is that we train the model jointly. Experiments on the GLUE benchmark show that our proposed approach beat the current state-of-the-art large BERT and set a new state-of-the-art results with a smaller parameter size.
In this paper, we present a Multi-Task Deep Neural Network (MT-DNN) for learning representations across multiple natural language understanding (NLU) tasks. MT-DNN not only leverages large amounts of cross-task data, but also benefits from a regularization effect that leads to more general representations to help adapt to new tasks and domains. MT-DNN extends the model proposed in \citet{liu2015mtl} by incorporating a pre-trained bidirectional transformer language model, known as BERT \citep{bert2018}. 
%MT-DNN in this study combines tasks of multi-label single-sentence classification, pairwise text classification, text similarity scoring and relevance ranking, and is easy to extend to incorporate new tasks.
MT-DNN obtains new state-of-the-art results on ten NLU tasks, including SNLI, SciTail, and eight out of nine GLUE tasks, pushing the GLUE benchmark to 82.7\% (2.2\% absolute improvement) \footnote{As of February 25, 2019 on the latest GLUE test set.}. 
We also demonstrate using the SNLI and SciTail datasets that the representations learned by MT-DNN allow domain adaptation with substantially fewer in-domain labels than the pre-trained BERT representations.
%outperforms BERT in a set of domain adaptation experiments on ,    
The code and pre-trained models are publicly available at https://github.com/namisan/mt-dnn.
\end{abstract}

\section{Introduction}
\label{sec:intro}

Language modeling is among the important problems that require modeling long-term dependency, with successful applications such as unsupervised pretraining~\citep{dai2015semi,peters2018deep,radford2018improving,devlin2018bert}.
However, it has been a challenge to equip neural networks with the capability to model long-term dependency in sequential data.
Recurrent neural networks (RNNs), in particular Long Short-Term Memory (LSTM) networks~\citep{hochreiter1997long}, have been a standard solution to language modeling and obtained strong results on multiple benchmarks.
Despite the wide adaption, RNNs are difficult to optimize due to gradient vanishing and explosion~\citep{hochreiter2001gradient}, and the introduction of gating in LSTMs and the gradient clipping technique~\citep{graves2013generating} might not be sufficient to fully address this issue.
% ,pascanu2012understanding
Empirically, previous work has found that LSTM language models use 200 context words on average~\citep{khandelwal2018sharp}, indicating room for further improvement.

On the other hand, the direct connections between long-distance word pairs baked in attention mechanisms might ease optimization and enable the learning of long-term dependency~\citep{bahdanau2014neural,vaswani2017attention}.
Recently, \citet{al2018character} designed a set of auxiliary losses to train deep Transformer networks for character-level language modeling, which outperform LSTMs by a large margin.
Despite the success, the LM training in~\citet{al2018character} is performed on separated fixed-length segments of a few hundred characters, without any information flow across segments.
As a consequence of the fixed context length, the model cannot capture any longer-term dependency beyond the predefined context length.
In addition, the fixed-length segments are created by selecting a consecutive chunk of symbols without respecting the sentence or any other semantic boundary.
Hence, the model lacks necessary contextual information needed to well predict the first few symbols, leading to inefficient optimization and inferior performance.
We refer to this problem as \textit{context fragmentation}.

%However, the context length is fixed to hundreds of characters and thus it is not possible to model longer-term dependency. Moreover, it is not clear how the model performs on word-level language modeling data, as the granularity changes.

% Moreover, using auxiliary losses brings additional challenges such as properly tuning the mixture weights and the loss decay schedule.

To address the aforementioned limitations of fixed-length contexts, we propose a new architecture called Transformer-XL (meaning extra long).
We introduce the notion of recurrence into our deep self-attention network. In particular, instead of computing the hidden states from scratch for each new segment, we reuse the hidden states obtained in previous segments.
The reused hidden states serve as memory for the current segment, which builds up a recurrent connection between the segments.
As a result, modeling very long-term dependency becomes possible because information can be propagated through the recurrent connections.
Meanwhile, passing information from the previous segment can also resolve the problem of context fragmentation.
More importantly, we show the necessity of using relative positional encodings rather than absolute ones, in order to enable state reuse without causing temporal confusion.
Hence, as an additional technical contribution, we introduce a simple but more effective relative positional encoding formulation that generalizes to attention lengths longer than the one observed during training.

Transformer-XL obtained strong results on five datasets, varying from word-level to character-level language modeling.
Transformer-XL is also able to generate relatively coherent long text articles with \textit{thousands of} tokens (see Appendix \ref{sec:gen}), trained on only 100M tokens.
% Transformer-XL improves the previous state-of-the-art (SoTA) results from 1.06 to 0.99 in bpc on enwiki8, from 1.13 to 1.08 in bpc on text8, from 20.5 to 18.3 in perplexity on WikiText-103, and from 23.7 to 21.8 in perplexity on One Billion Word.
% Transformer-XL improves the previous state-of-the-art (SoTA) results to 0.99 in bpc on enwiki8, 1.08 in bpc on text8, 18.3 in perplexity on WikiText-103, and 21.8 in perplexity on One Billion Word.
% On small data, Transformer-XL also achieves a perplexity of 54.5 on Penn Treebank without finetuning, which is SoTA when comparable settings are considered.

Our main technical contributions include introducing the notion of recurrence in a purely self-attentive model and deriving a novel positional encoding scheme. These two techniques form a complete set of solutions, as any one of them alone does not address the issue of fixed-length contexts. Transformer-XL is the first self-attention model that achieves substantially better results than RNNs on both character-level and word-level language modeling.

% On WikiText-103, Transformer-XL improves the previous state-of-the-art (SoTA) results from 33 perplexity to 24, with a relative reduction of 27\%. On enwiki8 character-level language modeling, Transformer-XL achieves a SoTA bpc of 1.03, which outperforms \cite{al2018character} by 0.03 with 60+\% fewer parameters. Given a more common model size with 40+M parameters, Transformer-XL achieves a bpc of 1.06, compared to 1.11 by \cite{al2018character}. Transformer-XL also achieves perplexities of 54.5 on Penn Treebank and 29.4 on One Billion Word, which are SoTA when comparable settings are considered.

% Due to the ability of modeling long-range context, our best model uses attention lengths of 1,600 and 3,800 on WikiText-103 and enwiki8 respectively. We also devise a metric called \textit{Relative Effective Context Length} (RECL) that aims to fairly compare the ability of long-range dependency modeling.
% % perform a fair comparison of the gains brought by increasing the context lengths for different models.
% In this setting, Transformer-XL learns a RECL of 900 words on WikiText-103, while the numbers for recurrent networks and Transformer are only 500 and 128.

% We use two methods to quantitatively study the effective lengths of Transformer-XL and the baselines. Similar to \cite{khandelwal2018sharp}, we gradually increase the attention length at test time until no further noticeable improvement ($\sim$0.1\% relative gains) can be observed. Our best model in this settings use attention lengths of 1,600 and 3,800 on WikiText-103 and enwiki8 respectively.
% %In addition, since the effective context length of Transformer-XL can be longer than the attention length due to our recurrent formulation, we devise a metric called \textit{Relative Effective Context Length} (RECL) that aims to perform a fair comparison of the gains brought by increasing the context lengths for different models.
% In addition, we devise a metric called \textit{Relative Effective Context Length} (RECL) that aims to perform a fair comparison of the gains brought by increasing the context lengths for different models.
% In this setting, Transformer-XL learns a RECL of 900 words on WikiText-103, while the numbers for recurrent networks and Transformer are only 500 and 128.


\section{Tasks}
\label{sec:tasks}

The MT-DNN model combines four types of NLU tasks: single-sentence classification, pairwise text classification, text similarity scoring, and relevance ranking. For concreteness, we describe them using the NLU tasks defined in the GLUE benchmark as examples. 
%The GLUE tasks are summarized in Table 1.

\paragraph{Single-Sentence Classification:} 
Given a sentence\footnote{In this study, a sentence can be an arbitrary span of contiguous text or word sequence, rather than a linguistically plausible sentence.}, the model labels it using one of the pre-defined class labels. For example, the \textbf{CoLA} task  is to predict whether an English sentence is grammatically plausible. The \textbf{SST-2} task is to determine whether the sentiment of a sentence extracted from movie reviews is positive or negative.

\paragraph{Text Similarity:}
This is a regression task. Given a pair of sentences, the model predicts a real-value score indicating the semantic similarity of the two sentences. \textbf{STS-B} is the only example of the task in GLUE. 

\paragraph{Pairwise Text Classification:} 
Given a pair of sentences, the model determines the relationship of the two sentences based on a set of pre-defined labels. 
For example, both \textbf{RTE} and \textbf{MNLI} are language inference tasks, where the goal is to predict whether a sentence is an \emph{entailment}, \emph{contradiction}, or \emph{neutral} with respect to the other. 
% \textbf{WNLI} is a natural language inference task that requires commonsense reasoning. But as noted in the GLUE webpage \footnote{https://gluebenchmark.com/faq}, there are issues in the dataset, and every submitted system has performed same or worse than the majority voting baseline whose accuracy is 65.1. MT-DNN also achieves the accuracy of 65.1, making WNLI the only GLUE task where MT-DNN does not create a new state of the art result.
\textbf{QQP} and \textbf{MRPC} are paraphrase datasets that consist of sentence pairs. The task is to predict whether the sentences in the pair are semantically equivalent.

\paragraph{Relevance Ranking:}
Given a query and a list of candidate answers, the model ranks all the candidates in the order of relevance to the query. 
\textbf{QNLI} is a version of Stanford Question Answering Dataset \citep{rajpurkar2016squad}. 
The task involves assessing whether a sentence contains the correct answer to a given query. 
Although QNLI is defined as a binary classification task in GLUE, in this study we formulate it as a pairwise ranking task, where the model is expected to rank the candidate that contains the correct answer higher than the candidate that does not. 
We will show that this formulation leads to a significant improvement in accuracy over binary classification.

%is derive from the Stanford Question Answering Dataset \citep{rajpurkar2016squad}, and consists  which has been converted to a binary classification task in GLUE. A query-candidate-answer pair is labeled as positive if the candidate contains the correct answer to the query, and negative otherwise. In this study, however, we formulate QNLI as a pairwise ranking task, where the model is expected to rank the candidate that contains the correct answer higher than the candidate that does not. We will show that our formulation leads to a significant improvement in accuracy over binary classification. 

\section{The Proposed MT-DNN Model}
\label{sec:mt-dnn}
\begin{figure*}
	\centering
	\vspace{-1mm}
	% \adjustbox{trim={0.0\width} {0.71\height} {0.\width} {0.01\height},clip}
    {
	\includegraphics[width=0.92\textwidth]{fig/mt-dnn.png}
    }
	%\vspace{-2mm}
	\caption{Architecture of the MT-DNN model for representation learning. The lower layers are shared across all tasks while the top layers are task-specific. The input $X$ (either a sentence or a pair of sentences) is first represented as a sequence of embedding vectors, one for each word, in $l_1$. Then the Transformer encoder captures the contextual information for each word and generates the shared contextual embedding vectors in $l_2$. Finally, for each task, additional task-specific layers generate task-specific representations, followed by operations necessary for classification, similarity scoring, or relevance ranking.}
    %\vspace{-4mm}
	\label{fig:mt-dnn}
\end{figure*}


The architecture of the MT-DNN model is shown in Figure \ref{fig:mt-dnn}. The lower layers are shared across all tasks, while the top layers represent task-specific outputs. The input $X$, which is a word sequence (either a sentence or a pair of sentences packed together) is first represented as a sequence of embedding vectors, one for each word, in $l_1$. Then the transformer encoder captures the contextual information for each word via self-attention, and generates a sequence of contextual embeddings in $l_2$. This is the shared semantic representation that is trained by our multi-task objectives.  In what follows, we elaborate on the model in detail.

\paragraph{Lexicon Encoder ($l_1$):} 
The input $X=\{x_1,...,x_m\}$ is a sequence of tokens of length $m$. Following \citet{bert2018}, the first token $x_1$ is always the \texttt{[CLS]} token. 
If $X$ is packed by a sentence pair $(X_1, X_2)$, we separate the two sentences with a special token \texttt{[SEP]}. The lexicon encoder maps $X$ into a sequence of input embedding vectors, one for each token, constructed by summing the corresponding word, segment, and positional embeddings.

\paragraph{Transformer Encoder ($l_2$):}
We use a multi-layer bidirectional Transformer encoder \citep{vaswani2017attention} to map the input representation vectors ($l_1$) into a sequence of contextual embedding vectors 
$\mathbf{C} \in \mathbb{R}^{d \times m}$. 
This is the shared representation across different tasks. Unlike the BERT model \citep{bert2018} that learns the representation via pre-training,
% and adapts it to each individual task via fine-tuning, 
MT-DNN learns the representation using multi-task objectives, in addition to pre-training.

Below, we will describe the task specific layers using the NLU tasks in GLUE as examples, although in practice we can incorporate arbitrary natural language tasks such as text generation where the output layers are implemented as a neural decoder.

\paragraph{Single-Sentence Classification Output:}
Suppose that $\mathbf{x}$ is the contextual embedding ($l_2$) of the token \texttt{[CLS]}, which can be viewed as the semantic representation of input sentence $X$. Take the SST-2 task as an example. The probability that $X$ is labeled as class $c$ (i.e., the sentiment) is predicted by a logistic regression with softmax:
\begin{equation}
P_r(c|X)= \text{softmax} (\mathbf{W}_{SST}^\top \cdot \mathbf{x}),
\label{eqn:single-sent-classification}
\end{equation}
where $\mathbf{W}_{SST}$ is the task-specific parameter matrix.

\paragraph{Text Similarity Output:}
Take the STS-B task as an example. Suppose that $\mathbf{x}$ is the contextual embedding ($l_2$) of \texttt{[CLS]} which can be viewed as the semantic representation of the input sentence pair $(X_1, X_2)$. We introduce a task-specific parameter vector $\mathbf{w}_{STS}$ to compute the similarity score as: 
\begin{equation}
\text{Sim} (X_1, X_2)= \mathbf{w}_{STS}^\top \cdot \mathbf{x},
\label{eqn:text-sim}
\end{equation}
where $\text{Sim} (X_1, X_2)$ is a real value of the range (-$\infty$, $\infty$).

%where $g(z)= \frac{1}{1+\exp{(-z)}}$ is a sigmoid function that maps the score to a real value of the range $[0, 1]$.

\paragraph{Pairwise Text Classification Output:}
Take natural language inference (NLI) as an example. The NLI task defined here involves a premise $P = (p_1,...,p_m)$ of $m$ words and a hypothesis $H = (h_1,..., h_n)$  of $n$ words, and aims to find a logical relationship $R$ between $P$ and $H$. The design of the output module follows the answer module of the stochastic answer network (SAN) \citep{liu2018san4nli}, a state-of-the-art neural NLI model. SAN's answer module uses multi-step reasoning. Rather than directly predicting the entailment given the input, it maintains a state and iteratively refines its predictions.

The SAN answer module works as follows. We first construct the working memory of premise $P$ by concatenating the contextual embeddings of the words in $P$, which are the output of the transformer encoder, denoted as $\mathbf{M}^p \in \mathbb{R}^{d \times m}$, and similarly the working memory of hypothesis $H$, denoted as $\mathbf{M}^h \in \mathbb{R}^{d \times n}$. 
Then, we perform $K$-step reasoning on the memory to output the relation label, where $K$ is a hyperparameter.
At the beginning,  the initial state $\mathbf{s}^0$ is the summary of $\mathbf{M}^h$: 
$\mathbf{s}^0 = \sum_j \alpha_j\mathbf{M}_j^h$, 
where $\alpha_j = \frac {\exp(\mathbf{w}_1^\top \cdot \mathbf{M}_j^h)} {\sum_i \exp(\mathbf{w}_1^\top \cdot \mathbf{M}_i^h)}$. 
At time step $k$ in the range of $\{1,2,…,K-1\}$, the state is defined by 
$\mathbf{s}^k = \text{GRU} (\mathbf{s}^{k-1}, \mathbf{x}^k)$. 
Here, $\mathbf{x}^k$ is computed from the previous state $\mathbf{s}^{k-1}$ and memory $\mathbf{M}^p$: $\mathbf{x}^k = \sum_j \beta_j\mathbf{M}_j^p$ and $\beta_j = \text{softmax} (\mathbf{s}^{k-1} \mathbf{W}_2^\top \mathbf{M}^p )$. 
A one-layer classifier is used to determine the relation at each step $k$:
\begin{equation}
P_r^k = \text{softmax} (\mathbf{W}_3^\top [\mathbf{s}^k ; \mathbf{x}^k ; |\mathbf{s}^k - \mathbf{x}^k|; \mathbf{s}^k \cdot \mathbf{x}^k ]).
\label{eqn:pairwise-text-classification}
\end{equation}
 
At last, we utilize all of the $K$ outputs by averaging the scores:
\begin{equation}
P_r = \text{avg} ([P_r^0, P_r^1, ..., P_r^{K-1}]).
\label{eqn:pairwise-text-classification-avg}
\end{equation}

Each $P_r$ is a probability distribution over all the relations $R \in \mathcal{R}$. 
During training, we apply \emph{stochastic prediction dropout} \citep{liu2018san} before the above averaging operation. 
During decoding, we average all outputs to improve robustness.

\paragraph{Relevance Ranking Output:}
Take QNLI as an example. Suppose that $\mathbf{x}$ is the contextual embedding vector of \texttt{[CLS]} which is the semantic representation of a pair of question and its candidate answer $(Q, A)$. 
%We introduce a task-specific parameter vector $\mathbf{w}_{QNLI}$ to compute the relevance score as: 
We compute the relevance score as: 
\begin{equation}
\text{Rel} (Q, A)= g (\mathbf{w}_{QNLI}^\top \cdot \mathbf{x}),
\label{eqn:rel-score}
\end{equation}
For a given $Q$, we rank all of its candidate answers based on their relevance scores computed using Equation \ref{eqn:rel-score}.

\subsection{The Training Procedure}

The training procedure of MT-DNN consists of two stages: pretraining and multi-task learning. The pretraining stage follows that of the BERT model \citep{bert2018}. The parameters of the lexicon encoder and Transformer encoder are learned using two unsupervised prediction tasks: masked language modeling and next sentence prediction.\footnote{In this study we use the pre-trained BERT models released by the authors.}

In the multi-task learning stage, we use mini-batch based stochastic gradient descent (SGD) to learn the parameters of our model (i.e., the parameters of all shared layers and task-specific layers) as shown in Algorithm \ref{algo:mtdnn}.  In each epoch, a mini-batch $b_t$ is selected(e.g., among all 9 GLUE tasks), and the model is updated according to the task-specific objective for the task $t$. This approximately optimizes the sum of all multi-task objectives. 
\begin{algorithm}[ht!]
 \SetAlgoLined
Initialize model parameters $\Theta$ randomly.  \\
Pre-train the shared layers (i.e., the lexicon encoder and the transformer encoder). \\
Set the max number of epoch: $epoch_{max}$.
\textit{//Prepare the data for $T$ tasks.}\\
\For{$t$ in $1,2,...,T$ }
{
    Pack the dataset $t$ into mini-batch: $D_t$.
}

 \For{$epoch$ in $1,2,...,epoch_{max}$}{
     1. Merge all the datasets: $D =D_1 \cup D_2 ... \cup D_T$ \\
     2. Shuffle $D$ \\
     \For{$b_t$ in D}{
        \textit{//$b_t$ is a mini-batch of task $t$.} \\
        % Note that if the task $t$ is a classification task, Equation~\ref{eqn:cross-entropy-loss} is used; if the task $t$ is a regression task, Equation~\ref{eqn:msq-loss} is used; and if the task $t$ is a ranking task, Eq~\ref{eqn:ranking-loss} is used.}
     3. Compute loss : $L(\Theta)$ \\
        \hspace{0.3cm} $L(\Theta)=$ Eq.~\ref{eqn:cross-entropy-loss} for classification \\
        \hspace{0.3cm} $L(\Theta)=$ Eq.~\ref{eqn:msq-loss} for regression \\
        \hspace{0.3cm} $L(\Theta)=$ Eq.~\ref{eqn:ranking-loss} for ranking \\
     4. Compute gradient: $\nabla(\Theta)$ \\
     5. Update model: $\Theta = \Theta - \epsilon \nabla(\Theta)$ \\
     }
 }
 \caption{\label{algo:mtdnn} Training a MT-DNN model.}
%\vspace{-0.3cm} 
 % \algorithmfootnote{Note that $\Theta$ denotes the model parameters and T is the number of tasks.}
\end{algorithm} %\vspace{-0.2cm} 

% \begin{algorithm}[ht!]
%  \SetAlgoLined
% Initialize model parameters $\Theta$ randomly  \\
% %Set M \quad\textit{//the number of updates for the shared layer} \\
% %\textit{Counter} = 0\\
%  \For{$iteration$ in $0 ... \infty$}{
%  	 %1. \textit{Counter} += 1\\
%      1. Pick a task $t$ randomly \\
%      2. Pick sample(s) from task $t$, i.e., \\
%      \hspace{0.4cm}$(Q,C=\{0,1\})$ for classification \\
%      \hspace{0.4cm}$(Q, D)$ for ranking\\
%      3. Compute loss: $L(\Theta)$, i.e.,\\
%      \hspace{0.4cm} the \textit{cross-entropy} for classification \\
%      \hspace{0.4cm} the ranking loss for ranking\cite{learning-to-rank2005burges}\\

%      4. Compute gradient: $\nabla(\Theta)$ \\
%      5. Update model: $\Theta = \Theta - \epsilon \nabla(\Theta)$ \quad\textit{}
%      % \eIf{Counter $<$ M}{
%   	 %5. Update model: $\Theta = \Theta - \epsilon \nabla(\Theta)$ \quad\textit{//update both $\Theta^s$ and $\Theta^t$} \\
%   %}{
%   	% 6. Update model: $\Theta^t = \Theta^t - \epsilon \nabla(\Theta^t)$ 
%   %}
%  }
%  \caption{\label{algo:mtdnn} Training a MT-DNN model.}
%  \algorithmfootnote{Note that $\Theta$ denotes the model parameters. \textcolor{red}{TODO: update alg based on task defination.}}
% \end{algorithm}

For the classification tasks (i.e., single-sentence or pairwise text classification), we use the cross-entropy loss as the objective:
\begin{equation}
-\sum_c \mathbbm{1}(X,c) \log(P_r(c|X)),
\label{eqn:cross-entropy-loss}
\end{equation}
where $\mathbbm{1}(X,c)$ is the binary indicator (0 or 1) if class label $c$ is the correct classification for $X$, and $P_r(.)$ is defined by e.g., Equation \ref{eqn:single-sent-classification} or \ref{eqn:pairwise-text-classification-avg}.

For the text similarity tasks, such as STS-B, where each sentence pair is annotated with a real-valued score $y$, we use the mean squared error as the objective:
\begin{equation}
(y - \text{Sim}(X_1, X_2))^2,
\label{eqn:msq-loss}
\end{equation}
where $\text{Sim}(.)$ is defined by Equation \ref{eqn:text-sim}.

The objective for the relevance ranking tasks follows the pairwise learning-to-rank paradigm \citep{learning-to-rank2005burges,huang2013dssm}. Take QNLI as an example. Given a query $Q$, we obtain a list of candidate answers $\mathcal{A}$ which contains a positive example $A^+$ that includes the correct answer, and $|\mathcal{A}|-1$ negative examples. We then minimize the negative log likelihood of the positive example given queries across the training data
\begin{equation}
-\sum_{(Q,A^+)} P_r(A^+ | Q),
\label{eqn:ranking-loss}
\end{equation}
%where the probability of a given answer $A^+$ is computed as
\begin{equation}
P_r(A^+ | Q) = \frac{\exp(\gamma \text{Rel}(Q,A^+))}{\sum_{A^{'} \in \mathcal{A}} \exp(\gamma \text{Rel}(Q,A^{'}))},
\label{eqn:ranking-prob}
\end{equation}
where $\text{Rel}(.)$ is defined by Equation \ref{eqn:rel-score} and $\gamma$ is a tuning factor determined on held-out data. In our experiment, we simply set $\gamma$ to 1.

%Similar to the BERT model, we can adapt the trained MT-DNN to any new task via fine-tuning on task-specific data. 
%After the multi-task training, we could directly use the multi-task model to perform prediction on each task. Alternative, this model can be further fine-tuned using task-specific data for each individual task.  The detailed empirical study of these two approaches will be provided in Section \ref{sec:exp}.


% !TEX root = ../multi_task.tex

We evaluate the presented MTL method on a number of problems. First, we use MultiMNIST \citep{multi_mnist}, an MTL adaptation of MNIST \citep{mnist}. Next, we tackle multi-label classification on the CelebA dataset \citep{celeba} by considering each label as a distinct binary classification task. These problems include both classification and regression, with the number of tasks ranging from 2 to 40. Finally, we experiment with scene understanding, jointly tackling the tasks of semantic segmentation, instance segmentation, and depth estimation on the Cityscapes dataset \citep{cityscapes}. We discuss each experiment separately in the following subsections.

The baselines we consider are (i) \textbf{uniform scaling:} minimizing a uniformly weighted sum of loss functions \mbox{$\frac{1}{T}\sum_t \lL^t$}, \mbox{(ii) \textbf{single task:}} solving tasks independently, \mbox{(iii) \textbf{grid search:}} exhaustively trying various values from $\{ c^t \in [0,1] | \sum_t c^t = 1\}$ and optimizing for $\frac{1}{T}\sum_t c^t \lL^t$, \mbox{(iv) \textbf{\citet{Kendall2018}:}} using the uncertainty weighting proposed by \citet{Kendall2018}, and \mbox{(v) \textbf{GradNorm:}} using the normalization proposed by \citet{Chen2018}.



\subsection{MultiMNIST}
\label{sec:multi_mnist_exp}

Our initial experiments are on MultiMNIST, an MTL version of the MNIST dataset \citep{multi_mnist}. In order to convert digit classification into a multi-task problem, \citet{multi_mnist} overlaid multiple images together. We use a similar construction. For each image, a different one is chosen uniformly in random. Then one of these images is put at the top-left and the other one is at the bottom-right. The resulting tasks are: classifying the digit on the top-left (task-L) and classifying the digit on the bottom-right (task-R). We use 60K examples and directly apply existing single-task MNIST models. The MultiMNIST dataset is illustrated in the supplement.

We use the LeNet architecture \citep{mnist}. We treat all layers except the last as the representation function $g$ and put two fully-connected layers as task-specific functions (see the supplement for details). We visualize the performance profile as a scatter plot of accuracies on task-L and task-R in Figure~\ref{fig:multi_mnist_performance_curve}, and list the results in Table~\ref{tab:multi_mnist}.

In this setup, any static scaling results in lower accuracy than solving each task separately (the single-task baseline). The two tasks appear to compete for model capacity, since increase in the accuracy of one task results in decrease in the accuracy of the other. Uncertainty weighting \citep{Kendall2018} and GradNorm \citep{Chen2018} find solutions that are slightly better than grid search but distinctly worse than the single-task baseline. In contrast, our method finds a solution that efficiently utilizes the model capacity and yields accuracies that are as good as the single-task solutions. This experiment demonstrates the effectiveness of our method as well as the necessity of treating MTL as multi-objective optimization. Even after a large hyper-parameter search, \emph{any} scaling of tasks does not approach the effectiveness of our method.



\subsection{Multi-Label Classification}

\begin{figure}[t]
\includegraphics[width=\textwidth]{radar_full_new}
\vspace{1mm}
\caption{Radar charts of percentage error per attribute on CelebA \citep{celeba}. Lower is better. We divide attributes into two sets for legibility: easy on the left, hard on the right. Zoom in for details.}
\label{fig:multi_label_radar}
\end{figure}


\begin{wraptable}{r}{0.3\textwidth}
%\vspace{-4mm}
\captionof{table}{Mean of error per category of MTL algorithms in multi-label classification on CelebA \citep{celeba}.}
\begin{tabular}{r@{\hspace{2mm}}c@{}}
\toprule
& Average  \\
&  error \\
\midrule
Single task & $8.77$ \\
Uniform scaling & $9.62$ \\
\citealt{Kendall2018} & $9.53$ \\
GradNorm & $8.44$ \\
Ours & $\mathbf{8.25}$  \\
\bottomrule
\end{tabular}
\label{table:multi_label_bar}
%\vspace{-5mm}
\end{wraptable}

Next, we tackle multi-label classification. Given a set of attributes, multi-label classification calls for deciding whether each attribute holds for the input. We use the CelebA dataset \citep{celeba}, which includes 200K face images annotated with 40 attributes. Each attribute gives rise to a binary classification task and we cast this as a 40-way MTL problem. We use ResNet-18 \citep{resnet} without the final layer as a shared representation function, and attach a linear layer for each attribute (see the supplement for further details).


We plot the resulting error for each binary classification task as a radar chart in Figure~\ref{fig:multi_label_radar}. The average over them is listed in Table~\ref{table:multi_label_bar}. We skip grid search since it is not feasible over 40 tasks. Although uniform scaling is the norm in the multi-label classification literature, single-task performance is significantly better. Our method outperforms baselines for significant majority of tasks and achieves comparable performance in rest. This experiment also shows that our method remains effective when the number of tasks is high.


\subsection{Scene Understanding}

To evaluate our method in a more realistic setting, we use scene understanding. Given an RGB image, we solve three tasks: semantic segmentation (assigning pixel-level class labels), instance segmentation (assigning pixel-level instance labels), and monocular depth estimation (estimating continuous disparity per pixel). We follow the experimental procedure of \citet{Kendall2018} and use an encoder-decoder architecture. The encoder is based on ResNet-50 \citep{resnet} and is shared by all three tasks. The decoders are task-specific and are based on the pyramid pooling module \citep{pspnet} (see the supplement for further implementation details).

Since the output space of instance segmentation is unconstrained (the number of instances is not known in advance), we use a proxy problem as in \citet{Kendall2018}. For each pixel, we estimate the location of the center of mass of the instance that encompasses the pixel. These center votes can then be clustered to extract the instances. In our experiments, we directly report the MSE in the proxy task. Figure~\ref{fig:cityscapes_performance_profile} shows the performance profile for each pair of tasks, although we perform all experiments on all three tasks jointly. The pairwise performance profiles shown in Figure~\ref{fig:cityscapes_performance_profile} are simply 2D projections of the three-dimensional profile, presented this way for legibility. The results are also listed in Table~\ref{tab:cityscapes_results}.

MTL outperforms single-task accuracy, indicating that the tasks cooperate and help each other. Our method outperforms all baselines on all tasks.


\subsection{Role of the Approximation}

In order to understand the role of the approximation proposed in Section~\ref{sec:approximation}, we compare the final performance and training time of our algorithm with and without the presented approximation in Table~\ref{tab:approximation_tradeoff} (runtime measured on a single Titan Xp GPU). For a small number of tasks (3 for scene understanding), training time is reduced by 40\%. For the multi-label classification experiment (40 tasks), the presented approximation accelerates learning by a factor of 25.

On the accuracy side, we expect both methods to perform similarly as long as the full-rank assumption is satisfied. As expected, the accuracy of both methods is very similar. Somewhat surprisingly, our approximation results in slightly improved accuracy in all experiments. While counter-intuitive at first, we hypothesize that this is related to the use of SGD in the learning algorithm. Stability analysis in convex optimization suggests that if gradients are computed with an error $\hat{\nabla}_\btheta \mathcal{L}^t = \nabla_\btheta \mathcal{L}^t + \mathbf{e}^t$ ($\btheta$ corresponds to $\btheta^{sh}$ in (\ref{eq:kkt_opt})), as opposed to $\mathbf{Z}$ in the approximate problem in \ref{eq:approx}, the error in the solution is bounded as $\|\hat{\mathbf{\alpha}} - \mathbf{\alpha} \|_2 \leq \mathcal{O}(\max_t \|\mathbf{e}^t\|_2)$. Considering the fact that the gradients are computed over the full parameter set (millions of dimensions) for the original problem and over a smaller space for the approximation (batch size times representation which is in the thousands), the dimension of the error vector is significantly higher in the original problem. We expect the $l_2$ norm of such a random vector to depend on the dimension.

In summary, our quantitative analysis of the approximation suggests that (i) the approximation does not cause an accuracy drop and (ii) by solving an equivalent problem in a lower-dimensional space, our method achieves both better computational efficiency and higher stability.

  {\small
  \begin{table}[t]
%  \vspace{-4mm}
  \caption{Effect of the MGDA-UB approximation. We report the final accuracies as well as training times for our method with and without the approximation.}
  %\vspace{1mm}
  \centering
  \begin{tabular}{@{}r@{\hspace{3mm}}c@{\hspace{3mm}}c@{\hspace{2mm}}c@{\hspace{2mm}}c@{}c@{\hspace{5mm}}c@{\hspace{2mm}}c@{}}
  \toprule
  & \multicolumn{4}{c}{Scene understanding (3 tasks)} &  & \multicolumn{2}{c}{Multi-label (40 tasks)}  \\
  \cmidrule(r){2-5} \cmidrule(lr){7-8}
                  & Training & Segmentation & Instance  & Disparity      & & Training & Average \\
                 & time     &  mIoU [\%]       & error [px] & error [px] & & time (hour)      & error \\
  \midrule
  Ours (w/o approx.) & $38.6$ & $66.13$ & $10.28$ & $2.59$ & & $429.9$ & $8.33$ \\
  Ours & $\mathbf{23.3}$ & $\mathbf{66.63}$ & $\mathbf{10.25}$ & $\mathbf{2.54}$  & & $\mathbf{16.1}$ & $\mathbf{8.25}$ \\
  \bottomrule
  \end{tabular}
  %\vspace{-2mm}
  \label{tab:approximation_tradeoff}
  \end{table}}


\section{Conclusion}
\label{sec:con}
In this work we proposed a model called MT-DNN to combine multi-task learning and language model pre-training for language representation learning. 
MT-DNN obtains new state-of-the-art results on ten NLU tasks across three popular benchmarks: SNLI, SciTail, and GLUE.
MT-DNN also demonstrates an exceptional generalization capability in domain adaptation experiments. 
%The fewer the training data, the improvement introduced by MT-DNN is larger.

There are many future areas to explore to improve MT-DNN, including a deeper understanding of model structure sharing in MTL, a more effective training method that leverages relatedness among multiple tasks, for both fine-tuning and pre-training \cite{unilm2019}, and ways of incorporating the linguistic structure of text in a more explicit and controllable manner. At last, we also would like to verify whether MT-DNN is resilience against adversarial attacks \cite{breaknli2019acl,talman2018testing,liu2019mt-dnn-kd}.
%how to combine multi-task and language model into a singe training objective. Both of these could be our future direction.

% Neural approaches are now transforming the field of NLP and IR where symbolic approaches have been dominating for decades \cite{gao2018neural}. Language representation learning is core to all neural approaches.  



\paragraph{3D Object Detection from RGB-D Data} Researchers have approached the 3D detection problem by taking various ways to represent RGB-D data.

\emph{Front view image based methods:} ~\cite{chen2016monocular, mousavian20163d, xiang2015data} take monocular RGB images and shape priors or occlusion patterns to infer 3D bounding boxes. ~\cite{li2016vehicle, deng2017amodal} represent depth data as 2D maps and apply CNNs to localize objects in 2D image. In comparison we represent depth as a point cloud and use advanced 3D deep networks (PointNets) that can exploit 3D geometry more effectively.

\emph{Bird's eye view based methods:} MV3D~\cite{cvpr17chen} projects LiDAR point cloud to bird's eye view and trains a region proposal network (RPN~\cite{ren2015faster}) for 3D bounding box proposal. However, the method lags behind in detecting small objects, such as pedestrians and cyclists and cannot easily adapt to scenes with multiple objects in vertical direction.
%Our method shares the idea with~\cite{cvpr17chen} in reducing 3D search cost by 2D search first. What differentiates our method from \cite{cvpr17chen} is that, \hao{???} instead of projecting point cloud to images costing loss in 3D geometry, we directly apply PointNet to point clouds that correspond to the 2D regions. % Besides, our method and MV3D can potentially be combined in the bird's eye setting. 3D proposals from our frustum-based PointNet and MV3D can be combined and our 3D network can also be used for bounding box estimation for point cloud in the bird's eye 2D region.

\emph{3D based methods:} ~\cite{wang2015voting, song2014sliding} train 3D object classifiers by SVMs on hand-designed geometry features extracted from point cloud and then localize objects using sliding-window search. \cite{engelcke2017vote3deep} extends ~\cite{wang2015voting} by replacing SVM with 3D CNN on voxelized 3D grids. \cite{ren2016three} designs new geometric features for 3D object detection in a point cloud. \cite{song2016deep, li20163d} convert a point cloud of the entire scene into a volumetric grid and use 3D volumetric CNN for object proposal and classification. Computation cost for those method is usually quite high due to the expensive cost of 3D convolutions and large 3D search space.
%In comparison, we use 2D region proposals from RGB images to reduce the search space from the entire 3D scenes into 3D frustums. Since the points cloud in the frustums have largely varying depth ranges and can be very sparse, it's not applicable to apply CNN on bird's eye view or apply 3D CNN in grids. Our frustum-based PointNet, on the other hand, suits well for this type of data and is able to accurately estimate 3D bounding box with good efficiency.
Recently, \cite{lahoud20172d} proposes a 2D-driven 3D object detection method that is similar to ours in spirit. However, they use hand-crafted features (based on histogram of point coordinates) with simple fully connected networks to regress 3D box location and pose, which is sub-optimal in both speed and performance. In contrast, we propose a more flexible and effective solution with deep 3D feature learning (PointNets).
%In addition we also get 3D instance segmentation as intermediate outputs. Evaluated on SUN-RGBD we show our method is \emph{8.9\%} better than theirs in mAP and \emph{34x} faster at the same time.


% \begin{enumerate}
%     \item ZOOX~\cite{mousavian20163d} image based
%     \item Vote3Deep~\cite{engelcke2017vote3deep} 3d cnn. Recent LIDAR-based methods place 3D windows in 3D voxel grids to score the point cloud
%     \item Voting for Voting~\cite{wang2015voting} Recent LIDAR-based methods place 3D windows in 3D voxel grids to score the point cloud. apply SVM classifers on 3D grids encoded with geometry features
%     \item MV3D~\cite{cvpr17chen}
%     \item VeloFCN~\cite{li2016vehicle} apply convolutional networks to the front view point map in a dense box prediction scheme
%     \item 3DOP~\cite{chen20153d} image based. reconstructs depth from stereo images and uses an energy minimization approach to generate 3D box proposals, which are fed to an R-CNN [10] pipeline for object recognition
%     \item Mono3D~\cite{chen2016monocular} image based. shares the same pipeline with 3DOP, it generates 3D proposals from monocular images.
%     \item 3DFCN~\cite{li20163d} 3d cnn.
%     \item 3DVP~\cite{xiang2015data} introduces 3D voxel patterns and employ a set of ACF detectors to do 2D detection and 3D pose estimation
%     \item Are Cars just 3D Box?~\cite{zeeshan2014cars} fit model to image patch
%     \item ~\cite{zia2013detailed} fit model to image patch
% \end{enumerate}
% \begin{enumerate}
%     \item SlidingShapes~\cite{song2014sliding} apply SVM classifers on 3D grids encoded with geometry features
%     \item DeepSlidingShapes~\cite{song2015sun} 3d cnn.
%     \item 2D-driven~\cite{lahoud20172d}
%     \item ~\cite{deng2017amodal} rgb-d images
%     \item COG feature~\cite{ren2016three}
%     \item Align 3D model in RGB-D~\cite{gupta2015aligning}
% \end{enumerate}

\paragraph{Deep Learning on Point Clouds}
Most existing works convert point clouds to images or volumetric forms before feature learning. \cite{wu20153d, maturana2015voxnet, qi2016volumetric} voxelize point clouds into volumetric grids and generalize image CNNs to 3D CNNs. ~\cite{li2016fpnn, riegler2016octnet, wang2017cnn, engelcke2017vote3deep} design more efficient 3D CNN or neural network architectures that exploit sparsity in point cloud.
However, these CNN based methods still require quantitization of point clouds with certain voxel resolution.
Recently, a few works~\cite{qi2017pointnet,qi2017pointnetplusplus} propose a novel type of network architectures (PointNets) that directly consumes raw point clouds without converting them to other formats. While PointNets have been applied to single object classification and semantic segmentation, our work explores how to extend the architecture for the purpose of 3D object detection.

%\section*{Acknowledgments}

% include your own bib file like this:
%\bibliographystyle{acl}
%\bibliography{acl2018}
\bibliography{acl_snli}
\bibliographystyle{acl_natbib}
%\appendix
%
%\section{Supplemental Material}
%\label{sec:supplemental}

\end{document}

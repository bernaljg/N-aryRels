%%%%%%%%%%%%%%%%%%%%%%%%%%%%%%%%%%%%%%%%%%%%%%%%%%%%%%%%%%%%%%%%%%
%%%%%%%% ICML 2016 EXAMPLE LATEX SUBMISSION FILE %%%%%%%%%%%%%%%%%
%%%%%%%%%%%%%%%%%%%%%%%%%%%%%%%%%%%%%%%%%%%%%%%%%%%%%%%%%%%%%%%%%%

% Use the following line _only_ if you're still using LaTeX 2.09.
%\documentstyle[icml2016,epsf,natbib]{article}
% If you rely on Latex2e packages, like most moden people use this:
\documentclass{article}

% use Times
\usepackage{times}
% For figures
\usepackage{graphicx} % more modern
%\usepackage{epsfig} % less modern
\usepackage{subfigure}

% For citations
\usepackage{natbib}

% For algorithms
\usepackage{algorithm}
\usepackage{algorithmic}

% As of 2011, we use the hyperref package to produce hyperlinks in the
% resulting PDF.  If this breaks your system, please commend out the
% following usepackage line and replace \usepackage{icml2016} with
% \usepackage[nohyperref]{icml2016} above.
\usepackage[draft]{hyperref}

% Packages hyperref and algorithmic misbehave sometimes.  We can fix
% this with the following command.
\newcommand{\theHalgorithm}{\arabic{algorithm}}

% Employ the following version of the ``usepackage'' statement for
% submitting the draft version of the paper for review.  This will set
% the note in the first column to ``Under review.  Do not distribute.''
% \usepackage{icml2016}

\usepackage{url}
\usepackage{graphicx}
\usepackage{wrapfig}
\usepackage{amsmath,amssymb,amsthm}

% Employ this version of the ``usepackage'' statement after the paper has
% been accepted, when creating the final version.  This will set the
% note in the first column to ``Proceedings of the...''
\usepackage[accepted]{icml2016}

\usepackage[algo2e, ruled, noline]{algorithm2e}
\providecommand{\SetAlgoLined}{\SetLine}
\providecommand{\DontPrintSemicolon}{\dontprintsemicolon}
\DontPrintSemicolon
\makeatletter
\newcommand{\pushline}{\Indp}% Indent
\newcommand{\popline}{\Indm}
\makeatother


\setlength{\bibsep}{0.5em}

% The \icmltitle you define below is probably too long as a header.
% Therefore, a short form for the running title is supplied here:
\icmltitlerunning{Dueling Network Architectures for Deep Reinforcement Learning}

\newcommand{\fix}{\marginpar{FIX}}
\newcommand{\new}{\marginpar{NEW}}

%\iclrfinalcopy % Uncomment for camera-ready version

\input{mydef}  % Macros from Kevin Murphy's book.
\DeclareMathOperator*{\argmin}{arg\,min}
\DeclareMathOperator*{\argmax}{arg\,max}

\begin{document}

\twocolumn[
\icmltitle{Dueling Network Architectures for Deep Reinforcement Learning}

% from ICLR
%\author{Ziyu Wang, Nando de Freitas \& Marc Lanctot\\
% \thanks{ Use footnote for providing further information
% about author (webpage, alternative address)---\emph{not} for acknowledging
% funding agencies.  Funding acknowledgements go at the end of the paper.} \\
%Google DeepMind\\
%London, UK\\
%\texttt{\{ziyu,nandodefreitas,lanctot\}@google.com} \\
%}

% It is OKAY to include author information, even for blind
% submissions: the style file will automatically remove it for you
% unless you've provided the [accepted] option to the icml2016
% package.
\icmlauthor{Ziyu Wang}{ziyu@google.com}
% \icmladdress{Google DeepMind, London, UK}
\icmlauthor{Tom Schaul}{schaul@google.com}
% \icmladdress{Google DeepMind, London, UK}
\icmlauthor{Matteo Hessel}{mtthss@google.com}
% \icmladdress{Google DeepMind, London, UK}
\icmlauthor{Hado van Hasselt}{hado@google.com}
% \icmladdress{Google DeepMind, London, UK}
\icmlauthor{Marc Lanctot}{lanctot@google.com}
% \icmladdress{Google DeepMind, London, UK}
\icmlauthor{Nando de Freitas}{nandodefreitas@gmail.com}
\icmladdress{Google DeepMind, London, UK}


% You may provide any keywords that you
% find helpful for describing your paper; these are used to populate
% the "keywords" metadata in the PDF but will not be shown in the document
\icmlkeywords{}

\vskip 0.3in
]

\begin{abstract}
In recent years there have been many successes of using deep representations in reinforcement learning. Still, many of these applications use conventional architectures, such as convolutional networks, LSTMs, or auto-encoders. In this paper, we present a new neural network architecture for model-free reinforcement learning. Our dueling network represents two separate estimators: one for the state value function and one for the state-dependent action advantage function. The main benefit of this factoring is to generalize learning across actions without imposing any change to the underlying reinforcement learning algorithm. Our results show that this architecture leads to better policy evaluation in the presence of many similar-valued actions. Moreover, the dueling architecture enables our RL agent to outperform the state-of-the-art on the Atari 2600 domain. 
%Double DQN method of~\citet{vanHasselt:2015} in 46 out of 57 Atari games.
\end{abstract}


\section{Introduction}
\label{sec:introduction}

\section{Introduction}
\label{sec:intro}

Language modeling is among the important problems that require modeling long-term dependency, with successful applications such as unsupervised pretraining~\citep{dai2015semi,peters2018deep,radford2018improving,devlin2018bert}.
However, it has been a challenge to equip neural networks with the capability to model long-term dependency in sequential data.
Recurrent neural networks (RNNs), in particular Long Short-Term Memory (LSTM) networks~\citep{hochreiter1997long}, have been a standard solution to language modeling and obtained strong results on multiple benchmarks.
Despite the wide adaption, RNNs are difficult to optimize due to gradient vanishing and explosion~\citep{hochreiter2001gradient}, and the introduction of gating in LSTMs and the gradient clipping technique~\citep{graves2013generating} might not be sufficient to fully address this issue.
% ,pascanu2012understanding
Empirically, previous work has found that LSTM language models use 200 context words on average~\citep{khandelwal2018sharp}, indicating room for further improvement.

On the other hand, the direct connections between long-distance word pairs baked in attention mechanisms might ease optimization and enable the learning of long-term dependency~\citep{bahdanau2014neural,vaswani2017attention}.
Recently, \citet{al2018character} designed a set of auxiliary losses to train deep Transformer networks for character-level language modeling, which outperform LSTMs by a large margin.
Despite the success, the LM training in~\citet{al2018character} is performed on separated fixed-length segments of a few hundred characters, without any information flow across segments.
As a consequence of the fixed context length, the model cannot capture any longer-term dependency beyond the predefined context length.
In addition, the fixed-length segments are created by selecting a consecutive chunk of symbols without respecting the sentence or any other semantic boundary.
Hence, the model lacks necessary contextual information needed to well predict the first few symbols, leading to inefficient optimization and inferior performance.
We refer to this problem as \textit{context fragmentation}.

%However, the context length is fixed to hundreds of characters and thus it is not possible to model longer-term dependency. Moreover, it is not clear how the model performs on word-level language modeling data, as the granularity changes.

% Moreover, using auxiliary losses brings additional challenges such as properly tuning the mixture weights and the loss decay schedule.

To address the aforementioned limitations of fixed-length contexts, we propose a new architecture called Transformer-XL (meaning extra long).
We introduce the notion of recurrence into our deep self-attention network. In particular, instead of computing the hidden states from scratch for each new segment, we reuse the hidden states obtained in previous segments.
The reused hidden states serve as memory for the current segment, which builds up a recurrent connection between the segments.
As a result, modeling very long-term dependency becomes possible because information can be propagated through the recurrent connections.
Meanwhile, passing information from the previous segment can also resolve the problem of context fragmentation.
More importantly, we show the necessity of using relative positional encodings rather than absolute ones, in order to enable state reuse without causing temporal confusion.
Hence, as an additional technical contribution, we introduce a simple but more effective relative positional encoding formulation that generalizes to attention lengths longer than the one observed during training.

Transformer-XL obtained strong results on five datasets, varying from word-level to character-level language modeling.
Transformer-XL is also able to generate relatively coherent long text articles with \textit{thousands of} tokens (see Appendix \ref{sec:gen}), trained on only 100M tokens.
% Transformer-XL improves the previous state-of-the-art (SoTA) results from 1.06 to 0.99 in bpc on enwiki8, from 1.13 to 1.08 in bpc on text8, from 20.5 to 18.3 in perplexity on WikiText-103, and from 23.7 to 21.8 in perplexity on One Billion Word.
% Transformer-XL improves the previous state-of-the-art (SoTA) results to 0.99 in bpc on enwiki8, 1.08 in bpc on text8, 18.3 in perplexity on WikiText-103, and 21.8 in perplexity on One Billion Word.
% On small data, Transformer-XL also achieves a perplexity of 54.5 on Penn Treebank without finetuning, which is SoTA when comparable settings are considered.

Our main technical contributions include introducing the notion of recurrence in a purely self-attentive model and deriving a novel positional encoding scheme. These two techniques form a complete set of solutions, as any one of them alone does not address the issue of fixed-length contexts. Transformer-XL is the first self-attention model that achieves substantially better results than RNNs on both character-level and word-level language modeling.

% On WikiText-103, Transformer-XL improves the previous state-of-the-art (SoTA) results from 33 perplexity to 24, with a relative reduction of 27\%. On enwiki8 character-level language modeling, Transformer-XL achieves a SoTA bpc of 1.03, which outperforms \cite{al2018character} by 0.03 with 60+\% fewer parameters. Given a more common model size with 40+M parameters, Transformer-XL achieves a bpc of 1.06, compared to 1.11 by \cite{al2018character}. Transformer-XL also achieves perplexities of 54.5 on Penn Treebank and 29.4 on One Billion Word, which are SoTA when comparable settings are considered.

% Due to the ability of modeling long-range context, our best model uses attention lengths of 1,600 and 3,800 on WikiText-103 and enwiki8 respectively. We also devise a metric called \textit{Relative Effective Context Length} (RECL) that aims to fairly compare the ability of long-range dependency modeling.
% % perform a fair comparison of the gains brought by increasing the context lengths for different models.
% In this setting, Transformer-XL learns a RECL of 900 words on WikiText-103, while the numbers for recurrent networks and Transformer are only 500 and 128.

% We use two methods to quantitatively study the effective lengths of Transformer-XL and the baselines. Similar to \cite{khandelwal2018sharp}, we gradually increase the attention length at test time until no further noticeable improvement ($\sim$0.1\% relative gains) can be observed. Our best model in this settings use attention lengths of 1,600 and 3,800 on WikiText-103 and enwiki8 respectively.
% %In addition, since the effective context length of Transformer-XL can be longer than the attention length due to our recurrent formulation, we devise a metric called \textit{Relative Effective Context Length} (RECL) that aims to perform a fair comparison of the gains brought by increasing the context lengths for different models.
% In addition, we devise a metric called \textit{Relative Effective Context Length} (RECL) that aims to perform a fair comparison of the gains brought by increasing the context lengths for different models.
% In this setting, Transformer-XL learns a RECL of 900 words on WikiText-103, while the numbers for recurrent networks and Transformer are only 500 and 128.


\section{Background}
\label{sec:DRL}

In the simplest seq2seq scenario, we are given a collection of source-target
sequence pairs and tasked with learning to generate
target sequences from source sequences. For instance, we might view machine translation in this way, where in particular we attempt to generate English sentences from (corresponding) French sentences. Seq2seq models are part of the broader class of ``encoder-decoder'' models~\cite{cho14on}, which first use an encoding model to transform a source object into an encoded representation $\boldx$. Many different sequential
(and non-sequential) encoders have proven to be effective for
different source domains. In this work we are agnostic to the
form of the encoding model, and simply assume an abstract source
representation $\boldx$. %In experiments we utilize an attention-based LSTM encoder \cite{} which has shown to be effective for many tasks \cite{}.

Once the input sequence is encoded, seq2seq models generate a target
sequence using a \textit{decoder}. The decoder is tasked with
generating a target sequence of words from a target vocabulary $\mcV$. In particular, words are generated sequentially by conditioning on the input representation $\boldx$ and on the previously generated words or \textit{history}. We use the notation $\pfx{T}$ to refer to an arbitrary word sequence of length $T$, and the notation $\goldpfx{T}$ to refer to the \textit{gold} (i.e., correct) target word sequence for an input $\boldx$. 

Most seq2seq systems utilize a recurrent neural network (RNN) for the decoder model. Formally, a recurrent neural network is a parameterized non-linear
function $\RNN$ that recursively maps a sequence of vectors to a
sequence of hidden states. Let $\boldm_1, \ldots, \boldm_T$ be a
sequence of $T$ vectors, and let $\boldh_0$ be some initial state
vector. Applying an RNN to any such sequence yields hidden states
$\boldh_t$ at each time-step $t$, as follows:
\begin{align*}
\boldh_t \gets \RNN(\boldm_t, \boldh_{t-1}; \btheta),
\end{align*}
where $\btheta$ is the set of model parameters, which are shared over time. In this work, the vectors $\boldm_t$ will always correspond to the embeddings of a target word sequence $\pfx{T}$, and so we will also write $\boldh_t \gets \RNN(w_t, \boldh_{t-1}; \btheta)$, with $w_t$ standing in for its embedding.
 
%To back-propagate errors through a recurrent neural network, we accumulate the 
%gradients of each state with respect to subsequent states by running a backward procedure we will refer to as $\BRNN$ at each time-step (starting at the penultimate step): 
%\begin{align*}
%\nabla_{\boldh_t} \mcL \gets \BRNN(y_{t+1}, \boldh_{t},\nabla_{\boldh_{t+1}} \mcL),
%\end{align*}
%$\BRNN$ takes into account $\boldh_t$'s contribution to any loss incurred from its next-step prediction, as well as to any loss incurred through $\boldh_{t+1}$. In what follows, we will often abbreviate $\nabla_{\boldh_t} \mcL$ as $\nabla_{\boldh_t}$.  
%%\begin{align*}
%%\nabla_{\boldh_t} \mcL \gets \nabla_{\boldh_t} \mcL + \BRNN(\nabla_{\boldh_{t+1}} \mcL, \boldm_t, \boldh_{t}).
%%\end{align*}
%%Note that $\boldm_t$ is the embedding corresponding to output word $w_t$. 
%Running this $\BRNN$ procedure from $t \niceq T$ to $t \niceq 1$ is known as back-propagation through time (BPTT).

%\textbf{something about BPTT}

%  which takes the form of a recurrent
% neural network (RNN). 

% where a
% decoder RNN generates a target sequence of T
% words w1 · · · wT (such as a translation or summary),
% from an

% As RNN decoding is the main focus of this work,
% we now describe this process in greater detail.  

RNN decoders are typically trained to act as conditional language
models. That is, one attempts to model the probability of the $t$'th target
word conditioned on $\boldx$ and the target history by stipulating that $p(w_{t} | \pfx{t-1}, \boldx) \niceq g(w_{t},
\boldh_{t-1}, \boldx)$, for some parameterized function $g$ typically computed with an affine layer followed by a softmax. In computing these probabilities, the state $\boldh_{t-1}$ represents the target history, and $\boldh_0$ is typically set to be some function of $\boldx$. The complete model (including encoder) is trained,
analogously to a neural language model, to minimize the cross-entropy
loss at each time-step while conditioning on the gold history in the
training data. That is, the model is trained to minimize $-\ln \prod_{t=1}^{T} p(y_{t} |\goldpfx{t-1}, \boldx)$.

Once the decoder is trained, discrete sequence generation can be
performed by approximately maximizing the probability of the target
sequence under the conditional distribution,
$\hat{y}_{1:T} \niceq \mathrm{argbeam}_{w_{1:T}} \prod_{t=1}^{T} p(w_t |\pfx{t-1}, \boldx)$, where we use the notation $\mathrm{argbeam}$ to emphasize that the decoding process requires heuristic search, since the RNN model is non-Markovian. In practice, a simple beam search
procedure that explores $K$ prospective histories at each time-step
has proven to be an effective decoding approach. However, as noted above,
decoding in this manner after conditional language-model style training \textit{potentially} suffers from the issues of exposure bias and label bias, which motivates the work of this paper.

% However we note that this procedure potentially
% suffers from the issues 


% and we will often omit the
% $\boldx$ argument to $f$ when there is only a single $\boldx$ in
% question.


  

% , which often takes the form of a recurrent
% neural network. 



% For the sake of this work the sequential form of the input sequence is
% actually 


% Seq2seq is highly related to the corresponding 
% \textit{encoder-decoder} approached  



%  $w_{1:s}$ 
% $w_{1:t}$


% It has become popular in recent years to use RNNs within an
% ``encoder-decoder'' framework, where a decoder RNN generates a target
% sequence of $T$ words $\longpfx{T}$ (such as a translation or
% summary), from an 


%  The methods we describe below are designed
% specifically for encoder-decoder scenarios where the decoder is an
% RNN; we make no assumption about the encoder.



% \noindent \textbf{RNNs:} A recurrent neural network is a parameterized
% non-linear function $\RNN$ that recursively maps a sequence of vectors
% to a sequence of hidden states. Let $\boldm_1, \ldots, \boldm_t$ be a
% sequence of $t$ vectors, and let $\boldh_0$ be some initial state
% vector. Applying an RNN to any such sequence yields hidden states
% $\boldh_t$ at each time-step, as follows:
% %{\small
% \begin{align*}
% \boldh_t \gets \RNN(\boldm_t, \boldh_{t-1}; \btheta),
% \end{align*}
% %}
% \noindent where $\btheta$ is the set of model parameters, which are shared over time. 


%Accordingly, we consider the generation of target word sequences $\longpfx{T}$ of length $T$, where we have used $\cdot$ as the concatenation operator, and where each word token $w_j$ comes from our target vocabulary $\mcV$. We denote by $\boldx$ the input representation on which the target generation conditions. We refer to the \textit{gold} (i.e., correct) output word sequence for an input $\boldx$ as $\longgoldpfx{T}$. We will often abbreviate sequences  $\longpfx{T}$ as $\pfx{T}$. % (and similarly for $\longgoldpfx{T}$ and $\goldpfx{T}$).\\ %, and we refer to set of all possible $\boldx$'s as $\mcX$.  \\

% When using an RNN decoder, it is typical to model the probability of
% the $t\,{+}\,1$'st target word's type being $w$ given the preceding
% words and the input as a function of $\boldh_t$. That is, one
% stipulates that $p(w_{t+1} \niceq w|\pfx{t}, \boldx) \propto g(w,
% \boldh_t, \boldx)$, for some function $g$ that examines the hidden
% state at time $t$ and $\boldx$. It is then natural to train such a
% model with a cross-entropy loss at each time-step. In this paper we
% will instead be interested in modeling non-probabilistic scores of
% arbitrary \textit{sequences} formed from the target vocabulary
% $\mcV$. We will accordingly define the score of an entire
% \textit{prefix} $\pfx{t}$ followed by a single word $w$ as
% \begin{align} \label{eq:score}
% \score(\pfx{t} \cdot w) \triangleq f(w, \boldh_t, \boldx),
% \end{align} 
% where, analogously, $f$ is some function examining the current hidden-state of the relevant RNN at time $t$ as well as the input representation $\boldx$. Note that we use $\cdot$ as the concatenation operator.  



\section{The Dueling Network Architecture}
\label{sec:dueling}

%!TEX root = main.tex
%TODO(@Ziyu): Where are the nonlinearities? Also, did you use pooling? Did you explain more precisely in results section?

The key insight behind our new architecture, as illustrated in Figure~\ref{fig:saliency}, is that for many states, it is unnecessary to estimate the value of each action choice. For example, in the Enduro game setting, knowing whether to move left or right only matters when a collision is eminent. In some states, it is of paramount importance to know which action to take, but in many other states the choice of action has no repercussion on what happens. For bootstrapping based algorithms, however, the estimation of state values is of great importance for every state.

To bring this insight to fruition, we design a single $Q$-network architecture, as illustrated in Figure~\ref{fig:duelnet}, which we refer to as the dueling network. The lower layers of the dueling network are convolutional as in the original DQNs \cite{Mnih:2015}. However, instead of following the convolutional layers with a single sequence of fully connected layers, we instead use two sequences (or streams) of fully connected layers. The streams are constructed such that they have they have the capability of providing separate estimates of the value and advantage functions. Finally, the two streams are combined to produce a single output $Q$ function. As in \cite{Mnih:2015}, the output of the network is a set of $Q$ values, one for each action.

Since the output of the dueling network is a $Q$ function, it can be trained with the many existing algorithms, such as DDQN and SARSA. In addition, it can take advantage of any improvements to these algorithms, including better replay memories, better exploration policies, intrinsic motivation, and so on. 

The module that combines the two streams of fully-connected layers to output a $Q$ estimate requires very thoughtful design.  

From the expressions for advantage $Q^{\pi}(s,a) = V^{\pi}(s) + A^{\pi}(s,a)$ and state-value $V^{\pi}(s) = \expectQ{Q^{\pi}(s,a)}{a \sim \pi(s)}$, it follows that $\expectQ{A^{\pi}(s,a)}{a \sim \pi(s)} = 0$. Moreover, for a deterministic policy, $a^* = \argmax_{a' \in \mathcal{A}} Q(s,a')$, it follows that $Q(s,a^*) = V(s)$ and hence $A(s,a^*)=0$.

Let us consider the dueling network shown in Figure~\ref{fig:duelnet}, where we make one stream of fully-connected layers output a scalar ${V}(s;\theta,\beta)$, and the other stream output an $|\mathcal{A}|$-dimensional vector ${A}(s,a;\theta,\alpha)$. Here, $\theta$ denotes the parameters of the convolutional layers, while $\alpha$ and $\beta$ are the parameters of the two streams of fully-connected layers.


Using the definition of advantage, we might be tempted to construct the aggregating module as follows:
\be
Q(s,a;\theta,\alpha,\beta) =  {V}(s;\theta,\beta) + {A}(s,a;\theta,\alpha),
\label{eq:combo1}
\ee
Note that this expression applies to all $(s,a)$ instances; that is, to express equation~(\ref{eq:combo1}) in matrix form we need to replicate the scalar, ${V}(s;\theta,\beta)$, $|\mathcal{A}|$ times.

However, we need to keep in mind that $Q(s,a;\theta,\alpha,\beta)$ is only a parameterized estimate of the true $Q$-function. Moreover, it would be wrong to conclude
that ${V}(s;\theta,\beta)$ is a good estimator of the state-value function, or likewise that ${A}(s,a;\theta,\alpha)$ provides a reasonable estimate of the advantage function. 

Equation~(\ref{eq:combo1}) is unidentifiable in the sense that given $Q$ we cannot recover ${V}$ and ${A}$ uniquely. To see this, add a constant to ${V}(s;\theta,\beta)$ and subtract the same constant from ${A}(s,a;\theta,\alpha)$. This constant cancels out resulting in the same $Q$ value. 
% It is therefore not necessarily true that $V(s;\theta,\beta) = \max_a Q(s,a;\theta,\alpha,\beta)$ when acting according to the policy $Q(s,a;\theta,\alpha,\beta)$. 
This lack of identifiability is mirrored by poor practical performance when this equation is used directly.

To address this issue of identifiability, we can force the advantage function estimator to have zero advantage at the chosen action. That is, we let the last module of the network implement the forward mapping
\begin{multline}
Q(s,a;\theta,\alpha,\beta) =  {V}(s;\theta,\beta)~+\\
\left({A}(s,a;\theta,\alpha) - \max_{a' \in |\mathcal{A}|}  {A}(s, a' ;\theta,\alpha) \right).
\label{eq:combo3}
\end{multline}
Now, for $a^* = \argmax_{a' \in \mathcal{A}} Q(s,a';\theta,\alpha,\beta) = \argmax_{a' \in \mathcal{A}} A(s,a';\theta,\alpha)$, we obtain $Q(s,a^*;\theta,\alpha,\beta) =  {V}(s;\theta,\beta)$. Hence, the stream ${V}(s;\theta,\beta)$ provides an estimate of the value function, while the other stream produces an estimate of the advantage function.

An alternative module replaces the max operator with an average:
\begin{multline}
Q(s,a;\theta,\alpha,\beta) =  {V}(s;\theta,\beta)~+\\
\left({A}(s,a;\theta,\alpha) - \frac{1}{|\mathcal{A}|} \sum_{a'} {A}(s, a' ;\theta,\alpha) \right).
\label{eq:combo2}
\end{multline}
On the one hand this loses the original semantics of $V$ and $A$ because they are now off-target by a constant, but on the other hand it increases the stability of the optimization: with (\ref{eq:combo2}) the advantages only need to change as fast as the mean, instead of having to compensate any change to the optimal action's advantage in (\ref{eq:combo3}). We also experimented with a softmax version of equation (\ref{eq:combo3}), but found it to deliver similar results to the simpler module of equation (\ref{eq:combo2}). Hence, all the experiments reported in this paper use the module of equation (\ref{eq:combo2}). 

Note that while subtracting the mean in equation (\ref{eq:combo2}) helps with identifiability, it does not change the relative rank of the ${A}$ (and hence $Q$) values, preserving any greedy or $\epsilon$-greedy policy based on $Q$ values from equation (\ref{eq:combo1}).
When acting, it suffices to evaluate the advantage stream to make decisions.

It is important to note that equation (\ref{eq:combo2}) is viewed and implemented as part of the network and not as a separate algorithmic step. Training of the dueling architectures, as with standard $Q$ networks (e.g. the deep $Q$-network of \citet{Mnih:2015}), requires only back-propagation. The estimates ${V}(s;\theta,\beta)$ and ${A}(s,a;\theta,\alpha)$ are computed automatically without any extra supervision or algorithmic modifications. 

As the dueling architecture shares the same input-output interface with standard $Q$ networks, 
we can recycle all learning algorithms with $Q$ networks (\emph{e.g.}, DDQN and SARSA) to train the dueling architecture.

% With the dueling network, we learn the value stream with every update to the $Q$ values.
% This frequent updating, enables the dueling network to approximate the state
% values better.
% State value estimation is especially important to bootstrapping-based RL algorithms 
% \cite{SuttonBarto:1998}.

% The dueling architecture could also preserve the advantage orders
% while allowing changes in state value.



%\be
%Q(s,a;\theta,\alpha,\beta) =  V(s;\theta,\beta) + \left( A(s,a;\theta,\alpha) - \sum_{a'} \sigma[Q(s,a;\theta,%\alpha,\beta)] A(s,a';\theta,\alpha) \right) \label{eq:combo3} 
%\ee

%\be
%Q(s,a;\theta,\alpha,\beta) =  V(s;\theta,\beta) + \left( A(s,a;\theta,\alpha) - \frac{1}{N_a} \sum_{a'} A(s,a;\theta,\alpha) \right)
%\label{eq:combo4}
%\ee

%TODO: NNET EQUATION AND DIAGRAM

%TODO: GRADIENT CLIPPING, SOFTMAX, MAX, ETC.


\section{Experiments}
\label{sec:experiments}


% !TEX root = ../multi_task.tex

We evaluate the presented MTL method on a number of problems. First, we use MultiMNIST \citep{multi_mnist}, an MTL adaptation of MNIST \citep{mnist}. Next, we tackle multi-label classification on the CelebA dataset \citep{celeba} by considering each label as a distinct binary classification task. These problems include both classification and regression, with the number of tasks ranging from 2 to 40. Finally, we experiment with scene understanding, jointly tackling the tasks of semantic segmentation, instance segmentation, and depth estimation on the Cityscapes dataset \citep{cityscapes}. We discuss each experiment separately in the following subsections.

The baselines we consider are (i) \textbf{uniform scaling:} minimizing a uniformly weighted sum of loss functions \mbox{$\frac{1}{T}\sum_t \lL^t$}, \mbox{(ii) \textbf{single task:}} solving tasks independently, \mbox{(iii) \textbf{grid search:}} exhaustively trying various values from $\{ c^t \in [0,1] | \sum_t c^t = 1\}$ and optimizing for $\frac{1}{T}\sum_t c^t \lL^t$, \mbox{(iv) \textbf{\citet{Kendall2018}:}} using the uncertainty weighting proposed by \citet{Kendall2018}, and \mbox{(v) \textbf{GradNorm:}} using the normalization proposed by \citet{Chen2018}.



\subsection{MultiMNIST}
\label{sec:multi_mnist_exp}

Our initial experiments are on MultiMNIST, an MTL version of the MNIST dataset \citep{multi_mnist}. In order to convert digit classification into a multi-task problem, \citet{multi_mnist} overlaid multiple images together. We use a similar construction. For each image, a different one is chosen uniformly in random. Then one of these images is put at the top-left and the other one is at the bottom-right. The resulting tasks are: classifying the digit on the top-left (task-L) and classifying the digit on the bottom-right (task-R). We use 60K examples and directly apply existing single-task MNIST models. The MultiMNIST dataset is illustrated in the supplement.

We use the LeNet architecture \citep{mnist}. We treat all layers except the last as the representation function $g$ and put two fully-connected layers as task-specific functions (see the supplement for details). We visualize the performance profile as a scatter plot of accuracies on task-L and task-R in Figure~\ref{fig:multi_mnist_performance_curve}, and list the results in Table~\ref{tab:multi_mnist}.

In this setup, any static scaling results in lower accuracy than solving each task separately (the single-task baseline). The two tasks appear to compete for model capacity, since increase in the accuracy of one task results in decrease in the accuracy of the other. Uncertainty weighting \citep{Kendall2018} and GradNorm \citep{Chen2018} find solutions that are slightly better than grid search but distinctly worse than the single-task baseline. In contrast, our method finds a solution that efficiently utilizes the model capacity and yields accuracies that are as good as the single-task solutions. This experiment demonstrates the effectiveness of our method as well as the necessity of treating MTL as multi-objective optimization. Even after a large hyper-parameter search, \emph{any} scaling of tasks does not approach the effectiveness of our method.



\subsection{Multi-Label Classification}

\begin{figure}[t]
\includegraphics[width=\textwidth]{radar_full_new}
\vspace{1mm}
\caption{Radar charts of percentage error per attribute on CelebA \citep{celeba}. Lower is better. We divide attributes into two sets for legibility: easy on the left, hard on the right. Zoom in for details.}
\label{fig:multi_label_radar}
\end{figure}


\begin{wraptable}{r}{0.3\textwidth}
%\vspace{-4mm}
\captionof{table}{Mean of error per category of MTL algorithms in multi-label classification on CelebA \citep{celeba}.}
\begin{tabular}{r@{\hspace{2mm}}c@{}}
\toprule
& Average  \\
&  error \\
\midrule
Single task & $8.77$ \\
Uniform scaling & $9.62$ \\
\citealt{Kendall2018} & $9.53$ \\
GradNorm & $8.44$ \\
Ours & $\mathbf{8.25}$  \\
\bottomrule
\end{tabular}
\label{table:multi_label_bar}
%\vspace{-5mm}
\end{wraptable}

Next, we tackle multi-label classification. Given a set of attributes, multi-label classification calls for deciding whether each attribute holds for the input. We use the CelebA dataset \citep{celeba}, which includes 200K face images annotated with 40 attributes. Each attribute gives rise to a binary classification task and we cast this as a 40-way MTL problem. We use ResNet-18 \citep{resnet} without the final layer as a shared representation function, and attach a linear layer for each attribute (see the supplement for further details).


We plot the resulting error for each binary classification task as a radar chart in Figure~\ref{fig:multi_label_radar}. The average over them is listed in Table~\ref{table:multi_label_bar}. We skip grid search since it is not feasible over 40 tasks. Although uniform scaling is the norm in the multi-label classification literature, single-task performance is significantly better. Our method outperforms baselines for significant majority of tasks and achieves comparable performance in rest. This experiment also shows that our method remains effective when the number of tasks is high.


\subsection{Scene Understanding}

To evaluate our method in a more realistic setting, we use scene understanding. Given an RGB image, we solve three tasks: semantic segmentation (assigning pixel-level class labels), instance segmentation (assigning pixel-level instance labels), and monocular depth estimation (estimating continuous disparity per pixel). We follow the experimental procedure of \citet{Kendall2018} and use an encoder-decoder architecture. The encoder is based on ResNet-50 \citep{resnet} and is shared by all three tasks. The decoders are task-specific and are based on the pyramid pooling module \citep{pspnet} (see the supplement for further implementation details).

Since the output space of instance segmentation is unconstrained (the number of instances is not known in advance), we use a proxy problem as in \citet{Kendall2018}. For each pixel, we estimate the location of the center of mass of the instance that encompasses the pixel. These center votes can then be clustered to extract the instances. In our experiments, we directly report the MSE in the proxy task. Figure~\ref{fig:cityscapes_performance_profile} shows the performance profile for each pair of tasks, although we perform all experiments on all three tasks jointly. The pairwise performance profiles shown in Figure~\ref{fig:cityscapes_performance_profile} are simply 2D projections of the three-dimensional profile, presented this way for legibility. The results are also listed in Table~\ref{tab:cityscapes_results}.

MTL outperforms single-task accuracy, indicating that the tasks cooperate and help each other. Our method outperforms all baselines on all tasks.


\subsection{Role of the Approximation}

In order to understand the role of the approximation proposed in Section~\ref{sec:approximation}, we compare the final performance and training time of our algorithm with and without the presented approximation in Table~\ref{tab:approximation_tradeoff} (runtime measured on a single Titan Xp GPU). For a small number of tasks (3 for scene understanding), training time is reduced by 40\%. For the multi-label classification experiment (40 tasks), the presented approximation accelerates learning by a factor of 25.

On the accuracy side, we expect both methods to perform similarly as long as the full-rank assumption is satisfied. As expected, the accuracy of both methods is very similar. Somewhat surprisingly, our approximation results in slightly improved accuracy in all experiments. While counter-intuitive at first, we hypothesize that this is related to the use of SGD in the learning algorithm. Stability analysis in convex optimization suggests that if gradients are computed with an error $\hat{\nabla}_\btheta \mathcal{L}^t = \nabla_\btheta \mathcal{L}^t + \mathbf{e}^t$ ($\btheta$ corresponds to $\btheta^{sh}$ in (\ref{eq:kkt_opt})), as opposed to $\mathbf{Z}$ in the approximate problem in \ref{eq:approx}, the error in the solution is bounded as $\|\hat{\mathbf{\alpha}} - \mathbf{\alpha} \|_2 \leq \mathcal{O}(\max_t \|\mathbf{e}^t\|_2)$. Considering the fact that the gradients are computed over the full parameter set (millions of dimensions) for the original problem and over a smaller space for the approximation (batch size times representation which is in the thousands), the dimension of the error vector is significantly higher in the original problem. We expect the $l_2$ norm of such a random vector to depend on the dimension.

In summary, our quantitative analysis of the approximation suggests that (i) the approximation does not cause an accuracy drop and (ii) by solving an equivalent problem in a lower-dimensional space, our method achieves both better computational efficiency and higher stability.

  {\small
  \begin{table}[t]
%  \vspace{-4mm}
  \caption{Effect of the MGDA-UB approximation. We report the final accuracies as well as training times for our method with and without the approximation.}
  %\vspace{1mm}
  \centering
  \begin{tabular}{@{}r@{\hspace{3mm}}c@{\hspace{3mm}}c@{\hspace{2mm}}c@{\hspace{2mm}}c@{}c@{\hspace{5mm}}c@{\hspace{2mm}}c@{}}
  \toprule
  & \multicolumn{4}{c}{Scene understanding (3 tasks)} &  & \multicolumn{2}{c}{Multi-label (40 tasks)}  \\
  \cmidrule(r){2-5} \cmidrule(lr){7-8}
                  & Training & Segmentation & Instance  & Disparity      & & Training & Average \\
                 & time     &  mIoU [\%]       & error [px] & error [px] & & time (hour)      & error \\
  \midrule
  Ours (w/o approx.) & $38.6$ & $66.13$ & $10.28$ & $2.59$ & & $429.9$ & $8.33$ \\
  Ours & $\mathbf{23.3}$ & $\mathbf{66.63}$ & $\mathbf{10.25}$ & $\mathbf{2.54}$  & & $\mathbf{16.1}$ & $\mathbf{8.25}$ \\
  \bottomrule
  \end{tabular}
  %\vspace{-2mm}
  \label{tab:approximation_tradeoff}
  \end{table}}



\section{Discussion}
\label{sec:discussion}
The advantage of the dueling architecture lies partly in its ability to learn the state-value function efficiently.
With every update of the $Q$ values in the dueling architecture,
the value stream ${V}$ is updated -- this contrasts with the updates in a single-stream architecture where only the value for one of the actions is updated, the values for all other actions remain untouched. This more frequent updating of the value stream in our approach allocates more resources to $V$, and thus allows for better approximation of the state values, which in turn need to be accurate for temporal-difference-based methods like Q-learning to work~\cite{SuttonBarto:1998}.
This phenomenon is reflected in the experiments, where the advantage of the dueling architecture over single-stream 	$Q$ networks grows when the number of actions is large.

Furthermore, the differences between $Q$-values for a given state are often very small relative to the magnitude of $Q$.
For example, after training with DDQN on the game of Seaquest, the average action gap (the gap between the $Q$ values of the best and the second best action in a given state) across visited states is roughly $0.04$, whereas the average state value across those states is about $15$.
This difference in scales can lead to small amounts of noise in the updates can lead to reorderings of the actions, and thus make the nearly greedy policy switch abruptly.
The dueling architecture with its separate advantage stream is robust to such effects.


% Max vs. Mean


\section{Conclusions}
\label{sec:conclusion}

We introduced a new neural network architecture that decouples value and advantage in deep $Q$-networks, while sharing a common feature learning module. The new dueling architecture, in combination with some algorithmic improvements, leads to dramatic improvements over existing approaches for deep RL in the challenging Atari domain. The results presented in this paper are the new state-of-the-art in this popular domain. 

% Add after accepted
%\subsubsection*{Acknowledgments}
%We would like to thank Hado van Hasselt, Arthur Guez, Vlad Mnih, Nicolas Hess, Marc Bellemare, Georg Ostrovski, Tom Schaul and all the folks at Google DeepMind for making this possible.


\bibliography{deeprl}
\bibliographystyle{icml2016}

\appendix

\onecolumn
\section{Double DQN Algorithm}
\label{sec:ddqn_alg}
%!TEX root = main.tex
\begin{algorithm2e}[h!]
\small
\SetKwInOut{Input}{input}\SetKwInOut{Output}{output}
\Input{$\mathcal{D}$ -- empty replay buffer; $\theta$ -- initial network parameters, $\theta^-$ -- copy of $\theta$}
\Input{$N_r$ -- replay buffer maximum size; $N_b$ -- training batch size; $N^-$ -- target network replacement freq.}
\For{\mbox{episode} $e \in \{ 1, 2, \ldots, M$ \} } {
  Initialize frame sequence $\mathbf{x} \leftarrow ()$ \;
  \For{$t \in \{ 0, 1, \ldots \} $} {
    Set state $s \leftarrow \mathbf{x}$, sample action $a \sim \pi_\mathcal{B}$ \;
    Sample next frame $x^t$ from environment $\mathcal{E}$ given $(s,a)$ and receive reward $r$, and append $x^t$ to $\mathbf{x}$ \;
    {\bf if} {$|\mathbf{x}| > N_f$} {\bf then} delete oldest frame $x_{t_{min}}$ from $\mathbf{x}$ {\bf end} \;
    Set $s' \leftarrow \mathbf{x}$, and add transition tuple $(s, a, r, s')$ to $\mathcal{D}$,\\~~~~~~~~~replacing the oldest tuple if $|\mathcal{D}| \ge N_r$ \;
    Sample a minibatch of $N_b$ tuples $(s, a, r, s') \sim \mbox{Unif}(\mathcal{D})$\; %uniformly from $\mathcal{D}$ \;
    Construct target values, one for each of the $N_b$ tuples: \;
    Define $a^{\max{}}(s'; \theta) = \argmax_{a'} Q(s', a'; \theta)$\;
    $y_j = \left\{ \begin{array}{ll}
         r & \mbox{if $s'$ is terminal}\\
         r + \gamma Q(s', a^{\max{}}(s'; \theta); \theta^-), & \mbox{otherwise}. \end{array} \right.$ \;
    Do a gradient descent step with loss $\|y_j - Q(s, a ; \theta )\|^2$ \;
    Replace target parameters $\theta^- \leftarrow \theta$ every $N^-$ steps\;
  }
}
\caption{Double DQN Algorithm.}\label{alg:ddqn}
\end{algorithm2e}


\section{Positive Definiteness of~$K$}\label{sec:appendixA}
To show that the kernel~$K$ defined in~(\ref{eq:kernel}) is positive definite
(p.d.), we simply use elementary rules from the kernel literature described in
Sections 2.3.2 and 3.4.1 of~\cite{shawe2004}.  A linear combination of p.d. kernels with non-negative weights is also p.d. (see Proposition 3.22
of\cite{shawe2004}), and thus it is sufficient to show that for all $\z,\z'$
in~$\Omega$, the following kernel on $\Omega \to \HH$ is p.d.:
\begin{displaymath}
   (\varphi,\varphi') \mapsto \big\|\varphi(\z)\big\|_\HH  \normH{\varphi'(\z')} e^{-\frac{1}{2\sigma^2} \normH{\tildephi(\z)-\tildephi'(\z')}^2}.
\end{displaymath}
Specifically, it is also sufficient to
show that the following kernel on $\HH$ is p.d.:
\begin{displaymath}
   (\phi,\phi') \mapsto \big\|{\phi}\big\|_\HH  \normH{\phi'} e^{-\frac{1}{2\sigma^2} \normH{\frac{\phi}{\|\phi\|_\HH}-\frac{\phi'}{\|\phi'\|_\HH}}^2}.
\end{displaymath}
with the convention $\phi/\|\phi\|_\HH=0$ if~$\phi=0$.
This is a pointwise product of two kernels and is p.d. when each of the two
kernels is p.d. The first one is obviously p.d.: $(\phi,\phi') \mapsto
\|{\phi}\|_\HH  \normH{\phi'}$. The second one is a composition of the Gaussian
kernel---which is p.d.---, with feature maps $\phi/\|\phi\|_\HH$ of a
normalized linear kernel in~$\HH$.  This composition is p.d. according to
Proposition 3.22, item (v) of~\cite{shawe2004} since the normalization does
not remove the positive-definiteness property.

\section{List of Architectures Reported in the Experiments}\label{appendix:arch}
We present in details the architectures used in the paper in Table~\ref{table:arch}.
\begin{table}[hbtp]
   \centering
   \begin{tabular}{|*{9}{c|}}
      \hline
      Arch. & $N$ & $m_1$  & $p_1$  &  $\gamma_1$ & $m_2$ &  $p_2$ & $S$  &  $\sharp$ param\\
      \hline
      \hline
      \multicolumn{9}{|c|}{MNIST} \\
      \hline
      CKN-GM1 & 2 &  $1 \times 1$  &  12  & 2 &  $3 \times 3$ &  50 &  $4 \times 4$ & $5\,400$\\
      \hline
      CKN-GM2 & 2 &  $1 \times 1$  &  12  & 2 &  $3 \times 3$ &  400 &  $3 \times 3$& $43\,200$ \\
      \hline
      CKN-PM1 & 1 &  $5 \times 5$  &  200  & 2 &  - &  - &  $4 \times 4$  & $5\,000$ \\
      \hline
      CKN-PM2 & 2 &  $5 \times 5$  &  50  & 2 &  $2 \times 2$ &  200 &  $6 \times 6$ & $41\,250$ \\
      \hline
      \hline
      \multicolumn{9}{|c|}{CIFAR-10} \\
      \hline
      CKN-GM & 2 &  $1 \times 1$  &  12  & 2 &  $2 \times 2$ & 800 &  $4 \times 4$ & $38\,400$\\
      \hline
      CKN-PM & 2 &  $2 \times 2$  &  100  & 2 &  $2 \times 2$ &  800 &  $4 \times 4$ & $321\,200$\\
      \hline
      \hline
      \multicolumn{9}{|c|}{STL-10} \\
      \hline
      CKN-GM & 2 &  $1 \times 1$  &  12  & 2 &  $3 \times 3$ & 800 &  $4 \times 4$ & $86\,400$\\
      \hline
      CKN-PM & 2 &  $3 \times 3$  &  50  & 2 &  $3 \times 3$ &  800 &  $3 \times 3$ & $361\,350$\\
      \hline

   \end{tabular}
   \caption{List of architectures reported in the paper. $N$ is the number of layers; $p_1$ and~$p_2$ represent the number of filters are each layer; $m_1$ and~$m_2$ represent the size of the patches~$\NN_1$ and~$\NN_2$ that are of size~$m_1 \times m_1$ and~$m_2 \times m_2$ on their respective feature maps~$\zeta_1$ and~$\zeta_2$; $\gamma_1$ is the subsampling factor between layer 1 and layer 2; $S$ is the size of the output feature map, and the last column indicates the number of parameters that the network has to learn.}
   \label{table:arch}
\end{table}


% %\newpage
% \section{Score Tables}
% %!TEX root = main.tex
\label{sec:appen}

\begin{table}[h!]
\vspace{-0.25cm}
\caption{Normalized scores, starting with {\bf 30 no-op} actions.}
\footnotesize
\begin{center}
\begin{tabular}{l|rrr}

              Games &      DQN &     DDQN &     Duel \\
              \hline 
               Alien &    20.2\% &    51.0\% &{\bf61.4\%}\\
              Amidar &    56.7\% &   104.3\% &{\bf137.1\%}\\
             Assault &   781.0\% &{\bf995.1\%}&   846.5\% \\
             Asterix &    50.0\% &   206.8\% &{\bf337.4\%}\\
           Asteroids &     1.4\% &     0.0\% &{\bf4.5\%}\\
            Atlantis &  1651.2\% &   576.1\% &{\bf2285.3\%}\\
          Bank Heist &    59.7\% &   137.6\% &{\bf216.2\%}\\
         Battle Zone &    79.1\% &    84.2\% &{\bf99.9\%}\\
          Beam Rider &    49.9\% &{\bf81.0\%}&    71.2\% \\
             Berzerk &    18.4\% &    44.0\% &{\bf53.8\%}\\
             Bowling &    19.8\% &{\bf32.7\%}&    30.8\% \\
              Boxing &   732.5\% &   762.1\% &{\bf827.1\%}\\
            Breakout &  1334.5\% &{\bf1449.2\%}&  1194.5\% \\
           Centipede &    25.9\% &    33.4\% &{\bf55.1\%}\\
     Chopper Command &    80.8\% &    76.0\% &{\bf158.2\%}\\
       Crazy Climber &   399.1\% &   425.2\% &{\bf530.1\%}\\
            Defender &   131.3\% &   205.3\% &{\bf248.8\%}\\
        Demon Attack &   659.6\% &  3182.8\% &{\bf3335.0\%}\\
         Double Dunk &   557.7\% &   607.9\% &{\bf866.5\%}\\
              Enduro &    84.7\% &   140.8\% &{\bf262.4\%}\\
       Fishing Derby &   163.8\% &   202.4\% &{\bf260.7\%}\\
             Freeway &   104.0\% &{\bf112.5\%}&     0.1\% \\
           Frostbite &    17.1\% &    37.9\% &{\bf107.9\%}\\
              Gopher &   395.4\% &   676.7\% &{\bf717.5\%}\\
            Gravitar &     9.4\% &     7.5\% &{\bf13.1\%}\\
            H.E.R.O. &    65.1\% &    64.1\% &{\bf66.4\%}\\
          Ice Hockey &    76.9\% &    70.0\% &{\bf96.4\%}\\
          James Bond &   270.1\% &{\bf485.4\%}&   468.8\% \\
            Kangaroo &   241.6\% &   433.8\% &{\bf496.2\%}\\
               Krull &   639.3\% &   592.3\% &{\bf923.1\%}\\
      Kung-Fu Master &   114.8\% &   131.0\% &{\bf151.4\%}\\
 Montezuma's Revenge &{\bf0.0\%}&     0.0\% &     0.0\% \\
         Ms. Pac-Man &    41.8\% &    36.2\% &{\bf89.9\%}\\
      Name This Game &   102.8\% &   144.6\% &{\bf168.1\%}\\
             Phoenix &   119.2\% &   177.3\% &{\bf344.5\%}\\
            Pitfall! &    -0.8\% &     3.0\% &{\bf3.4\%}\\
                Pong &   114.0\% &   117.8\% &{\bf118.2\%}\\
         Private Eye &{\bf0.2\%}&     0.2\% &     0.1\% \\
              Q*Bert &    97.5\% &   112.3\% &{\bf143.4\%}\\
          River Raid &    38.3\% &    85.8\% &{\bf125.6\%}\\
         Road Runner &   504.7\% &   563.2\% &{\bf887.4\%}\\
            Robotank &   631.5\% &   643.7\% &{\bf645.1\%}\\
            Seaquest &    13.8\% &    39.0\% &{\bf119.5\%}\\
              Skiing &    31.6\% &    63.3\% &{\bf64.6\%}\\
             Solaris &{\bf20.3\%}&    16.5\% &     9.1\% \\
      Space Invaders &   101.6\% &   156.3\% &{\bf412.9\%}\\
         Star Gunner &   559.3\% &   620.5\% &{\bf924.0\%}\\
           Surround  &    26.5\% &    43.2\% &{\bf86.9\%}\\
              Tennis &{\bf231.3\%}&     6.8\% &   186.2\% \\
          Time Pilot &    78.4\% &   287.2\% &{\bf487.5\%}\\
           Tutankham &    36.3\% &{\bf132.5\%}&   128.1\% \\
         Up and Down &    84.7\% &   201.1\% &{\bf397.9\%}\\
             Venture &    13.7\% &     8.3\% &{\bf41.9\%}\\
       Video Pinball &  1113.7\% &{\bf1754.3\%}&   555.9\% \\
       Wizard Of Wor &    51.0\% &   165.2\% &{\bf173.9\%}\\
       Yars' Revenge &    29.1\% &    16.7\% &{\bf90.4\%}\\
              Zaxxon &    58.3\% &   110.8\% &{\bf141.3\%}\\
            \hline
            {\bf Mean}   &    227.9\% & 307.3\% &  {\bf373.1}\% \\
            {\bf Median}   &79.1\% & 117.8\% & {\bf151.4}\% \\
\end{tabular}
\end{center}
\end{table}





\end{document}


% This document was modified from the file originally made available by
% Pat Langley and Andrea Danyluk for ICML-2K. This version was
% created by Lise Getoor and Tobias Scheffer, it was slightly modified
% from the 2010 version by Thorsten Joachims & Johannes Fuernkranz,
% slightly modified from the 2009 version by Kiri Wagstaff and
% Sam Roweis's 2008 version, which is slightly modified from
% Prasad Tadepalli's 2007 version which is a lightly
% changed version of the previous year's version by Andrew Moore,
% which was in turn edited from those of Kristian Kersting and
% Codrina Lauth. Alex Smola contributed to the algorithmic style files.
